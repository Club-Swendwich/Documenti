\section{Ordine del giorno}

\begin{itemize}
	\item Presentazione di altre domande riguardo il capitolato
\end{itemize}

\section{Resoconto}

\noindent 
Inizio della call: 16:32 \\

\noindent La call si è tenuta su zoom con il responsabile del capitolato Gregorio Piccoli. Prima della call, il responsabile aveva già fornito di un simil-DB con all'interno i dati per aiutarc.

\subsection{Domande}

Il gruppo ha esposto al responsabile di capitolato le seguenti domande:

\begin{enumerate}
	\item Come è meglio suddividere il tempo tra analisi, realizzazione, test e documentazione?
	\begin{itemize}
		\item Essendo un lavoro di tipo conoscitivo, fissare dei tempi specifici è complesso, proprio per la natura del progetto.
		\item Il responsabile del capitolato ha consigliato di attenersi ad un 20-25\% di analisi, 30-35\% codifica, 20\% test e il resto documentazione. Chiaramente la divisione degli orari da persona a persona è a nostra descrizione.
		\item Inoltre il responsabile ha consigliato di informarsi prima di incominciare ad analizzare i dati, per una piu' efficace visualizzazione
	\end{itemize}

	\item Nei documenti era descritta una analisi dei dati di diverse applicazioni, mentre i dati forniti riguardano solo una applicazione RM.
	\begin{itemize}
		\item Sono piu' applicazioni che usano come portale RM; Le applicazioni utilizzate sono interconnesse tramite esso
		\item Verrà inoltre fornita una versione aggiornata del DB con modifiche per proteggere l'anonimato degli utenti
	\end{itemize}

	\item Con questa tipologia di dati, esistono delle convenzioni per rappresentare tipologie di dati (es Locazione utente tramite IP)?
	\begin{itemize}
		\item La locazione degli IP puo' essere rappresentata tramite apposita mappa, il plotting e rappresentazione dei dati è a carico degli studenti
		\item I provider hanno dei sistemi in grado di mascherare la provenienza dell'utente (es fastweb)
		\item Il blocco dell'IP per provenienza è superfluo, dato che nel caso di un attacco la posizione viene mascherata
	\end{itemize}

	\item L'analisi verrà effettuata ad-hoc per ogni utente?
	\begin{itemize}
		\item La analisi puo' essere eseguita sia in maniera individuale sia in maniera generale, cioè confrontarlo con il resto dei dati
	\end{itemize}
\end{enumerate}

\subsection{Ulteriori suggerimenti}

Il responsabile Gregorio Piccoli ha risposto in maniera velocce e chiara alle varie domande poste, inoltre ci ha suggerito di attendere i dati aggiornati e di fare nota delle tecnologie:

\begin{itemize}
		\item Google fingerprinting: Come google steso è in grado di discernere se l'accesso alla piattaforma è avvenuto tramite dispositivo conosciuto
\end{itemize}


\noindent
Termine della call: 17:08
