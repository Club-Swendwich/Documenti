\section{Ordine del giorno}

\begin{itemize}
	\item Riunione sprint settimanale.
    \item Discussione e assegnazione "to do"$^{G}$ 
    \item Discussione ore individuali
\end{itemize}

\section{Resoconto}

\noindent
Inizio della call: 21:30. \\
\noindent La call si è tenuta su Discord$^{G}$.
\section{Rapporto Settimanale}
Ogni membro del gruppo ha fatto rapporto riguardo a cosa ha lavorato nel decorso della settimana, dove ha rilevato delle problematiche e su cosa ha intenzione lavorare prossimamente.
Questo avviene su consiglio del professor Tullio Vardanega e aiuta a tenere traccia del progresso in maniera regolare.

\subsection{Rapporto}
Durante il meeting il gruppo ha discusso riguardo a:
\begin{itemize}
	\item Avanzamento del prodotto 
	\item Creazione di un prodotto finito e dimostrabile per l'Avanzamento rispettando le specifiche di progetto
	\item Sviluppo del viewcomposer Scatterplot
	\item Portare il renderer del Sankey all'interno della architettura
\end{itemize}
Nello specifico si è tenuta una discussione sulle ore produttive che ogni membro del gruppo ha dedicato
e come questo influenzerà l'avanzamento del progetto. Dopo l'PB ci sarà un cambiamento sulle ore che ogni membro
del gruppo potrà dedicare al progetto, rispettando le ore produttive già date.

\subsection{Discussione "to do"$^{G}$}
Il gruppo, dopo aver discusso sui "to do"$^{G}$ su Trello$^{G}$ ha segnalato le difficoltà che ha riportato e su cosa può migliorare. In particolare:
\begin{itemize}
	\item Creazione di un viewcomposer dello Scatterplot$^{G}$
	\item Utilizzo del csv all'interno del viewcomposer Scatterplot$^{G}$
\end{itemize}

\subsection{Assegnazione "to do"$^{G}$}
\begin{itemize}
	\item I "to do"$^{G}$ sono stati assegnati con lo scopo di avanzare la creazione del viewcomposer e la documentazione.
	\item Ogni elemento del gruppo si è incaricato di continuare a sviluppare le funzionalità e documenti.
	\item Nello specifico la priorità è passata alla finalizzazione del viewcomposer dello Scatterplot$^{G}$.
\end{itemize}

Il meeting è terminato alle 22:10

