\section{Ordine del giorno}

\begin{itemize}
	\item Revisione delle correzioni fornite dai docenti.
	\item Identificazione individuali correzioni.
	\item Assegnazione lavori individuali.
\end{itemize}

\section{Resoconto}
\label{sec:Resoconto}

\noindent 
Inizio della call: 18:30. \\
\noindent La call si è tenuta su Discord$^{G}$.

\subsection{Domande}
Il gruppo ha valutato ogni le parti meno buone segnalate dal docente, per poi correggere e riflettere su 
possibili miglioramenti in futuro
\begin{enumerate}
	\item La consegna include pochi verbali
	\begin{itemize}
		\item Nel futuro sarà quindi previsto un più alto livello d'interazione con 
			  il proponente.
	\end{itemize}
	\item Voci riportate nel registro modifiche devono essere ordinate dal più recente al più antico
	\begin{itemize}
		\item Riportata issue su GitHub e poi automaticamente su Trello.
		\item Nel futuro le modifiche verranno riportate come indicato. 
	\end{itemize}
	\item I documenti ufficiali di progetto devono essere devono in premessa le informazioni riguardanti
		  obiettivo e struttura.
	\begin{itemize}
		\item Riportata issue su GitHub e poi automaticamente su Trello.
		\item I documenti verranno corretti in linea con la correzione.
	\end{itemize}
	\item Il conto economico del preventivo a finire deve essere calcolato sulla base del raggiungimento
		  degli obiettivi e il consumo di risorse utilizzato, invece di sottrarre lo speso al preventivo iniziale.
	\begin{itemize}
		\item Riportata issue su GitHub e poi automaticamente su Trello.
	\end{itemize}	
	\item Cruscotto di valutazione poco popolato d'informazioni e modesto il corredo di 
		  metriche e obiettivi di qualità nel PdQ.
	\begin{itemize}
    	\item Le verifiche verranno eseguite in futuro in più parti del processo esecutivo,
			  assicurando un maggior numero di verifica di obiettivi della qualità
	\end{itemize}	
\end{enumerate}
Il gruppo inoltre ha deciso di correggere anche le raccomandazioni aggiuntive fornite dal docente, per
una correzione completa degli elaborati.
\begin{enumerate}
	\item È preferibile che le date di calendario nel nome di file abbiano il formato aa-mm-gg.
	\begin{itemize}
		\item Riportata issue su GitHub e poi automaticamente su Trello.
		\item Nel futuro ci si accorgerà di utilizzare il formato aa-mm-gg.
	\end{itemize}
	\item La modalità d'ipertesto dei collegamenti PDF è esteticamente sgradevole.
	\item \begin{itemize}
		\item Riportata issue su GitHub e poi automaticamente su Trello.
		\item Nel futuro ci si accorgerà di utilizzare un metodo d'ipertesto più gradevole.
	\end{itemize}
	\item Il termine Scrum va riportato come la iniziale maiuscola.
	\item \begin{itemize}
		\item Riportata issue su GitHub e poi automaticamente su Trello.
		\item Nel futuro ci si accorgerà di utilizzare la grammatica corretta.
	\end{itemize}
	\item Accenti in LaTeX sono spesso scorretti.
	\item \begin{itemize}
		\item Riportata issue su GitHub e poi automaticamente su Trello.
		\item Nel futuro ci si accorgerà di utilizzare la grammatica corretta.
	\end{itemize}
\end{enumerate}

\noindent Termine della call: 19:40.

\section{Assegnamento incarichi}
La assegnazione degli incarichi avviene in modalità Kanban, su Trello ogni membro si assegna individualmente e poi adempie ai compiti assegnati. 
