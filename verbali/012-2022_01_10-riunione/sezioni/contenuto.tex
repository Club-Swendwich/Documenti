\section{Ordine del giorno}

\begin{itemize}
	\item Riunione di gruppo in merito a quanto tratto dalla call con Zucchetti.
\end{itemize}

\section{Resoconto}
\label{sec:Resoconto}

\noindent 
Inizio del meeting: 15:41. \\
\noindent Alla riunione sono rimasti presenti tutti i membri del gruppo precedentemente connessi nel meeting con Zucchetti. \\
Si sono discussi i seguenti argomenti:
\begin{itemize}
	\item Suddivisione del lavoro nei documenti mancanti: Piano di Progetto e Piano di Qualifica.
	\item Discussi i Requisiti funzionali all'interno del documento Analisi dei Requisiti. È stata riscontrata all'interno del gruppo poca chiarezza riguardo a quali elencare e quali invece fossero superflui.
	\item Sono state dunque decise le domande da rivolgere ai docenti Cardin e Vardanega riguardo i documenti svolti.
	\item Si sono discussi gli standard da seguire per la Qualità di Processo e Qualità di Progetto.
\end{itemize}

\noindent Termine del meeting: 16:40.


\section{Tracciamento decisioni}

\begin{itemize}
	\item Il gruppo ha così ripartito il carico di lavoro:
	\begin{itemize}
		\item Beni Valentina, Bustaffa Marco, Barilla Gianmarco, Canel Alessandro, Pozzebon Samuele: termine del documento "Piano di Progetto". Successiva stesura del documento Piano di Qualifica. 
		\item Ferrarini Alessio: aggiunta della parte di caricamento del dato all'interno del "Proof of Concept$^{G}$".
	\end{itemize}
	\item Si è deciso di fissare un meeting con il docente Vardanega per porre le seguenti domande:
	\begin{enumerate}
		\item Gli UC elencati tra i requisiti funzionali sono corretti? Bisogna aggiungere altri UC impliciti come ad esempio il parsing del dataset$^{G}$?
		\item I requisiti funzionali come andranno suddivisi? In requisiti atomici?
		\item Riguardo agli standard di progetto: sono sensati elementi come la velocità di rendering$^{G}$?
		\item Su quali standard ci consiglia di concentrarci?
		\item Conosce qualche standard relativo alle interfacce grafiche?
	\end{enumerate}
	\item Si è deciso di chiedere a lezione al docente Cardin le seguenti domande:
	\begin{enumerate}
		\item È corretto utilizzare gli \textit{extend} per la gestione degli errori negli UC\#
		\item È corretto l'utilizzo del participio passato nei nomi degli UC\#
	\end{enumerate}
	\item È stato aggiornato Trello$^{G}$ con quanto stabilito sopra.
\end{itemize}