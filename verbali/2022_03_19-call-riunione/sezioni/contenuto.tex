\section{Ordine del giorno}
\begin{itemize}
	\item Discussione architettura del prodotto LoginWarriors
	\item Discussione inerente alle tecnologie da utilizzare
	\item Discussione inerente alla realizzazione delle parti fondamentali del 
	prodotto, in maniera particolare la vista
\end{itemize}
\section{Resoconto}
Inizio della call: 9:00. \newline
La call è avvenuta tramite Discord.
\subsection{Architettura}
Nella prima parte della chiamata si è creato, tramite l'utilizzo di un programma
online di disegno UML, un abbozzo della architettura del software. \newline
In particolare questa parte si è divisa in:
\begin{itemize}
	\item Accordo in riguardo a un design pattern Model-view/View-model
	\item Creazione di un primo abbozzo della architettura
	\item Discussione sulle possibili problematiche inerenti a essa
\end{itemize}

\subsection{Tecnologie}
Durante la seconda parte della chiamata si è discusso sulle tecnologie che faranno
parte del progetto e di come implementarle al meglio.\newline
\begin{itemize}
	\item Il gruppo ha introdotto la programmazione reattiva e ha discusso su come implementarla al meglio
	\item È stata delucidato il metodo renderizzare i grafici e come questo influisce sulle altre parti del prodotto 
\end{itemize}

\subsection{Vista}
Infine il gruppo si è soffermato sulla parte del design pattern riguardante la vista. \newline
Nello specifico si è discusso su come:
\begin{itemize}
	\item I grafici devono essere visibili come da specifica con la vista di default iniziale
	\item Si è discusso come associare al meglio le dimensioni estrapolate dai dati ai corrispettivi assi
	\item Infine il gruppo ha ragionato sulle personalizzazioni inerenti ai grafici e come implementarle, considerando
		  le correzioni avvenute durante la correzione post-RTB.
\end{itemize}
\noindent Termine della call: 18:20.