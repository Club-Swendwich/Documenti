\section{Ordine del giorno}

\begin{itemize}
	\item Definizione dei documenti da produrre.
	\item Divisione del lavoro tra i membri del gruppo.
	\item Inizio sviluppo documento ``Analisi dei requisiti''.
\end{itemize}

\section{Resoconto}

\noindent 
Inizio della call: 15:00. \\
\noindent La call si è tenuta su Discord con tutti i componenti del gruppo presenti.
\begin{itemize}
	\item Il gruppo ha ragionato sui primi passi da svolgere per la redazione dei documenti richiesti nella prima revisione di avanzamento, ``Requirements and Technology Baseline''.
	\item Basandosi su slide fornite dal docente e appunti presi a lezione è stato evidenziato come miglior approccio quello di iniziare svolgendo l'analisi dei requisiti e, solo in seguito, procedere ai successivi documenti richiesti.
	\item \' E stata quindi riconfermata la suddivisione tra i membri (già emersa nel canale ufficiale di comunicazione del gruppo) per la redazione di:``Piano di progetto'' e ``Piano di qualifica''.
	\item Il gruppo ha quindi iniziato a valutare i primi casi d'uso da inserire all'interno del documento ``Analisi dei requisiti''.
\end{itemize}

\subsection{Casi d'uso}
\begin{enumerate}
	\item Caricamento dataset da file.
	\item Errore formato del file dataset.
	\item Errore struttura dataset.
	\item Errore entry dataset.
	\item Selezione tipo di grafico.
	\item Selezione set di dati da visualizzare.
	\begin{enumerate}
		\item[6.1] Ritorno a visualizzazione di default.
	\end{enumerate}
	\item Personalizzazione dati dei grafici.
	\item Personalizzazione visiva dei grafici.
\end{enumerate} 

\noindent I casi d'uso finora individuati si riferiscono solo alla parte di progetto obbligatoria. Quella opzionale verrà affrontata in un successivo incontro. \\

\noindent Termine della call: 16:30.

\section{Tracciamento decisioni}

\begin{itemize}
	\item Il gruppo ha confermato la suddivisione del lavoro tra i membri per quanto concerne i documenti ``Piano di progetto'' e ``Piano di qualifica''.
	\item L'ordine stabilito nella produzione dei documenti è il seguente:
		\begin{enumerate}
		\item Analisi dei requisiti
		\item Piano di progetto e Piano di qualifica
		\end{enumerate}
	\item \' E stata individuata una prima lista di casi d'uso da inserire all'interno del documento ``Analisi dei requisiti''.
	\item \' E stata fissata la data per la prossima riunione di progetto: 01/12/2021 ore 15.00.
\end{itemize}