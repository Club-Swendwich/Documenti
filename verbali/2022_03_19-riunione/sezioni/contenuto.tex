\section{Ordine del giorno}

\begin{itemize}
	\item Riunione di gruppo in merito a quanto notato durante la prima iterazione del progetto.
\end{itemize}

\section{Resoconto}
\label{sec:Resoconto}

\noindent 
Inizio del meeting: 18:40. \\
\noindent Alla riunione sono rimasti presenti tutti i membri del gruppo. \\
Si sono discussi i seguenti argomenti:
\begin{itemize}
	\item Difficoltà riportate durante le prime iterazione.
	\item Discussione delle possibili soluzioni a tali problematiche.
	\item Discussione riguardo alla creazione dei diagrammi di attività.
	\item Discussione riguardo all'assegnazione delle ore
\end{itemize}

\noindent Termine del meeting: 19:40.


\section{Difficoltà riportate durante la prima iterazione}

\begin{itemize}
	\item Il gruppo ha rilevato le seguenti problematiche:
	\begin{itemize}
		\item Parti del gruppo non si sono dimostrate abbastanza propositive, sia nella comunicazione con il professore attraverso le mail, ne come contenuto
		di esse. 
		\item Durante le scorse iterazioni, la responsabilità d'inviare le mail è ricaduta su pochi elementi del gruppo, i quali hanno comunicato un desiderio di .
	\end{itemize}
	\item Si è deciso di fissare un meeting con il docente Vardanega per porre le seguenti domande:
	\begin{enumerate}
		\item Gli UC elencati tra i requisiti funzionali sono corretti? Bisogna aggiungere altri UC impliciti come ad esempio il parsing del dataset$^{G}$?
		\item I requisiti funzionali come andranno suddivisi? In requisiti atomici?
		\item Riguardo agli standard di progetto: sono sensati elementi come la velocità di rendering$^{G}$?
		\item Su quali standard ci consiglia di concentrarci?
		\item Conosce qualche standard relativo alle interfacce grafiche?
	\end{enumerate}
	\item Si è deciso di chiedere a lezione al docente Cardin le seguenti domande:
	\begin{enumerate}
		\item È corretto utilizzare gli \textit{extend} per la gestione degli errori negli UC\#
		\item È corretto l'utilizzo del participio passato nei nomi degli UC\#
	\end{enumerate}
	\item È stato aggiornato Trello$^{G}$ con quanto stabilito sopra.
\end{itemize}