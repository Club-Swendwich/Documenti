\section{Ordine del giorno}

\begin{itemize}
    \item Scelta nome e logo del gruppo.
    \item Discussione dei capitolati proposti.
    \item Scelta domande per i proponenti.
    \item Stesura mail per fissare gli appuntamenti con i proponenti.
\end{itemize}

\section{Resoconto}

\begin{itemize}
    \item Vi è stata una votazione tra il nome "Squid Squad" e "Club Swendwich".
    \item Il capitolato C1 risulta essere interessante per il gruppo.
    \item Il capitolato C2 risulta essere interessante per il gruppo.
    \item Il capitolato C3 non risulta essere interessante per il gruppo.
    \item Il capitolato C4 non risulta essere interessante per il gruppo.
    \item Il capitolato C5 risulta essere interessante per il gruppo.
    \item Il capitolato C6 risulta essere interessante per il gruppo.
    \item Sono stati stesi pro e contro dei capitolati interessanti di cui un riassunto: \\
        \begin{tabular}{|p{0.30\linewidth} |p{0.30\linewidth} p{0.30 \linewidth}|}
        \hline
        Capitolato & Pro & Contro                                                                                                                                             \\
        \hline
        C2         & Proponente molto disponibile, Interesse molto alto da tutti i membri del gruppo & Tecnologia Solidity da imparare, Difficile definire MVP \\
        \hline
        C1         & Minore quantità di nuove tecnologie, Facile individuare MVP & Progetto di minore interesse tra i membri del gruppo \\
        \hline
        C5         & Tanti MVP e facili da individuare, Proponente molto disponibile & Nozioni di statistica da apprendere \\
        \hline
        C6         & Capitolato molto chiaro & Non sono una software house e può avere problemi riguardo il supporto, Tecnologie totalemente sconosciute \\
        \hline
        \end{tabular}
    \item Per quanto riguarda il capitolato C6 sono sorti dei dubbi riguardo: librerie già esistenti per la comunicazione con l'UPS, velocità e complessità dei dati trasmessi (limite tempo di render), utilizzo emulatore per testare il software e su quale piattaforme può essere eseguito.
    \item Per quanto riguarda il capitolato C2 sono sorti dei dubbi riguardo: individuazione di MVP, scope del database postgre in un' applicazione distribuita, dati utente salvabili, tecnologia blockchain.
    \item Per quanto riguarda il capitolato C1 sono sorti dei dubbi riguardo: livello di sicurezza richiesto dall'applicazione, complessità del sistema per la comprensione del linguaggio.
    \item Per quanto riguarda il capitolato C5 sono sorti dei dubbi riguardo: conoscenze statistiche necessarie, gestione di grandi quantità di dati, complessità della sezione dedicata al machine learning.
    \item Sono state inviate mail ai proponenti dei capitolati C6, C2, C1 e C5.
\end{itemize}
\section{Tracciamento decisioni}
\begin{itemize}
    \item Il nome del gruppo è "Club Swendwich", è stato deciso il logo.
    \item Si è deciso di proporre un incontro conoscitivo con i proponenti dei capitolati C1, C2, C5, C6.
    \item Ogni membro del gruppo è incaricato di pensare a possibili domande da porre ai vari proponenti.
\end{itemize}