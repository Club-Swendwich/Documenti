\section{Ordine del giorno}

\begin{itemize}
	\item Presentazione di ulteriori domande riguardo il capitolato
\end{itemize}

\section{Resoconto}

\noindent 
Inizio della call: 16:32.\\
\noindent La call si è tenuta su zoom con il responsabile del capitolato Gregorio Piccoli. In precedenza il responsabile aveva già fornito un file CSV contenente dati fax-simile a quelli da analizzare.

\subsection{Domande}
Il gruppo ha esposto al responsabile di capitolato le seguenti domande:

\begin{enumerate}
	\item Come è meglio suddividere il tempo tra analisi, realizzazione, test e documentazione?
	\begin{itemize}
		\item Essendo un lavoro di tipo conoscitivo, fissare dei tempi specifici è complesso, proprio per la natura del progetto
		\item Il proponente, data la natura innovativa del progetto ha consigliato un approccio agile, con tanta verifica e a passi incrementali
		\item Il proponente del capitolato ha inoltre consigliato di attenersi ad un 20-25\% di analisi, 30-35\% codifica, 20\% test e il resto documentazione. Chiaramente la divisione degli orari da persona a persona è a nostra discrezione.
		\item Il proponente ha consigliato di informarsi prima di cominciare ad analizzare i dati, per una piu' efficace visualizzazione
	\end{itemize}

	\item Nei documenti era descritta una analisi dei dati di diverse applicazioni, mentre i dati forniti riguardano solo una applicazione RM.
	\begin{itemize}
		\item Sono piu' applicazioni che usano come portale RM; Le applicazioni utilizzate sono interconnesse tramite esso
		\item Verrà inoltre fornita una versione aggiornata del DB con modifiche per proteggere l'anonimato degli utenti
	\end{itemize}

	\item Esistono delle convenzioni per rappresentare le tipologie di dati considerate dal capitolato? (es. Locazione utente tramite IP)
	\begin{itemize}
		\item La locazione degli IP puo' essere rappresentata tramite apposita mappa, il plotting e la rappresentazione dei dati sono a carico degli studenti
		\item I provider hanno dei sistemi in grado di mascherare la provenienza dell'utente (es fastweb)
		\item Il blocco dell'IP per provenienza è superfluo, dato che nel caso di un attacco la posizione viene mascherata (IP spoofing)
	\end{itemize}

	\item L'analisi verrà effettuata ad-hoc per ogni utente?
	\begin{itemize}
		\item L' analisi puo' avere sia carattere generale che specifico, cioè è possibile considerare i dati sia come unico insieme che come elementi singoli
	\end{itemize}
\end{enumerate}

\subsection{Ulteriori suggerimenti}

Il responsabile Gregorio Piccoli ha risposto in maniera chiara e veloce alle varie domande poste, ha inoltre suggerito di attendere i dati aggiornati e di osservare il funzionamento delle seguenti tecnologie:

\begin{itemize}
		\item Google fingerprinting: Come Google stesso è in grado di discernere se l'accesso alla piattaforma è avvenuto tramite dispositivo conosciuto o meno
\end{itemize}


\noindent
Termine della call: 17:08
