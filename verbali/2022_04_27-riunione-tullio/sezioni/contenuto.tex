\section{Ordine del giorno}
\begin{itemize}
	\item Incontro con professore Tullio Vardanega
	\item Riunione di gruppo dopo Incontro
\end{itemize}

\section{Incontro con professor Tullio Vardanega} 
Nel lunedì antecedente al meeting, il professore ha inviato una mail al gruppo 
in riguardo all'avanzamento architetturale, in quanto era da settimane che non avevamo notizie.\newline 
Il gruppo comunicò che ci sono stati delle importanti problematiche che hanno causato un rallentamento 
nel progetto, e acconsentì a una chiamata il mercoledì stesso, il 27/04/2022. \newline 
 
La chiamata è incominciata alle 14:30 
 
\subsection{Svolgimento chiamata} 
Il professore ha domandato riguardo alle difficoltà che il gruppo ha incontrato, con lo scopo di  
migliorare e correggere le dinamiche che hanno portato al ritardo. 
Il gruppo ha esposto le problematiche di affaticamento e rallentamento dovute alle tecnologie utilizzate e  
una percepita scarsa collaborazione di membri del gruppo. \newline 
Una volta ascoltato il rapporto della situazione da ogni membro del gruppo, il professore ha esposto  
diversi consigli mirati al miglioramento della realizzazione del progetto. I suggerimenti sono i seguenti: 
\begin{itemize} 
 \item La suddivisione equa delle parti di lavoro porta a un rallentamento generale delle parti di lavoro, 
 dato che il gruppo dipende dal proseguimento di ogni parte ripartita per proseguire, portando a rallentamenti 
 \item Una soluzione più adatta sarebbe ripartire con metodologia Kanban$^{G}$ le parti da sviluppare, e che ogni  
 membro del gruppo prenda un "to do"$^{G}$ da svolgere, rendendo l'approccio al lavoro molto più agile 
 \item Il professore ha inoltre commentato sul fatto che una ripartizione meno equa delle ore possa in realtà  
 beneficiare il gruppo, in quanto una tale ripartizione aiuta ogni membro a essere 
 più efficace in quello in cui è più abile 
 \item Il professore ha inoltre suggerito di effettuare una chiamata con il proponente per diminuire la quantità di lavoro, in 
 quanto essendo in cinque la capacità di portare a termine il progetto negli orari prestabiliti è più arduo
 \item Infine il professore ha suggerito di sfoltire la mole dei "to do"$^{G}$ in linea con la chiamata da effettuare
 con il relatore 
\end{itemize} 
La chiamata è finita alle 15:40
\newpage
 
\section {Chiamata gruppo} 
La chiamata è incominciata alle 20:30 . \newline

Nella riunione del gruppo avvenuta dopo si è fatto il punto della situazione, dove il gruppo ha 
riflettuto sulla riunione avvenuta durante il primo pomeriggio. \newline 
 
\subsection{Argomenti chiamata}
Si è particolarmente riflettuto su: 
\begin{itemize} 
 \item Su come l'aiuto del professore sia stato costruttivo e su come ci abbia aiutato ad affrontare 
 il progetto da ora in poi 
 \item Come poter frazionare meglio il lavoro 
 \item Utilizzare di più la board Scrum$^{G}$ per fare spazio alla nuova strategia suggerita dal professore
 \item Sul bisogno di stillare una lista di "to do"$^{G}$ 
 \item Aggiornare i documenti in maniera da riflettere lo stato di avanzamento corrente
 \item Sul organizzare una riunione con il referente della Zucchetti 
 \item Organizzare una riunione Lunedì 02/05/2022, dove ognuno elencherà i "to do"$^{G}$ per poi organizzarsi al meglio 
\end{itemize} 
Il gruppo ha poi considerato la proposizione di lasciare due grafici (Sankey e Parallel Coordinates) da parte, 
per concentrarsi sul portare il resto del prodotto avanti. \newline

La chiamata è terminata alle 21:00 . \newline
 
