\section{Ordine del giorno}

\begin{itemize}
	\item Domande relative ai dubbi sorti durante la stesura della documentazione.
\end{itemize}

\section{Resoconto}
\label{sec:Resoconto}

\noindent 
Inizio della call: 10:00. \\
\noindent La call si è tenuta su Zoom$^{G}$.

\subsection{Domande}
Il gruppo ha esposto al Prof. Vardanega le seguenti domande:

\begin{enumerate}
	\item Quanto nel dettaglio è necessario spingersi per ricercare i casi d'uso e i requisiti funzionali?
	\begin{itemize}
		\item Bisogna spingersi nel dettaglio fino al punto in cui i requisiti risultano ovvi e la loro descrizione può rimanere implicita.
			  Di conseguenza casi diversi possono avere diversi livelli di dettaglio.
	\end{itemize}
	\item Tra i diversi standard IEEE presenti ce n'è qualcuno su cui conviene concentrarsi?
	\begin{itemize}
		\item Nel piano di qualifica si valutano efficienza ed efficacia relativi allo sviluppo del progetto. 
		È importante trovare il metodo di valutazione più idoneo alle esigenze del gruppo, in modo da avere una visione qualitativa sull'andamento del progetto.
	\end{itemize}
	\item Nel Piano di Progetto sono state parallelizzate la parte di Analisi e quella di Technology baseline, è corretto?
	\begin{itemize}
		\item È ragionevole che dentro ad uno o più periodi ci siano obbiettivi paralleli se il loro conseguimento è funzionale per un altro obbiettivo.
	\end{itemize}

	
\end{enumerate}


\noindent Termine della call: 10:50.

\section{Tracciamento decisioni}


\noindent Durante la riunione, il gruppo ha potuto trovare alcuni nuovi requisiti funzionali: 
\begin{itemize}
	\item Caricamento file CSV$^{G}$
	\begin{itemize}
		\item Parsing$^{G}$ file CSV$^{G}$
		\item Controllo di colonne possibilmente derivabili
		\item Creazione colonne derivate
	\end{itemize}
	\item Render dei grafici
	\begin{itemize}
		\item Render$^{G}$ di Scatter Plot$^{G}$
		\item Render$^{G}$ di Parallel Coordinates$^{G}$
		\item Render$^{G}$ di Forced-directed Graph$^{G}$
		\item Render$^{G}$ di Sankey Diagram$^{G}$
		\item Normalizzazione dei valori per gli assi
	\end{itemize}
	\item Caricamento viste$^{G}$
	\begin{itemize}
		\item Parsing$^{G}$ viste$^{G}$
		\item Serializzazione delle viste$^{G}$
	\end{itemize}
\end{itemize}
