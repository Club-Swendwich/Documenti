\section{Ordine del giorno}

\begin{itemize}
	\item Presentazione di dubbi e domande inerenti al progetto, sorti durante la redazione del documento ``Analisi dei requisiti''.
	\item Proposta di utillizzo di mezzi comunicativi alternativi alla e-mail.
	\item Richiesta di libreria ML$^{G}$ per analisi dei dati.
\end{itemize}

\section{Resoconto}

\noindent 
Inizio della call: 15:00. \\
\noindent La call si è tenuta su Zoom$^{G}$ con tutti i componenti del gruppo presenti.
\subsection{Domande}
Il gruppo ha esposto al responsabile di capitolato$^{G}$ le seguenti domande:
\begin{enumerate}
	\item È possibile creare dei ``\textit{preset}'', decisi a seguito dell'analisi dei dati, nei quali le dimensioni che ha senso plottare sono già definite, lasciando così all'utente finale (sistemista) soltanto la possibilità di personalizzazione estetica, oppure è meglio lasciare all'utente anche la possibilità di scegliere le dimensioni?
	\begin{itemize}
		\item L'idea del preset è corretta, tuttavia sarebbe meglio che gli utenti avessero anche la possibilità di esplorare ed in questo modo creare dei propri preset.
		\item L'utente dovrebbe inoltre poter salvare il proprio preset.
	\end{itemize}
	\item È possibile aggiungere una casella di testo per permettere all'utente di inserire/modificare i dati correnti tramite codice \textit{javascript} ed esplorare così nuove combinazioni?
	\begin{itemize}
		\item Sarebbe meglio evitarlo, in quanto rappresenta una possibile falla di sicurezza in caso di attacchi \textit{Cross-Site Scripting}.
	\end{itemize}
	\item È possibile utilizzare un metodo di comunicazione alternativo (più rapido delle e-mail) come ad esempio \textit{Slack}?
	\begin{itemize}
		\item No, è preferibile continuare una corrispondenza mediante posta elettronica.
	\end{itemize}
	
\end{enumerate}

Termine della call: 15.40.

\section{Tracciamento decisioni}

\begin{itemize}
	\item Il gruppo ha evidenziato i seguenti nuovi casi d'uso:
	\begin{itemize}
		\item UC - Creazione di una nuova vista$^{G}$.
		\item UC - Salvataggio vista$^{G}$.
		\item UC - Eliminazione vista$^{G}$.
	\end{itemize}
	\item Il gruppo ha ricevuto via mail la libreria ML$^{G}$ richiesta.
\end{itemize}