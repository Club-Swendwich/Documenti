\section{Ordine del giorno}
\begin{itemize}
    \item Conoscenza con l'azienda committente.
    \item Chiarimento dei dubbi sorti sul capitolato$^{G}$.
\end{itemize}

\section{Resoconto}
Inizio della call: 16:05.\\
I membri del gruppo sono stati colpiti dalla grande disponibilità dei rappresentanti aziendali a 
rispondere alle domande poste e nel fornire la documentazione necessaria per la valutazione del capitolato$^{G}$.

\subsection{Questioni poste e risposte}
\begin{enumerate}
    \item Esistono librerie per poter interagire con l'UPS?
    \begin{itemize} 
        \item La comunicazione avviene tramite protocollo Modbus, in modalità master-slave.
        \item Le informazioni vengono trasmesse in word da 128 bit di cui ci fornirebbero la documentazione per l'interpretazione e un emulatore per il testing.
    \end{itemize}
    \item Come decidere quali informazioni rappresentare e quali non sono rilevanti?
    \begin{itemize}
        \item Fornirebbero la lista delle informazioni rilevanti.
        \item Fornirebbero gli asset per la rappresentazione dei dati.
        \item Fornirebbero gli screenshot di un loro prototipo per permetterci di capire come rappresentare i dati.
        \item Forniti degli esempi sulla rappresentazione dei dati tramite apposito emulatore.
    \end{itemize}
    \item Esiste una documentazione fruibile per la comprensione del database fornito?
    \begin{itemize}
        \item Fornirebbero la documentazione necessaria per l'utilizzo del database, alla quale la stessa azienda si riferisce.
    \end{itemize}
    \item Come gestire la bassa latenza richiesta?
    \begin{itemize}
        \item La latenza del protocollo Modbus è molto bassa ed è possibile richiedere tutti i dati necessari entro i tempi stabiliti.
        \item Suggeriscono un aggiornamento intelligente dei dati, dato che non tutti i parametri cambiano frequentemente.
    \end{itemize}
    \item Che SO è necessario per l'esecuzione dell'emulatore?
    \begin{itemize}
        \item Windows (min. Vista).
        \item Se necessario, cercherebbero di renderlo portabile.
    \end{itemize}
    \item Chiarimenti sull'obiettivo opzionale della videochiamata con il tecnico.
    \begin{itemize}
        \item Il tecnico deve avere un feedback visivo.
        \item Il tecnico deve poter accedere alle informazioni sullo stato dell'UPS. Tali dati \emph{non devono} partire dall'UPS.
    \end{itemize}
\end{enumerate}
Ci hanno inoltre informati che l'emulatore non è abilitato alla simulazione della comunicazione tramite Bluetooth ma solo a quella tramite ethernet e, pertanto, l'obiettivo opzionale relativo alla comunicazione tramite protocollo Bluetooth potrebbe essere abbandonato.\\
Termine della call: 16:45

\section{Tracciamento decisioni}
\begin{itemize}
    \item Il gruppo ha deciso di considerare provvisoriamente il capitolato$^{G}$ come seconda scelta, in attesa dell'incontro con il rappresentante della Zucchetti SpA.
\end{itemize}
