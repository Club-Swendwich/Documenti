\section{Ordine del giorno}

\begin{itemize}
	\item Nuova divisione dei ruoli.
	\item Domande riguardo il documento ``Analisi dei requisiti''.
\end{itemize}

\section{Resoconto}
\label{sec:Resoconto}

\noindent 
Inizio della call: 18:00. \\
\noindent La call si è tenuta su Discord, un membro del gruppo era assente per problemi di salute.
\begin{itemize}
	\item Il gruppo ha valutato l'attuale stato di avanzamento del progetto constatando la necessità di parallelizzare maggiormente il lavoro, in modo tale da portare a termine gli attuali documenti con maggior celerità. Sono dunque stati rivisti i ruoli di ogni componente.
	\item Sono stati discussi i dubbi inerenti al documento ``Analisi dei requisiti'' con le seguenti conclusioni:
		\begin{enumerate}
			\item Alcuni casi d'uso presentano uno ``Scenario generale'' che necessita di essere suddiviso in punti.
			\item Il grafico del caso d'uso UC8 presenta un errore nella generalizzazione.
			\item \' E stata decisa l'ipotetica interfaccia utente per il caso d'uso ``UC9 - creazione di una nuova vista''.
		\end{enumerate}
\end{itemize}

\noindent Termine della call: 19:00.

\section{Tracciamento decisioni}

\begin{itemize}
	\item Il gruppo ha così ripartito il carico di lavoro:
	\begin{itemize}
		\item Beni Valentina e Bustaffa Marco: termine del documento ``Analisi dei requisiti''. Successivo aiuto nella stesura dei documenti Piano di Progetto e Piano di Qualifica. 
		\item Barilla Gianmarco e Canel Alessandro: prima bozza del documento ``Piano di progetto''.
		\item Ferrarini Alessio: inizio sviluppo del ``Proof of concept$^{G}$''.
		\item Pozzebon Samuele: aggiornamento ``Norme di Progetto''.
	\end{itemize}
	\item Risolti i dubbi inerenti al documento ``Analisi dei requisiti'' come evidenziato nel dettaglio nella sezione precedente \hyperref[sec:Resoconto]{Resoconto}.
\end{itemize}