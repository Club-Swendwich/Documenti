\section{Ordine del giorno}

\begin{itemize}
	\item Esposizione delle scelte architetturali al proponente.
\end{itemize}

\section{Resoconto}
\label{sec:Resoconto}

\noindent
Inizio della call: 16:30. \\
\noindent La call si è tenuta su Zoom$^{G}$ con tutti i membri del gruppo presenti.

\subsection{Problemi individuati e soluzioni}
\begin{enumerate}
	\item Il proponente ha suggerito la creazione di due diagrammi \textit{ER} uno finalizzato all'architettura logica dell'applicativo ed uno finalizzato all'implementazione.
		  \begin{itemize}
			  \item Il gruppo ha deciso di chiedere al professor Cardin consiglio sul da farsi.
			  \item Il gruppo decide di programmare un ulteriore call con il proponente per discutere i progressi della progettazione.
		  \end{itemize}
	\item Il proponente ha trovato confusionaria la sezione del diagramma \textit{ER} dedicata alla composizione delle varie parti del grafico a causa della mancanza di \textit{generics} sulla classe \textit{complete view}.
		  \begin{itemize}
			  \item Il gruppo ha deciso di esplorare nuovi strumenti che espongano questa funzionalità, in caso non vengano trovati allora verrà utilizzato un software di disegno per modificare i grafici generati da \textit{StarUML}.
		  \end{itemize}
	\item Il proponente ha sottolineato che quello che gli interessa principalmente è la separazione della business logic dalla presentazione.
		  \begin{itemize}
			  \item Il gruppo ha rassicurato il proponente che è già intenzione del team quella di separare business logic dalla presentazione, spiegando come il pattern \textit{MVVM} lo enfatizzi.
		  \end{itemize}
\end{enumerate}

\noindent Termine della call: 17:30.

\section{Tracciamento decisioni}
Il gruppo seguendo le indicazioni del proponente si è subito attivato alla correzione delle problematiche tramite l'assegnazione delle card Trello relative ad esse e l'aggiunta al relativo backlog.\\
Il gruppo decide di chiedere al professor Cardin delucidazioni sul punto \textit{1} alla fine della lezione di ingegneria del software del \textit{04/04/2022}.
