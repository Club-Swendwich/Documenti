\section{Ordine del giorno}

\begin{itemize}
	\item Domande sul "Proof of Concept$^{G}$".
	\item Domande sui casi d'uso sviluppati.
\end{itemize}

\section{Resoconto}
\label{sec:Resoconto}

\noindent 
Inizio della call con Zucchetti: 15:00. \\
\noindent La call si è tenuta su Zoom$^{G}$ con tutti i membri del gruppo presenti.

\subsection{Domande}
\begin{enumerate}
	\item Nel PoC$^{G}$ realizzato finora è presente un solo grafico con varie funzionalità di personalizzazione, è sufficiente oppure vorrebbe vedere maggiori funzionalità?
	\begin{itemize}
		\item Un solo grafico è sufficiente per il PoC$^{G}$, tuttavia si possono aggiungere le seguenti funzionalità:
		\begin{itemize}
			\item Necessario: caricamento del dataset$^{G}$ .csv$^{G}$ .
			\item Opzionale: caricamento e salvataggio delle viste$^{G}$.
		\end{itemize}
	\end{itemize}
	\item Per la realizzazione del PoC$^{G}$ si è scelto di utilizzare il \textit{framework} React$^{G}$. Rappresenta un problema?
	\begin{itemize}
		\item L'importante è che non faccia utilizzo di server.
	\end{itemize}
	\item Nel salvataggio di una vista$^{G}$ si è deciso di utilizzare un file .json$^{G}$, è corretto?
	\begin{itemize}
		\item \' E corretto, tuttavia bisognerebbe valutare uno dei seguenti scenari per contestualizzare il dato salvato:
		\begin{itemize}
			\item Salvare il file .json$^{G}$ e in un riga specificare a quale dataset$^{G}$ fa riferimento.
			\item Salvare il file .json$^{G}$ con anche il dataset$^{G}$ utilizzato collegato.
		\end{itemize}
	\end{itemize}
\end{enumerate}

\noindent
Si è poi proceduto a mostrare i casi d'uso finali realizzati nel documento "Analisi dei Requisiti".
I \textbf{punti di revisione} identificati sono i seguenti:
\begin{itemize}
	\item Rivedere l'utilizzo degli \textit{extend} utilizzati nei casi di errore degli UC.
	\item Rivedere i tempi verbali degli UC, il participio passato non è corretto.
\end{itemize}
Nel complesso i casi d'uso realizzati coprono le necessità dell'applicativo e sono corretti.

\subsection{Ulteriori suggerimenti}
\begin{itemize}
	\item Testare all'interno del PoC$^{G}$ il caricamento di file .csv$^{G}$ con una mole di dati maggiore.
	\item Nel caso del \textit{timestamp} è possibile associare dei punti del grafico a singole parti di esso, ad esempio prendendone solo i giorni della settimana. Valutarne l'aggiunta all'interno dell'Analisi dei Requisiti.
\end{itemize}

\noindent Termine della call con Zucchetti: 15:40.



\section{Tracciamento decisioni}
\begin{itemize}
	\item Il gruppo ha ritenuto fondamentale svolgere una riunione a seguito della seguente call, per sciogliere i dubbi riguardanti: 
	\begin{itemize}
		\item Requisiti funzionali;
		\item Nuova suddivisione del lavoro;
		\item Possibilità di un incontro con il docente Vardanega.
	\end{itemize}
\end{itemize}