\section{Ordine del giorno}

\begin{itemize}
	\item Call conoscitiva con l'azienda Zucchetti
	\item Presentazione di alcune domande riguardo il capitolato
\end{itemize}

\section{Resoconto}

La call si è tenuta su zoom con il responsabile del capitolato Gregorio Piccoli. 
I membri del gruppo sono rimasti pienamente soddisfatti dalla disponibilità dimostrata sia nel rispondere alle domande, sia nel fornire tempestivamente materiale utile per il capitolato.

\subsection{Domande}

Il gruppo ha esposto al responsabile di capitolato le seguenti domande:

\begin{enumerate}
	\item Come selezionare le informazioni utili dai dati forniti?
	\begin{itemize}
		\item Esempio timestamp: analizzare in primo luogo i login effettuati ad esempio in giorni lavorativi e in giorni festivi; analizzare l'ora in cui sono stati effettuati (durante la mezzanotte? Durante orario lavorativo?)
		\item Applicare in seguito dei tag ai dati in base all'analisi svolta.
	\end{itemize}

	\item Che linguaggio è preferibile utilizzare per la parte di Machine Learning?
	\begin{itemize}
		\item Preferenza per Javascript ma libertà di scelta
		\item \'E stata citata la possibilità di esplorazione dati mediante Python (libreria Scikit-learn) con visualizzazione in linguaggio differente.
	\end{itemize}

	\item Sarà fornito un framework?
	\begin{itemize}
		\item Non ci sarà bisogno di utilizzare framework.
	\end{itemize}

	\item A quale target si deve pensare per la realizzazione del sito?
	\begin{itemize}
		\item Il sito da realizzare si pone ad un target di sistemisti.
		\item La componente front-end sarà improntata all'usabilità e all'efficienza nel mostrare i grafici.
		\end{itemize}

\end{enumerate}

\subsection{Ulteriori suggerimenti}

Il responsabile Gregorio Piccoli è stato molto disponibile nel mostrare esempi sulla parte da svolgere in Machine Learning. In particolare si è potuto osservare:

\begin{itemize}
		\item Grafana: open source, consigliata per l'analisi interattiva di dati. Sono state mostrate varie tipologie di approssimazione, in particolare sono risultate molto efficienti quelle gaussiane.
\end{itemize}

\noindent 
\'E stata inoltre offerta la possibilità di visionare su richiesta un sample di dati per comprendere meglio le richieste nel capitolato.

\section{Tracciamento decisioni}

\begin{itemize}
	\item Il gruppo ha deciso di considerare provvisoriamente Login Warrior come prima scelta, in concorrenza con ShopChain.
	\item Il gruppo ha iniziato a riflettere sulla parte di analisi dati richiesta.
	\item \'E stato deciso di scrivere un'ulteriore mail al responsabile Gregorio Piccoli per poter visionare un sample di dati.
\end{itemize}