\section{Ordine del giorno}

\begin{itemize}
	\item Domande sul "Proof of Concept".
	\item Domande sui casi d'uso sviluppati.
	\item Riunione di gruppo riepilogativa.
\end{itemize}

\section{Resoconto}
\label{sec:Resoconto}

\subsection{Call Zucchetti}
\noindent 
Inizio della call con Zucchetti: 15:00. \\
\noindent La call si è tenuta su Zoom con tutti i membri del gruppo presenti.

\subsubsection{Domande}
\begin{enumerate}
	\item Nel PoC realizzato finora è presente un solo grafico con varie funzionalità di personalizzazione, è sufficiente oppure vorrebbe vedere maggiori funzionalità?
	\begin{itemize}
		\item Un solo grafico è sufficiente per il PoC, tuttavia si possono aggiungere le seguenti funzionalità:
		\begin{itemize}
			\item Necessario: caricamento del dataset .csv .
			\item Opzionale: caricamento e salvataggio delle viste.
		\end{itemize}
	\end{itemize}
	\item Per la realizzazione del PoC si è scelto di utilizzare il \textit{framework} React. Rappresenta un problema?
	\begin{itemize}
		\item L'importante è che non faccia utilizzo di server.
	\end{itemize}
	\item Nel salvataggio di una vista si è deciso di utilizzare un file .json, è corretto?
	\begin{itemize}
		\item \' E corretto, tuttavia bisognerebbe valutare uno dei seguenti scenari per contestualizzare il dato salvato:
		\begin{itemize}
			\item Salvare il file .json e in un riga specificare a quale dataset fa riferimento.
			\item Salvare il file .json con anche il dataset utilizzato collegato.
		\end{itemize}
	\end{itemize}
\end{enumerate}

\noindent
Si è poi proceduto a mostrare i casi d'uso finali realizzati nel documento "Analisi dei Requisiti".
I \textbf{punti di revisione} identificati sono i seguenti:
\begin{itemize}
	\item Rivedere l'utilizzo degli \textit{extend} utilizzati nei casi di errore degli UC.
	\item Rivedere i tempi verbali degli UC, il participio passato non è corretto.
\end{itemize}
Nel complesso i casi d'uso realizzati coprono le necessità dell'applicativo e sono corretti.

\subsubsection{Ulteriori suggerimenti}
\begin{itemize}
	\item Testare all'interno del PoC il caricamento di file .csv con una mole di dati maggiore.
	\item Nel caso del \textit{timestamp} è possibile associare dei punti del grafico a singole parti di esso, ad esempio prendendone solo i giorni della settimana. Valutarne l'aggiunta all'interno dell'Analisi dei Requisiti.
\end{itemize}

\noindent Termine della call con Zucchetti: 15:40.

\subsection{Meeting su Discord}
\noindent 
Inizio del meeting: 15:41. \\
\noindent Al meeting successivo sono rimasti presenti tutti i membri del gruppo. \\
Si sono discussi i seguenti argomenti:
\begin{itemize}
	\item Suddivisione del lavoro nei documenti mancanti: Piano di Progetto e Piano di Qualifica.
	\item Discussi i Requisiti funzionali all'interno del documento Analisi dei Requisiti. \' E stata riscontrata all'interno del gruppo poca chiarezza riguardo a quali elencare e quali invece fossero superflui.
	\item Sono state dunque decise le domande da rivolgere ai docenti Cardin e Vardanega riguardo i documenti svolti.
	\item Si sono discussi gli standard da seguire per la Qualità di Processo e Qualità di Progetto.
\end{itemize}

\noindent Termine del meeting: 16:40.


\section{Tracciamento decisioni}

\begin{itemize}
	\item Il gruppo ha così ripartito il carico di lavoro:
	\begin{itemize}
		\item Beni Valentina, Bustaffa Marco, Barilla Gianmarco, Canel Alessandro, Pozzebon Samuele: termine del documento "Piano di Progetto". Successiva stesura del documento Piano di Qualifica. 
		\item Ferrarini Alessio: aggiunta della parte di caricamento del dato all'interno del "Proof of Concept".
	\end{itemize}
	\item Si è deciso di fissare un meeting con il docente Vardanega per porre le seguenti domande:
	\begin{enumerate}
		\item Gli UC elencati tra i requisiti funzionali sono corretti? Bisogna aggiungere altri UC impliciti come ad esempio il parsing del dataset?
		\item I requisiti funzionali come andranno suddivisi? In requisiti atomici?
		\item Riguardo agli standard di progetto: sono sensati elementi come la velocità di render?
		\item Su quali standard ci consiglia di concentrarci?
		\item Conosce qualche standard relativo alle interfacce grafiche?
	\end{enumerate}
	\item Si è deciso di chiedere a lezione al docente Cardin le seguenti domande:
	\begin{enumerate}
		\item \' E corretto utilizzare gli \textit{extend} per la gestione degli errori negli UC?
		\item \' E corretto l'utilizzo del participio passato nei nomi degli UC?
	\end{enumerate}
	\item \' E stato aggiornato Trello con quanto stabilito sopra.
\end{itemize}