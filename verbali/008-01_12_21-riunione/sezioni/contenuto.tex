\section{Ordine del giorno}

\begin{itemize}
	\item Inizio formalizzazione casi d'uso in comune
	\item Accordarci sul software per la creazione di grafici
	\item Creazione dei primi grafici
	\item Formalizzazione di domande e dubbi da esporre in preparazione all'incontro con il professore del 10-12-2021
\end{itemize}

\section{Resoconto}

\noindent 
Inizio della call: 15:20. \\
\noindent La call si è tenuta su Discord con tutti i componenti del gruppo presenti.
\begin{itemize}
	\item Il gruppo ha ragionato sulla interfaccia utente inerente alla rappresentazione e caricamento dei dataset$^{G}$. La rappresentazione si estende al tipo, colorazione e dimensione nella UI dei vari grafici inerenti l progetto
	\item Modifica dei Casi d'uso, dato che il punto 6 e 7 sono risultati ridondanti al gruppo
	\item Il gruppo ha concluso il primo punto dei Casi d'uso (Caricamento dataset$^{G}$) nel documento "Casi d'uso"
	\item Il gruppo ha infine discusso sulle possibili domande da proporre in data 10-12-2021
\end{itemize}

\subsection{Casi d'uso finali}
\begin{enumerate}
	\item Caricamento dataset$^{G}$ da file.
	\item Errore formato del file dataset$^{G}$.
	\item Errore struttura dataset$^{G}$.
	\item Errore entry dataset$^{G}$.
	\item Selezione tipo di grafico.
	\item Selezione set di dati da visualizzare.
	\item Personalizzazione visiva dei grafici.
	\item Visualizzazione di default
\end{enumerate}

\subsection{Domande proposte}
\begin{enumerate}
	\item Domandare se il registro delle modifiche da noi impostato (es. documento impegni o scelta del capitolato$^{G}$) è corretto oppure se necessita di modifiche.
	\item Domandare se, nella stestura dei requisiti progettuali, è piu' fruibile una stesura a punti o tabellare.
	\item Infine sul da farsi sulla problematica mancanza di contatti da parte della zucchetti.
\end{enumerate}

\section{Tracciamento decisioni}
\begin{itemize}
	\item Il gruppo ha confermato la suddivisione del lavoro tra i membri per quanto concerne i Casi d'uso
	\item Il gruppo ha steso delle domande da riportare al professore durante l'incontro del 10-12-2021
\end{itemize}