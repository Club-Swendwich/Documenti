\section{Ordine del giorno}

\begin{itemize}
	\item Revisione delle correzioni fornite dai docenti.
	\item Redazione di una lista contente i punti da correggere.
	\item Assegnazione lavori individuali.
\end{itemize}

\section{Resoconto}
\label{sec:Resoconto}

\noindent 
Inizio della call: 18:30. \\
\noindent La call si è tenuta su Discord$^{G}$ con tutti i membri del gruppo presenti.

\subsection{Problemi individuati e soluzioni}
Il gruppo, dopo una rilettura delle valutazioni ricevute, si è soffermato sulle parti meno buone segnalate dai docenti, così da stilare una lista dei punti da correggere e migliorare.
I problemi riscontrati dal docente Vardanega sono i seguenti:
\begin{enumerate}
	\item La consegna include pochi verbali esterni
	\begin{itemize}
		\item Il gruppo si impegnerà a ricercare un più alto livello d'interazione con il proponente.
	\end{itemize}
	\item Le voci riportate nel registro modifiche devono essere ordinate per cronologia inversa
	\begin{itemize}
		\item \'E stato creato un \textit{issue} su GitHub, automaticamente associato ad una card di Trello, per la correzione del problema.
		\item Nel futuro le modifiche verranno riportate come indicato. 
	\end{itemize}
	\item I documenti ufficiali di progetto riportano in premessa le informazioni riguardanti obiettivo e struttura.
	\begin{itemize}
		\item \'E stato creato un \textit{issue} su GitHub, automaticamente associato ad una card di Trello, per la correzione del problema.
	\end{itemize}
	\item Il conto economico del preventivo a finire deve essere calcolato sulla base del raggiungimento degli obiettivi e il consumo di risorse utilizzato, invece di sottrarre lo speso al preventivo iniziale.
	\begin{itemize}
		\item \'E stato creato un \textit{issue} su GitHub, automaticamente associato ad una card di Trello, per la correzione del problema.
	\end{itemize}	
	\item Il cruscotto di valutazione è poco popolato d'informazioni e il corredo di metriche e obiettivi di qualità nel PdQ risulta modesto.
	\begin{itemize}
    	\item Le verifiche verranno eseguite in futuro in più parti del processo esecutivo, assicurando un maggior numero di verifica di obiettivi della qualità.
	\end{itemize}
	\item L'organizzazione delle Norme di Progetto non è pienamente conforme con la gerarchia "processo-attività-procedure[-strumenti]"
	\begin{itemize}
		\item \'E stato creato un \textit{issue} su GitHub, automaticamente associato ad una card di Trello, per la correzione del problema.
	\end{itemize}
	\item Migiorare attualizzazione dei rischi, in particolare nei casi in cui le misure correttive non siano state efficaci
	\begin{itemize}
		\item \'E stato creato un \textit{issue} su GitHub, automaticamente associato ad una card di Trello, per la correzione del problema.
		\item \'E stato deciso di rivedere nel dettaglio i piani di contingenza adottati e la loro effettiva efficacia
	\end{itemize}	
\end{enumerate}
\noindent
Il gruppo inoltre ha deciso di correggere anche le raccomandazioni aggiuntive fornite dal docente, per
una correzione completa degli elaborati.
\begin{enumerate}
	\item È preferibile che le date di calendario nel nome di file abbiano il formato aa-mm-gg.
	\begin{itemize}
		\item \'E stato creato un \textit{issue} su GitHub, automaticamente associato ad una card di Trello, per la correzione del problema.
		\item Il gruppo adotterà dunque il formato aa-mm-gg anche per i verbali futuri.
	\end{itemize}
	\item La modalità d'ipertesto dei collegamenti PDF è esteticamente sgradevole.
	\begin{itemize}
		\item \'E stato creato un \textit{issue} su GitHub, automaticamente associato ad una card di Trello, per la correzione del problema.
		\item Il gruppo ha scelto di adottare un'evidenziazione dei link con hover del mouse.
	\end{itemize}
	\item Il termine Scrum va riportato come la iniziale maiuscola.
	\begin{itemize}
		\item \'E stato creato un \textit{issue} su GitHub, automaticamente associato ad una card di Trello, per la correzione del problema.
		\item \'E stato aggiunto il termine corretto nella lista di controllo per le verifiche.
	\end{itemize}
	\item Accenti in LaTeX sono spesso scorretti.
	\begin{itemize}
		\item \'E stato creato un \textit{issue} su GitHub, automaticamente associato ad una card di Trello, per la correzione del problema.
		\item Nel futuro ci si accorgerà di utilizzare la grammatica corretta.
	\end{itemize}
\end{enumerate}
\noindent
Il gruppo è quindi passato all'analisi dei problemi individuati dal docente Cardin, e per ognuno è stato creato un \textit{issue} su GitHub, automaticamente associato ad una card di Trello, per la correzione:
\begin{enumerate}
	\item UC9 e suoi sottocasi: la funzionalità non è il fatto che il mouse passi sopra un grafico, ma cosa questo scatena.
	\item UC7 / UC8: sono funzionalità mutualmente esclusive; dovrebbero essere ulteriormente
	suddivisi in Zoom-in e Zoom-out.
	\item UC4 non è chiaro: bisogna spiegare meglio che funzionalità rappresenta.
	\item I casi d’uso di personalizzazione dello stile dei grafici devono essere descritti in maggior dettaglio.
\end{enumerate}

\noindent Termine della call: 19:40.

\section{Tracciamento decisioni}
Seguendo le indicazioni del responsabile di progetto le attività di correzione sono state spartite tra i membri del gruppo tramite assegnazione delle card su Trello.\\
\'E stata inoltre decisa la scadenza dello sprint per la correzione dei problemi sopra citati al 15/03/22.\\
Infine sono state assegnate le attività di pianificazione e preventivo per la fase corrente.
