\section{Ordine del giorno}

\begin{itemize}
	\item Riunione sprint settimanale.
    \item Discussione e assegnazione "to do"$^{G}$ 
\end{itemize}

\section{Resoconto}

\noindent
Inizio della call: 21:00. \\
\noindent La call si è tenuta su Discord$^{G}$.
\section{Rapporto Settimanale}
Ogni membro del gruppo ha fatto rapporto riguardo a cosa ha lavorato nel decorso della settimana, dove ha rilevato delle problematiche e su cosa ha intenzione lavorare prossimamente.
Questo avviene su consiglio del professor Tullio Vardanega e aiuta a tenere traccia del progresso in maniera regolare.

\subsection{Rapporto}
Durante il meeting il gruppo ha discusso riguardo a:
\begin{itemize}
	\item Completamento di un render Scatterplot$^{G}$ funzionante e reattivo
	\item Caricamento del CSV e funzionalità "Drag and Drop"
	\item Abbellimento del viewcomposer dello Scatterplot$^{G}$
\end{itemize}

\subsection{Discussione "to do"$^{G}$}
Il gruppo, dopo aver discusso sui "to do"$^{G}$ su Trello$^{G}$ ha segnalato le difficoltà che ha riportato e su cosa può migliorare. In particolare:
\begin{itemize}
	\item Abbellimento dello Scatterplot$^{G}$
	\item Completamento viewcomposer del Sankey$^{G}$
	\item Utilizzo delle dimensioni del Sankey$^{G}$ nel viewcomposer
\end{itemize}

\subsection{Assegnazione "to do"$^{G}$}
\begin{itemize}
	\item I "to do"$^{G}$ sono stati assegnati con lo scopo di avanzare la creazione del viewcomposer Scatterplot, Sankey e la documentazione.
	\item Ogni elemento del gruppo si è incaricato di continuare a sviluppare le funzionalità e documenti.
	\item Nello specifico la priorità è passata alla vista dello Scatterplot$^{G}$ e il viewcomposer del Sankey.
\end{itemize}

Il meeting è terminato alle 21:30

