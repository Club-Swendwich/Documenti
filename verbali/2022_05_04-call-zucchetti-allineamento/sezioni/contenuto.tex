\section{Ordine del giorno}
Richiesta ridimensionamento carico di lavoro in funzione della situazione attuale del gruppo.

\section{Resoconto}
\noindent 
Call svolta in data 04/05/2022, alle ore 14:30. \\
\noindent La call si è tenuta su Zoom$^{G}$ con il responsabile del capitolato$^{G}$ Gregorio Piccoli. 

Il meeting è iniziato con un rapporto iniziale da parte del gruppo, dove sono state riferite le problematiche relative al procedimento del progetto. Il proponente ha suggerito di sfruttare, per le comunicazioni future, il membro del gruppo Alessio Ferrarini, dato che sta correntemente svolgendo il tirocinio presso l'azienda proponente Zucchetti.

Il gruppo ha segnalato un ritardo difficile da recuperare e ha proposto, secondo le indicazioni del Prof. Vardanega derivate dalla perdita di un membro del gruppo che si è laureato, un ridimensionamento degli obiettivi del progetto. Considerata la dimensione del CSV da utilizzare, il proponente ha discusso in merito al suo campionamento e come diversi gruppi non abbiano usufruito di tutti i dati del database, bensì solo una parte, in maniera da renderne più agevole l'elaborazione e la visualizzazione.

In merito alla natura dell'approccio scelto, il proponente ha evidenziato come certe viste risulterebbero illeggibili o difficilmente realizzabili, senza previo campionamento. I grafici interessati da questi problemi sono il Force Directed Graph e il Parallel Coordinates. Risulta complessa la visualizzazione dei nodi e i rispettivi archi dal lato computazionale per il FDG e la visualizzazione di troppe linee nel Parallel Coordinates ne renderebbe difficile la lettura. Invece lo Scatter Plot e il Sankey Diagram si prestano ad una visualizzazione del complessivo dei dati forniti, in quanto lo Scatter Plot ha già dimostrato di essere adatto al lavoro nel PoC e il Sankey Diagram funziona attraverso proporzionalità, e quindi il numero di dati non interferisce con il suo funzionamento.


\section{Tracciamento decisioni}
Il proponente ha quindi suggerito di interrompere lo sviluppo dei grafici Force Directed Graph e Parallel Coordinates, così da alleggerire il gruppo e mantenere l'aspetto interessante della visualizzazione complessiva dei dati. In futuro, una volta che il gruppo avrà requisiti del progetto rimasti, potrà considerare l'implementazione anche  dei grafici sospesi.

\vspace{1.8em}
\FirmaData{Gregorio Piccoli}
\vspace{1.8em}
\FirmaData{Tullio Vardanega}
\vspace{1.8em}
\FirmaData{Referente Club Swendwich}