\section{Qualità di prodotto}
Dallo standard ISO/IEC 9126, il gruppo ha identificato la qualità che vuole applicare all'intero progetto.
\subsection{Documenti}
I documenti oltre a essere grammaticamente corretti devono pure essere facilmente fruibili, infatti devono 
poter essere letti da chiunque.
\subsubsection{Metriche}
\begin{itemize}
    \item MPD1 - Indice Gulpease
    \item MPD2 - Errori ortografici
\end{itemize}
\subsubsection{Valori ammissibili}
{\renewcommand{\arraystretch}{1.5}
\begin{longtable}{p{0.12\linewidth}p{0.30\linewidth}p{0.30\linewidth}}
	\rowcolor[RGB]{33, 73, 50}
	\textcolor{white}{\textbf{Metrica}} & \textcolor{white}{\textbf{Valore accettabile}} & \textcolor{white}{\textbf{Valore ottimale}}\\
    \rowcolor[RGB]{216, 235, 171}
    MPD1 & $ \geq 60$ & $ \geq 80$\\
    \rowcolor[RGB]{233, 245, 206}
    MPD2 & 100\% corretto & 100\% corretto.\\ 
    \caption{Metriche inerenti alla correttezza dei documenti}
\end{longtable}	
}
\subsection{Software}
la qualità del prodotto è data da diversi punti:
\subsubsection{Adeguatezza funzionale}
Capacità del prodotto di soddisfare tutte le funzioni presenti nell'\textit{Analisi dei requisiti}. 
In particolare gli obbiettivi sono:
\begin{itemize}
    \item \textbf{Completezza}: Il prodotto dovrà adempiere a tutti i requisiti funzionali
    \item \textbf{Correttezza}: Il prodotto deve adempiere ai requisiti superando una treshold di test definiti in questo documento
\end{itemize}    
\subsubsection{Metrica}
\begin{itemize}
    \item MPD3 - Copertura dei requisiti.
\end{itemize}
\subsubsection{Valori ammissibili}
\renewcommand{\arraystretch}{1.5}
\begin{longtable}{p{0.12\linewidth}p{0.30\linewidth}p{0.30\linewidth}}
	\rowcolor[RGB]{33, 73, 50}
	\textcolor{white}{\textbf{Metrica}} & \textcolor{white}{\textbf{Valore accettabile}} & \textcolor{white}{\textbf{Valore ottimale}}\\
    \rowcolor[RGB]{233, 245, 206}
    MPD3 & 100\% requisiti obbligatori & 100\% di tutti i requisiti\\ 
    \caption{Metriche inerenti alla correttezza dei documenti}
\end{longtable}	
\subsubsection{Efficenza prestazionale}
L'Efficenza è la capacità di portare a termine un lavoro con il minimo utilizzo di tempo e energia. Esso dovrà essere efficente:
\begin{itemize}
    \item \textbf{Nel tempo}: I grafici dovranno essere renderizzati in un tempo limite
    \item \textbf{Nelle altre risorse}: Il plotting e rendering dei grafici non deve essere troppo intensivo sulle risorse del client.
\end{itemize}    
\subsubsection{Metrica}
\begin{itemize}
    \item MPD4 - Copertura dei requisiti.
\end{itemize}
\subsubsection{Valori ammissibili}
\renewcommand{\arraystretch}{1.5}
\begin{longtable}{p{0.12\linewidth}p{0.30\linewidth}p{0.30\linewidth}}
	\rowcolor[RGB]{33, 73, 50}
	\textcolor{white}{\textbf{Metrica}} & \textcolor{white}{\textbf{Valore accettabile}} & \textcolor{white}{\textbf{Valore ottimale}}\\
    \rowcolor[RGB]{233, 245, 206}
    MPD4 & 3 Secondi & 2 Secondi\\ 
    \caption{Metriche inerenti alla Efficenza prestazionale}
\end{longtable}	
%\subsubsection{Compatibilità}
%\begin{itemize}
%    \item Coesistenza
%    \item Interoperabilità
%\end{itemize}  
\subsubsection{Usabilità}
Il prodotto deve essere comprensibile a colpo d'occhio e UX deve essere piacevole.
\begin{itemize}
    % \item Evidenza di appropriatezza
    \item \textbf{Apprendibilità}: Deve essere facile apprendere le funzioni basilari del prodotto
    \item \textbf{Operabilità}: Le funzioni del prodotto devono rappresentare le aspettative dell'utente
    \item \textbf{Protezione da Errori}: L'user deve essere protetto dalla possibilità di rimanere bloccato
    \item \textbf{User experience}: L'user experience deve essere piacevole
    \item \textbf{Accessibilità}: Il prodotto deve essere di facile visione e utilizzo
\end{itemize}  
\subsubsection{Metrica}
\begin{itemize}
    \item MPD5 - Facilità utilizzo.
    \item MPD6 - Facilità apprendimento funzione.
    \item MPD7 - Livello complessità UX. %Qui si usa Average Cyclomatic Complexity, più ricerca necessaria
\end{itemize}
\subsubsection{Valori ammissibili}
\renewcommand{\arraystretch}{1.5}
\begin{longtable}{p{0.12\linewidth}p{0.30\linewidth}p{0.30\linewidth}}
	\rowcolor[RGB]{33, 73, 50}
	\textcolor{white}{\textbf{Metrica}} & \textcolor{white}{\textbf{Valore accettabile}} & \textcolor{white}{\textbf{Valore ottimale}}\\
    \rowcolor[RGB]{216, 235, 171}
    MPD5 & 3 Secondi & 2 Secondi\\ 
    \rowcolor[RGB]{233, 245, 206}
    MPD6 & 3 Secondi & 2 Secondi\\
    \rowcolor[RGB]{216, 235, 171}
    MPD7 & 3 Secondi & 2 Secondi\\ 
    \caption{Metriche inerenti alla usabilità}
\end{longtable}	
%\subsubsection{Affidabilità}
%\begin{itemize}
%    \item Maturità
%    \item Disponibilità
%    \item Tolleranza ai guasti
%    \item Riparabilità
%\end{itemize}
%\subsubsection{Sicurezza}
%\begin{itemize}
%    \item Confidenzialità
%    \item Integrità
%    \item Non-ripiudiabilità
%   \item Tracciabilità
%    \item Autenticità
%\end{itemize}
\subsubsection{Manutenibilità}
Un prodotto, per avere la capacità di essere corretto o espanso nel future, deve essere strutturato in una maniera tale che non si rischi di compromettere
l'intero progetto. Le caratteristiche del prodotto devono quindi essere:
\begin{itemize}
    \item \textbf{Modularità}: Il prodotto deve essere modulare
    %\item \textbf{Riusabilità}: 
    \item \textbf{Analizzabilità}: L'individuazione degli errori deve essere semplice
    \item \textbf{Modificabilità}: La aggiunta e/o rimozione di parti deve essere permessa, inoltre la modifica deve essere più semplice possibile
    \item \textbf{Verificabilità}: I test sulle modifiche devono essere facili da eseguire
\end{itemize}
\subsubsection{Metrica}
\begin{itemize}
    \item MPD8 - Comprensione del codice.
\end{itemize}
\subsubsection{Valori ammissibili}
\renewcommand{\arraystretch}{1.5}
\begin{longtable}{p{0.12\linewidth}p{0.30\linewidth}p{0.30\linewidth}}
	\rowcolor[RGB]{33, 73, 50}
	\textcolor{white}{\textbf{Metrica}} & \textcolor{white}{\textbf{Valore accettabile}} & \textcolor{white}{\textbf{Valore ottimale}}\\
    \rowcolor[RGB]{233, 245, 206}
    MPD8 & 60-100\% & 80-100\%\\ 
    \caption{Metriche inerenti alla comprensibilità del codice}
\end{longtable}	
\subsubsection{Portabilità}
E' la capacità di poter funzionare in diversi ambienti. Nel nostro specifico caso l'utilizzo del
prodotto sul maggior numero di piattaforme web è di rilevanza.
Per essere portabile il prodotto dovrà avere queste caratteristiche:
\begin{itemize}
    \item \textbf{Adattabilità}: Il software dovrà essere utilizzabile in diversi browser
    %\item \textbf{Installabilità}:
    \item \textbf{Rimpiazzabilità}: Il software deve poter sostituire un prodotto con lo stesso fine e che viene eseguito sullo stesso browser.
\end{itemize}
\subsubsection{Metrica}
\begin{itemize}
    \item MPD9 - Versioni browser supportate.
\end{itemize}
\subsubsection{Valori ammissibili}
\renewcommand{\arraystretch}{1.5}
\begin{longtable}{p{0.12\linewidth}p{0.30\linewidth}p{0.30\linewidth}}
	\rowcolor[RGB]{33, 73, 50}
	\textcolor{white}{\textbf{Metrica}} & \textcolor{white}{\textbf{Valore accettabile}} & \textcolor{white}{\textbf{Valore ottimale}}\\
    \rowcolor[RGB]{233, 245, 206}
    MPD9 & 60-100\% & 80-100\%\\ 
    \caption{Metriche inerenti alla Portabilità del progetto}
\end{longtable}	
\newpage
