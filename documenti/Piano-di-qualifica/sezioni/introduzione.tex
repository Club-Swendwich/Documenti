\section{Introduzione}

\subsection{Scopo del documento}

L'obbiettivo di questo documento è di illustrare tutte le informazioni relative al controllo di qualità del software, implementando dei test standardizzati che permettano che il software sia migliorabile e funzionale.

\subsection{Scopo capitolato}
Il capitolato C5 si pone l'obiettivo di creare un'applicazione per la visualizzazione di dati
di login per facilitarne l'analisi e l'interpretazione.\\
Il prodotto fornirà all'utente la possibilità di scegliere tra diverse modalità di visualizzazione personalizzabili 
e algoritmi per l'analisi dei dati. Attraverso questo processo esplorativo, EDA (\textit{Exploratory Data Analysis}), 
l'utente finale potrà studiare i dati in modo più fruibile ed immediato: obbiettivo ultimo di questo capitolato 
è lo studio di dati di login per indivudare potenziali accessi non autorizzati.

\subsection{Glossario}
Al fine di chiarire possibili ambiguità relative alla terminologia utilizzata all'interno dei
documenti è stato redatto il Glossario contenente i termini di particolare rilievo, questi
sono contrassegnati con una "G" ad apice.

\subsection{Riferimenti}
\subsubsection{Riferimenti normativi}
\begin{itemize}
    \item Norme di Progetto v2.0.0
\end{itemize}

\subsubsection{Riferimenti informativi}
\begin{itemize}
    \item Slide T12 del corso Ingegneria del Software -  \href{https://www.math.unipd.it/~tullio/IS-1/2021/Dispense/T12.pdf}{Qualità di prodotto}
    \item Slide T13 del corso Ingegneria del Software - \href{https://www.math.unipd.it/~tullio/IS-1/2021/Dispense/T13.pdf}{Qualità di processo}  
\end{itemize}
\newpage