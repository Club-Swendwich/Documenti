\section{Introduzione}
\subsection{Scopo del documento}
L'obbiettivo di questo documento è di illustrare tutte le informazioni relativa al controllo di qualità del software, implementando dei test standardizzati che permettano che il software sia migliorabile e funzionale.
\subsection{Scopo capitolato}
Il capoitolato Login warrior ha lo scopo di rendere fruibile una grande mole di dati ai sistemisti, rendendo così possibile l'individuazione di login non autorizzati. %Fix this mess
\subsection{Riferimenti}
\begin{itemize}
    \item Slide T12 del corso Ingegneria del Software -  \href{https://www.math.unipd.it/~tullio/IS-1/2020/Dispense/L12.pdf}{Qualità di prodotto}
    \item Slide T13 del corso Ingegneria del Software - \href{https://www.math.unipd.it/~tullio/IS-1/2020/Dispense/L13.pdf}{Qualità di processo}  
    \item Slide T14 del corso Ingegneria del Software - \href{https://www.math.unipd.it/~tullio/IS-1/2020/Dispense/L14.pdf}{Verifica e validazione}
\end{itemize}
\newpage