\section{Introduzione}

\subsection{Scopo del documento}

L'obbiettivo di questo documento è d'illustrare tutte le informazioni relative al controllo di qualità del software, implementando dei test standardizzati che permettano al software di essere sia migliorabile che funzionale.

\subsection{Scopo capitolato}
Il capitolato$^{G}$ C5 si pone l'obiettivo di creare un'applicazione per la visualizzazione di dati
di login per facilitarne l'analisi e l'interpretazione.\\
Il prodotto fornirà all'utente la possibilità di scegliere tra diverse modalità di visualizzazione personalizzabili 
e algoritmi per l'analisi dei dati. Attraverso questo processo esplorativo, EDA$^{G}$ (\textit{Exploratory Data Analysis}), 
l'utente finale potrà studiare i dati in modo più fruibile e immediato: obbiettivo ultimo di questo capitolato 
è lo studio di dati di login$^{G}$ per individuare potenziali accessi non autorizzati.


\subsection{Glossario}
Al fine di chiarire possibili ambiguità relative alla terminologia utilizzata all'interno dei
documenti è stato redatto il \textit{Glossario} contenente i termini di particolare rilievo, questi
sono contrassegnati con una "G" ad apice.
\newpage