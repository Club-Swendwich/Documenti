\section{Test}
I test sono fondamentali per un acurata e mirata verifica del prodotto. Esistono vari tipi diversi di test.
Il gruppo ha deciso di peseguire la correttezza continua del prodotto, il processo di verifica avverrà quindi in parallelo
con il processo di sviluppo. Per evitare che lo sviluppo rallenti troppo è quinid necessario che la integrazione dei test sia unitaria, continua,
veloci e automatici con l'uso di vari software.
%
% POTREMMO PENSARE DI FARE UN GRAFICO A MO' https://www.math.unipd.it/~tullio/IS-1/2020/Dispense/L14.pdf slide 15
%
\subsection{Test di unità}
I test di unità sono necessarie per controllare singole unità di software, più specifico testare il minimo componente del nostro
programma o metodo.
\subsection{Test di integrazione}
Quando due o più unità testate vengono aggregate in una struttura più grande, rappresentano l'estensione del test di unità
Questi moduli testati assieme formano i testi di integrazione
\subsection{Test di sistema}
I test di sistema sono test olistici che prendono in consideazione l'insieme dei test, per comprendere la seguente
tabella le sigle utilizzate sono:
\begin{itemize}
    \item \textbf{I}: test implementato
    \item \textbf{NI}: test non implementato
\end{itemize}
\newpage

%Attenzione! TS2F2 tipo non sono congruenti
\renewcommand{\arraystretch}{1.5}
\begin{longtable}{p{0.12\linewidth}p{0.68\linewidth}p{0.12\linewidth}}
	\rowcolor[RGB]{33, 73, 50}
	\textcolor{white}{\textbf{Codice}} & \textcolor{white}{\textbf{Descrizione}} & \textcolor{white}{\textbf{Stato}}\\
    \rowcolor[RGB]{233, 245, 206}
    TS1F1 &
    L'utente deve poter caricare dei dati nel sistema tramite file. 
    Se tali operazioni non sono eseguite
    correttamente il caricamento fallisce. Verificare che l'utente possa:
    \begin{itemize}
        \item Visualizzare schermata inserimento dei dati
        \item Inserire dati attraverso portabile
        \item Il file deve essere corretto
        \item L'interrogazione sarà effettuata correttamente
        \item Visualizzare un messaggio di conferma di esito della operazione
        \item In caso di operazione fallita l'utente deve essere in grado di ri-effettuare l'operazione
    \end{itemize}
    & NI\\ 
    \rowcolor[RGB]{216, 235, 171}
    TS2F2 &
    L'utente nella pagina principale è in grado di:
    \begin{itemize}
        \item Localizzare le viste poroposte dal prodotto
        \item Localizzare i widget 
    \end{itemize}
    & NI\\
    \rowcolor[RGB]{233, 245, 206}
    TS1F3 &
    Dobbiamo verificare ch i widget siano:
    \begin{itemize}
        \item Facilmente localizzabili dall'utente
        \item Che siano utili all'utente
    \end{itemize}
    & NI\\
    \rowcolor[RGB]{216, 235, 171}
    TS2F2 &
    L'utente deve poter essere in grado di selezionare:
    \begin{itemize}
        \item Le diverse tipologie di grafico a lui disponibile, elencate nell'Analisi dei requisiti
        \item Le dimensioni e le ripesttive associazioni con gli assi dei grafici in Visualizzarte
        \item Le dimensioni e le loro associazioni siano salvate nel sistema
    \end{itemize}
    & NI\\
\end{longtable}	
\newpage
\subsection{Test di accettazione}
I test di accettazione avvengono al collaudo del software, comparando le esigenze del
proponente con la realtà dei fatti. Sono l'unione di tutti i test eseguiti nelle
parti pregedenti. \newline
I test in questione sono quelli riportati nella tabella Test di sistema.

\subsection{Test di regressione}
I test di regressione subentrano ove avvengono modifiche su nuove versioni del prodotto,
in questo documento non si è in grado di fornire test di regressione pertinenti.

\newpage

%Si incomincia resoconto attività di verifica
\subsection{A   Resoconto di attività di verifica}
\subsubsection{Periodo di Analisi}
