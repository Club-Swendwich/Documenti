\section{Qualità di processo}
\subsection{Introduzione}
Per garantire la qualità di processo il gruppo \textit{Club Swendwich} 
ha scelto di adottare lo standard \textbf{ISO/IEC 12207:1995}.
Il seguente standard divide i processi in tre principali macrocategorie:
\begin{itemize}
    \item \hyperref[sec:PCP]{Processi primari}
    \item \hyperref[sec:PCO]{Processi organizzativi}
    \item \hyperref[sec:PCS]{Processi di supporto}
\end{itemize}
Ogni raggruppamento si divide in attività, che a loro volta si suddividono
in \textit{task}, per questo motivo il seguente standard propone una valida gestione della
qualità dei processi anche in un modello di sviluppo \textit{Agile}.\\
\noindent
Nelle successive sottosezioni del documento si sono scelte alcune attività ritenute di valore all'interno
delle macrocategorie dello standard \textbf{ISO/IEC 12207:1995}, esse vengono quindi analizzate
secondo le \textbf{metriche di processo} adeguate.\\
\\
\noindent
Riferimenti tecnici:
\begin{itemize}
    \item Lo standard \textbf{ISO/IEC 12207:1995} è approfondito nel documento \textit{"Norme di Progetto"},
    sezione \textit{Appendice A}.
    \item Le \textbf{metriche} e le \textbf{formule} utilizzate sono descritte nel documento \textit{"Norme di Progetto"},
    sezione xx.xx.
\end{itemize}

\bigskip

\subsection{Processi primari}
\label{sec:PCP}

\begin{table}[htb]
    \centering
    \small
    \begin{tabular}{>{\raggedright\arraybackslash}m{0.23\linewidth}|m{0.43\linewidth}|m{0.26\linewidth}}
    \rowcolor[RGB]{33, 73, 50}
    \multicolumn{1}{>{\centering\arraybackslash}m{0.23\linewidth}|}{\textcolor{white}{\textbf{Processo}}} 
        & \multicolumn{1}{>{\centering\arraybackslash}m{0.43\linewidth}|}{\textcolor{white}{\textbf{Descrizione}}} 
        & \multicolumn{1}{>{\centering\arraybackslash}m{0.26\linewidth}|}{\textcolor{white}{\textbf{Metrica di processo}}}\\
    \rowcolor[RGB]{216, 235, 171}
        \textbf{Fornitura} 
        & Il processo ha lo scopo di identificare procedure e risorse atte a soddisfare i requisiti di progetto. 
        & MPC1 MPC2 MPC3 \par MPC4 MPC5\\
    \rowcolor[RGB]{233, 245, 206}
        \textbf{Sviluppo} 
        & Il processo si occupa delle attività per la realizzazione del prodotto. 
        & MPC6 MPC7\\
    \end{tabular}
    \caption{Processi primari e metriche utilizzate}
\end{table}

\subsubsection{Valori di riferimento}

\begin{itemize}
    \item La sigla \textbf{BAC} fa riferimento alla metrica \textit{Budget at Completion}.
    \item La sigla \textbf{BCWS} fa riferimento alla metrica \textit{Budgeted Cost of Work Scheduled}.
\end{itemize}

{\renewcommand{\arraystretch}{1.5}
\footnotesize
\begin{longtable}{>{\raggedright\arraybackslash}m{0.20\linewidth}m{0.10\linewidth}m{0.20\linewidth}m{0.20\linewidth}}
	\rowcolor[RGB]{33, 73, 50}
    \multicolumn{1}{>{\centering\arraybackslash}m{0.20\linewidth}}{\textcolor{white}{\textbf{Metrica}}} 
    & \multicolumn{1}{>{\centering\arraybackslash}m{0.10\linewidth}}{\textcolor{white}{\textbf{Codice}}} 
    & \multicolumn{1}{>{\centering\arraybackslash}m{0.20\linewidth}}{\textcolor{white}{\textbf{Valore ottimale}}}
    & \multicolumn{1}{>{\centering\arraybackslash}m{0.20\linewidth}}{\textcolor{white}{\textbf{Valore accettabile}}}\\
    
    \rowcolor[RGB]{47, 106, 73}
    \multicolumn{4}{>{\centering\arraybackslash}m{0.779\linewidth}}{\textcolor{white}{\textbf{Fornitura}}}\\

    \rowcolor[RGB]{216, 235, 171}
        \centering \textbf{Schedule Variance \\ (SV)} 
        & \centering \textbf{MPC1} 
        & \multicolumn{1}{>{\centering\arraybackslash}m{0.20\linewidth}}{$\leq 0\% $}
        & \multicolumn{1}{>{\centering\arraybackslash}m{0.20\linewidth}}{$\geq -10\% $}\\
    \rowcolor[RGB]{233, 245, 206}
        \centering \textbf{Budget Variance \\ (BV)} 
        & \centering \textbf{MPC2} 
        & \multicolumn{1}{>{\centering\arraybackslash}m{0.20\linewidth}}{$\leq 0\% $}
        & \multicolumn{1}{>{\centering\arraybackslash}m{0.20\linewidth}}{$\geq -10\% $}\\
    \rowcolor[RGB]{216, 235, 171}
        \centering \textbf{Estimated at Completion \\ (EAC)} 
        & \centering \textbf{MPC3} 
        & \multicolumn{1}{>{\centering\arraybackslash}m{0.20\linewidth}}{EAC $=$ BAC}
        & \multicolumn{1}{>{\centering\arraybackslash}m{0.20\linewidth}}{BAC $-$ 5\% $\leq EAC $\par EAC $\leq $ BAC $+$ 5\%}\\
    \rowcolor[RGB]{233, 245, 206}
        \centering \textbf{Actual Cost of Work Performed \\ (ACWP)} 
        & \centering \textbf{MPC4} 
        & \multicolumn{1}{>{\centering\arraybackslash}m{0.20\linewidth}}{ACWP $\leq$ BCWS}
        & \multicolumn{1}{>{\centering\arraybackslash}m{0.20\linewidth}}{BCWS $+$ 5\%}\\
    \rowcolor[RGB]{216, 235, 171}
        \centering \textbf{Budgeted Cost of Work Performed \\ (BCWP)} 
        & \centering \textbf{MPC5} 
        & \multicolumn{1}{>{\centering\arraybackslash}m{0.20\linewidth}}{BCWP $=$ BCWS}
        & \multicolumn{1}{>{\centering\arraybackslash}m{0.20\linewidth}}{BCWP$\leq$BCWS $+$ 5\%}\\

        \rowcolor[RGB]{47, 106, 73}
        \multicolumn{4}{>{\centering\arraybackslash}m{0.779\linewidth}}{\textcolor{white}{\textbf{Sviluppo}}}\\    

    \rowcolor[RGB]{216, 235, 171}
        \centering \textbf{Requirements stability index \\ (RSI)} 
        & \centering \textbf{MPC6} 
        & \multicolumn{1}{>{\centering\arraybackslash}m{0.20\linewidth}}{100\%}
        & \multicolumn{1}{>{\centering\arraybackslash}m{0.20\linewidth}}{$\geq$ 80\%}\\
    \rowcolor[RGB]{233, 245, 206}
        \centering \textbf{Satisfied obligatory requirements \\ (SOR) } 
        & \centering \textbf{MPC7} 
        & \multicolumn{1}{>{\centering\arraybackslash}m{0.20\linewidth}}{100\%}
        & \multicolumn{1}{>{\centering\arraybackslash}m{0.20\linewidth}}{100\%}\\
        \caption{Valori di riferimento per le metriche dei "Processi primari"}
\end{longtable}


%---------------------------------------------------------------------------------

\subsection{Processi organizzativi}
\label{sec:PCO}

\begin{table}[htb]
    \centering
    \small
    \begin{tabular}{>{\raggedright\arraybackslash}m{0.23\linewidth}|m{0.43\linewidth}|m{0.26\linewidth}}
    \rowcolor[RGB]{33, 73, 50}
    \multicolumn{1}{>{\centering\arraybackslash}m{0.23\linewidth}|}{\textcolor{white}{\textbf{Processo}}} 
        & \multicolumn{1}{>{\centering\arraybackslash}m{0.43\linewidth}|}{\textcolor{white}{\textbf{Descrizione}}} 
        & \multicolumn{1}{>{\centering\arraybackslash}m{0.26\linewidth}|}{\textcolor{white}{\textbf{Metrica di processo}}}\\
    \rowcolor[RGB]{216, 235, 171}
        \textbf{Gestione organizzativa} 
        & Il processo ha lo scopo di organizzare, monitorare
        e controllare l'avvio e le prestazioni di un processo
        per il raggiungimento degli obiettivi in accordo.
        & MPC8\\
    \end{tabular}
    \caption{Processi organizzativi e metriche utilizzate}
\end{table}

\subsubsection{Valori di riferimento}

{\renewcommand{\arraystretch}{1.5}
\footnotesize
\begin{longtable}{>{\raggedright\arraybackslash}m{0.20\linewidth}m{0.10\linewidth}m{0.20\linewidth}m{0.20\linewidth}}
	\rowcolor[RGB]{33, 73, 50}
    \multicolumn{1}{>{\centering\arraybackslash}m{0.20\linewidth}}{\textcolor{white}{\textbf{Metrica}}} 
    & \multicolumn{1}{>{\centering\arraybackslash}m{0.10\linewidth}}{\textcolor{white}{\textbf{Codice}}} 
    & \multicolumn{1}{>{\centering\arraybackslash}m{0.20\linewidth}}{\textcolor{white}{\textbf{Valore ottimale}}}
    & \multicolumn{1}{>{\centering\arraybackslash}m{0.20\linewidth}}{\textcolor{white}{\textbf{Valore accettabile}}}\\
    
    \rowcolor[RGB]{47, 106, 73}
    \multicolumn{4}{>{\centering\arraybackslash}m{0.779\linewidth}}{\textcolor{white}{\textbf{Gestione organizzativa}}}\\

    \rowcolor[RGB]{216, 235, 171}
        \centering \textbf{Non-calculated Risk \\ (NCR)} 
        & \centering \textbf{MPC8} 
        & \multicolumn{1}{>{\centering\arraybackslash}m{0.20\linewidth}}{0 rischi avvenuti}
        & \multicolumn{1}{>{\centering\arraybackslash}m{0.20\linewidth}}{$\leq$ 5 rischi avvenuti}\\
        \caption{Valori di riferimento per le metriche dei "Processi organizzativi"}
\end{longtable}

%--------------------------------------------------------------------------

\subsection{Processi di supporto}
\label{sec:PCS}

\begin{table}[htb]
    \centering
    \small
    \begin{tabular}{>{\raggedright\arraybackslash}m{0.23\linewidth}|m{0.43\linewidth}|m{0.26\linewidth}}
    \rowcolor[RGB]{33, 73, 50}
    \multicolumn{1}{>{\centering\arraybackslash}m{0.23\linewidth}|}{\textcolor{white}{\textbf{Processo}}} 
        & \multicolumn{1}{>{\centering\arraybackslash}m{0.43\linewidth}|}{\textcolor{white}{\textbf{Descrizione}}} 
        & \multicolumn{1}{>{\centering\arraybackslash}m{0.26\linewidth}|}{\textcolor{white}{\textbf{Metrica di processo}}}\\
    \rowcolor[RGB]{216, 235, 171}
        \textbf{Verifica} 
        & Il processo ha lo scopo di determinare se i prodotti
        software di un'attività soddisfano i requisiti o le
        condizioni a loro imposti. 
        & MPC9 MPC10 MCP11 \\
    \rowcolor[RGB]{233, 245, 206}
        \textbf{Assicurazione della qualità} 
        & Il processo ha lo scopo di assicurare che tutti i
        prodotti di fase siano conformi con i piani e gli
        standard definiti.         
        & MPC12 \\
    \end{tabular}
    \caption{Processi di supporto e metriche utilizzate}
\end{table}

\subsubsection{Valori di riferimento}

{\renewcommand{\arraystretch}{1.5}
\footnotesize
\begin{longtable}{>{\raggedright\arraybackslash}m{0.20\linewidth}m{0.10\linewidth}m{0.20\linewidth}m{0.20\linewidth}}
	\rowcolor[RGB]{33, 73, 50}
    \multicolumn{1}{>{\centering\arraybackslash}m{0.20\linewidth}}{\textcolor{white}{\textbf{Metrica}}} 
    & \multicolumn{1}{>{\centering\arraybackslash}m{0.10\linewidth}}{\textcolor{white}{\textbf{Codice}}} 
    & \multicolumn{1}{>{\centering\arraybackslash}m{0.20\linewidth}}{\textcolor{white}{\textbf{Valore ottimale}}}
    & \multicolumn{1}{>{\centering\arraybackslash}m{0.20\linewidth}}{\textcolor{white}{\textbf{Valore accettabile}}}\\

    \rowcolor[RGB]{47, 106, 73}
    \multicolumn{4}{>{\centering\arraybackslash}m{0.779\linewidth}}{\textcolor{white}{\textbf{Verifica}}}\\    

    \rowcolor[RGB]{216, 235, 171}
        \centering \textbf{Code coverage \\ (CC)} 
        & \centering \textbf{MPC9} 
        & \multicolumn{1}{>{\centering\arraybackslash}m{0.20\linewidth}}{$\geq 90\% $}
        & \multicolumn{1}{>{\centering\arraybackslash}m{0.20\linewidth}}{$\geq 80\% $}\\
    \rowcolor[RGB]{233, 245, 206}
        \centering \textbf{Passed test cases percentage \\ (PTCP) } 
        & \centering \textbf{MPC10} 
        & \multicolumn{1}{>{\centering\arraybackslash}m{0.20\linewidth}}{100\%}
        & \multicolumn{1}{>{\centering\arraybackslash}m{0.20\linewidth}}{$\geq 90\% $}\\
    \rowcolor[RGB]{216, 235, 171}
        \centering \textbf{Failed test cases percentage \\ (FTCP)} 
        & \centering \textbf{MPC11} 
        & \multicolumn{1}{>{\centering\arraybackslash}m{0.20\linewidth}}{0\%}
        & \multicolumn{1}{>{\centering\arraybackslash}m{0.20\linewidth}}{$\leq 10\% $}\\

        \rowcolor[RGB]{47, 106, 73}
        \multicolumn{4}{>{\centering\arraybackslash}m{0.779\linewidth}}{\textcolor{white}{\textbf{Assicurazione della qualità}}}\\    

    \rowcolor[RGB]{233, 245, 206}
        \centering \textbf{Quality Metrics Satisfied \\ (QMS)} 
        & \centering \textbf{MPC12} 
        & \multicolumn{1}{>{\centering\arraybackslash}m{0.20\linewidth}}{100\%}
        & \multicolumn{1}{>{\centering\arraybackslash}m{0.20\linewidth}}{$\geq 90\% $}\\
        \caption{Valori di riferimento per le metriche dei "Processi di supporto"}
\end{longtable}


\newpage
