% -----------------------
% AAAAAAAAAAAAAAAAAA
%------------------------
\section{A}

\newpage
% -----------------------
% BBBBBBBBBBBBBBBBBB
%------------------------
\section{B}

\newpage
% -----------------------
% CCCCCCCCCCCCCCCCCC
%------------------------
\section{C}
\subsection{Capitolato}
Il capitolato d'appalto è un documento tecnico, in genere allegato a un contratto di appalto, a cui si fa riferimento per definire in quella sede le specifiche tecniche delle opere che andranno a eseguirsi per effetto del contratto stesso, di cui è solitamente parte integrante.

\subsection{CSV}
Il comma-separated values è un formato di file basato su file di testo utilizzato per l'importazione ed esportazione di una tabella di dati.

\newpage
% -----------------------
% DDDDDDDDDDDDDDDDDD
%------------------------
\section{D}
\subsection{Dataset}
\'E una collezioni di dati organizzata in forma tabellare dove, ogni colonna rappresenta una possibile dimensione e ogni riga un tentativo di login.

\subsection{Dimensione}
Con \textit{dimensione} di un grafico si intende un asse o un riferimento logico al quale l'utente può associare un insieme di dati. Ad esempio nel grafico Scatter plot si hanno a disposizione due assi, associando però i dati anche a riferimenti logici come: il colore, il tempo, ecc; si può ottenere un grafico a sei dimensioni. 

\subsection{D3.js}
\'E una libreria JavaScript per creare visualizzazioni dinamiche e interattive partendo da dati organizzati, visibili attraverso un comune browser.

\newpage
% -----------------------
% EEEEEEEEEEEEEEEEEEE
%------------------------
\section{E}
\subsection{Exploratory Data Analysis}
In statistica, l'analisi esplorativa dei dati è un approccio all'analisi dei set di dati per riassumerne le caratteristiche principali, spesso utilizzando grafici statistici e altri metodi di visualizzazione dei dati.

\newpage
% -----------------------
% FFFFFFFFFFFFFFFFFFFF
%------------------------
\section{F}
\subsection{Force-directed graph}
\'E un grafo in cui i nodi vengono posizionati in uno spazio bidimensionale o tridimensionale. Una delle sue caratteristiche principali è che gli archi, ovvero i collegamenti tra i nodi, sono tutti circa della stessa lunghezza e si incrociano tra loro il minor numero di volte possibili.

\newpage
% -----------------------
% GGGGGGGGGGGGGGGGGGGG
%------------------------
\section{G}
\subsection{Git}
Software di controllo di versione distribuito utilizzabile da interfaccia a riga di comando, creato da Linus Torvalds nel 2005.

\subsection{GitHub}
Servizio di hosting per sviluppatori. Fornisce uno strumento di controllo versione e permette lo sviluppo distribuito del software.

\newpage
% -----------------------
% HHHHHHHHHHHHHHHHHHHH
%------------------------
\section{H}
\subsection{HDBSCAN}
\'E un algoritmo di clustering sviluppato da Campello, Moulavi, and Sander. Estende l'algoritmo DBSCAN convertendolo in un algoritmo di clustering gerarchico.

\newpage
% -----------------------
% IIIIIIIIIIIIIIIIIIII
%------------------------
\section{I}
\subsection{Isolation forest}
L'Isolation forest è un algoritmo per rilevare anomalie. Identifica le anomalie utilizzando il principio d'isolamento, infatti si deve prima di tutto definire un profilo di ciò che si considera "normale", dopo di che l'algoritmo isola e identifica tutti i casi non conformi al profilo costruito. 

\newpage
% -----------------------
% JJJJJJJJJJJJJJJJJJJJ
%------------------------
\section{J}
\subsection{Json}
In informatica, nell'ambito della programmazione web, JSON, acronimo di JavaScript Object Notation, è un formato adatto all'interscambio di dati fra applicazioni client/server.

\newpage
% -----------------------
% KKKKKKKKKKKKKKKKKKKK
%------------------------
\section{K}

\newpage
% -----------------------
% LLLLLLLLLLLLLLLLLLLL
%------------------------
\section{L}
\subsection{Login}
Il login è la procedura di accesso a un sistema informatico o a un'applicazione informatica.

\newpage
% -----------------------
% MMMMMMMMMMMMMMMMMMMM
%------------------------
\section{M}
\subsection{Machine learning}
L'apprendimento automatico è una branca dell'intelligenza artificiale che raccoglie metodi sviluppati negli ultimi decenni del XX secolo in varie comunità scientifiche.\\
L'apprendimento automatico è strettamente legato al riconoscimento di pattern e alla teoria computazionale dell'apprendimento, ed esplora lo studio e la costruzione di algoritmi che possano apprendere da un insieme di dati, e fare delle predizioni su questi.

\subsection{Milestone}
Una milestone è un riferimento temporale utilizzato spesso per indicare degli obbiettivi intermedi da raggiungere nel corso dello svolgimento di un progetto. Frequentemente le milestones hanno una durata prefissata in giorni.

\newpage
% -----------------------
% NNNNNNNNNNNNNNNNNNNN
%------------------------
\section{N}

\newpage
% -----------------------
% OOOOOOOOOOOOOOOOOOOO
%------------------------
\section{O}
\subsection{Open source}
Con open source, in informatica, si indica un tipo di software o il suo modello di sviluppo o distribuzione. Un software open source è reso tale per mezzo di una licenza attraverso cui i detentori dei diritti favoriscono la modifica, lo studio, l'utilizzo e la redistribuzione del codice sorgente.

\newpage
% -----------------------
% PPPPPPPPPPPPPPPPPPPP
%------------------------
\section{P}
\subsection{Parallel coordinates}
Il grafico a coordinate parallele è un sistema comunemente usato per visualizzare spazi ad n dimensioni. Per mostrare un insieme di punti in uno spazio a n dimensioni, vengono disegnate n linee parallele, solitamente verticali e poste a uguale distanza l'una dall'altra.

\newpage
% -----------------------
% QQQQQQQQQQQQQQQQQQQQ
%------------------------
\section{Q}

\newpage
% -----------------------
% RRRRRRRRRRRRRRRRRRRR
%------------------------
\section{R}
\subsection{Repository}
Ambiente di un sistema informativo, in cui vengono gestiti i metadati, attraverso tabelle relazionali;
l’insieme di tabelle, regole e motori di calcolo tramite cui si gestiscono i metadati prende il nome di
metabase.

\newpage
% -----------------------
% SSSSSSSSSSSSSSSSSSSS
%------------------------
\section{S}
\subsection{Sankey Diagram}
Il diagramma di Sankey è un particolare tipo di diagramma di flusso in cui l'ampiezza delle frecce è disegnata in maniera proporzionale alla quantità di flusso.
Esso è usualmente utilizzato per indicare trasferimenti di energia, materiali, costi o dati in un processo.

\subsection{Scatter Plot}
Il grafico a dispersione o scatter plot è un tipo di grafico in cui due variabili di un set di dati sono riportate su uno spazio cartesiano.
I dati sono visualizzati tramite una collezione di punti ciascuno con una posizione sull'asse orizzontale determinato da una variabile e sull'asse verticale determinato dall'altra. 

\subsection{Scikit-learn}
Scikit-learn è una libreria open source di apprendimento automatico per il linguaggio di programmazione Python.

\subsection{Self organizing map}
Le self-organizing map (SOM) sono un tipo di organizzazione di processi di informazione in rete analoghi alle reti neurali artificiali. Le SOM sono particolarmente utili per la visualizzazione di dati di dimensione elevata.

\newpage
% -----------------------
% TTTTTTTTTTTTTTTTTTTTT
%------------------------
\section{T}
\subsection{Trello}
Trello è un servizio online gratuito, disponibile anche sotto forma di applicazione per sistemi mobile e desktop, che permette di organizzare e gestire i propri progetti (sia personali che professionali) tramite la creazione di bacheche tematiche condivisibili con altri utenti.

\subsection{t-SNE}
t-distributed stochastic neighbor embedding è un algoritmo di riduzione della dimensionalità sviluppato da Geoffrey Hinton e Laurens van der Maaten, ampiamente utilizzato come strumento di apprendimento automatico in molti ambiti di ricerca.

\newpage
% -----------------------
% UUUUUUUUUUUUUUUUUUUUU
%------------------------
\section{U}
\subsection{UMAP}
Uniform Manifold Approximation and Projection è un algoritmo di riduzione dimensionale che può essere usato per visualizzazioni simili a quelle proposte dal \textit{t-SNE} ma anche per  riduzioni di dimensioni non lineari generiche.

\newpage
% -----------------------
% VVVVVVVVVVVVVVVVVVVVV
%------------------------
\section{V}
\subsection{Vista}
Una vista è un sottoinsieme del dataset, di dati base o derivati, associati a un grafico e alle dimensioni che esso mette a disposizione. Una vista tiene inoltre traccia delle personalizzazioni grafiche che l'utente ha deciso di adottare. Esistono viste create di default, ma ogni utente ha la possibilità di creare la propria.

\newpage
% -----------------------
% WWWWWWWWWWWWWWWWWWWWW
%------------------------
\section{W}

\newpage
% -----------------------
% XXXXXXXXXXXXXXXXXXXXX
%------------------------
\section{X}

\newpage
% -----------------------
% YYYYYYYYYYYYYYYYYYYYY
%------------------------
\section{Y}

\newpage
% -----------------------
% ZZZZZZZZZZZZZZZZZZZZZ
%------------------------
\section{Z}