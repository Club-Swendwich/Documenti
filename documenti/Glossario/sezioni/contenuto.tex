% -----------------------
% AAAAAAAAAAAAAAAAAA
%------------------------
\section{A}

\newpage
% -----------------------
% BBBBBBBBBBBBBBBBBB
%------------------------
\section{B}

\newpage
% -----------------------
% CCCCCCCCCCCCCCCCCC
%------------------------
\section{C}
\subsection{Capitolato}
Il capitolato d'appalto è un documento tecnico, in genere allegato a un contratto di appalto, a cui si fa riferimento per definire, in quella sede, le specifiche tecniche delle opere che andranno a eseguirsi per effetto del contratto stesso, di cui è solitamente parte integrante.

\subsection{CSV}
Il comma-separated values è un formato di file basato su file di testo utilizzato per l'importazione ed esportazione di una tabella di dati.

\newpage
% -----------------------
% DDDDDDDDDDDDDDDDDD
%------------------------
\section{D}
\subsection{Dataset}
\`E una collezione di dati organizzata in forma tabellare dove ogni colonna rappresenta una possibile dimensione$^{G}$ e ogni riga un tentativo di login$^{G}$.

\subsection{Dimensione}
Con \textit{dimensione} di un grafico si intende un asse o un riferimento logico al quale l'utente può associare un insieme di dati. Ad esempio nel grafico Scatter plot$^{G}$ si hanno a disposizione due assi, associando però i dati anche a riferimenti logici (es. colore, tempo, ecc.) è possibile ottenere un grafico a sei dimensioni. 

\subsection{Discord}
Discord è una piattaforma statunitense di VoIP, messaggistica istantanea e distribuzione digitale. Gli utenti comunicano con chiamate vocali, videochiamate, messaggi di testo, media e file in chat private o come membri di un server.

\subsection{D3.js}
\`E una libreria JavaScript per creare visualizzazioni dinamiche e interattive partendo da dati organizzati, visibili attraverso un comune browser.



\newpage
% -----------------------
% EEEEEEEEEEEEEEEEEEE
%------------------------
\section{E}
\subsection{Exploratory Data Analysis}
In statistica, l'analisi esplorativa dei dati è un approccio all'analisi dei set di dati per riassumerne le caratteristiche principali, spesso utilizzando grafici statistici e altri metodi di visualizzazione dei dati.

\newpage
% -----------------------
% FFFFFFFFFFFFFFFFFFFF
%------------------------
\section{F}
\subsection{Force-directed graph}
È un grafo in cui i nodi vengono posizionati in uno spazio bidimensionale o tridimensionale. Una delle sue caratteristiche principali è che gli archi, ovvero i collegamenti tra i nodi, sono tutti circa della stessa lunghezza e si incrociano tra loro il minor numero di volte possibili.

\subsection{Framework}
È lo standard utilizzato per lo sviluppo del software. Il Software Process Framework è la fondazione di un processo completo di sviluppo
del software. Esso comprende un set ti attività annidate.

\newpage
% -----------------------
% GGGGGGGGGGGGGGGGGGGG
%------------------------
\section{G}
\subsection{Git}
Software di controllo di versione distribuito utilizzabile da interfaccia a riga di comando, creato da Linus Torvalds nel 2005.

\subsection{GitHub}
Servizio di hosting per sviluppatori. Fornisce uno strumento di controllo versione e permette lo sviluppo distribuito del software.

\newpage
% -----------------------
% HHHHHHHHHHHHHHHHHHHH
%------------------------
\section{H}
\subsection{HDBSCAN}
È un algoritmo di clustering sviluppato da Campello, Moulavi, and Sander. Estende l'algoritmo DBSCAN convertendolo in un algoritmo di clustering gerarchico.

\newpage
% -----------------------
% IIIIIIIIIIIIIIIIIIII
%------------------------
\section{I}
\subsection{Isolation forest}
L'Isolation forest è un algoritmo per rilevare anomalie. Identifica le anomalie utilizzando il principio d'isolamento, infatti si deve prima di tutto definire un profilo di ciò che si considera "normale", dopo di che l'algoritmo isola e identifica tutti i casi non conformi al profilo costruito. 

\subsection{Issue}
Una Issue$^{G}$ in GitHub$^{G}$ è un modo per tenere in ordine e assegnare le varie features da completare a scopo di collaborazione.

\newpage
% -----------------------
% JJJJJJJJJJJJJJJJJJJJ
%------------------------
\section{J}
\subsection{Json}
In informatica, nell'ambito della programmazione web, JSON, acronimo di JavaScript Object Notation, è un formato adatto all'interscambio di dati fra applicazioni client/server.

\newpage
% -----------------------
% KKKKKKKKKKKKKKKKKKKK
%------------------------
\section{K}

\newpage
% -----------------------
% LLLLLLLLLLLLLLLLLLLL
%------------------------
\section{L}
\subsection{Login}
Il login è la procedura di accesso a un sistema informatico o a un'applicazione informatica.

\newpage
% -----------------------
% MMMMMMMMMMMMMMMMMMMM
%------------------------
\section{M}
\subsection{Machine learning}
L'apprendimento automatico è una branca dell'intelligenza artificiale che raccoglie metodi sviluppati negli ultimi decenni del XX secolo in varie comunità scientifiche.\\
L'apprendimento automatico è strettamente legato al riconoscimento di pattern e alla teoria computazionale dell'apprendimento, ed esplora lo studio e la costruzione di algoritmi che possano apprendere da un insieme di dati, e fare delle predizioni su questi.

\subsection{Milestone}
Una milestone è un riferimento temporale utilizzato spesso per indicare degli obbiettivi intermedi da raggiungere nel corso dello svolgimento di un progetto. Frequentemente le milestones hanno una durata prefissata in giorni.

\newpage
% -----------------------
% NNNNNNNNNNNNNNNNNNNN
%------------------------
\section{N}

\newpage
% -----------------------
% OOOOOOOOOOOOOOOOOOOO
%------------------------
\section{O}
\subsection{Open-source}
Con open-source, in informatica, si indica un tipo di software o il suo modello di sviluppo o distribuzione. Un software open-source è reso tale per mezzo di una licenza attraverso cui i detentori dei diritti favoriscono la modifica, lo studio, l'utilizzo e la redistribuzione del codice sorgente.

\newpage
% -----------------------
% PPPPPPPPPPPPPPPPPPPP
%------------------------
\section{P}
\subsection{Parallel coordinates}
Il grafico a coordinate parallele è un sistema comunemente usato per visualizzare spazi ad n dimensioni. Per mostrare un insieme di punti in uno spazio a n dimensioni, vengono disegnate n linee parallele, solitamente verticali e poste a uguale distanza l'una dall'altra.

\subsection{Parsing}
Il parsing è un processo che analizza un input per determinare la correttezza della struttura rispetto a delle regole 
predeterminate.

\subsection{Proof of Concept}
Una prova di concetto (POC) è una realizzazione di una bozza del progetto al fine di dimostrarne la fattibilità. \newline

\newpage
% -----------------------
% QQQQQQQQQQQQQQQQQQQQ
%------------------------
\section{Q}

\newpage
% -----------------------
% RRRRRRRRRRRRRRRRRRRR
%------------------------
\section{R}

\subsection{Rendering/Render}
Il rendering indica il processo di generazione di un'immagine a partire da una descrizione matematica applicata ad un set di dati.

\subsection{Repository}
Ambiente di un sistema informatico in cui vengono gestiti i metadati attraverso tabelle relazionali.
L’insieme di tabelle, regole e motori di calcolo tramite cui si gestiscono i metadati prende il nome di
metabase.

\newpage
% -----------------------
% SSSSSSSSSSSSSSSSSSSS
%------------------------
\section{S}
\subsection{Sankey Diagram}
Il diagramma di Sankey è un particolare tipo di diagramma di flusso in cui l'ampiezza delle frecce è disegnata in maniera proporzionale alla quantità di flusso.
Esso è usualmente utilizzato per indicare trasferimenti di energia, materiali, costi o dati in un processo.

\subsection{Scaling}
Lo scaling si riferisce alla funzione che permette di trasformare una colonna del file .csv$^{G}$ in una dimensione$^{G}$ associabile con un asse selezionato.

\subsection{Scatter Plot}
Il grafico a dispersione o scatter plot è un tipo di grafico in cui due variabili di un set di dati sono riportate su uno spazio cartesiano.
I dati sono visualizzati tramite una collezione di punti ciascuno con una posizione sull'asse orizzontale determinato da una variabile e sull'asse verticale determinato dall'altra. 

\subsection{Scikit-learn}
Scikit-learn è una libreria open-source$^{G}$ di apprendimento automatico per il linguaggio di programmazione Python.

\subsection{Scrum}
Scrum è un framework$^{G}$ agile, incrementale e iterativo.\newline
Si possono identificare all'interno del framework$^{G}$ tre responsabilità distinte:
\begin{itemize}
    \item \textbf{Product Owner}
    \item \textbf{Scrum Master}
    \item \textbf{Developer}
\end{itemize}
Insieme queste figure sono in grado di compiere ogni incremento richiesto dalla pianificazione.
\newline
Lo Scrum si suddivide in quattro parti
\begin{itemize}
    \item \textbf{Sprint Planning:} Prima dello Sprint$^{G}$ (o iterazione), è un evento di pianificazione
    \item \textbf{Daily Scrum:} Un evento giornaliero in cui gli sviluppatori aggiustano i loro piani per raggiungere l'obbiettivo dello Sprint$^{G}$
    \item \textbf{Sprint Review:} Un evento necessario per controllare il lavoro svolto
    \item \textbf{Sprint Retrospective:} Un ultimo meeting nello Sprint$^{G}$ con l'obbiettivo d'ispezionare gli aspetti legati alla collaborazione
\end{itemize}

\subsection{Self organizing map}
Le self-organizing map (SOM) sono un tipo di organizzazione di processi di informazione in rete, analoghi alle reti neurali artificiali. Le SOM sono particolarmente utili per la visualizzazione di dati di dimensione elevata.

\subsection{Sprint}
Lo sprint è un breve periodo di tempo dove il team di sviluppo lavora per un obiettivo comune. Gli sprint, chiamati anche
"iterazioni", essenzialmente frammentano il progetto nei blocchi di tempo in cui si possono eseguire gli obiettivi pianificati.

\subsection{Stakeholder}
Lo stakeholder è colui che ha una qualsiasi tipo di relazione/interesse nel progetto. Per l'esattezza
un Software Project Stakeholder si riferisce a una persona, gruppo o compagnia che è direttamente o indirettamente coinvolta
del progetto e che potrebbe essere affetta dal risultato del progetto. 

\newpage
% -----------------------
% TTTTTTTTTTTTTTTTTTTTT
%------------------------
\section{T}
\subsection{Telegram}
Telegram è un servizio di messaggistica istantanea e broadcasting basato su cloud ed erogato senza fini di lucro dalla società Telegram LLC.

\subsection{Template}
Il template è un punto di partenza per creare un documento da utilizzare come modello.

\subsection{Trello}
Trello è un servizio online gratuito, disponibile anche sotto forma di applicazione per sistemi mobile e desktop, che permette di organizzare e gestire i propri progetti (sia personali che professionali) tramite la creazione di bacheche tematiche condivisibili con altri utenti.

\subsection{t-SNE}
T-distributed Stochastic Neighbor Embedding è un algoritmo di riduzione della dimensionalità sviluppato da Geoffrey Hinton e Laurens van der Maaten, ampiamente utilizzato come strumento di apprendimento automatico in molti ambiti di ricerca.

\newpage
% -----------------------
% UUUUUUUUUUUUUUUUUUUUU
%------------------------
\section{U}
\subsection{UMAP}
Uniform Manifold Approximation and Projection è un algoritmo di riduzione dimensionale che può essere usato per visualizzazioni simili a quelle proposte dal \textit{t-SNE} ma anche per  riduzioni di dimensioni non lineari generiche.

\newpage
% -----------------------
% VVVVVVVVVVVVVVVVVVVVV
%------------------------
\section{V}
\subsection{Vista}
Una vista è un sottoinsieme del dataset$^{G}$, di dati base o derivati, associati a un grafico e alle dimensioni che esso mette a disposizione. Una vista tiene inoltre traccia delle personalizzazioni grafiche che l'utente ha deciso di adottare. Esistono viste create di default ma ogni utente ha la possibilità di creare la propria.

\newpage
% -----------------------
% WWWWWWWWWWWWWWWWWWWWW
%------------------------
\section{W}

\newpage
% -----------------------
% XXXXXXXXXXXXXXXXXXXXX
%------------------------
\section{X}

\newpage
% -----------------------
% YYYYYYYYYYYYYYYYYYYYY
%------------------------
\section{Y}

\newpage
% -----------------------
% ZZZZZZZZZZZZZZZZZZZZZ
%------------------------
\section{Z}
\subsection{Zapier}
Zapier è un sistema di automazione in genere utilizzato per gestire le task in modelli di tipo agile.

\subsection{Zoom}
Zoom (ufficialmente Zoom Video Communications) è una società di servizi di teleconferenza con sede a San Jose in California; fornisce servizi di conferenza remota che combina videoconferenza, riunioni online, chat e collaborazione mobile.