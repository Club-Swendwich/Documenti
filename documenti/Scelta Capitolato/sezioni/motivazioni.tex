\section{Motivazioni scelta}
La scelta riguardante il capitolato ha seguito le seguenti fasi:
\begin{enumerate}
	\item Incontro iniziale in cui sono state evidenziate le preferenze del gruppo seguendo la maggioranza, è stata individuata così una prima lista di preferenze
	\item Incontro con i proponenti dei capitolati favoriti per approfondire tecnologie e fasi di sviluppo
	\item Sondaggio all'interno del canale di comunicazione del gruppo per stilare la classifica finale di preferenza
	\item Ulteriore incontro di gruppo con valutazione ``pro e contro'' a seguito di un risultato di pareggio tra due capitolati con scelta definitiva
\end{enumerate}
Il capitolato scelto è quindi ``Login Warrior'' dell'azienda Zucchetti per le seguenti motivazioni:
\begin{itemize}
	\item Durante il primo colloquio con il proponente Gregorio Piccoli è stato evidenziato come svolgere l'analisi dati richiesta che, nonostante la mole di dati fosse stata inserita come punto sfavorevole in un primo momento, sembra rivelarsi comunque alla portata del gruppo.
	\item Il progetto è chiaro nelle sue fasi di svolgimento e nei suoi MVP, risultando quindi più semplice stimarne costi e tempi
	\item I requisiti obbligatori rappresentano una parte contenuta del progetto, che prende invece ampio spazio nelle sue parti opzionali con l'implementazione di Machine Learning; in questo modo il progetto non solo risulta più interessante in quanto più innovativo, ma consente anche di poter occupare maggiori risorse in quest'utima parte
	\item Le tecnologie proposte sono già familiari alla maggior parte del gruppo, inoltre non essendo perentorie è stata anche immaginata una possibile via alternativa che preveda l'uso di Python
	\item Infine il proponente si è mostrato da subito molto disponibile e celere nel fornire risposte e informazioni, fattore importante in quanto predisponne ad un miglior ambiente lavorativo per il gruppo
\end{itemize}