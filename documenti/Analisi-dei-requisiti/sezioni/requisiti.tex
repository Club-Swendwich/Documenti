\section{Requisiti}
\subsection{Introduzione}
Il team \textit{clubswendwich} ha classificato e assegnato i requisiti secondo quanto definito nelle \textit{Norme di Progetto}. 
\subsection{Requisiti funzionali}
{\renewcommand{\arraystretch}{1.5}
\begin{tabular}{p{0.12\linewidth}p{0.15\linewidth}p{0.50\linewidth}p{0.15\linewidth}}
	\rowcolor[RGB]{33, 73, 50}
	\textcolor{white}{\textbf{Codice}} & \textcolor{white}{\textbf{Classe}} & \textcolor{white}{\textbf{Descrizione}} &
    \textcolor{white}{\textbf{Riferimenti}}\\
    \rowcolor[RGB]{216, 235, 171}
    R1F1 & Obbligatorio & L'utente ha la possibilità di caricare i dati nel sistema & UC1\\
    \rowcolor[RGB]{233, 245, 206}
    R1F2 & Obbligatorio & L'utente può scegliere la tipologia di grafico che vuole visualizzare & UC2\\
    \rowcolor[RGB]{216, 235, 171}
    R1F2.2 & Obbligatorio & L'applicativo deve fornire la visualizzazione \textit{Scatter Plot} & UC2.1\\
    \rowcolor[RGB]{233, 245, 206}
    R1F2.2.1 & Obbligatorio & L'utente può vedere i valori di ciascun punto passandoci sopra con il mouse & Verbale\\
    \rowcolor[RGB]{216, 235, 171}
    R1F2.3 & Obbligatorio & L'applicativo deve fornire la visualizzazione \textit{Parallels coordinates} & UC2.2\\
    \rowcolor[RGB]{233, 245, 206}
    R2F3 & Obbligatorio & L'utente deve poter tornare alla visualizzazione di default & UC4\\
    \rowcolor[RGB]{216, 235, 171}
    R1F4 & Obbligatorio & L'utente deve poter scegliere quale vista di dati visualizzare & UC3\\
\end{tabular}	
}
