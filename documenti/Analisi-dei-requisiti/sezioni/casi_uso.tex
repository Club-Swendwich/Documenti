\section{Casi d'uso}
\subsection{Scopo}
L'obbiettivo di questa sezione è quello di presentare e descrivere tutti i casi d'uso, individuati dal team, in riferimento alle funzionalità richieste nell'applicazione.
\subsection{Attori}
Non essendo richiesto nessun tipo di servizio di autenticazione o più in generale una distinzione gerarchica degli utenti, è presente un solo attore nella gerarchia ovvero: l'utente generico.
\begin{figure}[h!]
    \centering
    \includegraphics[scale=0.50]{../../assets/Utente.png}
    \caption{Gerarchia attori}
\end{figure}

% --------------------------------------------------------------------
% INIZIO CASI D'USO
% CARICAMENTO DATASET
% --------------------------------------------------------------------

\subsection{UC1 - Caricamento dataset}
\begin{figure}[h!]
    \centering
    % Sistema immagine inserendo tag UC e abbelliscila dc
    \includegraphics[scale=0.60]{../../assets/Caricamento_dataset.png}
    \caption{UC1 - Caricamento ${\mathrm{dataset^{G}}}$}
\end{figure}
\begin{itemize}
    \item \textbf{Attore primario:} Utente.
    \item \textbf{Precondizioni:} Il sistema è funzionante
    \item \textbf{Postcondizioni:} Viene visualizzato un messaggio che avvisa l'utente del corretto caricamento dei dati e della loro validità. 
                                   I dati vengono caricati nel sistema
    \item \textbf{Scenario principale:}
          \begin{enumerate}
              \item L'utente seleziona il file da caricare
              \item L'utente carica il file
          \end{enumerate}
    \item \textbf{Estensioni:}
    \begin{itemize}
        \item   Nel caso in cui l'utente carichi un file in un formato non supportato
                \begin{enumerate}
                    \item I dati non vengono caricati
                    \item Viene visualizzato un messaggio di errore esplicativo [\hyperref[sec:UC14 - Errore formato file]{UC14}] % To Do: metti il link alla sezione   
                \end{enumerate}
        \item   Nel caso in cui l'utente carichi un file non correttamente strutturato
                \begin{enumerate}
                    \item I dati non vengono caricati
                    \item Viene visualizzato un messaggio di errore esplicativo [\hyperref[sec:UC15 - Errore struttura dataset]{UC15}]
                \end{enumerate}
        \item   Nel caso in cui i dati di una o più righe non siano validi
                \begin{enumerate}
                    \item Viene visualizzato un messaggio di errore esplicativo [\hyperref[sec:UC16 - Errore validità riga]{UC16}] 
                \end{enumerate}
    \end{itemize} 
\end{itemize}
\newpage

% --------------------------------------------------------------------
% SCELTA TIPO GRAFICO
% --------------------------------------------------------------------

\subsection{UC2 - Scelta tipologia grafico}
\label{sec:UC2}
\begin{figure}[h!]
    \centering
    \includegraphics[scale=0.60]{../../assets/Selezione_tipo_grafico.png}
    \caption{UC2 - Scelta tipologia grafico}
\end{figure}
\begin{itemize}
    \item \textbf{Attore primario}: Utente.
    \item \textbf{Precondizioni}: Il ${\mathrm{dataset^{G}}}$ è stato caricato correttamente.
    \item \textbf{Postcondizioni}: Viene mostrata la visualizzazione scelta.
    \item \textbf{Scenario principale}:
          \begin{enumerate}
              \item L'utente seleziona il tipo di grafico da visualizzare tra le tipologie disponibili.
          \end{enumerate}
    \item \textbf{Generalizzazioni}:
    \begin{itemize}
        \item L'utente seleziona una delle seguenti opzioni:
                \begin{enumerate}
                    \item \textit{$Scatter$ $Plot^{G}$} \hyperref[sec:UC2.1]{UC2.1}.
                    \item \textit{$Parallel$ $Coordinates^{G}$} \hyperref[sec:UC2.2]{UC2.2}.
                    \item \textit{$Force-directed$ $Graph^{G}$} \hyperref[sec:UC2.3]{UC2.3}.
                    \item \textit{$Sankey$ $Diagram^{G}$} \hyperref[sec:UC2.4]{UC2.4}.
                \end{enumerate}
    \end{itemize} 
\end{itemize}

\subsubsection{UC2.1 - Selezione Scatter Plot}
\label{sec:UC2.1}
\begin{itemize}
    \item \textbf{Attore primario}: Utente.
    \item \textbf{Precondizioni}: Il ${\mathrm{dataset^{G}}}$ è stato caricato correttamente [UC2].
    \item \textbf{Postcondizioni}: Viene mostrata la visualizzazione $Scatter$ $Plot^{G}$ scelta dall'utente con possibilità di selezionare una diversa vista$^{G}$ \hyperref[sec:UC3]{UC3} e personalizzare grafico e stile \hyperref[sec:UC5.1]{UC5.1}. %inserire il numero corretto dello stile e delle dimensioni
    \item \textbf{Scenario principale}:
          \begin{enumerate}
              \item L'utente seleziona il grafico $Scatter$ $Plot^{G}$ e il sistema ritorna tale grafico con conseguente possibilità di personalizzazione. 
          \end{enumerate}
\end{itemize}

\subsubsection{UC2.2 - Selezione Parallel Coordinates}
\label{sec:UC2.2}
\begin{itemize}
    \item \textbf{Attore primario}: Utente.
    \item \textbf{Precondizioni}: Il ${\mathrm{dataset^{G}}}$ è stato caricato correttamente [UC2].
    \item \textbf{Postcondizioni}: Viene mostrata la visualizzazione $Parallel$ $Coordinates^{G}$ scelta dall'utente con possibilità di selezionare una diversa vista$^{G}$ \hyperref[sec:UC3]{UC3} e personalizzare grafico e stile \hyperref[sec:UC5.2]{UC5.2}. %inserire il numero corretto dello stile e delle dimensioni
    \item \textbf{Scenario principale}:
          \begin{enumerate}
              \item L'utente seleziona il grafico $Parallel$ $Coordinates^{G}$ e il sistema ritorna tale grafico con conseguente possibilità di personalizzazione. 
          \end{enumerate}
\end{itemize}

\subsubsection{UC2.3 - Selezione Force-directed Graph}
\label{sec:UC2.3}
\begin{itemize}
    \item \textbf{Attore primario}: Utente.
    \item \textbf{Precondizioni}: Il ${\mathrm{dataset^{G}}}$ è stato caricato correttamente [UC2].
    \item \textbf{Postcondizioni}: Viene mostrata la visualizzazione $Force-directed$ $Graph^{G}$ scelta dall'utente con possibilità di selezionare una diversa vista$^{G}$ \hyperref[sec:UC3]{UC3} e personalizzare grafico e stile \hyperref[sec:UC5.3]{UC5.3}. %inserire il numero corretto dello stile e delle dimensioni
    \item \textbf{Scenario principale}:
          \begin{enumerate}
              \item L'utente seleziona il grafico $Force-directed$ $Graph^{G}$ e il sistema ritorna tale grafico con conseguente possibilità di personalizzazione. 
          \end{enumerate}
\end{itemize}

\subsubsection{UC2.4 - Selezione Sankey Diagram}
\label{sec:UC2.4}
\begin{itemize}
    \item \textbf{Attore primario}: Utente.
    \item \textbf{Precondizioni}: Il ${\mathrm{dataset^{G}}}$ è stato caricato correttamente [UC2].
    \item \textbf{Postcondizioni}: Viene mostrata la visualizzazione $Sankey$ $Diagram^{G}$ scelta dall'utente con possibilità di selezionare una diversa vista$^{G}$ \hyperref[sec:UC3]{UC3} e personalizzare grafico e stile \hyperref[sec:UC5.4]{UC5.4}. %inserire il numero corretto dello stile e delle dimensioni
    \item \textbf{Scenario principale}:
          \begin{enumerate}
              \item L'utente seleziona il grafico $Sankey$ $Diagram^{G}$ e il sistema ritorna tale grafico con conseguente possibilità di personalizzazione. 
          \end{enumerate}
\end{itemize}
\newpage

%               _ _              _       _                 _
%  ___  ___ ___| | |_ __ _    __| | __ _| |_ __ _ ___  ___| |_
% / __|/ __/ _ \ | __/ _` |  / _` |/ _` | __/ _` / __|/ _ \ __|
% \__ \ (_|  __/ | || (_| | | (_| | (_| | || (_| \__ \  __/ |_
% |___/\___\___|_|\__\__,_|  \__,_|\__,_|\__\__,_|___/\___|\__|


\subsection{UC3 - Eliminazione Vista}
\label{sec:UC3}
\begin{figure}[h!]
    \centering
    % Controlla che UC abbia il numero corretto
    \includegraphics[scale=0.60]{../../assets/eliminazione_vista.png}
    \caption{UC3 - Eliminazione di una vista$^{G}$}
\end{figure}
\begin{itemize}
    \item \textbf{Attore primario}: Utente.
    \item \textbf{Precondizioni}: L'utente sta visualizzando una vista$^{G}$.
    \item \textbf{Postcondizioni}: Viene eliminata la vista selezionata.
    \item \textbf{Scenario principale}:
          \begin{enumerate}
              \item L'utente preme il pulsante elimina vista$^{G}$.
              \item La vista$^{G}$ viene eliminata.
              \item L'utente viene riportato alla visualizzazione di default.
          \end{enumerate}
\end{itemize}

\subsubsection{UC3.1 - Visualizzazione messaggio di conferma eliminazione}
\label{sec:UC3.1}
\begin{itemize}
	\item \textbf{Attore primario:} Utente.
	\item \textbf{Precondizioni:} Il ${\mathrm{dataset^{G}}}$ è presente e caricato correttamente. L'utente ha cliccato il pulsante per l'eliminazione di una vista$^{G}$ \hyperref[sec:UC3]{UC3}.
	\item \textbf{Postcondizioni:} Viene eliminata la vista$^{G}$, oppure l'operazione viene annullata.
	\item \textbf{Scenario principale:}
		\begin{enumerate}
		\item Viene visualizzato un messaggio di conferma per l'eliminazione della vista$^{G}$.
		\item L'utente sceglie una delle due opzioni:
			\begin{itemize}
				\item L'utente vuole ripristinare eliminare la vista$^{G}$ e clicca ``Ok''
				\item L'utente vuole annullare l'operazione e clicca ``Annulla''
			\end{itemize}
		\end{enumerate}
\end{itemize}

\newpage

% --------------------------------------------------------------------
% VISUALIZZAZIONE DI DEFAULT DEI GRAFICI
% --------------------------------------------------------------------

\subsection{UC4 - Visualizzazione di default dei grafici}
\label{sec:UC4}
\begin{figure}[h!]
    \centering
    \includegraphics[scale=0.60]{../../assets/visualizzazione_default.png}
	\caption{UC4 - Visualizzazione di default dei grafici}
\end{figure}

\begin{itemize}
	\item \textbf{Attore primario:} Utente.
	\item \textbf{Precondizioni:} Il ${\mathrm{dataset^{G}}}$ è presente e caricato correttamente.
	\item \textbf{Postcondizioni:} Viene mostrato il messaggio relativo al ripristino delle impostazioni di default \hyperref[sec:UC4.1]{UC4.1}.
	\item \textbf{Scenario principale:}
	\begin{itemize}
		\item L'utente clicca sul pulsante ``Visualizzazione di default''.
		\item L'utente visualizza un messaggio di conferma della scelta.
	\end{itemize}
\end{itemize}

\subsubsection{UC4.1 - Visualizzazione messaggio di conferma scelta}
\label{sec:UC4.1}
\begin{itemize}
	\item \textbf{Attore primario:} Utente.
	\item \textbf{Precondizioni:} Il ${\mathrm{dataset^{G}}}$ è presente e caricato correttamente, visualizzata nella finestra ci può essere un qualsivoglia grafico con qualsiasi tipo di personalizzazione applicata dall'utente.
    L'utente ha cliccato il pulsante per il ripristino delle impostazioni di default \hyperref[sec:UC4]{UC4}.
	\item \textbf{Postcondizioni:} Viene ripristinato il grafico con le impostazioni di default, il che implica che il grafico ritorna alle impostazioni
    di personalizzazione prima delle modifiche dell'utente, oppure l'operazione viene annullata.
	\item \textbf{Scenario principale:}
		\begin{enumerate}
		\item Viene visualizzato un messaggio di conferma per tornare alla visualizzazione di default del grafico.
		\item L'utente sceglie una delle due opzioni:
			\begin{itemize}
				\item L'utente vuole ripristinare le impostazioni di default e clicca ``Ok''
				\item L'utente vuole annullare l'operazione e clicca ``Annulla''
			\end{itemize}
		\end{enumerate}
        \item Nel caso l'utente abbia cliccato l'opzione di tornare alle impostazioni di visualizzazione di default,
              il grafico in questione ritorna alle impostazioni di personalizzazione prima delle modifiche di personalizzazione
              selezionate dell'utente.
        \item Nel caso invece in cui l'utente abbia cliccato l'operazione ``Annulla'', ritornerà alle impostazioni di personalizzazione 
              precedentemente impostate.
\end{itemize}

\newpage

% --------------------------------------------------------------------
% SEZIONE PERSONALIZZAZIONE DEI GRAFICI
% --------------------------------------------------------------------

\subsection{UC5 - Personalizzazione dei grafici}
\begin{figure}[h!]
	\centering
	\includegraphics[scale=0.60]{../../assets/personalizzazioneVisivaGrafici.png}
	\caption{UC5 - Personalizzazione dei grafici}
\end{figure}
\begin{itemize}
	\item \textbf{Attore primario:} Utente.
	\item \textbf{Precondizioni:} L'utente ha scelto uno dei grafici a disposizione nell'interfaccia \hyperref[sec:UC2]{UC2}.
	\item \textbf{Postcondizioni:} Il grafico viene ridisegnato secondo le scelte selezionate dall'utente.
	\item \textbf{Scenario principale:}
	\begin{enumerate}
		\item Per ogni grafico l'utente ha a disposizione delle modifiche specifiche su: ${\mathrm{dimensione^{G}}}$, ordinamento e numero degli assi, scelta della funzione di forza, dell'algoritmo di integrazione, e ordinamento dei dati.
    		\item L'utente modifica anche lo stile del grafico.
    		\item Nel caso si scelga di tornare alla vista$^{G}$ di default, le modifiche eseguite torneranno ai valori standard.
    \end{enumerate}
    \item \textbf{Generalizzazioni}:
    \begin{itemize}
        \item L'utente seleziona una delle seguenti opzioni:
                \begin{enumerate}
                    \item \textit{Personalizza $Scatter$ $Plot^{G}$} \hyperref[sec:UC5.1]{UC5.1}.
                    \item \textit{Personalizza $Parallel$ $Coordinates^{G}$} \hyperref[sec:UC5.2]{UC5.2}.
                    \item \textit{Personalizza $Force-directed$ $Graph^{G}$} \hyperref[sec:UC5.3]{UC5.3}.
                    \item \textit{Personalizza $Sankey$ $Diagram^{G}$} \hyperref[sec:UC5.4]{UC5.4}.
                \end{enumerate}
    \end{itemize} 
\end{itemize}

% SCATTER PLOT
\newpage
\subsubsection{UC5.1 - Personalizzazione Scatter Plot}
\label{sec:UC5.1}
\begin{figure}[h!]
	\centering
	\includegraphics[scale=0.60]{../../assets/personalizzazioneScatterPlot.png}
	\caption{UC5.1 - Personalizzazione $Scatter$ $Plot^{G}$}
\end{figure}
\begin{itemize}
    \item \textbf{Attore primario:} Utente.
	\item \textbf{Precondizioni:} L'utente ha scelto il grafico $Scatter$ $Plot^{G}$. \hyperref[sec:UC2.1]{UC2.1}.
	\item \textbf{Postcondizioni:} 
	Il grafico viene ridisegnato secondo le scelte selezionate dall'utente.
	\item \textbf{Scenario principale:} L'utente decide:
	\begin{enumerate}
        \item Associazione delle ${\mathrm{dimensioni^{G}}}$ agli assi \hyperref[sec:UC5.1.1]{UC5.1.1}.
        \item Personalizzazione dello stile \hyperref[sec:UC5.1.2]{UC5.1.2}.
    \end{enumerate}
\end{itemize}
\paragraph{UC5.1.1 - Associazione delle dimensioni agli assi}
\label{sec:UC5.1.1}
    \begin{itemize}
        \item \textbf{Attore primario:} Utente.
        \item \textbf{Precondizioni:} L'utente ha selezionato il grafico $Scatter$ $Plot^{G}$ \hyperref[sec:UC2.1]{UC2.1}.
	    \item \textbf{Postcondizioni:} L'utente ha associato le ${\mathrm{dimensioni^{G}}}$ disponibili con gli assi del grafico.
	    \item \textbf{Scenario principale:} 
	    \begin{enumerate}
	    		\item L'utente sceglie quali ${\mathrm{dimensioni^{G}}}$ associare agli assi del grafico.
		\end{enumerate}
	    \item \textbf{Estensioni:} Nel caso l'utente non abbia associato correttamente le ${\mathrm{dimensioni^{G}}}$ agli assi del grafico:
              \begin{itemize}
                  \item Non viene visualizzato nessun grafico.
                  \item Viene visualizzato un errore di personalizzazione grafico \hyperref[sec:UC17 - Errore di personalizzazione]{UC17}.
              \end{itemize}
    \end{itemize}
\paragraph{UC5.1.2 - Personalizzazione dello stile}
\label{sec:UC5.1.2}
    \begin{itemize}
        \item \textbf{Attore primario:} Utente.
        \item \textbf{Precondizioni:} L'utente ha selezionato il grafico $Scatter$ $Plot^{G}$ \hyperref[sec:UC2.1]{UC2.1}.
	    \item \textbf{Postcondizioni:} L'utente ha associato nuovi colori e forme ai punti.
	    \item \textbf{Scenario principale:} 
	    \begin{enumerate}
	    		\item L'utente visualizza le opzioni di colori per la personalizzazione di ogni specifica del grafico, il che comprende ma
                non è limitato a: forma dei punti, dimensione dei punti, colore dei punti.
	    		\item Nel caso l'utente non li modifichi rimangono le impostazione di personalizzazione di default.
		\end{enumerate}
    \end{itemize}

% PARALLEL CORDINATES 
\subsubsection{UC5.2 - Personalizzazione Parallel Coordinates}
\label{sec:UC5.2}
\begin{figure}[h!]
	\centering
	\includegraphics[scale=0.60]{../../assets/personalizzazioneParallelCoordinates.png}
	\caption{UC5.2 - Personalizzazione $Parallel$ $Coordinates^{G}$}
\end{figure}
\begin{itemize}
    \item \textbf{Attore primario:} Utente.
	\item \textbf{Precondizioni:} L'utente ha scelto il grafico $Parallel$ $Coordinates^{G}$ \hyperref[sec:UC2.2]{UC2.2}.
	\item \textbf{Postcondizioni:} Il grafico viene ridisegnato secondo le scelte selezionate dall'utente.
	\item \textbf{Scenario principale:} L'utente decide:
	\begin{enumerate}
        \item Associazione delle ${\mathrm{dimensioni^{G}}}$ \hyperref[sec:UC5.2.1]{UC5.2.1}.
        \item Ordine degli assi \hyperref[sec:UC5.2.2]{UC5.2.2}.
        \item \textit{Scaling} dei dati \hyperref[sec:UC5.2.3]{UC5.2.3}.
        \item Personalizzazione dello stile \hyperref[sec:UC5.2.4]{UC5.2.4}.
    \end{enumerate}
\end{itemize}

\paragraph{UC5.2.1 - Associazione delle dimensioni}
\label{sec:UC5.2.1}
    \begin{itemize}
        \item \textbf{Attore primario:} Utente.
        \item \textbf{Precondizioni:} L'utente ha selezionato $Parallel$ $Coordinates^{G}$ \hyperref[sec:UC2.2]{UC2.2}.
	    \item \textbf{Postcondizioni:} L'utente ha associato le ${\mathrm{dimensioni^{G}}}$ disponibili con gli assi del grafico.
	    \item \textbf{Scenario principale:}
	    \begin{enumerate}
	    		\item L'utente sceglie quali ${\mathrm{dimensioni^{G}}}$ associare agli assi del grafico.
		\end{enumerate}
	    \item \textbf{Estensioni:} Nel caso l'utente non abbia associato correttamente le ${\mathrm{dimensioni^{G}}}$ agli assi del grafico:
              \begin{itemize}
                  \item Non viene visualizzato nessun grafico.
                  \item Viene visualizzato un errore di personalizzazione grafico \hyperref[sec:UC17 - Errore di personalizzazione]{UC17}.
              \end{itemize}
    \end{itemize}
\paragraph{UC5.2.2 - Ordine e numero degli assi}
\label{sec:UC5.2.2}
    \begin{itemize}
        \item \textbf{Attore primario:} Utente.
        \item \textbf{Precondizioni:} L'utente ha selezionato $Parallel$ $Coordinates^{G}$ \hyperref[sec:UC2.2]{UC2.2}.
	    \item \textbf{Postcondizioni:} L'utente ha modificato l'ordine verticale degli assi e il numero, il grafico quindi riduce o aumenta gli intrecci paralleli.
	    \item \textbf{Scenario principale:} 
	    \begin{enumerate}
	    		\item L'utente sceglie ordine e numero degli assi.
		\end{enumerate}
	    \item \textbf{Estensioni:} Nel caso l'utente abbia creato nuovi assi ma non abbia associato nessuna ${\mathrm{dimensione^{G}}}$ a essi:
              \begin{itemize}
                  \item Non viene visualizzato nessun grafico.
                  \item Viene visualizzato un errore di personalizzazione grafico.
              \end{itemize}
    \end{itemize}
    
\paragraph{UC5.2.3 - Scaling dei dati}
\label{sec:UC5.2.3}
    \begin{itemize}
        \item \textbf{Attore primario:} Utente.
        \item \textbf{Precondizioni:} L'utente ha selezionato $Parallel$ $Coordinates^{G}$ \hyperref[sec:UC2.2]{UC2.2}.
	    \item \textbf{Postcondizioni:} L'utente ha ridimensionato i dati in una nuova scala comune tra le variabili o viceversa.
	    \item \textbf{Scenario principale:} 
	    \begin{enumerate}
	    		\item L'utente sceglie lo \textit{scaling} dei dati.
		\end{enumerate}
    \end{itemize}

\paragraph{UC5.2.4 - Personalizzazione dello stile}
\label{sec:UC5.2.4}
    \begin{itemize}
        \item \textbf{Attore primario:} Utente.
        \item \textbf{Precondizioni:} L'utente ha selezionato $Parallel$ $Coordinates^{G}$ \hyperref[sec:UC2.2]{UC2.2}.
	    \item \textbf{Postcondizioni:} L'utente ha evidenziato uno specifico gruppo di linee di interesse modificandone i colori.
	    \item \textbf{Scenario principale:} 
	    \begin{enumerate}
	    		\item L'utente visualizza le opzioni di colori per la personalizzazione, le quali comprendono ma non sono limitate a:
                      dimensione e colore dei punti sugli assi, colore dei collegamenti tra nodi, opacità dei collegamenti tra nodi.
	    		\item Nel caso l'utente non li modifichi rimangono quelli di default.
		\end{enumerate}
    \end{itemize}


% FORCE-DIRECTED GRAPH 
\subsubsection{UC5.3 - Personalizzazione Force-directed Graph}
\label{sec:UC5.3}
\begin{figure}[h!]
	\centering
	\includegraphics[scale=0.60]{../../assets/personalizzazioneForce-directedGraph.png}
	\caption{UC5.3 - Personalizzazione $Force-directed$ $Graph^{G}$}
\end{figure}
\begin{itemize}
    \item \textbf{Attore primario:} Utente.
	\item \textbf{Precondizioni:} L'utente ha scelto il grafico $Force-directed$ $Graph^{G}$ \hyperref[sec:UC2.3]{UC2.3}.
	\item \textbf{Postcondizioni:} Il grafico viene ridisegnato secondo le scelte selezionate dall'utente.
	\item \textbf{Scenario principale:}L'utente decide:
	\begin{enumerate}
        \item Scelta del tipo di algoritmo d'integrazione \hyperref[sec:UC5.3.1]{UC5.3.1}.
        \item Scelta della funzione di forza \hyperref[sec:UC5.3.2]{UC5.3.2}.
        \item Personalizzazione dello stile \hyperref[sec:UC5.3.3]{UC5.3.3}.
    \end{enumerate}
\end{itemize}
\paragraph{UC5.3.1 - Scelta tipo di algoritmo d'integrazione}
\label{sec:UC5.3.1}
    \begin{itemize}
        \item \textbf{Attore primario:} Utente.
        \item \textbf{Precondizioni:} L'utente ha selezionato $Force-directed$ $Graph^{G}$ \hyperref[sec:UC2.3]{UC2.3}.
	    \item \textbf{Postcondizioni:} L'utente ha scelto l'algoritmo d'integrazione che desidera.
	    \item \textbf{Scenario principale:}
	    \begin{enumerate}
	    		\item L'utente sceglie l'algoritmo d'integrazione che desidera.
		\end{enumerate}
    \end{itemize}
\paragraph{UC5.3.2 - Scelta della funzione di forza}
\label{sec:UC5.3.2}
    \begin{itemize}
        \item \textbf{Attore primario:} Utente.
        \item \textbf{Precondizioni:} L'utente ha selezionato $Force-directed$ $Graph^{G}$ \hyperref[sec:UC2.3]{UC2.3}.
	    \item \textbf{Postcondizioni:} L'utente ha deciso la funzione di forza per i nodi.
	    \item \textbf{Scenario principale:} 
	    \begin{enumerate}
	    		\item L'utente sceglie i parametri per la rappresentazione dei rapporti tra i nodi.
		\end{enumerate}
	    \item \textbf{Estensioni:} Nel caso i parametri non rientrino nei limiti fissati:
              \begin{itemize}
                  \item Non viene visualizzato nessun grafico.
                  \item Viene visualizzato un errore di personalizzazione grafico \hyperref[sec:UC17 - Errore di personalizzazione]{UC17}.
              \end{itemize}
    \end{itemize}
\paragraph{UC5.3.3 - Personalizzazione dello stile}
\label{sec:UC5.3.3}
    \begin{itemize}
        \item \textbf{Attore primario:} Utente.
        \item \textbf{Precondizioni:} L'utente ha selezionato $Force-directed$ $Graph^{G}$ \hyperref[sec:UC2.3]{UC2.3}.
	    \item \textbf{Postcondizioni:} L'utente ha selezionato le personalizzazioni per i \textit{cluster} dei nodi.
	    \item \textbf{Scenario principale:} 
	    \begin{enumerate}
	    		\item L'utente visualizza le opzioni per la personalizzazione di ogni specifica del grafico, che comprendono ma non sono limitati da:
                dimensione nodi, colore dei \textit{cluster} di nodi, forma dei nodi.
	    		\item Nel caso l'utente non li modifichi rimangono quelli di default.
		\end{enumerate}
    \end{itemize}

\newpage
% SANKEY DIAGRAM
\subsubsection{UC5.4 - Personalizzazione Sankey Diagram}
\label{sec:UC5.4}
\begin{figure}[h!]
	\centering
	\includegraphics[scale=0.60]{../../assets/personalizzazioneSankey.png}
	\caption{UC5.4 - Personalizzazione $Sankey$ $Diagram^{G}$}
\end{figure}
\begin{itemize}
    \item \textbf{Attore primario:} Utente.
	\item \textbf{Precondizioni:} L'utente ha scelto il grafico $Sankey$ $Diagram^{G}$ \hyperref[sec:UC2.4]{UC2.4}.
	\item \textbf{Postcondizioni:} Il grafico viene ridisegnato secondo le scelte selezionate dall'utente.
	\item \textbf{Scenario principale:}L'utente decide:
	\begin{enumerate}
        \item Ordinamento dei dati \hyperref[sec:UC5.4.1]{UC5.4.1}.
        \item Personalizzazione dello stile \hyperref[sec:UC5.4.2]{UC5.4.2}.
    \end{enumerate}
\end{itemize}
\paragraph{UC5.4.1 - Ordinamento dei dati}
\label{sec:UC5.4.1}
    \begin{itemize}
        \item \textbf{Attore primario:} Utente.
        \item \textbf{Precondizioni:} L'utente ha selezionato $Sankey$ $Diagram^{G}$ \hyperref[sec:UC2.4]{UC2.4}.
	    \item \textbf{Postcondizioni:} L'utente ha deciso le ${\mathrm{dimensioni^{G}}}$ da rappresentare.
	    \item \textbf{Scenario principale:}
	    \begin{enumerate}
	    		\item L'utente sceglie i parametri per la rappresentazione dei rapporti tra i nodi.
		\end{enumerate}
	    \item \textbf{Estensioni:} Se non viene selezionata almeno un'origine tra i dati:
              \begin{itemize}
                  \item Non viene visualizzato nessun grafico.
                  \item Viene visualizzato un errore di personalizzazione grafico \hyperref[sec:UC17 - Errore di personalizzazione]{UC17}.
              \end{itemize}
    \end{itemize}
\paragraph{UC5.4.2 - Personalizzazione dello stile}
\label{sec:UC5.4.2}
\begin{itemize}
    \item \textbf{Attore primario:} Utente.
    \item \textbf{Precondizioni:} L'utente ha selezionato $Sankey$ $Diagram^{G}$ \hyperref[sec:UC2.4]{UC2.4}.
	\item \textbf{Postcondizioni:} L'utente ha selezionato i colori per le fasce del grafico.
	\item \textbf{Scenario principale:}
	\begin{enumerate}
		\item L'utente visualizza le opzioni per la personalizzazione di ogni specifica del grafico, che sono comprese ma non limitate a:
        colore e forma dei nodi, opacità dei collegamenti tra nodi, colore collegamento tra i nodi, spessore e opacità collegamento tra nodi.
		\item Nel caso l'utente non li modifichi rimangono quelli di default.
	\end{enumerate}
\end{itemize}

\newpage

% --------------------------------------------------------------------
% VISUALIZZAZIONE GRAFICO A TUTTO SCHERMO
% --------------------------------------------------------------------

\subsection{UC6 - Visualizzazione di un grafico a schermo intero}
\begin{figure}[h!]
	\centering
	\includegraphics[scale=0.60]{../../assets/visualizzazione_fullscreen.png}
	\caption{UC6 - Visualizzazione di un grafico a schermo intero}
\end{figure}

\begin{itemize}
	\item \textbf{Attore primario:} Utente.
	\item \textbf{Precondizioni:} Il ${\mathrm{dataset^{G}}}$ è stato caricato correttamente. L'utente visualizza almeno un grafico sulla pagina. 
	\item \textbf{Postcondizioni:} 
	Viene mostrato il grafico scelto in \textit{fullscreen}.
	\item \textbf{Scenario principale:}
		\begin{enumerate}
			\item L'utente clicca sul pulsante di \textit{fullscreen} di un grafico.
			\item Il grafico viene visualizzato a schermo intero.
		\end{enumerate}
\end{itemize}
\newpage

% --------------------------------------------------------------------
% SEZIONE ZOOM INTERATTIVO
% --------------------------------------------------------------------

\subsection{UC7 - Zoom con selezione}
\label{sec:UC7}
\begin{figure}[h!]
    \centering
    \includegraphics[scale=0.60]{../../assets/zoom_selezione.png}
    \caption{UC7 - Zoom con selezione sul grafico}
\end{figure}
\begin{itemize}
    \item \textbf{Attore primario}: Utente.
    \item \textbf{Precondizioni}: Il ${\mathrm{dataset^{G}}}$ è stato caricato correttamente. \par L'utente ha scelto la seguente tipologia di grafico:
    \begin{itemize}
    		\item $Scatter$ $Plot^{G}$.
    \end{itemize}
    \item \textbf{Postcondizioni}: Viene mostrato uno zoom della sezione selezionata dal mouse.
    \item \textbf{Scenario principale}:
          \begin{enumerate}
              \item L'utente seleziona con il mouse l'area all'interno del grafico che vuole ingrandire.
              \item L'area selezionata viene ingrandita.
          \end{enumerate}
\end{itemize}

\subsection{UC8 - Zoom con scroll del mouse}
\label{sec:UC8}
\begin{figure}[h!]
    \centering
    \includegraphics[scale=0.60]{../../assets/zoom_mouse.png}
    \caption{UC8 - Zoom con lo scroll del mouse}
\end{figure}
\begin{itemize}
    \item \textbf{Attore primario}: Utente.
    \item \textbf{Precondizioni}: Il ${\mathrm{dataset^{G}}}$ è stato caricato correttamente. \par L'utente ha scelto la seguente tipologia di grafico:
    \begin{itemize}
          \item $Force-directed$ $Graph^{G}$.
    \end{itemize}
    \item \textbf{Postcondizioni}: Viene mostrato uno zoom dove si trova il puntatore del mouse.
    \item \textbf{Scenario principale}:
          \begin{enumerate}
              \item L'utente posiziona il puntatore del mouse sulla zona di interesse.
              \item Per zoomare, l'utente esegue lo \textit{scroll} del mouse o utilizza le \textit{gesture} per lo zoom del \textit{trackpad}.
              \item L'area selezionata viene ingrandita.
          \end{enumerate}
\end{itemize}

\newpage

% --------------------------------------------------------------------
% SEZIONE MOUSE HOVER
% --------------------------------------------------------------------
\subsection{UC9 - Visualizzazione dettagli aggiuntivi}
\label{sec:UC9}
\begin{figure}[h!]
    \centering
    % Controlla che UC abbia il numero corretto
    \includegraphics[scale=0.60]{../../assets/UC9-DettagliAggiuntivi.png}
    \caption{UC9 - Visualizzazione dettagli aggiuntivi}
\end{figure}
\begin{itemize}
    \item \textbf{Attore primario:} Utente.
    \item \textbf{Precondizioni:} Il ${\mathrm{dataset^{G}}}$ è stato caricato correttamente. L'utente ha scelto quale tipologia di grafico visualizzare e ha selezionato una vista$^{G}$.
    \item \textbf{Postcondizioni:} Vengono mostrate all'utente informazioni utili relative all'elemento su cui il mouse sta eseguendo l'hover.
    \item \textbf{Scenario principale}: 
    \begin{enumerate}
		\item L'utente passa con il mouse sopra ad un elemento di interesse. 
		\item Visualizza delle informazioni utili.
		\item L'elemento viene evidenziato. 
	\end{enumerate}
	Le tipologie di informazioni fornite dipendono dal grafico con cui l'utente interagisce, per questo vengono riportate le seguenti generalizzazioni.
    \item \textbf{Generalizzazioni:} \begin{enumerate}
                                        \item Dettagli aggiuntivi su \textit{$Scatter$ $Plot^{G}$} [\hyperref[sec:UC9.1]{UC9.1}]
                                        \item Dettagli aggiuntivi su \textit{$Parallel$ $Coordinates^{G}$} [\hyperref[sec:UC9.2]{UC9.2}]
                                        \item Dettagli aggiuntivi su \textit{$Force-directed$ $Graph^{G}$} [\hyperref[sec:UC9.3]{UC9.3}]
                                        \item Dettagli aggiuntivi su \textit{$Sankey$ $Diagram^{G}$} [\hyperref[sec:UC9.4]{UC9.4}]
                                    \end{enumerate}
\end{itemize}

\subsubsection{UC9.1 - Dettagli aggiuntivi su Scatter Plot}
\label{sec:UC9.1}
\begin{itemize}
    \item \textbf{Attore primario:} Utente.
    \item \textbf{Precondizioni:} Il ${\mathrm{dataset^{G}}}$ è stato caricato correttamente. L'utente ha scelto di visualizzare il grafico $Scatter$ $Plot^{G}$ e ha selezionato una vista$^{G}$.
    \item \textbf{Postcondizioni:} L'utente passando sopra con il mouse ad un punto sul grafico ottiene più informazioni relative a quel punto e lo evidenzia.
    \item \textbf{Scenario principale}: 
    \begin{enumerate}
		\item L'utente passa con il mouse sopra ad un punto sul grafico.
		\item Visualizza un'etichetta che contiene informazioni più complete relative a quel punto, come ad esempio: valore delle coordinate delle x, valore delle coordinate delle y. 
		\item Viene evidenziato il punto rispetto a tutti gli altri.
	\end{enumerate}
\end{itemize}

\subsubsection{UC9.2 - Dettagli aggiuntivi su Parallel Coordinates}
\label{sec:UC9.2}
\begin{itemize}
    \item \textbf{Attore primario:} Utente.
    \item \textbf{Precondizioni:} Il ${\mathrm{dataset^{G}}}$ è stato caricato correttamente. L'utente ha scelto di visualizzare il grafico $Parallel$ $Coordinates^{G}$ e ha selezionato una vista$^{G}$.
    \item \textbf{Postcondizioni:} L'utente passando sopra con il mouse a una linea, a un gruppo di linee o ad un asse ottiene maggiori informazioni a riguardo.
    \item \textbf{Scenario principale}:
    \begin{enumerate}
		\item L'utente passa con il mouse sopra ad una linea, ad un gruppo di linee o ad un asse sul grafico.
		\item Visualizza un'etichetta che contiene informazioni più complete relative a quella linea, a quel gruppo di linee o a quell'asse.
		\item Vengono evidenziati la linea, il gruppo intero o l'asse stesso.
	\end{enumerate}
\end{itemize}

\subsubsection{UC9.3 - Dettagli aggiuntivi su Force-directed Graph}
\label{sec:UC9.3}
\begin{itemize}
    \item \textbf{Attore primario:} Utente.
    \item \textbf{Precondizioni:} Il ${\mathrm{dataset^{G}}}$ è stato caricato correttamente. L'utente ha scelto di visualizzare il grafico $Force-directed$ $Graph^{G}$ e ha selezionato una vista$^{G}$.
    \item \textbf{Postcondizioni:} L'utente passando sopra con il mouse a un nodo del grafo lo evidenzia ed ottiene maggiori informazioni a riguardo.
    \item \textbf{Scenario principale}:
    \begin{enumerate}
		\item L'utente passa con il mouse sopra ad un nodo del grafo.
		\item Visualizza un \textit{tooltip} che mostra maggiori informazioni a riguardo. Ad esempio se un nodo rappresentasse il singolo tentativo di login, passandoci sopra con il mouse potrei visulizzare informazioni come l'ora in cui è avvenuto, l'indirizzo ip da cui si è verificato etc.
		\item Viene evidenziato il nodo.
	\end{enumerate}
\end{itemize}

\subsubsection{UC9.4 - Dettagli aggiuntivi su Sankey Diagram}
\label{sec:UC9.4}
\begin{itemize}
    \item \textbf{Attore primario:} Utente.
    \item \textbf{Precondizioni:} Il ${\mathrm{dataset^{G}}}$ è stato caricato correttamente. L'utente ha scelto di visualizzare il grafico $Sankey$ $Diagram^{G}$ e ha selezionato una vista$^{G}$.
    \item \textbf{Postcondizioni:} L'utente passando sopra con il mouse ad elemento nel grafico lo evidenzia.
    \item \textbf{Scenario principale}: 
    \begin{enumerate}
		\item L'utente passa con il mouse sopra ad un nodo del grafo.
		\item Vengono evidenziati il nodo ed i collegamenti da lui entranti e uscenti.
		\item L'utente passa con il mouse sopra ad un collegamento.
		\item Viene evidenziato il collegamento.
	\end{enumerate}
\end{itemize}

\newpage


% --------------------------------------------------------------------
% SEZIONE UC10 - DRAG DI UN NODO - FORCE DIRECTED
% --------------------------------------------------------------------
\subsection{UC10 - Drag di un nodo}
\label{sec:UC10}
\begin{figure}[h!]
    \centering
    \includegraphics[scale=0.60]{../../assets/drag_nodo.png}
    \caption{UC10 - Drag di un nodo nel grafico $Force-directed$ $Graph^{G}$}
\end{figure}
\begin{itemize}
    \item \textbf{Attore primario}: Utente.
    \item \textbf{Precondizioni}: Il ${\mathrm{dataset^{G}}}$ è stato caricato correttamente. \par L'utente ha scelto la seguente tipologia di grafico:
    \begin{itemize}
    		\item $Force-directed$ $Graph^{G}$.
    \end{itemize}
    \item \textbf{Postcondizioni}: Il nodo viene trascinato secondo il movimento del mouse dell'utente.
    \item \textbf{Scenario principale}:
          \begin{enumerate}
              \item L'utente seleziona con il mouse il nodo da trascinare e lo clicca.
              \item Tenendo il nodo cliccato lo trascina a suo piacimento.
          \end{enumerate}
\end{itemize}

\newpage

% --------------------------------------------------------------------
% SEZIONE UC11 - CREAZIONE NUOVA VISTA
% --------------------------------------------------------------------
\subsection{UC11 - Creazione nuova vista}
\label{sec:UC11}
\begin{figure}[h!]
    \centering
    \includegraphics[scale=0.60]{../../assets/creazione_vista.png}
    \caption{UC11 - Creazione di una nuova vista$^{G}$}
\end{figure}
\begin{itemize}
    \item \textbf{Attore primario}: Utente.
    \item \textbf{Precondizioni}: Il ${\mathrm{dataset^{G}}}$ è stato caricato correttamente.
    \item \textbf{Postcondizioni}: L'utente ha creato una nuova vista$^{G}$ scegliendo quali colonne associare agli assi e quali righe visualizzare, oppure ha filtrato determinate celle da visualizzare.
    \item \textbf{Scenario principale}:
          \begin{enumerate}
              \item L'utente clicca il pulsante di creazione di una nuova vista$^{G}$.
              \item Il sistema visualizza la schermata di creazione.
              \item L'utente sceglie se selezionare tutte le colonne e righe del ${\mathrm{dataset^{G}}}$ o se selezionare solo determinate celle [\hyperref[sec:UC11.1]{UC11.1}].
              \item L'utente inserisce le informazioni richieste dalla schermata.
          \end{enumerate}
  \item \textbf{Generalizzazioni:} \begin{enumerate}
                                        \item Nuova vista$^{G}$ su $Scatter$ $Plot^{G}$ [\hyperref[sec:UC11.2]{UC11.2}]
                                        \item Nuova vista$^{G}$ su \textit{$Parallel$ $Coordinates^{G}$} [\hyperref[sec:UC11.3]{UC11.3}]
                                        \item Nuova vista$^{G}$ su \textit{$Force-directed$ $Graph^{G}$} [\hyperref[sec:UC11.4]{UC11.4}]
                                        \item Nuova vista$^{G}$ su \textit{$Sankey$ $Diagram^{G}$} [\hyperref[sec:UC11.5]{UC11.5}]
                                    \end{enumerate}
\end{itemize}

%Scelta colonne

\subsubsection{UC11.1 - Scelta delle colonne/righe/celle}
\label{sec:UC11.1}
\begin{figure}[H]
    \centering
    \includegraphics[scale=0.60]{../../assets/scelta_colonne_righe.png}
    \caption{UC11.1 - Scelta delle colonne/righe/celle}
\end{figure}

\begin{itemize}
    \item \textbf{Attore primario:} Utente.
    \item \textbf{Precondizioni:} Il ${\mathrm{dataset^{G}}}$ è stato caricato correttamente. L'utente ha cliccato il pulsante ``Creazione nuova vista$^{G}$''.
    \item \textbf{Postcondizioni:} L'utente ha scelto quali righe e colonne utilizzare filtrando così le celle del ${\mathrm{dataset^{G}}}$ caricato.
    \item \textbf{Scenario principale}:
    \begin{enumerate}
		\item L'utente decide di creare una nuova vista$^{G}$.
		\item L'utente sceglie quali colonne, righe o multiple celle visualizzare.
		\item L'utente può anche scegliere di selezionarle tutte.
	\end{enumerate}
	\item \textbf{Estensioni:} Nel caso in cui l'utente non abbia selezionato un numero sufficiente di celle o non ne abbia selezionata alcuna:
              \begin{itemize}
                  \item Non viene visualizzato nessun grafico.
                  \item Viene visualizzato un errore di scelta celle insufficienti \hyperref[sec:UC19 - Errore-celle-insufficienti]{UC19}.
              \end{itemize}
\end{itemize}


%Scatter Plot

\subsubsection{UC11.2 - Nuova vista su Scatter Plot}
\label{sec:UC11.2}
\begin{figure}[H]
	\centering
	\includegraphics[scale=0.60]{../../assets/creazionevista_scatter_plot.png}
	\caption{UC11.2 - Nuova vista$^{G}$ su $Scatter$ $Plot^{G}$}
\end{figure}
\begin{itemize}
    \item \textbf{Attore primario:} Utente.
    \item \textbf{Precondizioni:} Il ${\mathrm{dataset^{G}}}$ è stato caricato correttamente. L'utente ha cliccato il pulsante ``Creazione nuova vista$^{G}$'' scegliendo quali celle/righe/colonne visualizzare (\hyperref[sec:UC11.1]{UC11.1}) e ha scelto il grafico Scatter Plot.
    \item \textbf{Postcondizioni:} Il grafico viene mostrato secondo le scelte selezionate dall'utente.
    \item \textbf{Scenario principale}:
    \begin{enumerate}
		\item L'utente decide di creare una nuova vista$^{G}$ per il grafico $Scatter$ $Plot^{G}$.
		\item L'utente può dunque associare le ${\mathrm{dimensioni^{G}}}$ desiderate agli assi del grafico. \hyperref[sec:UC11.2.1]{UC11.2.1}.
	\end{enumerate}
\end{itemize}

\paragraph{UC11.2.1 - Associazione delle dimensioni agli assi}
\label{sec:UC11.2.1}
    \begin{itemize}
        \item \textbf{Attore primario:} Utente.
        \item \textbf{Precondizioni:} Valgono le precondizioni di \hyperref[sec:UC11.2]{UC11.2}.
	    \item \textbf{Postcondizioni:} L'utente ha associato le ${\mathrm{dimensioni^{G}}}$ disponibili con gli assi del grafico.
	    \item \textbf{Scenario principale:} 
	    \begin{enumerate}
	    		\item L'utente sceglie quali dimensioni$^{G}$ associare agli assi del grafico.
		\end{enumerate}
	    \item \textbf{Estensioni:} Nel caso l'utente non abbia associato correttamente le ${\mathrm{dimensioni^{G}}}$ agli assi del grafico:
              \begin{itemize}
                  \item Non viene visualizzato nessun grafico.
                  \item Viene visualizzato un errore di personalizzazione grafico \hyperref[sec:UC17 - Errore di personalizzazione]{UC17}.
              \end{itemize}
    \end{itemize}

\newpage
%Parallel Coordinates

\subsubsection{UC11.3 - Nuova vista su Parallel Coordinates}
\label{sec:UC11.3}
\begin{figure}[h!]
	\centering
	\includegraphics[scale=0.60]{../../assets/creazionevista_parallel_coordinates.png}
	\caption{UC11.3 - Nuova vista$^{G}$ su $Parallel$ $Coordinates^{G}$}
\end{figure}
\begin{itemize}
    \item \textbf{Attore primario:} Utente.
    \item \textbf{Precondizioni:} Il ${\mathrm{dataset^{G}}}$ è stato caricato correttamente. L'utente ha cliccato il pulsante ``Creazione nuova vista$^{G}$'' scegliendo quali celle/righe/colonne visualizzare (\hyperref[sec:UC11.1]{UC11.1}) e ha scelto il grafico $Parallel$ $Coordinates^{G}$.
    \item \textbf{Postcondizioni:} Il grafico viene mostrato secondo le scelte selezionate dall'utente.
    \item \textbf{Scenario principale}:
    \begin{enumerate}
		\item L'utente decide di creare una nuova vista$^{G}$ per il grafico $Parallel$ $Coordinates^{G}$.
		\item L'utente può associare le ${\mathrm{dimensioni^{G}}}$ desiderate agli assi \hyperref[sec:UC11.3.1]{UC11.3.1}.
		\item L'utente può decidere ordine e numero degli assi \hyperref[sec:UC11.3.2]{UC11.3.2}.
		\item L'utente può decidere lo \textit{scaling} dei dati \hyperref[sec:UC11.3.3]{UC11.3.3}.
	\end{enumerate}
\end{itemize}

\paragraph{UC11.3.1 - Associazione delle dimensioni agli assi}
\label{sec:UC11.3.1}
    \begin{itemize}
        \item \textbf{Attore primario:} Utente.
        \item \textbf{Precondizioni:} Valgono le precondizioni di \hyperref[sec:UC11.3]{UC11.3}.
	    \item \textbf{Postcondizioni:} L'utente ha associato le ${\mathrm{dimensioni^{G}}}$ disponibili agli assi del grafico.
	    \item \textbf{Scenario principale:} 
	    \begin{enumerate}
	    		\item L'utente sceglie quali colonne associare agli assi del grafico.
		\end{enumerate}
	    \item \textbf{Estensioni:} Nel caso l'utente non abbia associato correttamente gli assi del grafico:
              \begin{itemize}
                  \item Non viene visualizzato nessun grafico.
                  \item Viene visualizzato un errore di personalizzazione grafico \hyperref[sec:UC17 - Errore di personalizzazione]{UC17}.
              \end{itemize}
    \end{itemize}
    
\paragraph{UC11.3.2 - Ordine e numero degli assi}
\label{sec:UC11.3.2}
    \begin{itemize}
        \item \textbf{Attore primario:} Utente.
        \item \textbf{Precondizioni:} Valgono le precondizioni di \hyperref[sec:UC11.3]{UC11.3}.
	    \item \textbf{Postcondizioni:} L'utente ha modificato l'ordine verticale degli assi e il numero.
	    \item \textbf{Scenario principale:} 
	    \begin{enumerate}
	    		\item L'utente sceglie ordine e numero degli assi.
		\end{enumerate}
	    \item \textbf{Estensioni:} Nel caso l'utente abbia creato nuovi assi ma non abbia associato nessuna ${\mathrm{dimen^{G}}}$ ad essi:
              \begin{itemize}
                  \item Non viene visualizzato nessun grafico.
                  \item Viene visualizzato un errore di visualizzazione personalizzazione grafico \hyperref[sec:UC17 - Errore di personalizzazione]{UC17}.
              \end{itemize}
    \end{itemize}
    
\paragraph{UC11.3.3 - Scaling dei dati}
\label{sec:UC11.3.3}
    \begin{itemize}
        \item \textbf{Attore primario:} Utente.
        \item \textbf{Precondizioni:} Valgono le precondizioni di \hyperref[sec:UC11.3]{UC11.3}.
	    \item \textbf{Postcondizioni:} L'utente ha ridimensionato i dati in una nuova scala comune tra le variabili o viceversa.
	    \item \textbf{Scenario principale:} 
	    \begin{enumerate}
	    		\item L'utente sceglie lo \textit{scaling} dei dati.
		\end{enumerate}
    \end{itemize}

\newpage
%Force-directed

\subsubsection{UC11.4 - Nuova vista su Force-directed Graph}
\label{sec:UC11.4}
\begin{figure}[h!]
	\centering
	\includegraphics[scale=0.60]{../../assets/creazionevista_force.png}
	\caption{UC11.4 - Nuova vista$^{G}$ su $Force-directed$ $Graph^{G}$}
\end{figure}
\begin{itemize}
    \item \textbf{Attore primario:} Utente.
    \item \textbf{Precondizioni:} Il ${\mathrm{dataset^{G}}}$ è stato caricato correttamente. L'utente ha cliccato il pulsante ``Creazione nuova vista$^{G}$'' scegliendo quali celle/righe/colonne visualizzare (\hyperref[sec:UC11.1]{UC11.1}) e ha scelto il grafico $Force-directed Graph^{G}$.
    \item \textbf{Postcondizioni:} Il grafico viene mostrato secondo le scelte selezionate dall'utente.
    \item \textbf{Scenario principale}:
    \begin{enumerate}
		\item L'utente decide di creare una nuova vista$^{G}$ per il grafico $Force-directed$ $Graph^{G}$.
		\item L'utente può decidere un algoritmo di integrazione \hyperref[sec:UC11.4.1]{UC11.4.1}.
		\item L'utente può decidere la funzione di forza \hyperref[sec:UC11.4.2]{UC11.4.2}.
	\end{enumerate}
\end{itemize}

\paragraph{UC11.4.1 - Scelta algoritmo di integrazione}
\label{sec:UC11.4.1}
    \begin{itemize}
        \item \textbf{Attore primario:} Utente.
        \item \textbf{Precondizioni:} Valgono le precondizioni di \hyperref[sec:UC11.4]{UC11.4}.
	    \item \textbf{Postcondizioni:} L'utente ha associato l'algoritmo di integrazione che desidera.
	    \item \textbf{Scenario principale:} 
	    \begin{enumerate}
	    		\item L'utente sceglie l'algoritmo di integrazione.
		\end{enumerate}
    \end{itemize}
    
\paragraph{UC11.4.2 - Scelta funzione di forza}
\label{sec:UC11.4.2}
    \begin{itemize}
        \item \textbf{Attore primario:} Utente.
        \item \textbf{Precondizioni:} Valgono le precondizioni di \hyperref[sec:UC11.4]{UC11.4}.
	    \item \textbf{Postcondizioni:} L'utente ha deciso la funzione di forza tra i nodi.
	    \item \textbf{Scenario principale:} 
	    \begin{enumerate}
	    		\item L'utente sceglie i parametri per la rappresentazione dei rapporti tra i nodi.
		\end{enumerate}
	    \item \textbf{Estensioni:} Nel caso l'utente abbia selezionato dei parametri che non rientrino nei limiti fissati:
              \begin{itemize}
                  \item Non viene visualizzato nessun grafico.
                  \item Viene visualizzato un errore di visualizzazione personalizzazione grafico \hyperref[sec:UC17 - Errore di personalizzazione]{UC17}.
              \end{itemize}
    \end{itemize}


%Sankey

\subsubsection{UC11.5 - Nuova vista su Sankey Diagram}
\label{sec:UC11.5}
\begin{figure}[h!]
	\centering
	\includegraphics[scale=0.60]{../../assets/creazionevista_sankey.png}
	\caption{UC11.5 - Nuova vista$^{G}$ su $Sankey$ $Diagram^{G}$}
\end{figure}
\begin{itemize}
    \item \textbf{Attore primario:} Utente.
    \item \textbf{Precondizioni:} Il ${\mathrm{dataset^{G}}}$ è stato caricato correttamente. L'utente ha cliccato il pulsante ``Creazione nuova vista$^{G}$'' scegliendo quali celle/righe/colonne visualizzare (\hyperref[sec:UC11.1]{UC11.1}) e ha scelto il grafico Sankey Diagram.
    \item \textbf{Postcondizioni:} Il grafico viene mostrato secondo le scelte selezionate dall'utente.
    \item \textbf{Scenario principale}:
    \begin{enumerate}
		\item L'utente decide di creare una nuova vista$^{G}$ per il grafico $Sankey$ $Diagram^{G}$.
		\item L'utente può decidere un ordinamento dei dati \hyperref[sec:UC11.5.1]{UC11.5.1}.
	\end{enumerate}
\end{itemize}

\paragraph{UC11.5.1 - Scelta ordinamento dei dati}
\label{sec:UC11.5.1}
    \begin{itemize}
        \item \textbf{Attore primario:} Utente.
        \item \textbf{Precondizioni:} Valgono le precondizioni di \hyperref[sec:UC11.5]{UC11.5}.
	    \item \textbf{Postcondizioni:} L'utente ha stabilito le ${\mathrm{dimensioni^{G}}}$ da rappresentare.
	    \item \textbf{Scenario principale:}
	    \begin{enumerate}
	    		\item L'utente sceglie l'ordinamento dei dati.
		\end{enumerate}
		\item \textbf{Estensioni:} Se non viene selezionata almeno un'origine nei dati:
              \begin{itemize}
                  \item Non viene visualizzato nessun grafico.
                  \item Viene visualizzato un errore di visualizzazione personalizzazione grafico \hyperref[sec:UC17 - Errore di personalizzazione]{UC17}.
              \end{itemize}
    \end{itemize}


\newpage
% --------------------------------------------------------------------
% SEZIONE SALVATAGGIO VISTA SU FILE
% --------------------------------------------------------------------

\subsection{UC12 - Salvataggio vista}
\label{sec:UC12}
\begin{figure}[h!]
    \centering
    \includegraphics[scale=0.60]{../../assets/salvataggio_vista.png}
    \caption{UC12 - Salvataggio della vista$^{G}$ su file}
\end{figure}
\begin{itemize}
    \item \textbf{Attore primario}: Utente.
    \item \textbf{Precondizioni}: Il ${\mathrm{dataset^{G}}}$ è stato caricato correttamente. L'utente ha scelto una tipologia di grafico e ha creato una propria vista$^{G}$ personalizzata.
    \item \textbf{Postcondizioni}: La vista$^{G}$ viene salvata all'interno di un file ${\mathrm{.json^{G}}}$.
    \item \textbf{Scenario principale}:
          \begin{enumerate}
              \item L'utente seleziona il pulsante di salvataggio della vista$^{G}$.
              \item Il file viene scaricato in locale in formato ${\mathrm{.json^{G}}}$.
          \end{enumerate}
\end{itemize}

\newpage

% --------------------------------------------------------------------
% SEZIONE CARICAMENTO VISTA
% --------------------------------------------------------------------

\subsection{UC13 - Caricamento vista}
\label{sec:UC13}
\begin{figure}[h!]
    \centering
    \includegraphics[scale=0.60]{../../assets/caricamento_vista.png}
    \caption{UC13 - Caricamento della vista$^{G}$ da file}
\end{figure}
\begin{itemize}
    \item \textbf{Attore primario}: Utente.
    \item \textbf{Precondizioni}: Il ${\mathrm{dataset^{G}}}$ è stato caricato correttamente.
    \item \textbf{Postcondizioni}: È selezionabile la nuova vista$^{G}$ nel menù viste$^{G}$ descritto da \hyperref[sec:UC3]{UC3}.
    \item \textbf{Scenario principale}:
          \begin{enumerate}
              \item L'utente preme il pulsante "carica vista$^{G}$".
              \item L'utente seleziona il file ${\mathrm{.json^{G}}}$ contenente la vista$^{G}$.
              \item L'applicativo rende disponibile la vista$^{G}$ appena caricata.
          \end{enumerate}
    \item \textbf{Estensioni:}
    \begin{itemize}
        \item Nel caso il ${\mathrm{.json^{G}}}$ contenente la vista$^{G}$ non sia ben formato: 
        \begin{enumerate}
            \item La vista$^{G}$ non viene caricata.
            \item Viene visualizzato un messaggio d'errore. [\hyperref[sec:UC18 - Errore caricamento vista]{UC18}]
        \end{enumerate}
    \end{itemize}
\end{itemize}

\newpage

% --------------------------------------------------------------------
% SEZIONE ERRORI
% --------------------------------------------------------------------

\section{Errori}
\subsection{UC14 - Errore formato file}
\label{sec:UC14 - Errore formato file}
\begin{itemize}
    \item \textbf{Attore primario:} Utente
    \item \textbf{Precondizioni:} L'utente carica un file in un formato non supportato.
    \item \textbf{Postcondizioni:} L'utente visualizza un messaggio di errore e il file non viene caricato.
    \item \textbf{Scenario principale:}
          \begin{enumerate}
              \item L'utente visualizza un messaggio di errore esplicativo
              \item L'utente clicca "OK" per proseguire.
          \end{enumerate}
\end{itemize}

\subsection{UC15 - Errore struttura dataset}
\label{sec:UC15 - Errore struttura dataset}
\begin{itemize}
    \item \textbf{Attore primario:} Utente
    \item \textbf{Precondizioni:} L'utente carica un file nel formato supportato ma che non è correttamente strutturato  
                                  (e.g. due o più colonne sono invertite oppure una o più colonne sono mancanti). 
    \item \textbf{Postcondizioni:} L'utente visualizza un messaggio di errore e il file non viene caricato.
    \item \textbf{Scenario principale:}
          \begin{enumerate}
              \item L'utente visualizza un messaggio di errore esplicativo.
              \item L'utente clicca "OK" per proseguire.
          \end{enumerate} 
\end{itemize}

\subsection{UC16 - Errore validità riga nel dataset}
\label{sec:UC16 - Errore validità riga}
\begin{itemize}
    \item \textbf{Attore primario:} Utente
    \item \textbf{Precondizioni:} L'utente carica un file nel formato supportato e ben strutturato ma una o più righe presentano un dato non valido (e.g. nella colonna relativa ai \textit{timestamp} il valore è negativo).  
    \item \textbf{Postcondizioni:} L'utente visualizza un messaggio di errore che chiede se ignorare la riga non valida, oppure continuare.
    \item \textbf{Scenario principale:}
    \begin{itemize}
        \item   L'utente sceglie di ignorare la riga:
                \begin{enumerate}
                    \item L'utente visualizza un messaggio di errore esplicativo.
                    \item L'utente clicca "IGNORA" per ignorare la riga non valida.
                    \item Il file viene caricato escludendo la riga precedentemente ignorata.
                \end{enumerate} 
        \item   L'utente sceglie di proseguire:
                \begin{enumerate}
                    \item L'utente visualizza un messaggio di errore esplicativo.
                    \item L'utente clicca "OK" per proseguire.
                    \item Il file non viene caricato.
                \end{enumerate} 
    \end{itemize}
\end{itemize}

\subsection{UC17 - Errore di personalizzazione grafico}
\label{sec:UC17 - Errore di personalizzazione}
\begin{itemize}
    \item \textbf{Attore primario:} Utente
    \item \textbf{Precondizioni:}
    		\begin{itemize}
    			\item L'utente associa non correttamente una o più ${\mathrm{dimensioni^{G}}}$ ad uno o più assi nel grafico.
    			\item L'utente inserisce dei parametri relativi alla funzione di forza non corretti.
    			\item L'utente sceglie dei parametri relativi all'ordinamento dei dati non corretti.
    		\end{itemize}
    \item \textbf{Postcondizioni:} L'utente visualizza un messaggio d'errore esplicativo relativo alla condizione che ha causato l'errore, e le modifiche da lui apportate non vengono eseguite.
    \item \textbf{Scenario principale:}
    \begin{enumerate}
        \item L'utente visualizza un messaggio di errore esplicativo.
        \item L'utente clicca "OK" per proseguire.
    \end{enumerate}
\end{itemize}

\subsection{UC18 - Errore caricamento vista}
\label{sec:UC18 - Errore caricamento vista}
\begin{itemize}
    \item \textbf{Attore primario:} Utente
    \item \textbf{Precondizioni:} L'utente carica un file vista$^{G}$ malformato.
    \item \textbf{Postcondizioni:} L'utente visualizza un messaggio d'errore esplicativo relativo alla condizione che ha causato l'errore, non viene caricata la vista$^{G}$.
    \item \textbf{Scenario principale:}
    \begin{enumerate}
        \item L'utente visualizza un messaggio di errore esplicativo.
        \item L'utente clicca "OK" per proseguire.
    \end{enumerate}
\end{itemize}

\subsection{UC19 - Errore scelta celle insufficienti}
\label{sec:UC19 - Errore-celle-insufficienti}
\begin{itemize}
    \item \textbf{Attore primario:} Utente
    \item \textbf{Precondizioni:} Il ${\mathrm{dataset^{G}}}$ è stato caricato correttamente. L'utente ha cliccato il pulsante ``Creazione nuova vista$^{G}$'' e ha inserito le colonne/righe/celle che vuole utilizzare (\hyperref[sec:UC9.1]{UC9.1}).
    \item \textbf{Postcondizioni:} L'utente visualizza un messaggio di errore per celle insufficienti. 
    \item \textbf{Scenario principale:}
          \begin{enumerate}
              \item L'utente ha selezionato un numero di celle insufficienti o nessuna.
              \item Viene visualizzato un messaggio di errore.
              \item L'utente clicca "OK" per proseguire.
          \end{enumerate} 
\end{itemize}

