
\section{Casi d'uso}
\subsection{Scopo}

\subsection{Attori}

\subsection{UC1 - Caricamento dataset}
\begin{figure}[h!]
    \centering
    % Sistema immagine inserendo tag UC e abbelliscila dc
    \includegraphics[scale=0.50]{../../assets/Caricamento_dataset.png}
    \caption{UC1 - Caricamento dataset}
\end{figure}
\begin{itemize}
    \item \textbf{Attore primario:} Utente.
    \item \textbf{Precondizioni:} Il sistema è funzionante
    \item \textbf{Postcondizioni:} Viene visualizzato un messaggio che avvisa l'utente del corretto caricamento dei dati e della loro validità. 
                                   I dati vengono caricati nel sistema
    \item \textbf{Scenario principale:}
          \begin{enumerate}
              \item L'utente seleziona il file da caricare
              \item L'utente carica il file
          \end{enumerate}
    \item \textbf{Estensioni:}
    \begin{itemize}
        \item   Nel caso in cui l'utente carichi un file in un formato non supportato
                \begin{enumerate}
                    \item I dati non vengono caricati
                    \item Viene visualizzato un messaggio di errore esplicativo [\hyperref[sec:UC - Errore formato file]{UC}] % To Do: metti il link alla sezione   
                \end{enumerate}
        \item   Nel caso in cui l'utente carichi un file non correttamente strutturato
                \begin{enumerate}
                    \item I dati non vengono caricati
                    \item Viene visualizzato un messaggio di errore esplicativo [\hyperref[sec:UC - Errore struttura dataset]{UC}]
                \end{enumerate}
        \item   Nel caso in cui i dati di una o più righe non siano validi
                \begin{enumerate}
                    \item Viene visualizzato un messaggio di errore esplicativo [\hyperref[sec:UC - Errore validita riga]{UC}] % To Do: metti il link alla sezione
                \end{enumerate}
    \end{itemize} 
\end{itemize}


% --------------------------------------------------------------------
% SEZIONE ERRORI
% --------------------------------------------------------------------

\subsection{UC - Errore formato file}
\label{sec:UC - Errore formato file}
\begin{itemize}
    \item \textbf{Attore primario:} Utente
    \item \textbf{Precondizioni:} L'utente carica un file in un formato non supportato.
    \item \textbf{Postcondizioni:} L'utente visualizza un messaggio di errore e il file non viene caricato.
    \item \textbf{Scenario principale:}
          \begin{enumerate}
              \item L'utente visualizza un messaggio di errore esplicativo
              \item L'utente clicca "OK" per proseguire.
          \end{enumerate}
\end{itemize}

\subsection{UC - Errore struttura dataset}
\label{sec:UC - Errore struttura dataset}
\begin{itemize}
    \item \textbf{Attore primario:} Utente
    \item \textbf{Precondizioni:} L'utente carica un file nel formato supportato ma che non è correttamente strutturato, 
                                  ad esempio due o più colonne sono invertite oppure una o più colonne sono mancanti. 
    \item \textbf{Postcondizioni:} L'utente visualizza un messaggio di errore e il file non viene caricato.
    \item \textbf{Scenario principale:}
          \begin{enumerate}
              \item L'utente visualizza un messaggio di errore esplicativo.
              \item L'utente clicca "OK" per proseguire.
          \end{enumerate} 
\end{itemize}

\subsection{UC - Errore validità riga nel dataset}
\label{sec:UC - Errore validita riga}
\begin{itemize}
    \item \textbf{Attore primario:} Utente
    \item \textbf{Precondizioni:} L'utente carica un file nel formato supportato e ben strutturato ma, una o più righe presentano
                                  un dato non valido, ad esempio nella colonna relativa ai \textit{timestamp} il valore è negativo.  
    \item \textbf{Postcondizioni:} L'utente visualizza un messaggio di errore che chiede o di ignorare la riga non valida, 
                                   oppure di ricaricare il file.
    \item \textbf{Scenario principale:}
    \begin{itemize}
        \item   L'utente sceglie di ignorare la riga
                \begin{enumerate}
                    \item L'utente visualizza un messaggio di errore esplicativo.
                    \item L'utente clicca "IGNORA" per ignorare la riga non valida.
                    \item Il file viene caricato escludendo la riga precedentemente ignorata.
                \end{enumerate} 
        \item   L'utente sceglie di ricaricare il file corretto.
                \begin{enumerate}
                    \item L'utente visualizza un messaggio di errore esplicativo.
                    \item L'utente clicca "OK" per proseguire.
                    \item Il file non viene caricato.
                \end{enumerate} 
    \end{itemize}
    
    
\end{itemize}