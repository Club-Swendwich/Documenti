\section{Casi d'uso}
\subsection{Scopo}
L'obbiettivo di questa sezione è quello di presentare e descrivere tutti i casi d'uso, individuati dal team, in riferimento alle funzionalità richieste nell'applicazione.
\subsection{Attori}
Non essendo richiesto nessun tipo di servizio di autenticazione o più in generale una distinzione gerarchica degli utenti, è presente un solo attore nella gerarchia ovvero: l'utente generico.
\begin{figure}[h!]
    \centering
    \includegraphics[scale=0.50]{../../assets/Utente.png}
    \caption{Gerarchia attori}
\end{figure}

% --------------------------------------------------------------------
% INIZIO CASI D'USO
% CARICAMENTO DATASET
% --------------------------------------------------------------------

\subsection{UC1 - Caricamento dataset}
\begin{figure}[h!]
    \centering
    % Sistema immagine inserendo tag UC e abbelliscila dc
    \includegraphics[scale=0.50]{../../assets/Caricamento_dataset.png}
    \caption{UC1 - Caricamento dataset}
\end{figure}
\begin{itemize}
    \item \textbf{Attore primario:} Utente.
    \item \textbf{Precondizioni:} Il sistema è funzionante
    \item \textbf{Postcondizioni:} Viene visualizzato un messaggio che avvisa l'utente del corretto caricamento dei dati e della loro validità. 
                                   I dati vengono caricati nel sistema
    \item \textbf{Scenario principale:}
          \begin{enumerate}
              \item L'utente seleziona il file da caricare
              \item L'utente carica il file
          \end{enumerate}
    \item \textbf{Estensioni:}
    \begin{itemize}
        \item   Nel caso in cui l'utente carichi un file in un formato non supportato
                \begin{enumerate}
                    \item I dati non vengono caricati
                    \item Viene visualizzato un messaggio di errore esplicativo [\hyperref[sec:UC - Errore formato file]{UC}] % To Do: metti il link alla sezione   
                \end{enumerate}
        \item   Nel caso in cui l'utente carichi un file non correttamente strutturato
                \begin{enumerate}
                    \item I dati non vengono caricati
                    \item Viene visualizzato un messaggio di errore esplicativo [\hyperref[sec:UC - Errore struttura dataset]{UC}]
                \end{enumerate}
        \item   Nel caso in cui i dati di una o più righe non siano validi
                \begin{enumerate}
                    \item Viene visualizzato un messaggio di errore esplicativo [\hyperref[sec:UC - Errore validita riga]{UC}] % To Do: metti il link alla sezione
                \end{enumerate}
    \end{itemize} 
\end{itemize}
\newpage

% --------------------------------------------------------------------
% SCELTA TIPO GRAFICO
% --------------------------------------------------------------------

\subsection{UC2 - Scelta tipologia grafico}
\label{sec:UC2}
\begin{figure}[h!]
    \centering
    % Controlla che UC abbia il numero corretto
    \includegraphics[scale=0.50]{../../assets/Selezione_tipo_grafico.png}
    \caption{UC2 - Scelta tipologia grafico}
\end{figure}
\begin{itemize}
    \item \textbf{Attore primario}: Utente.
    \item \textbf{Precondizioni}: Il dataset è stato caricato correttamente.
    \item \textbf{Postcondizioni}: Viene mostrata la visualizzazione scelta.
    \item \textbf{Scenario principale}:
          \begin{enumerate}
              \item L'utente seleziona il tipo di grafico da visualizzare tra le tipologie disponibili.
          \end{enumerate}
    \item \textbf{Generalizzazioni}:
    \begin{itemize}
        \item L'utente seleziona una delle seguenti opzioni:
                \begin{enumerate}
                    \item \textit{Scatter Plot} \hyperref[sec:UC2.1]{UC2.1}.
                    \item \textit{Parallel Coordinates} \hyperref[sec:UC2.2]{UC2.2}.
                    \item \textit{Force-directed Graph} \hyperref[sec:UC2.3]{UC2.3}.
                    \item \textit{Sankey Diagram} \hyperref[sec:UC2.4]{UC2.4}.
                \end{enumerate}
    \end{itemize} 
\end{itemize}

\subsubsection{UC2.1 - Selezionato Scatter Plot}
\label{sec:UC2.1}
\begin{itemize}
    \item \textbf{Attore primario}: Utente.
    \item \textbf{Precondizioni}: Il dataset è stato caricato correttamente [UC2].
    \item \textbf{Postcondizioni}: Viene mostrata la visualizzazione \textit{Scatter Plot} scelta dall'utente con possibilità di selezionare una diversa vista \hyperref[sec:UC6]{UC6} e personalizzare lo stile \hyperref[sec:UC2]{UC?}. %inserire il numero corretto dello stile e delle dimensioni
    \item \textbf{Scenario principale}:
          \begin{enumerate}
              \item L'utente seleziona il grafico \textit{Scatter Plot} e il sistema ritorna tale grafico con conseguente possibilità di personalizzazione. 
          \end{enumerate}
\end{itemize}

\subsubsection{UC2.2 - Selezionato Parallel Coordinates}
\label{sec:UC2.2}
\begin{itemize}
    \item \textbf{Attore primario}: Utente.
    \item \textbf{Precondizioni}: Il dataset è stato caricato correttamente [UC2].
    \item \textbf{Postcondizioni}: Viene mostrata la visualizzazione \textit{Parallel Coordinates} scelta dall'utente con possibilità di selezionare una diversa vista \hyperref[sec:UC6]{UC6} e personalizzare lo stile \hyperref[sec:UC2]{UC?}. %inserire il numero corretto dello stile e delle dimensioni
    \item \textbf{Scenario principale}:
          \begin{enumerate}
              \item L'utente seleziona il grafico \textit{Parallel Coordinates} e il sistema ritorna tale grafico con conseguente possibilità di personalizzazione. 
          \end{enumerate}
\end{itemize}

\subsubsection{UC2.3 - Selezionato Force-directed Graph}
\label{sec:UC2.3}
\begin{itemize}
    \item \textbf{Attore primario}: Utente.
    \item \textbf{Precondizioni}: Il dataset è stato caricato correttamente [UC2].
    \item \textbf{Postcondizioni}: Viene mostrata la visualizzazione \textit{Force-directed Graph} scelta dall'utente con possibilità di selezionare una diversa vista \hyperref[sec:UC6]{UC6} e personalizzare lo stile \hyperref[sec:UC2]{UC?}. %inserire il numero corretto dello stile e delle dimensioni
    \item \textbf{Scenario principale}:
          \begin{enumerate}
              \item L'utente seleziona il grafico \textit{Force-directed Graph} e il sistema ritorna tale grafico con conseguente possibilità di personalizzazione. 
          \end{enumerate}
\end{itemize}

\subsubsection{UC2.4 - Selezionato Sankey Diagram}
\label{sec:UC2.4}
\begin{itemize}
    \item \textbf{Attore primario}: Utente.
    \item \textbf{Precondizioni}: Il dataset è stato caricato correttamente [UC2].
    \item \textbf{Postcondizioni}: Viene mostrata la visualizzazione \textit{Sankey Diagram} scelta dall'utente con possibilità di selezionare una diversa vista \hyperref[sec:UC6]{UC6} e personalizzare lo stile \hyperref[sec:UC2]{UC?}. %inserire il numero corretto dello stile e delle dimensioni
    \item \textbf{Scenario principale}:
          \begin{enumerate}
              \item L'utente seleziona il grafico \textit{Sankey Diagram} e il sistema ritorna tale grafico con conseguente possibilità di personalizzazione. 
          \end{enumerate}
\end{itemize}
\newpage

%               _ _              _       _                 _
%  ___  ___ ___| | |_ __ _    __| | __ _| |_ __ _ ___  ___| |_
% / __|/ __/ _ \ | __/ _` |  / _` |/ _` | __/ _` / __|/ _ \ __|
% \__ \ (_|  __/ | || (_| | | (_| | (_| | || (_| \__ \  __/ |_
% |___/\___\___|_|\__\__,_|  \__,_|\__,_|\__\__,_|___/\___|\__|


\subsection{UC3 - Selezione Vista}
\label{sec:UC6}
\begin{figure}[h!]
    \centering
    % Controlla che UC abbia il numero corretto
    \includegraphics[scale=0.50]{../../assets/scelta_vista.png}
    \caption{UC6 - Scelta della vista dei dati da visualizzare}
\end{figure}
\begin{itemize}
    \item \textbf{Attore primario}: Utente.
    \item \textbf{Precondizioni}: Il dataset è stato caricato correttamente.
    \item \textbf{Postcondizioni}: Viene mostrata all'utente la vista dei dati selezionata.
    \item \textbf{Scenario principale}:
          \begin{enumerate}
              \item L'utente apre il menù a tendina contenente tutte le interpretazioni disponibili.
              \item L'utente seleziona la vista che vuole visualizzare.
          \end{enumerate}
\end{itemize}
\newpage

% --------------------------------------------------------------------
% VISUALIZZAZIONE DI DEFAULT DEI GRAFICI
% --------------------------------------------------------------------

\subsection{UC4 - Visualizzazione di default dei grafici}
\begin{figure}[h!]
    \centering
    \includegraphics[scale=0.70]{../../assets/visualizzazione_default.png}
	\caption{UC4 - Visualizzazione di default dei grafici}
\end{figure}

\begin{itemize}
	\item \textbf{Attore primario:} Utente.
	\item \textbf{Precondizioni:} Il dataset è presente e caricato correttamente.
	\item \textbf{Postcondizioni:} 
	Vengono riportati i grafici alla visualizzazione di default.
	\item \textbf{Scenario principale:}
	\begin{itemize}
		\item   L'utente sceglie di ripristinare le viste
	\begin{enumerate}
		\item L'utente clicca sul pulsante di ripristino
		\item Viene visualizzato un messaggio di riconferma
		\item L'utente clicca "OK" per confermare la scelta
		\item I grafici vengono resettati alla vista di default
	\end{enumerate}
		\item   L'utente sceglie di annullare l'operazione
	\begin{enumerate}
		\item L'utente clicca sul pulsante di ripristino
		\item Viene visualizzato un messaggio di riconferma
		\item L'utente clicca "ANNULLA" per annullare l'operazione
		\item Ritorno alla vista corrente senza modifiche apportate
	\end{enumerate}	
	\end{itemize}
\end{itemize}
\newpage

% --------------------------------------------------------------------
% SEZIONE PERSONALIZZAZIONE VISIVA DEI GRAFICI
% --------------------------------------------------------------------

\subsection{UC5 - Personalizzazione visiva dei grafici}
\begin{figure}[h!]
	\centering
	\includegraphics[scale=0.60]{../../assets/personalizzazioneVisivaGrafici.drawio.png}
	\caption{UC5 - Personalizzazione visiva dei grafici}
\end{figure}
\begin{itemize}
	\item \textbf{Attore primario:} Utente.
	\item \textbf{Precondizioni:} L'utente ha scelto uno di grafici a disposizione nell'interfaccia [UC2].
	\item \textbf{Postcondizioni:} Il grafico viene ridisegnato secondo le scelte selezionate dall'utente.
	\item \textbf{Scenario principale:}
    L'utente seleziona il tipo di grafico e i colori che desidera e, per ogni elemento cambiato, i valori di default del grafico
    vengono cambiati. Nel caso si scelta di tornare nella vista di default, le modifiche eseguite torneranno ai valori standard.
    \item \textbf{Generalizzazioni}:
    \begin{itemize}
        \item L'utente seleziona una delle seguenti opzioni:
                \begin{enumerate}
                    \item \textit{Personalizza Scatter Plot} \hyperref[sec:UC5.1]{UC5.1}.
                    \item \textit{Personalizza Parallel Coordinates} \hyperref[sec:UC5.2]{UC5.2}.
                    \item \textit{Personalizza Force-directed Graph} \hyperref[sec:UC5.3]{UC5.3}.
                    \item \textit{Personalizza Sankey Diagram} \hyperref[sec:UC5.4]{UC5.4}.
                \end{enumerate}
    \end{itemize} 
\end{itemize}
\subsubsection{UC5.1 - Personalizzazione Scatter plot}
\label{sec:UC5.1}
\begin{figure}[h!]
	\centering
	\includegraphics[scale=0.90]{../../assets/personalizzazioneScatterPlot.drawio.png}
	\caption{UC5 - Personalizzazione Scatter Plot}
\end{figure}
\begin{itemize}
    \item \textbf{Attore primario:} Utente.
	\item \textbf{Precondizioni:} L'utente ha scelto il grafico scatter plot. \hyperref[sec:UC2.1]{UC2.1}.
	\item \textbf{Postcondizioni:} 
	Il grafico viene ridisegnato secondo le scelte selezionate dall'utente.
	\item \textbf{Scenario principale:}L'utente decide:
	\begin{itemize}
        \item Associazione delle dimensioni agli assi \hyperref[sec:UC5.1.1]{UC5.1.1}.
        \item Personalizzazione dello stile \hyperref[sec:UC5.1.2]{UC5.1.2}.
    \end{itemize}
\end{itemize}
\paragraph{UC5.1.1 - Associazione delle dimensioni agli assi}
\label{sec:UC5.1.1}
    \begin{itemize}
        \item \textbf{Attore primario:} Utente.
        \item \textbf{Precondizioni:} L'utente ha selezionato lo scatter plot \hyperref[sec:UC2.1]{UC2.1}.
	    \item \textbf{Postcondizioni:} L'utente ha associato le dimensioni disponibili con gli assi del grafico.
	    \item \textbf{Scenario principale:} L'utente sceglie quali dimensioni associare agli assi del grafico.
	    \item \textbf{Estensioni:} Nel caso l'utente non abbia associato correttamente le dimensioni agli assi del grafico:
              \begin{itemize}
                  \item Non viene visualizzato nessun grafico.
                  \item Viene visualizzato un errore di personalizzazione grafico.
              \end{itemize}
    \end{itemize}
\paragraph{UC5.1.2 - Personalizzazione dello stile}
\label{sec:UC5.1.2}
    \begin{itemize}
        \item \textbf{Attore primario:} Utente.
        \item \textbf{Precondizioni:} L'utente ha selezionato lo scatter plot \hyperref[sec:UC2.1]{UC2.1}.
	    \item \textbf{Postcondizioni:} L'utente ha selezionato tra le opzioni disponibili lo stile che preferisce.
	    \item \textbf{Scenario principale:} L'utente visualizza le opzioni di colori per la personalizzazione di ogni specifica del grafico, nel caso l'utente non li modifichi rimangono quelli di default.
    \end{itemize}

\subsubsection{UC5.2 - Personalizzazione Parallel coordinates}
\label{sec:UC5.2}
\begin{figure}[h!]
	\centering
	\includegraphics[scale=0.90]{../../assets/personalizzazioneParallelCoordinates.drawio.png}
	\caption{UC5 - Personalizzazione Parallel coordinates}
\end{figure}
\begin{itemize}
    \item \textbf{Attore primario:} Utente.
	\item \textbf{Precondizioni:} L'utente ha scelto il grafico Parallel coordinates. \hyperref[sec:UC2.1]{UC2.1}.
	\item \textbf{Postcondizioni:} Il grafico viene ridisegnato secondo le scelte selezionate dall'utente.
	\item \textbf{Scenario principale:}L'utente decide:
	\begin{itemize}
        \item Associazione delle dimensioni \hyperref[sec:UC5.2.1]{UC5.2.1}.
        \item Ordine degli assi \hyperref[sec:UC5.2.2]{UC5.2.2}.
        \item Personalizzazione dello stile \hyperref[sec:UC5.2.3]{UC5.2.3}.
    \end{itemize}
\end{itemize}
\paragraph{UC5.2.1 - Associazione delle dimensioni}
\label{sec:UC5.2.1}
    \begin{itemize}
        \item \textbf{Attore primario:} Utente.
        \item \textbf{Precondizioni:} L'utente ha selezionato Parallel coordinates. \hyperref[sec:UC2.2]{UC2.2}.
	    \item \textbf{Postcondizioni:} L'utente ha associato le dimensioni disponibili con gli assi del grafico.
	    \item \textbf{Scenario principale:} L'utente sceglie quali dimensioni associare agli assi del grafico.
	    \item \textbf{Estensioni:} Nel caso l'utente non abbia associato correttamente le dimensioni agli assi del grafico:
              \begin{itemize}
                  \item Non viene visualizzato nessun grafico.
                  \item Viene visualizzato un errore di visualizzazione personalizzazione grafico.
              \end{itemize}
    \end{itemize}
\paragraph{UC5.2.2 - Ordine e numero degli assi}
\label{sec:UC5.2.2}
    \begin{itemize}
        \item \textbf{Attore primario:} Utente.
        \item \textbf{Precondizioni:} L'utente ha selezionato Parallel coordinates. \hyperref[sec:UC2.2]{UC2.2}.
	    \item \textbf{Postcondizioni:} L'utente ha deciso un ordine degli assi e il loro numero.
	    \item \textbf{Scenario principale:} L'utente sceglie ordine e numero degli assi.
	    \item \textbf{Estensioni:} Nel caso l'utente abbia creato nuovi assi ma non abbai associato nessuna dimensione con essi:
              \begin{itemize}
                  \item Non viene visualizzato nessun grafico.
                  \item Viene visualizzato un errore di visualizzazione personalizzazione grafico.
              \end{itemize}
    \end{itemize}
\paragraph{UC5.2.3 - Personalizzazione dello stile}
\label{sec:UC5.2.3}
    \begin{itemize}
        \item \textbf{Attore primario:} Utente.
        \item \textbf{Precondizioni:} L'utente ha selezionato Parallel coordinates \hyperref[sec:UC2.2]{UC2.2}.
	    \item \textbf{Postcondizioni:} L'utente ha selezionato tra le opzioni disponibili lo stile che preferisce.
	    \item \textbf{Scenario principale:} L'utente visualizza le opzioni di colori per la personalizzazione di ogni specifica del grafico, nel caso l'utente non li modifichi rimangono quelli di default.
    \end{itemize}

\subsubsection{UC5.3 - Personalizzazione Force-directed graph}
\label{sec:UC5.3.1}
\begin{figure}[h!]
	\centering
	\includegraphics[scale=0.90]{../../assets/personalizzazioneForce-directedGraph.drawio.png}
	\caption{UC5 - Personalizzazione Force-directed graph}
\end{figure}
\begin{itemize}
    \item \textbf{Attore primario:} Utente.
	\item \textbf{Precondizioni:} L'utente ha scelto il grafico Force-directed graph. \hyperref[sec:UC3.1]{UC3.1}.
	\item \textbf{Postcondizioni:} Il grafico viene ridisegnato secondo le scelte selezionate dall'utente.
	\item \textbf{Scenario principale:}L'utente decide:
	\begin{itemize}
        \item Associazione delle dimensioni \hyperref[sec:UC5.3.1]{UC5.3.1}.
        \item Ordine degli assi \hyperref[sec:UC5.3.2]{UC5.3.2}.
        \item Personalizzazione dello stile \hyperref[sec:UC5.3.3]{UC5.3.3}.
    \end{itemize}
\end{itemize}
\paragraph{UC5.3.1 - Scelta tipo di algoritmo d'integrazione}
\label{sec:UC5.3.1}
    \begin{itemize}
        \item \textbf{Attore primario:} Utente.
        \item \textbf{Precondizioni:} L'utente ha selezionato Force-directed graph. \hyperref[sec:UC2.3]{UC2.3}.
	    \item \textbf{Postcondizioni:} L'utente ha associato l'algoritmo d'integrazione che desidera.
	    \item \textbf{Scenario principale:} L'utente sceglie l'algoritmo d'integrazione che desidera.
    \end{itemize}
\paragraph{UC5.3.2 - Scelta della funzione di forza}
\label{sec:UC5.3.2}
    \begin{itemize}
        \item \textbf{Attore primario:} Utente.
        \item \textbf{Precondizioni:} L'utente ha selezionato Force-directed graph. \hyperref[sec:UC2.3]{UC2.3}.
	    \item \textbf{Postcondizioni:} L'utente ha deciso i parametri per la rappresentazione.
	    \item \textbf{Scenario principale:} L'utente sceglie i parametri per la rappresentazione dei rapporti tra i nodi.
	    \item \textbf{Estensioni:} Nel caso i parametri non rientrino nei limiti fissati:
              \begin{itemize}
                  \item Non viene visualizzato nessun grafico.
                  \item Viene visualizzato un errore di visualizzazione personalizzazione grafico.
              \end{itemize}
    \end{itemize}
\paragraph{UC5.3.3 - Personalizzazione dello stile}
\label{sec:UC5.3.3}
    \begin{itemize}
        \item \textbf{Attore primario:} Utente.
        \item \textbf{Precondizioni:} L'utente ha selezionato Force-directed graph \hyperref[sec:UC2.3]{UC2.3}.
	    \item \textbf{Postcondizioni:} L'utente ha selezionato tra le opzioni disponibili lo stile che preferisce.
	    \item \textbf{Scenario principale:} L'utente visualizza le opzioni di colori per la personalizzazione di ogni specifica del grafico, nel caso l'utente non li modifichi rimangono quelli di default.
    \end{itemize}


\subsubsection{UC5.4 - Personalizzazione Sankey diagram}
\label{sec:UC5.4}
\begin{figure}[h!]
	\centering
	\includegraphics[scale=0.90]{../../assets/personalizzazioneSankey.drawio.png}
	\caption{UC5 - Personalizzazione Sankey diagram}
\end{figure}
\begin{itemize}
    \item \textbf{Attore primario:} Utente.
	\item \textbf{Precondizioni:} L'utente ha scelto il grafico Personalizzazione Sankey diagram. \hyperref[sec:UC2.4]{UC2.4}.
	\item \textbf{Postcondizioni:} Il grafico viene ridisegnato secondo le scelte selezionate dall'utente.
	\item \textbf{Scenario principale:}L'utente decide:
	\begin{itemize}
        \item Ordinamento dei dati \hyperref[sec:UC5.4.1]{UC5.4.1}.
        \item Personalizzazione dello stile \hyperref[sec:UC5.4.2]{UC5.4.2}.
    \end{itemize}
\end{itemize}
\paragraph{UC5.4.1 - Ordinamento dei dati}
\label{sec:UC5.4.1}
    \begin{itemize}
        \item \textbf{Attore primario:} Utente.
        \item \textbf{Precondizioni:} L'utente ha selezionato Sankey diagram. \hyperref[sec:UC2.3]{UC2.3}.
	    \item \textbf{Postcondizioni:} L'utente ha deciso i gruppi di dati da rappresentare.
	    \item \textbf{Scenario principale:} L'utente sceglie i parametri per la rappresentazione dei rapporti tra i nodi.
	    \item \textbf{Estensioni:} Nel caso i parametri non rientrino nelle relazioni fissate:
              \begin{itemize}
                  \item Non viene visualizzato nessun grafico.
                  \item Viene visualizzato un errore di visualizzazione personalizzazione grafico.
              \end{itemize}
    \end{itemize}
\paragraph{UC5.4.2 - Personalizzazione dello stile}
\label{sec:UC5.4.2}
\begin{itemize}
    \item \textbf{Attore primario:} Utente.
    \item \textbf{Precondizioni:} L'utente ha selezionato Sankey diagram \hyperref[sec:UC2.3]{UC2.3}.
	\item \textbf{Postcondizioni:} L'utente ha selezionato tra le opzioni disponibili lo stile che preferisce.
	\item \textbf{Scenario principale:} L'utente visualizza le opzioni di colori per la personalizzazione di ogni specifica del grafico, nel caso l'utente non li modifichi rimangono quelli di default.
\end{itemize}

\newpage

% --------------------------------------------------------------------
% VISUALIZZAZIONE GRAFICO A TUTTO SCHERMO
% --------------------------------------------------------------------

\subsection{UC6 - Visualizzazione di un grafico a schermo intero}
\begin{figure}[h!]
	\centering
	\includegraphics[scale=0.70]{../../assets/visualizzazione_fullscreen.png}
	\caption{UC6 - Visualizzazione di un grafico a schermo intero}
\end{figure}

\begin{itemize}
	\item \textbf{Attore primario:} Utente.
	\item \textbf{Precondizioni:} È già mostrato almeno un grafico sulla pagina e l'utente ne seleziona la vista a schermo intero. 
	\item \textbf{Postcondizioni:} 
	Viene mostrato il grafico scelto in fullscreen.
	\item \textbf{Scenario principale:}
		\begin{enumerate}
			\item L'utente clicca sul pulsante di fullscreen di un grafico.
			\item Il grafico viene visualizzato a schermo intero.
		\end{enumerate}
\end{itemize}
\newpage

% --------------------------------------------------------------------
% SEZIONE ZOOM INTERATTIVO
% --------------------------------------------------------------------

\subsection{UC7 - Zoom interattivo}
\label{sec:UC7}
\begin{figure}[h!]
    \centering
    % Controlla che UC abbia il numero corretto
    \includegraphics[scale=0.50]{../../assets/zoom_interattivo.png}
    \caption{UC7 - Zoom su sezione di grafico}
\end{figure}
\begin{itemize}
    \item \textbf{Attore primario}: Utente.
    \item \textbf{Precondizioni}: Il dataset è stato caricato correttamente. L'utente ha scelto quale tipologiadi grafico visualizzare e ha selezionato una vista.
    \item \textbf{Postcondizioni}: Viene mostrato uno zoom sulla sezione selezionata dal mouse.
    \item \textbf{Scenario principale}:
          \begin{enumerate}
              \item L'utente seleziona con il mouse l'area all'interno del grafico che vuole ingrandire.
              \item L'area selezionata viene ingrandita.
          \end{enumerate}
\end{itemize}

\newpage


% --------------------------------------------------------------------
% SEZIONE ERRORI
% --------------------------------------------------------------------

\section{Errori}
\subsection{UC - Errore formato file}
\label{sec:UC - Errore formato file}
\begin{itemize}
    \item \textbf{Attore primario:} Utente
    \item \textbf{Precondizioni:} L'utente carica un file in un formato non supportato.
    \item \textbf{Postcondizioni:} L'utente visualizza un messaggio di errore e il file non viene caricato.
    \item \textbf{Scenario principale:}
          \begin{enumerate}
              \item L'utente visualizza un messaggio di errore esplicativo
              \item L'utente clicca "OK" per proseguire.
          \end{enumerate}
\end{itemize}

\subsection{UC - Errore struttura dataset}
\label{sec:UC - Errore struttura dataset}
\begin{itemize}
    \item \textbf{Attore primario:} Utente
    \item \textbf{Precondizioni:} L'utente carica un file nel formato supportato ma che non è correttamente strutturato  
                                  (e.g. due o più colonne sono invertite oppure una o più colonne sono mancanti). 
    \item \textbf{Postcondizioni:} L'utente visualizza un messaggio di errore e il file non viene caricato.
    \item \textbf{Scenario principale:}
          \begin{enumerate}
              \item L'utente visualizza un messaggio di errore esplicativo.
              \item L'utente clicca "OK" per proseguire.
          \end{enumerate} 
\end{itemize}
\subsection{UC - Errore validità riga nel dataset}
\label{sec:UC - Errore validita riga}
\begin{itemize}
    \item \textbf{Attore primario:} Utente
    \item \textbf{Precondizioni:} L'utente carica un file nel formato supportato e ben strutturato ma, una o più righe presentano
                                  un dato non valido (e.g. nella colonna relativa ai \textit{timestamp} il valore è negativo).  
    \item \textbf{Postcondizioni:} L'utente visualizza un messaggio di errore che chiede o di ignorare la riga non valida, 
                                   oppure di continuare.
    \item \textbf{Scenario principale:}
    \begin{itemize}
        \item   L'utente sceglie di ignorare la riga
                \begin{enumerate}
                    \item L'utente visualizza un messaggio di errore esplicativo.
                    \item L'utente clicca "IGNORA" per ignorare la riga non valida.
                    \item Il file viene caricato escludendo la riga precedentemente ignorata.
                \end{enumerate} 
        \item   L'utente sceglie di proseguire.
                \begin{enumerate}
                    \item L'utente visualizza un messaggio di errore esplicativo.
                    \item L'utente clicca "OK" per proseguire.
                    \item Il file non viene caricato.
                \end{enumerate} 
    \end{itemize}

\end{itemize}