\section{Vincoli di progettazione}

L'applicativo finale rispetterà necessariamente tutti i vincoli progettuali obbligatori identificati nella documentazione di \textit{Capitolato$^{G}$ C5 - Login Warrior}, ed integrerà inoltre alcuni vincoli opzionali, al fine di fornire un prodotto quanto più completo possibile. \\ \\
I vincoli \textbf{obbligatori} identificati sono i seguenti:
\begin{itemize}
	\item L’applicazione sarà sviluppata prevalentemente in tecnologia \textit{HTML/CSS/JavaScript}.
	\item La principale libreria per la creazione dei grafici sarà \textit{D3.js$^{G}$}.
	\item I dati da presentare dovranno essere in formato \textit{.CSV$^{G}$} e saranno forniti
			dall'azienda.
	\item “Login Warrior” dovrà presentare almeno le seguenti visualizzazioni:
		 \begin{enumerate}
                    \item \textit{Scatter Plot$^{G}$}.
                    \item \textit{Sankey Diagram$^{G}$}.
		\end{enumerate}
\end{itemize}

\noindent
Le richieste obbligatorie sono relative alla prima fase esplorativa dei dati (EDA$^{G}$), dove si cerca di fornire un aiuto alla visualizzazione. È tuttavia possibile sfruttare anche il \textit{Machine Learning$^{G}$} per derivare visualizzazioni più complete.

\noindent Tra i vincoli \textbf{opzionali} si annoverano dunque i seguenti punti:
\begin{itemize}
\item Possibilità di utilizzo di altre tecnologie come \textit{Python} e libreria \textit{Scikit-learn$^{G}$}.
\item Possibilità di utilizzo di altri dataset$^{G}$ simili recuperati da archivi aperti.
\item Algoritmi di preparazione del dato per la visualizzazione. Librerie d'interesse fornite dall'azienda:
	\begin{enumerate}
                    \item \textit{t-SNE$^{G}$}.
                    \item \textit{UMAP$^{G}$}.
                    \item \textit{Self Organizing Map$^{G}$}.
	\end{enumerate}
\item Algoritmi di analisi del dato. Librerie d'interesse fornite dall'azienda:
	\begin{enumerate}
                    \item \textit{Isolation Forest$^{G}$}.
                    \item \textit{One Class SVM$^{G}$}.
                    \item \textit{Gaussian MCD$^{G}$}.
                    \item \textit{Robust Gaussian$^{G}$}.
                    \item \textit{HDBSCAN$^{G}$}.
	\end{enumerate}
\item Altri algoritmi di analisi dei dati da librerie non fornite dall’azienda.
\item Altre visualizzazioni dalla libreria D3.js$^{G}$ o da altre librerie.
\end{itemize}
