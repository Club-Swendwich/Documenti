\section{Vincoli di progettazione}

L'applicativo finale rispetterà necessariamente tutti i vincoli progettuali obbligatori identificati nella documentazione di \textit{Capitolato C5 - Login Warrior}, ed integrerà inoltre alcuni vincoli opzionali, al fine di fornire un prodotto quanto più completo possibile. \\ \\
I vincoli \textbf{obbligatori} identificati sono i seguenti:
\begin{itemize}
	\item L’applicazione sarà sviluppata prevalentemente in tecnologia \textit{HTML/CSS/JavaScript}.
	\item La principale libreria per la creazione dei grafici sarà \textit{D3.js}.
	\item I dati da presentare dovranno essere in formato \textit{CSV} e saranno forniti 
			dall'azienda.
	\item “Login Warrior” dovrà presentare almeno le seguenti visualizzazioni:
		 \begin{enumerate}
                    \item \textit{Scatter Plot}.
                    \item \textit{Parallel Coordinates}.
                    \item \textit{Force-directed Graph}.
                    \item \textit{Sankey Diagram}.
		\end{enumerate}
\end{itemize}

\noindent
Le richieste obbligatorie sono relative alla prima fase esplorativa dei dati (EDA), dove si cerca di fornire un aiuto alla visualizzazione. \' E tuttavia possibile sfruttare anche il \textit{Machine Learning} per derivare visualizzazioni più complete.

\noindent Tra i vincoli \textbf{opzionali} si annoverano dunque i seguenti punti:
\begin{itemize}
\item Possibilità di utilizzo di altre tecnologie come \textit{Python} e libreria \textit{Scikit-learn}. 
\item Possibilità di utilizzo di altri dataset simili recuperati da archivi aperti.
\item Algoritmi di preparazione del dato per la visualizzazione. Librerie d'interesse fornite dall'azienda:
	\begin{enumerate}
                    \item \textit{t-SNE}.
                    \item \textit{UMAP}.
                    \item \textit{Self Organizing Map}.
	\end{enumerate}
\item Algoritmi di analisi del dato. Librerie di interesse fornite dall'azienda:
	\begin{enumerate}
                    \item \textit{Isolation Forest}.
                    \item \textit{One Class SVM}.
                    \item \textit{Gaussian MCD}.
                    \item \textit{Robust Gaussian}.
                    \item \textit{HDBSCAN}.
	\end{enumerate}
\item Altri algoritmi di analisi dei dati da librerie non fornite dall’azienda.
\item Altre visualizzazioni dalla libreria D3.js o da altre librerie.
\end{itemize}



