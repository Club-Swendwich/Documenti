\section{Descrizione Generale}
\subsection{Scopo dell'applicativo}

L'applicativo da sviluppare pone come obiettivo la realizzazione di una \textit{web application} per la visualizzazione di dati di login$^{G}$, mediante lo sviluppo di grafici che supportino la fase esplorativa del riconoscimento e distinzione di attività lecite e attività sospette.

\subsection{Funzioni dell'applicativo}

Per l'esplorazione dei dati si utilizzerà un approccio EDA$^{G}$ (\textit{Exploratory Data Analysis}), il quale promuove una visualizzazione grafica del dato.\\ 
Immagini e grafici risultano infatti più esplicativi rispetto a numeri e l'occhio umano percepisce più facilmente possibili configurazioni come forme, gruppi o punti isolati. \\ \\
L'applicativo permetterà quindi all'utente di caricare un proprio dataset$^{G}$ e, in assenza di errori, poter scegliere quali tipologie di grafico visualizzare. Ogni grafico scelto fornirà le seguenti opzioni:
\begin{itemize}
\item Diverse \textbf{viste$^{G}$}, le quali saranno preconfezionate in modo tale da risultare di interesse.
\item Possibilità di creare e salvare nuove \textbf{viste$^{G}$} personalizzate.
\item Possibilità di modifica dello \textbf{stile}, ed eventualmente di ritorno ad una visualizzazione di default.
\end{itemize}

\subsection{Caratteristiche degli utenti}

Il progetto è pensato per un pubblico aziendale, in particolare per un target tecnico di sistemisti.\\
Non si prevedono quindi fasi di login$^{G}$ o diverse tipologie di utenza (l'unica presente avrà accesso a tutte le funzionalità del prodotto).

\subsection{Piattaforme di esecuzione}

Il progetto sarà fornito via web e costituito da un insieme di pagine accessibili dai browser più comuni e recenti. Testeremo nel dettaglio la completa compatibilità con browser come:
\begin{itemize}
\item \textit{Google Chrome}
\item \textit{Mozilla Firefox}
\item \textit{Safari}
\item \textit{Microsoft Edge}
\end{itemize}

%Accessibilità alla pagina web anche da mobile? 