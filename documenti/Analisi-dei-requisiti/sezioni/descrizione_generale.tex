\section{Descrizione Generale}
\subsection{Scopo dell'applicativo}

L'applicativo da sviluppare pone come obiettivo la realizzazione di una \textit{web application} per la visualizzazione di dati di login, mediante lo sviluppo di grafici che supportino la fase esplorativa del riconoscimento e distinzione di attività lecite e attività sospette.

\subsection{Funzioni dell'applicativo}

Per l'esplorazione dei dati si utilizzerà un approccio EDA (\textit{Exploratory Data Analysis}), il quale permette una visualizzazione grafica del dato. \\ 
Immagini e grafici risultano infatti più esplicativi rispetto a numeri e l'occhio umano percepisce più facilmente possibili configurazioni come forme, gruppi o punti isolati. \\ \\
L'applicativo permetterà quindi all'utente di caricare un proprio dataset e, in assenza di errori, poter scegliere quali tipologie di grafico visualizzare. Ogni grafico scelto fornirà le seguenti opzioni:
\begin{itemize}
\item Diverse \textbf{viste}, le quali saranno preconfezionate in modo tale da risultare di interesse.
\item Possibilità di modifica dello \textbf{stile}, ed eventualmente di ritorno ad una visualizzazione di default.
\end{itemize}

\subsection{Caratteristiche degli utenti}

Il progetto è pensato per un pubblico aziendale, in particolare per un target tecnico di sistemisti.\\
Non si prevedono quindi fasi di login o diverse tipologie di utenza (l'unica presente avrà accesso a tutte le funzionalità del prodotto).

\subsection{Piattaforme di esecuzione}

Il progetto sarà fornito via web e costituito da un insieme di pagine accessibili dai browser più comuni e recenti. Testeremo nel dettaglio la completa compatibilità con browser come:
\begin{itemize}
\item \textit{Google Chrome}
\item \textit{Mozilla Firefox}
\item \textit{Safari}
\item \textit{Microsoft Edge}
\item \textit{Opera}
\end{itemize}

%Accessibilità alla pagina web anche da mobile? 

\subsection{Vincoli di progettazione}

L'applicativo finale rispetterà necessariamente tutti i vincoli progettuali obbligatori identificati nella documentazione di \textit{Capitolato C5 - Login Warrior}, ed integrerà inoltre alcuni vincoli opzionali, al fine di fornire un prodotto quanto più completo possibile. \\ \\
I vincoli \textbf{obbligatori} identificati sono i seguenti:
\begin{itemize}
	\item L’applicazione sarà sviluppata prevalentemente in tecnologia 				\textit{HTML/CSS/JavaScript}.
	\item La principale libreria per la creazione dei grafici sarà \textit{D3.js}.
	\item I dati da presentare dovranno essere in formato \textit{CSV} e saranno forniti 
			dall'azienda.
	\item “Login Warrior” dovrà presentare almeno le seguenti visualizzazioni:
		 \begin{enumerate}
                    \item \textit{Scatter Plot}.
                    \item \textit{Parallel Coordinates}.
                    \item \textit{Force-directed Graph}.
                    \item \textit{Sankey Diagram}. \\
		\end{enumerate}
\end{itemize}

\noindent
Le richieste obbligatorie sono relative alla prima fase esplorativa dei dati (EDA), dove si cerca di fornire un aiuto alla visualizzazione. \' E tuttavia possibile sfruttare anche il \textit{Machine Learning} per derivare visualizzazioni più complete.

\noindent Tra i vincoli \textbf{opzionali} annoveriamo dunque i seguenti punti:
\begin{itemize}
\item Possibilità di utilizzo di altre tecnologie come \textit{Python} e libreria \textit{Scikit-learn}. 
\item Possibilità di utilizzo di altri dataset simili recuperati da archivi aperti.
\item Algoritmi di preparazione del dato per la visualizzazione. Librerie di interesse fornite dall'azienda:
	\begin{enumerate}
                    \item \textit{t-SNE}.
                    \item \textit{UMAP}.
                    \item \textit{Self Organizing Map}.
	\end{enumerate}
\item Algoritmi di analisi del dato. Librerie di interesse fornite dall'azienda:
	\begin{enumerate}
                    \item \textit{Isolation Forest}.
                    \item \textit{One Class SVM}.
                    \item \textit{Gaussian MCD}.
                    \item \textit{Robust Gaussian}.
                    \item \textit{HDBSCAN}.
	\end{enumerate}
\item Altri algoritmi di analisi dei dati da librerie non fornite dall’azienda.
\item Altre visualizzazioni dalla libreria D3.js o da altre librerie.
\end{itemize}



