\section{Introduzione}
\subsection{Scopo del documento}

Il seguente documento ha lo scopo di illustrare il \textit{way of working} adottato dal gruppo, specificando procedure, strumenti e metriche di qualità in un documento.
Al suo interno vengono riportati:
\begin{itemize}
    \item processi primari;
    \item processi organizzativi;
    \item processi di supporto;
    \item appendici tecniche relative agli standard di qualità implementati.
\end{itemize}

\subsection{Scopo capitolato}
Il capitolato$^{G}$ C5 si pone l'obiettivo di creare un'applicazione per la visualizzazione di dati
di login$^{G}$ per facilitarne l'analisi e l'interpretazione.\\
Il prodotto fornirà all'utente la possibilità di scegliere tra diverse modalità di visualizzazione personalizzabili 
e algoritmi per l'analisi dei dati. Attraverso questo processo esplorativo, EDA$^{G}$ (\textit{Exploratory Data Analysis}), 
l'utente finale potrà studiare i dati in modo più fruibile e immediato: obbiettivo ultimo di questo capitolato$^{G}$ 
è lo studio di dati di login$^{G}$ per individuare potenziali accessi non autorizzati.

\subsection{Glossario}
Al fine di chiarire possibili ambiguità relative alla terminologia utilizzata all'interno dei
documenti è stato redatto il \textit{Glossario} contenente i termini di particolare rilievo, questi
sono contrassegnati con una "G" ad apice.
\newpage