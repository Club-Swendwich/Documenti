\section{Standard ISO/IEC 12207:1995}
\textit{ISO/IEC 12207:1995} è uno standard
internazionale per i processi del ciclo di vita del software.
Nonostante sia stato aggiornato, e abbia dunque versioni più recenti, il gruppo
ne ha individuato all'interno caratteristiche di valore per la determinazione
della qualità di processo.\\
\noindent
Nello standard sono definiti i processi del ciclo di vita del
software e, per ognuno di essi, le attività da svolgere e i
risultati, "\textit{outcomes}", da produrre.\\ \\
\noindent
I processi sono suddivisi in tre categorie:

\setlength\extrarowheight{5pt}

\begin{table}[h!]
    \centering
    \begin{tabular}{c|p{0.65\linewidth}}
    \rowcolor[RGB]{33, 73, 50}
    \multicolumn{1}{>{\centering\arraybackslash}c|}{\textcolor{white}{\textbf{Categoria}}}
        & \multicolumn{1}{>{\centering\arraybackslash}c}{\textcolor{white}{\textbf{Descrizione}}} \\[4pt]
        \rowcolor[RGB]{216, 235, 171}
	    	\textbf{Processi primari} & Attività direttamente legate allo sviluppo del software. \\[4pt]
        \rowcolor[RGB]{233, 245, 206}
	    	\textbf{Processi di supporto} & Attività di supporto ad altri processi legate al controllo della qualità. \\[4pt]
        \rowcolor[RGB]{216, 235, 171}
	    	\textbf{Processi organizzativi} & Attività per il miglioramento e la gestione delle risorse.\\[4pt]
    \end{tabular}
    \caption{Categorie di processo}
\end{table}

\noindent
Per ogni processo, oltre alla lista delle attività, sono riportati
gli obiettivi, i risultati e le responsabilità.\\
Le attività che di seguito vengono riportate in \textbf{grassetto} sono quelle analizzate all'interno del \textit{Piano di Qualifica}.

%-------------------------------------------------------------------------------------------------------------------

\subsection{Processi primari}

\begin{table}[h!]
    \centering
    \begin{tabular}{c|p{0.70\linewidth}}
    \rowcolor[RGB]{33, 73, 50}
    \multicolumn{1}{>{\centering\arraybackslash}c|}{\textcolor{white}{\textbf{Processo}}}
        & \multicolumn{1}{>{\centering\arraybackslash}c}{\textcolor{white}{\textbf{Descrizione}}} \\[4pt]
        \rowcolor[RGB]{216, 235, 171}
        Acquisizione
        & Il processo ha lo scopo di ottenere il prodotto/servizio che soddisfa
        le necessità del cliente. \\[4pt]
        \rowcolor[RGB]{233, 245, 206}
        \textbf{Fornitura}
        & Il processo ha lo scopo di identificare procedure e risorse
        atte a soddisfare i requisiti di progetto. \\[4pt]
        \rowcolor[RGB]{216, 235, 171}
        \textbf{Sviluppo}
        & Il processo si occupa delle attività per la
        realizzazione del prodotto. \\[4pt]
        \rowcolor[RGB]{233, 245, 206}
        Operation
        & Il processo è svolto simultaneamente al processo di manutenzione.
        Il processo ha lo scopo di mantenere operativo il sistema e di fornire
        il supporto agli utenti. \\[4pt]
        \rowcolor[RGB]{216, 235, 171}
        Manutenzione
        & Il processo è svolto simultaneamente al processo di operation.
        Il processo ha lo scopo di modificare il prodotto software dopo il suo rilascio per
        correggerne i difetti e migliorare le prestazioni. \\[4pt]
    \end{tabular}
    \caption{Processi primari}
\end{table}

\subsubsection{Fornitura}
Le procedure della seguente attività sono:
\begin{itemize}
    \item Preparazione della risposta
    \item Contratto
    \item Pianiicazione
    \item Esecuzione e Controllo
    \item Revisione e Valutazione
    \item Rilascio e Completamento
\end{itemize}
\subsubsection{Sviluppo}
Le procedure della seguente attività sono:
\begin{itemize}
    \item Estrazione dei requisiti
    \item Analisi dei requisiti di sistema
    \item Progettazione architettura di sistema
    \item Analisi dei requisiti software
    \item Progettazione software
    \item Codifica
    \item Integrazione dei componenti
    \item Testing del software
    \item Integrazione del sistema
    \item Testing del sistema
    \item Installazione software
\end{itemize}
%-----------------------------------------------------------------------------------------------------------------
\newpage
\subsection{Processi organizzativi}

\begin{table}[h!]
    \centering
    \begin{tabular}{c|p{0.58\linewidth}}
    \rowcolor[RGB]{33, 73, 50}
    \multicolumn{1}{>{\centering\arraybackslash}c|}{\textcolor{white}{\textbf{Categoria}}}
        & \multicolumn{1}{>{\centering\arraybackslash}c}{\textcolor{white}{\textbf{Descrizione}}} \\[4pt]
        \rowcolor[RGB]{216, 235, 171}
        Gestione dell’infrastruttura
        & Il processo ha lo scopo di mantenere una infrastruttura stabile ed affidabile
        per supportare le prestazioni di qualsiasi processo. \\[4pt]
        \rowcolor[RGB]{233, 245, 206}
        \textbf{Gestione organizzativa}
        & Il processo ha lo scopo di organizzare, monitorare e controllare l'avvio e le
        prestazioni di un processo per il raggiungimento degli obiettivi in accordo. \\[4pt]
        \rowcolor[RGB]{216, 235, 171}
        Miglioramento del processo
        & Il processo ha lo scopo di stabilire, valutare, misurare, controllare
        e migliorare il ciclo di vita del software. \\[4pt]
        \rowcolor[RGB]{233, 245, 206}
        Risorse Umane
        & Il processo ha lo scopo di mantenere le competenze del gruppo consistenti
        con le necessità di progetto. \\[4pt]
        \rowcolor[RGB]{216, 235, 171}
        Asset Management
        & Il processo ha lo scopo di gestire gli elementi definiti
        come "asset" per tutta la loro durata. \\[4pt]
        \rowcolor[RGB]{233, 245, 206}
        Reuse Program Management
        & Il processo ha lo scopo di pianificare, stabilire, gestire,
        controllare e monitorare il programma di riuso formulato
        dall'organizzazione \\[4pt]
        \rowcolor[RGB]{216, 235, 171}
        Domain Engineering
        & Il processo ha lo scopo di sviluppare e manutenere modelli,
        architetture e asset del dominio in cui opera il prodotto
        software sviluppato. \\[4pt]
    \end{tabular}
    \caption{Processi organizzativi}
\end{table}

\subsubsection{Gestione organizzativa}
Le procedure della seguente attività sono:
\begin{itemize}
    \item Inizializzazione e definizione dello scopo
    \item Pianificazione
    \item Esecuzione e controllo
    \item Revisione e Valutazione
    \item Chiusura
\end{itemize}
%-------------------------------------------------------------------------------------------------------------------
\newpage
\subsection{Processi di supporto}

\begin{table}[h!]
    \centering
    \begin{tabular}{c|p{0.56\linewidth}}
    \rowcolor[RGB]{33, 73, 50}
    \multicolumn{1}{>{\centering\arraybackslash}c|}{\textcolor{white}{\textbf{Categoria}}}
        & \multicolumn{1}{>{\centering\arraybackslash}c}{\textcolor{white}{\textbf{Descrizione}}} \\[4pt]
        \rowcolor[RGB]{216, 235, 171}
        \textbf{Gestione della documentazione}
        & Il processo garantisce lo sviluppo e la manutenzione delle informazioni prodotte e
        registrate relativamente al prodotto software. \\[4pt]
        \rowcolor[RGB]{233, 245, 206}
        Gestione della configurazione
        & Il processo ha lo scopo di definire e mantenere l'integrità di tutti i componenti
        della configurazione, e consentirne l'accesso a chi ne possiede l'autorizzazione. \\[4pt]
        \rowcolor[RGB]{216, 235, 171}
        Risoluzione problemi
        & Il processo di risoluzione dei problemi ha lo scopo di analizzare e risolvere
        i problemi. \\[4pt]
        \rowcolor[RGB]{233, 245, 206}
        \textbf{Assicurazione della qualità}
        & Il processo ha lo scopo di assicurare che tutti i prodotti di fase siano conformi con i piani e gli standard
        definiti. \\[4pt]
        \rowcolor[RGB]{216, 235, 171}
        \textbf{Verifica}
        & Il processo ha lo scopo di determinare se i prodotti software di un’attività soddisfano i
        requisiti o le condizioni a loro imposti. \\[4pt]
        \rowcolor[RGB]{233, 245, 206}
        Audit
        & Il processo ha lo scopo di determinare in maniera indipendente la conformità
        di prodotti e processi selezionati ai requisiti. \\[4pt]
        \rowcolor[RGB]{216, 235, 171}
        Revisione Congiunta
        & Il processo ha lo scopo di rivedere con gli \textit{stakeholders}$^G$ i processi
        eseguiti rispetto agli obiettivi definiti negli accordi, e le cose da fare per assicurare
        lo sviluppo di un prodotto che soddisfi i requisiti concordati. \\[4pt]
        \rowcolor[RGB]{233, 245, 206}
        Validazione
        & Il processo ha lo scopo di valutare che i requisiti siano rispettati quando,
        uno specifico work product, è utilizzato nell'ambiente destinatario. \\[4pt]
        \rowcolor[RGB]{216, 235, 171}
        Usabilità
        & Il processo ha lo scopo di assicurare che siano prese in considerazione, e opportunamente indirizzate,
        le considerazioni espresse dagli \textit{stakeholders}$^G$ relativamente alla facilità d'uso del prodotto finale. \\[4pt]
        \rowcolor[RGB]{233, 245, 206}
        Valutazione del Prodotto
        & Il processo ha lo scopo di assicurare, tramite esami e misure, che il prodotto soddisfi le necessità esplicite
        e implicite degli utilizzatori.  \\[4pt]
    \end{tabular}
    \caption{Processi di supporto}
\end{table}

\subsubsection{Gestione della documentazione}
Le procedure della seguente attività sono:
\begin{itemize}
    \item Pianificazione
    \item Progettazione e Sviluppo
    \item Manutenzione
\end{itemize}

\subsubsection{Assicurazione della qualità}
Le procedure della seguente attività sono:
\begin{itemize}
    \item Accertamento di prodotto
    \item Accertamento di processo
    \item Assicurazione della qualità di sistema
\end{itemize}

\subsubsection{Verifica}
Per l'efficacia dei costi e delle prestazioni, la verifica deve essere integrata il prima possibile con il processo
che la utilizza.\\
Le procedure della seguente attività sono:
\begin{itemize}
    \item Analisi
    \item Revisione
    \item Test
\end{itemize}

\setlength\extrarowheight{0pt}
