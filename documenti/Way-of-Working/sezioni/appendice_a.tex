\section{Standard ISO/IEC 12207:1995}
\textit{ISO/IEC 12207:1995} è uno standard 
internazionale per i processi del ciclo di vita del software.
Nonostante sia stato aggiornato, e abbia dunque versioni più recenti, il gruppo
ne ha individuato all'interno carattarestiche di valore per la determinazione
della qualità di processo.\\
\noindent
Nello standard sono definiti i processi del ciclo di vita del 
software e, per ognuno di essi, le attività da svolgere ed i
risultati, "\textit{outcomes}", da produrre.\\ \\
\noindent 
I processi sono suddivisi in tre categorie:

\setlength\extrarowheight{5pt}

\begin{table}[h!]
	\footnotesize
    \centering
    \begin{tabular}{>{\raggedright\arraybackslash}c|c}
    \rowcolor[RGB]{33, 73, 50}
    \multicolumn{1}{>{\centering\arraybackslash}c|}{\textcolor{white}{\textbf{Categoria}}} 
        & \multicolumn{1}{>{\centering\arraybackslash}c}{\textcolor{white}{\textbf{Descrizione}}} \\[4pt]
        \rowcolor[RGB]{216, 235, 171}
	    	\textbf{Processi primari} & Attività direttamente legate allo sviluppo del software. \\[4pt]
        \rowcolor[RGB]{233, 245, 206}
	    	\textbf{Processi di supporto} & Attività di supporto ad altri processi legate al controllo della qualità. \\[4pt]
        \rowcolor[RGB]{216, 235, 171}
	    	\textbf{Processi organizzativi} & Attività per il miglioramento e la gestione delle risorse.\\[4pt]
    \end{tabular}
    \caption{Categorie di processo}
\end{table}

\noindent
Per ogni processo, oltre alla lista delle attività, sono riportati
gli obiettivi, i risultati e le responsabilità.\\

%-------------------------------------------------------------------------------------------------------------------

\subsection{Processi primari}

\begin{table}[h!]
	\footnotesize
    \centering
    \begin{tabular}{>{\raggedright\arraybackslash}c|c}
    \rowcolor[RGB]{33, 73, 50}
    \multicolumn{1}{>{\centering\arraybackslash}c|}{\textcolor{white}{\textbf{Categoria}}} 
        & \multicolumn{1}{>{\centering\arraybackslash}c}{\textcolor{white}{\textbf{Descrizione}}} \\[4pt]
        \rowcolor[RGB]{216, 235, 171}
        P1 Acquisizione
        & - \\[4pt]
        \rowcolor[RGB]{233, 245, 206}
        \textbf{P2 Fornitura} 
        & \Centerstack{Il processo ha lo scopo di identificare procedure e risorse 
        atte a soddisfare\\ i requisiti di progetto}\\[4pt]
        \rowcolor[RGB]{216, 235, 171}
        \textbf{P3 Sviluppo} 
        & Il processo si occupa delle attività per la
        realizzazione del prodotto. \\[4pt]
        \rowcolor[RGB]{233, 245, 206}
        P4 Esecuzione
        & - \\[4pt]
        \rowcolor[RGB]{216, 235, 171}
        P5 Manutenzione
        & - \\[4pt]
    \end{tabular}
    \caption{Categorie di processo}
\end{table}

\subsubsection{Fornitura}
\subsubsection{Sviluppo}

%-----------------------------------------------------------------------------------------------------------------

\subsection{Processi organizzativi}

\begin{table}[h!]
	\footnotesize
    \centering
    \begin{tabular}{>{\raggedright\arraybackslash}c|c}
    \rowcolor[RGB]{33, 73, 50}
    \multicolumn{1}{>{\centering\arraybackslash}c|}{\textcolor{white}{\textbf{Categoria}}} 
        & \multicolumn{1}{>{\centering\arraybackslash}c}{\textcolor{white}{\textbf{Descrizione}}} \\[4pt]
        \rowcolor[RGB]{216, 235, 171}
        O1 Gestione dell’infrastruttura
        & - \\[4pt]
        \rowcolor[RGB]{233, 245, 206}
        O2 Management
        & - \\[4pt]
        \rowcolor[RGB]{216, 235, 171}
        O3 Miglioramento 
        & - \\[4pt]
        \rowcolor[RGB]{233, 245, 206}
        \textbf{O4 Risorse Umane}
        & \Centerstack{Il processo ha lo scopo di identificare procedure e 
        risorse atte\\ a soddisfare i requisiti di progetto} \\[4pt]
        \rowcolor[RGB]{216, 235, 171}
        O5 Asset Management
        & - \\[4pt]
        \rowcolor[RGB]{233, 245, 206}
        O6 Reuse Program Management
        & - \\[4pt]
        \rowcolor[RGB]{216, 235, 171}
        O7 Domain Engineering
        & - \\[4pt]
    \end{tabular}
    \caption{Categorie di processo}
\end{table}

\subsubsection{Risorse Umane}

%-------------------------------------------------------------------------------------------------------------------
\newpage
\subsection{Processi di supporto}

\begin{table}[h!]
	\footnotesize
    \centering
    \begin{tabular}{>{\raggedright\arraybackslash}c|c}
    \rowcolor[RGB]{33, 73, 50}
    \multicolumn{1}{>{\centering\arraybackslash}c|}{\textcolor{white}{\textbf{Categoria}}} 
        & \multicolumn{1}{>{\centering\arraybackslash}c}{\textcolor{white}{\textbf{Descrizione}}} \\[4pt]
        \rowcolor[RGB]{216, 235, 171}
        S1 Gestione della documentazione
        & - \\[4pt]
        \rowcolor[RGB]{233, 245, 206}
        S2 Gestione della configurazione
        & - \\[4pt]
        \rowcolor[RGB]{216, 235, 171}
        S3 Risoluzione problemi
        & - \\[4pt]
        \rowcolor[RGB]{233, 245, 206}
        \textbf{S4 Assicurazione della qualità}
        & \Centerstack{Il processo ha lo scopo di identificare procedure 
        e risorse atte\\ a soddisfare i requisiti di progetto} \\[4pt]
        \rowcolor[RGB]{216, 235, 171}
        \textbf{S5 Verifica}
        & \Centerstack{Il processo si occupa delle attività per la
        realizzazione del\\ prodotto} \\[4pt]
        \rowcolor[RGB]{233, 245, 206}
        S6 Audit
        & - \\[4pt]
        \rowcolor[RGB]{216, 235, 171}
        S7 Revisione Congiunta
        & - \\[4pt]
        \rowcolor[RGB]{233, 245, 206}
        S8 Validazione
        & - \\[4pt]
        \rowcolor[RGB]{216, 235, 171}
        S9 Usabilità
        & - \\[4pt]
        \rowcolor[RGB]{233, 245, 206}
        S10 Valutazione del Prodotto
        & - \\[4pt]
    \end{tabular}
    \caption{Categorie di processo}
\end{table}

\subsubsection{Assicurazione della qualità}
\subsubsection{Verifica}


\setlength\extrarowheight{0pt}

