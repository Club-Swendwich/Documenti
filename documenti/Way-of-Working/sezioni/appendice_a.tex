\section{Standard ISO/IEC 12207:1995}
Lo standard \textit{ISO/IEC 12207:1995} è uno standard 
internazionale per i processi del ciclo di vita del software.
Nonostante sia stato aggiornato, e abbia dunque versioni più recenti, il gruppo
ne ha individuato all'interno carattarestiche di valore per la determinazione
della qualità di processo.\\
\noindent
Nello standard sono definiti i processi del ciclo di vita del 
software e, per ognuno di essi, le attività da svolgere ed i
risultati, "\textit{outcomes}", da produrre.\\ \\
\noindent 
I processi sono suddivisi in tre categorie:

\setlength\extrarowheight{5pt}

\begin{table}[h!]
	\footnotesize
    \centering
    \begin{tabular}{>{\raggedright\arraybackslash}c|c}
    \rowcolor[RGB]{33, 73, 50}
    \multicolumn{1}{>{\centering\arraybackslash}c|}{\textcolor{white}{\textbf{Categoria}}} 
        & \multicolumn{1}{>{\centering\arraybackslash}c}{\textcolor{white}{\textbf{Descrizione}}} \\[4pt]
        \rowcolor[RGB]{216, 235, 171}
	    	Processi primari & Attività direttamente legate allo sviluppo del software. \\[4pt]
        \rowcolor[RGB]{233, 245, 206}
	    	Processi di supporto & Attività di supporto ad altri processi legate al controllo della qualità. \\[4pt]
        \rowcolor[RGB]{216, 235, 171}
	    	Processi organizzativi & Attività per il miglioramento e la gestione delle risorse.\\[4pt]
    \end{tabular}
    \caption{Categorie di processo}
\end{table}

\noindent
Per ogni processo, oltre alla lista delle attività, sono riportati
gli obiettivi, i risultati e le responsabilità.\\
Dalla lista delle attività ne sono state selezionate solo alcune 
ritenute di rilievo per il capitolato C5.



\setlength\extrarowheight{0pt}

