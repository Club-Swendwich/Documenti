\section{Codifica}

Per la produzione del codice, il gruppo ha deciso di adottare modello di sviluppo iterativo \textit{Scrum}. Per tracciare lo stato delle attività è quindi stata implementata una Scrum board su Trello.
La struttura della board è la seguente:
\begin{itemdescript}
    \item [Backlog] Lista delle card relative a issues che sono in attesa di essere assegnate ad uno sprint.
    \item [Next Sprint] Lista delle card assegnate allo sprint successivo. 
    \item [Sprint Backlog] Lista delle card in attesa di essere prese in carico.
    \item [Code Review] Lista delle card relative a eventuali bug o verifiche non superate in attesa di essere prese in carico non appena possibile.
    \item [In Progress] Lista delle card attualmente nella fase di codifica. A ogni card corrisponde un branch di lavoro.
    \item [Testing] Lista delle card in attesa o attualmente nella fase di verifica.
    \item [Waiting for approval] Lista delle card in attesa di approvazione e merge nel branch padre.
    \item [Done - Current Sprint] Lista delle card approvate e delle quali è stato effettuato il merge durante lo sprint attuale.
    \item [Done - All Time] Lista onnicomprensiva delle card approvate durante tutti gli sprint.
\end{itemdescript}
Tutto il codice prodotto deve essere mantenuto all'interno della repo git \href{}{hostata su GitHub (LINK)}.
La struttura della repository è la seguente:
\begin{itemize}
    \item \texttt{...}: ...
\end{itemize}

\subsection{Aggiunta e modifica di funzionalità}
Le funzionalità da aggiungere e modificare devono essere unità atomiche, cioè facilmente risolvibili, in modo da rispettare l'approccio Scrum dove ogni micro-obiettivo inserito nel backlog viene organizzato e suddiviso in sprint dove, assieme agli altri micro-obiettivi dello sprint, permette di raggiungere lo scopo fissato per lo sprint stesso.
\subsection{Creazione issue GitHub e gestione del workflow tramite card Trello}
Il primo passaggio da effettuare, quando si decide di aggiungere o modificare funzionalità al programma, è quello dell'apertura di un'issue.
Tale issue deve essere aperta su GitHub nell'apposita sezione "Issues", della repository ????? e deve essere compilata nel seguente modo:
\begin{itemize}
    \item Se si tratta di una nuova funzionalità, il titolo dell'issue deve essere: \texttt{Nome funzionalità}
    \item Se, invece, si tratta di una modifica, il titolo deve essere:
    \texttt{Nome funzionalità - Modifica da effettuare (in breve)}
    \item Nella descrizione è opportuno spiegare in breve in cosa consiste tale funzionalità o le modifiche che dovranno essere apportate
    \item È necessario definire, tramite label, se si tratta di una funzionalità obbligatoria, opzionale o gradita.
    \item Se necessario, assegnare l'issue a una milestone fissata.
\end{itemize}
Le issues create in questo modo sono automaticamente trasformate in \textit{Trello cards} grazie all'integrazione con Zapier.
\subsection{Gestione degli sprint}
Ogni 2 settimane inizia un nuovo sprint. Nello sprint precedente sarà già stato preparato, dal responsabile del gruppo, il backlog per lo sprint in partenza, e tali card saranno state posizionate nella lista \textit{Next Sprint}. \\
Con questa condizione soddisfatta, si comincia il nuovo sprint premendo il pulsante \textit{Next Sprint}. Questo causa lo spostamento delle card da \textit{Next Sprint} a \textit{Sprtint Backlog} e da \textit{Done - Current Sprint} a \textit{Done - All Time}. Queste ultime schede vengono inoltre rinominate automaticamente nel seguente modo: \texttt{Sprint \{data inizio sprint\}\footnote{La data corrisponde all'inizio del primo sprint di appartenenza della card dato che potrebbe accadere, per varie motivazioni, che una card sia incompleta alla fine dello sprint e venga inclusa in quello successivo.} to \{data fine sprint\} - \{titolo della card\}}.\\
A questo punto si procede con l'assegnazione delle cards (e delle corrispondenti issues) ai vari componenti.
\subsubsection{Gestione dei branch}
\begin{itemize}
    \item Quando le cards vengono spostate nello \textit{Sprint Backlog} gli viene assegnata in automatico l'etichetta \texttt{To Be Branched}, per ricordare al collaboratore di eseguire il branch prima di iniziare a lavorarci.
    \item Prima di procedere con la produzione del documento è fondamentale assegnare sia issue che card ai componenti che lo hanno preso in carico ed eseguire il branch.
    \item Il branch deve seguire lo standard \texttt{NomeUtente/Issue\#}, dove "\#" sarà sostituito dal numero dell'issue.
    \item Una volta eseguita tale operazione, è necessario cambiare l'etichetta della card in \texttt{Branched}, manualmente o tramite il bottone "Branched", che è possibile trovare all'interno della card. È consigliato usare il bottone della card, dato che sposta automaticamente la card in "\textit{In Progress}".
\end{itemize}
Per migliorare il workflow è possibile scegliere già verificatori e approvatori che, una volta terminate le modifiche, procederanno con la verifica e l'approvazione del documento.\\ 
Questi componenti potranno seguire la card (tramite il pulsante apposito), in modo da essere notificati su eventuali modifiche della stessa.
A questo punto è possibile procedere con la produzione del codice.
\subsection{Standard di codifica}
-- Definire standard di codifica --\\

\subsection{Richiedere la verifica di una funzionalità}
Per richiedere la verifica di una funzionalità è sufficiente spostare la card di Trello nella lista \textit{Testing}. I partecipanti, vengono tolti automaticamente, in modo da rendere chiaro agli altri componenti che l'issue ha bisogno di una verifica, e tale verifica non è ancora stata presa in carico. Il verificatore si assegna poi la card e procede con il lavoro.

\subsection{Verificare una funzionalità}
-- Definire standard di verifica --\\

\subsubsection{Segnalare un problema}
Se, durante la fase di verifica, si dovessero presentare dei problemi, bisognerà procedere nel seguente modo:
\begin{enumerate}
    \item Spostare la card nella lista \textit{Code Review}. Alla card verrà automaticamente assegnata la label \texttt{Bug}.
    \item Assegnare la card e l'issue corrisponde agli autori originali del codice. È possibile trovare tale informazione nei dettagli della card. In caso non siano visibili, premere il bottone \textit{Mostra Dettagli}.
    \item Commentare la card con una descrizione precisa dell'errore. Tale descrizione deve essere quanto più completa possibile, comprensiva di situazione in cui si verifica l'errore e metodo per ripetere l'errore. Se possibile allegare alla card anche un file di log dell'errore.
    \item È consigliato avvisare gli autori del problema rilevato tramite messaggio.
\end{enumerate}

\subsection{Richiedere l'approvazione di una funzionalità}
Per richiedere l'approvazione di una funzionalità è sufficiente spostare la card da \textit{Testing} a \textit{Waiting for approval}. I partecipanti vengono tolti automaticamente, in modo da rendere chiaro agli altri componenti che l'issue necessita dell'approvazione finale, e tale compito non è ancora stato preso in carico. 
L'approvatore si assegna poi la card e procede con il lavoro.\\
Automaticamente viene assegnata l'etichetta \texttt{To be merged} alla card (tale etichetta serve a ricordare all'approvatore di eseguire il merge del branch dell'issue, una volta approvato il documento).
Quando questa azione viene eseguita, si preme il pulsante "Merged" presente nella card: tale pulsante sostituisce la label \texttt{To be merged} con quella \texttt{Merged} e sposta la card nella lista \textit{Done - Current Sprint}, togliendo contemporaneamente tutti i componenti assegnati alla card.