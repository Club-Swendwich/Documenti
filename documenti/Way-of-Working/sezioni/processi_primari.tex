\section{Processi primari}

\subsection{Fornitura}
Nella seguente sottosezione vengono normati i seguenti aspetti per l'attività di \textit{Fornitura}:
\begin{itemize}
    \item \textbf{Procedure}
    \item \textbf{Strumenti}
\end{itemize}

\subsubsection{Procedure}

Le procedure individuate dallo standard \textit{ISO/IEC 12207:1995} in questa attività sono:
\begin{itemize}
    \item Preparazione della risposta
    \item Pianificazione
    \item Esecuzione e Controllo
    \item Revisione e Valutazione
    \item Rilascio e Completamento
\end{itemize}

\paragraph{Preparazione della risposta}
\mbox{} \\
Il gruppo \textit{Club Swendwich} deve redigere i seguenti documenti:
\begin{itemize}
    \item \textbf{Impegni}
    \item \textbf{Scelta Capitolato$^G$}
\end{itemize}
Deve inoltre essere redatta la prima versione del \textit{way of working} da adottare, documento \textbf{Norme di Progetto} v1.0.0.

\paragraph{Pianificazione}
\mbox{} \\
Il gruppo deve individuare i \textbf{requisiti} richiesti dal proponente.\\
Il gruppo deve redigere i documenti per la gestione dei costi attesi, obiettivi da raggiungere e qualità attesa:
\begin{itemize}
    \item \textbf{Piano di Progetto}
    \item \textbf{Piano di Qualifica}
\end{itemize}

\paragraph{Esecuzione e controllo}
\mbox{} \\
Per quanto concerne il \textit{Piano di Progetto} il gruppo deve controllare i seguenti aspetti:
\begin{itemize}
    \item Costi attesi (\textbf{Preventivo a finire})
    \item Costi osservati (\textbf{Consuntivo di periodo})
    \item Rischi e mitigazioni
    \item Obiettivi raggiunti e da raggiungere
\end{itemize}
Per quanto concerne il \textit{Piano di Qualifica} il gruppo deve controllare i seguenti aspetti:
\begin{itemize}
    \item Specifica degli obiettivi quantitativi di \textbf{qualità di prodotto} e di \textbf{processo}
    \item Misurazione del raggiungimento di tali obiettivi allo stato corrente (\textbf{Cruscotto})
    \item Esiti di retrospettive
\end{itemize}

Il gruppo deve inoltre monitorare l'avanzamento mediante uso di \textbf{milestones}$^G$ .

\paragraph{Revisione e Valutazione}
\mbox{} \\
Il gruppo deve eseguire verifica costante dei propri prodotti, secondo quanto riportato nella sezione \hyperref[sec:Verifica]{Processi di Supporto - Verifica}.

\paragraph{Rilascio e Completamento}
\mbox{} \\
Ogni rilascio sarà effettuato tramite \textit{release} nella \textit{repository}$^G$  di lavoro.
Dovrà essere rilasciato sia codice sorgente che documentazione.
La struttura interna della cartella che sarà rilasciata dovrà includere le seguenti sezioni:
\begin{itemize}
    \item Documenti interni
    \item Documenti esterni
\end{itemize}

\subsubsection{Strumenti}
Gli strumenti da utilizzare sono:

\begin{itemize}
    \item Google Sheets: per la creazione di tabelle e grafici \\
    \href{https://www.google.com/sheets/about/}{https://www.google.com/sheets/about/}
    \item Trello$^G$: per la gestione delle \textit{tasks} da svolgere nel progetto.\\
    \href{https://trello.com/it}{https://trello.com/it}
    \item Gantt Project: per la creazione dei grafici di Gantt relativi alla pianificazione di progetto.\\
    \href{https://www.ganttproject.biz/}{https://www.ganttproject.biz/}
    \item GitHub$^G$ : per il rilascio di una \textit{release} ad ogni fase di avanzamento\\
    \href{https://github.com/Club-Swendwich}{https://github.com/Club-Swendwich}
\end{itemize}

\newpage
\subsection{Sviluppo}
Nella seguente sottosezione vengono normati i seguenti aspetti per l'attività di \textit{Sviluppo}:
\begin{itemize}
    \item \textbf{Procedure}
    \item \textbf{Metriche di qualità}
    \item \textbf{Strumenti}
\end{itemize}

\subsubsection{Procedure}

Le procedure individuate dallo standard \textit{ISO/IEC 12207:1995} in questa attività sono:
\begin{itemize}
    \item Analisi dei requisiti
    \item Progettazione
    \item Codifica
\end{itemize}

\paragraph{Analisi dei requisiti}
\mbox{} \\
Dopo aver effettuato una prima analisi dei requisiti, il gruppo \textit{Club Swendwich}
deve procedere allo sviluppo del documento:
\begin{itemize}
    \item \textbf{Analisi dei requisiti}
\end{itemize}

\subparagraph{Casi d'uso}
\mbox{} \\
I casi d'uso presenti nel documento \textit{Analisi dei requisiti} vengono identificati univocamente attraverso un codice, la struttura di questo codice è così formata:
\begin{itemize}
    \item   \textbf{Codice identificativo:}
            \par \centerline{\textbf{UC[\# caso d'uso].[Eventuale caso d'uso figlio]-[Titolo]}}
\end{itemize}
\mbox{} \\
Per la realizzazione degli Use Case diagram si utilizza \textbf{Draw.io} con le seguenti norme:
\begin{itemize}
\item Carattere del testo: Helvetica 12pt;
\item Titolo frame: UC[\#];
\item Titolo Use Case: UC[\#], a capo "Nome del caso d'uso";
\item Descrizione Extend with Condition: "Condition: \{descrizione\}" a capo "Extension Point: visualizzazione (descrizione di cosa viene visualizzato)".
\end{itemize}
Le componenti grafiche del diagramma seguono lo standard Use Case di UML analizzato nella lezione P4 del docente Cardin. \\

\noindent Per la descrizione degli Use Case si utilizza sempre lo standard visto nella lezione P4:
\begin{itemize}
	\item \textbf{Attore primario:} Chi è l'attore primario del caso d'uso;
	\item \textbf{Precondizioni:} Condizioni necessarie all'inizio del caso d'uso;
	\item \textbf{Postcondizioni:} Effetti/garanzia di ciò che succede in seguito;
	\item \textbf{Scenario principale:} Sequenza di passi che descrive le interazioni, si utilizza una lista \textit{itemize} in LaTeX per descriverne le fasi;
	\item \textbf{Estensioni:} Descrizione dei possibili ``extend``;
	\item \textbf{Generalizzazioni:} Descrizione di possibili aggiunte o modifiche delle caratteristiche base.
\end{itemize}


\subparagraph{Struttura dei requisiti}
\mbox{} \\
La struttura delle tabelle relative alla sezione \textit{Requisiti} nel documento \textit{Analisi dei requisiti} è composta nel modo seguente:
\begin{itemize}
    \item   \textbf{Codice identificativo:}
            \par \centerline{\textbf{R[Peso][Tipo][Codice]}}
            Dove \textbf{Peso} indica un numero da 1 a 3 che rappresenta:
            \begin{enumerate}
                \item Requisito obbligatorio;
                \item Requisito apprezzabile ma non essenziale;
                \item Requisito opzionale.
            \end{enumerate}
            Con \textbf{Tipo} viene indicata la natura del requisito, le possibilità sono le seguenti:
            \begin{itemize}
                \item \textbf{V}: Vincolo;
                \item \textbf{F}: Funzionale;
                \item \textbf{Q}: Qualitativo;
                \item \textbf{P}: Prestazionale.
            \end{itemize}
            Con \textbf{Codice} si intende un identificativo univoco del requisito.
    \item \textbf{Classe:} Serve a rendere più esplicita la tabella, in essa viene infatti riportato il peso, già espresso nel codice identificativo, ma questa volta scritto a parole.
    \item \textbf{Descrizione:} Una breve descrizione del requisito.
    \item \textbf{Riferimenti:} Indica da quale fonte deriva il requisito in analisi.
\end{itemize}

\paragraph{Progettazione}
\mbox{} \\
Basandosi sul documento \textit{Analisi dei requisiti} il gruppo deve sviluppare
un applicativo di valore.

\subparagraph{Requirements and Technology Baseline}
\mbox{} \\
Per questa fase di avanzamento il progettista deve occuparsi di:
\begin{itemize}
    \item Scelta di tecnologie, framework$^G$, e librerie per la realizzazione del prodotto;
    \item Progettazione di un \textbf{Proof of Concept}$^G$ .
\end{itemize}

\paragraph{Codifica}

\subparagraph{Aggiunta e modifica di funzionalità}
\mbox{}\\
Tutte le funzionalità da aggiungere e modificare devono essere \textbf{unità atomiche}, cioè facilmente risolvibili, in modo da rispettare
l'approccio Scrum$^G$ :
\begin{itemize}
    \item Ogni micro-obiettivo inserito nel backlog viene organizzato e suddiviso in sprint$^G$;
    \item Assieme agli altri micro-obiettivi dello sprint$^G$ permette di raggiungere lo scopo fissato per lo sprint$^G$ stesso.
\end{itemize}

\subparagraph{Adesione allo standard}
\mbox{}\\
Si vuole garantire:
\begin{itemize}
    \item Uniformità nella produzione di codice da parte di tutto il team;
    \item Codice comprensibile da tutti;
    \item Automatizzazione di questi processi
\end{itemize}
Per ogni membro del team è obbligatorio verificare che il codice prodotto rispetti questi standard prima che il codice
venga fatto entrare all'interno della repo$^G$ .\\
Di seguito viene riportato per ogni linguaggio di programmazione le linee guida a cui fare riferimento e gli opportuni
stumenti per la verifica:

\subparagraph{TypeScript}
\mbox{}\\
Per la scrittura di tutto il code typescript che verrà prodotto dal team ci si impegna a seguire il coding
style standard:
\begin{itemize}
    \item \textit{AirBnB} \href{https://github.com/airbnb/javascript}{https://github.com/airbnb/javascript}
\end{itemize}
con annesse estensioni per il supporto al typesystem di Typescript.\\

Per garantire che venga rispettatto costantemente e in modo totalmente automatico in ogni progetto JavaScript
andrà installato il plugin:
\begin{itemize}
    \item \textit{Eslint} \href{https://eslint.org/}{https://eslint.org/} e impostato l'insieme di regole
    \begin{itemize}
        \item \textit{AirBnB} \href{https://www.npmjs.com/package/eslint-config-airbnb-typescript}{https://www.npmjs.com/package/eslint-config-airbnb-typescript}
    \end{itemize}
\end{itemize}

\subparagraph{Python}
\mbox{}\\
Per lo sviluppo degli automatismi all'interno dei repository$^G$  dell' organizzazione (esecuzione test automatici,
compilazione automatica documenti \LaTeX, controlli ortografici automatici) si utilizzarà il linguaggio:
\textit{Python3} seguendo le direttive ufficiali:
\begin{itemize}
    \item \textit{PEP-8} \href{https://pep8.org/}{https://pep8.org/}
\end{itemize}
Verrà verificato in modo automatico tramite lo strumento:
\begin{itemize}
    \item \textit{Pylint} \href{https://pylint.org/}{https://pylint.org/}
\end{itemize}

\newpage
\subsubsection{Metriche di qualità}
\paragraph{Funzionalità}
\setlength\extrarowheight{5pt}


\begin{center}
    \begin{longtable}{p{0.20\linewidth} c p{0.35\linewidth} c}
    \rowcolor[RGB]{33, 73, 50}
        \multicolumn{1}{>{\centering\arraybackslash}c}{\textcolor{white}{\textbf{Nome}}}
        & \multicolumn{1}{>{\centering\arraybackslash}c}{\textcolor{white}{\textbf{Codice}}}
        & \multicolumn{1}{>{\centering\arraybackslash}c}{\textcolor{white}{\textbf{Descrizione}}}
		& \multicolumn{1}{>{\centering\arraybackslash}c}{\textcolor{white}{\textbf{Formula}}}\\[4pt]

    \rowcolor[RGB]{216, 235, 171}
        \textbf{Copertura} \par \textbf{dei requisiti} &
        MPD3 &
        Indice di copertura dei requisiti all'interno del prodotto. \par
        \textbf{$R_M$ requisiti mancanti}. \par
        \textbf{$R_T$ requisiti totali}. &
        $(1- \frac{R_M}{R_T}) \cdot 100$ \\

        \caption{Metriche di funzionalità}
    \end{longtable}
\end{center}

\setlength\extrarowheight{0pt}

\paragraph{Affidabilità}
\setlength\extrarowheight{5pt}

\begin{center}
    \centering
    \begin{longtable}{p{0.25\linewidth} c p{0.35\linewidth} c}
    \rowcolor[RGB]{33, 73, 50}
        \multicolumn{1}{>{\centering\arraybackslash}c}{\textcolor{white}{\textbf{Nome}}}
        & \multicolumn{1}{>{\centering\arraybackslash}c}{\textcolor{white}{\textbf{Codice}}}
        & \multicolumn{1}{>{\centering\arraybackslash}c}{\textcolor{white}{\textbf{Descrizione}}}
		& \multicolumn{1}{>{\centering\arraybackslash}c}{\textcolor{white}{\textbf{Formula}}}\\[4pt]

    \rowcolor[RGB]{216, 235, 171}
        \textbf{Densità di failure} &
        MPD4 &
        Indice di percentuale di test falliti su test eseguiti. \par
        \textbf{$T_F$ test falliti}. \par
        \textbf{$T_E$ test eseguiti}.&
        $(\frac{T_F}{T_E}) \cdot 100$ \\

        \caption{Metriche di affidabilità}
    \end{longtable}
\end{center}

\setlength\extrarowheight{0pt}

\paragraph{Efficienza}
\setlength\extrarowheight{5pt}

\begin{center}
    \centering
    \begin{longtable}{p{0.20\linewidth} c p{0.35\linewidth} c}
    \rowcolor[RGB]{33, 73, 50}
        \multicolumn{1}{>{\centering\arraybackslash}c}{\textcolor{white}{\textbf{Nome}}}
        & \multicolumn{1}{>{\centering\arraybackslash}c}{\textcolor{white}{\textbf{Codice}}}
        & \multicolumn{1}{>{\centering\arraybackslash}c}{\textcolor{white}{\textbf{Descrizione}}}
		& \multicolumn{1}{>{\centering\arraybackslash}c}{\textcolor{white}{\textbf{Formula}}}\\[4pt]
    \rowcolor[RGB]{216, 235, 171}
    \textbf{Tempo medio} \par \textbf{di risposta} &
    MPD5 &
    Tempo medio impiegato dal \par prodotto nel rispondere ad una richiesta/svolgere un'attività di sistema. &
    - \\

    \caption{Metriche di efficienza}
    \end{longtable}
\end{center}

\setlength\extrarowheight{0pt}

\newpage
\paragraph{Usabilità}
\setlength\extrarowheight{5pt}

\begin{center}
    \centering
    \begin{longtable}{p{0.30\linewidth} c p{0.32\linewidth} p{0.16\linewidth}}
    \rowcolor[RGB]{33, 73, 50}
        \multicolumn{1}{>{\centering\arraybackslash}c}{\textcolor{white}{\textbf{Nome}}}
        & \multicolumn{1}{>{\centering\arraybackslash}c}{\textcolor{white}{\textbf{Codice}}}
        & \multicolumn{1}{>{\centering\arraybackslash}c}{\textcolor{white}{\textbf{Descrizione}}}
		& \multicolumn{1}{>{\centering\arraybackslash}c}{\textcolor{white}{\textbf{Formula}}}\\[4pt]

    \rowcolor[RGB]{216, 235, 171}
        \textbf{Average Cyclomatic} \par \textbf{Complexity} &
         MPD6 &
         Complessità condizionale,\par  misura il numero di cammini linearmente indipendenti &
         Grafo di \par controllo del \par flusso \\

    \rowcolor[RGB]{233, 245, 206}
        \textbf{Facilità di utilizzo} &
        MPD7 &
        Numero di clicli necessari \par al raggiungimento della funzionalità di interesse &
        - \\

    \rowcolor[RGB]{216, 235, 171}
        \textbf{Facilità apprendimento funzionalità} &
        MPD8 &
        Tempo necessario ( in \par minuti) all'utente per \par l'apprendimento delle \par funzionalità &
        - \\

        \caption{Metriche di usabilità}
    \end{longtable}
\end{center}

\setlength\extrarowheight{0pt}

\paragraph{Manutenibilità}
\setlength\extrarowheight{5pt}

\begin{center}
    \centering
    \begin{longtable}{p{0.20\linewidth} c p{0.35\linewidth} c}
    \rowcolor[RGB]{33, 73, 50}
        \multicolumn{1}{>{\centering\arraybackslash}c}{\textcolor{white}{\textbf{Nome}}}
        & \multicolumn{1}{>{\centering\arraybackslash}c}{\textcolor{white}{\textbf{Codice}}}
        & \multicolumn{1}{>{\centering\arraybackslash}c}{\textcolor{white}{\textbf{Descrizione}}}
		& \multicolumn{1}{>{\centering\arraybackslash}c}{\textcolor{white}{\textbf{Formula}}}\\[4pt]
    \rowcolor[RGB]{216, 235, 171}
        \textbf{Comprensione del codice} &
        MPD9 &
        Rapporto indicativo tra le linee di commento e di codice \par
        \textbf{$N_c$ numero linee commento} \par
        \textbf{$N_C$ numero linee di codice}&
        $(\frac{N_c}{N_C}) \cdot 100$ \\

        \caption{Metriche di manutenibilità}
    \end{longtable}
\end{center}

\setlength\extrarowheight{0pt}

\paragraph{Portabilità}
\setlength\extrarowheight{5pt}

\begin{center}
    \centering
    \begin{longtable}{p{0.25\linewidth} c p{0.35\linewidth} c}
    \rowcolor[RGB]{33, 73, 50}
        \multicolumn{1}{>{\centering\arraybackslash}c}{\textcolor{white}{\textbf{Nome}}}
        & \multicolumn{1}{>{\centering\arraybackslash}c}{\textcolor{white}{\textbf{Codice}}}
        & \multicolumn{1}{>{\centering\arraybackslash}c}{\textcolor{white}{\textbf{Descrizione}}}
		& \multicolumn{1}{>{\centering\arraybackslash}c}{\textcolor{white}{\textbf{Formula}}}\\[4pt]
    \rowcolor[RGB]{216, 235, 171}
        \textbf{Versioni del} \par \textbf{browser supportate} &
        MPD10 &
        Percentuale di browser supportati dal prodotto \par
        \textbf{$B_S$ browser supportati} \par
        \textbf{$B_I$ browser individuati} \par \textbf{nell'analisi dei requisiti} &
        $(\frac{B_S}{B_I}) \cdot 100$ \\

        \caption{Metriche di portabilità}
    \end{longtable}
\end{center}

\setlength\extrarowheight{0pt}

\subsubsection{Strumenti}
Gli strumenti da utilizzare sono:
\begin{itemize}
    \item D3.js$^G$: libreria JavaScript per la realizzazione del grafico Scatter Plot$^G$  nel \textit{Proof of Concept}$^G$ .\\
    \href{https://d3js.org/}{https://d3js.org/}
    \item React: la libreria utilizzata per la realizzazione del \textit{Proof of Concept}$^G$ .\\
    \href{https://it.reactjs.org/}{https://it.reactjs.org/}
    \item TypeScript: il linguaggio utilizzato per la realizzazione del \textit{Proof of Concept}$^G$ .\\
    \href{https://www.typescriptlang.org/}{https://www.typescriptlang.org/}
    \item Visual Studio Code: editor di codice utilizzato dal gruppo. \\
    \href{https://code.visualstudio.com/}{https://code.visualstudio.com/}
    \item Draw.io: strumento per la realizzazione dei grafici UML.\\
    \href{https://app.diagrams.net/}{https://app.diagrams.net/}
\end{itemize}
