\section{Processi primari}

\subsection{Fornitura}
Nella seguente sottosezione vengono normati i seguenti aspetti per il 
processo di \textit{Fornitura}:
\begin{itemize}
    \item \textbf{Procedure}
    \item \textbf{Metriche di qualità}
    \item \textbf{Strumenti}
\end{itemize}

\subsubsection{Procedure}

Le attività individuate dallo standard \textit{ISO/IEC 12207:1995} in questo processo sono:
\begin{itemize}
    \item Preparazione della risposta
    \item Pianificazione
    \item Esecuzione e Controllo
    \item Revisione e Valutazione
    \item Rilascio e Completamento
\end{itemize}

\paragraph{Preparazione della risposta}
\mbox{} \\
Il gruppo \textit{Club Swendwich} deve redigere i seguenti documenti:
\begin{itemize}
    \item \textbf{Impegni}
    \item \textbf{Scelta Capitolato}
\end{itemize}
Deve inoltre essere redatta la prima versione del \textit{way of working} da adottare, documento \textbf{Norme di Progetto} v1.0.0.

\paragraph{Pianificazione}
\mbox{} \\
Il gruppo deve individuare i \textbf{requisiti} richiesti dal proponente.\\
Il gruppo deve redigere i documenti per la gestione dei costi attesi, obiettivi da raggiungere e qualità attesa:
\begin{itemize}
    \item \textbf{Piano di Progetto}
    \item \textbf{Piano di Qualifica}
\end{itemize}

\paragraph{Esecuzione e controllo}
\mbox{} \\
Per quanto concerne il \textit{Piano di Progetto} il gruppo deve controllare i seguenti aspetti:
\begin{itemize}
    \item Costi attesi (\textbf{Preventivo a finire})
    \item Costi osservati (\textbf{Consuntivo di periodo})
    \item Rischi e mitigazioni
    \item Obiettivi raggiunti e da raggiungere
\end{itemize}
Per quanto concerne il \textit{Piano di Qualifica} il gruppo deve controllare i seguenti aspetti:
\begin{itemize}
    \item Specifica degli obiettivi quantitativi di \textbf{qualità di prodotto} e di \textbf{processo}
    \item Misurazione del raggiungimento di tali obiettivi allo stato corrente (\textbf{Cruscotto})
    \item Esiti di retrospettive
\end{itemize}

Il gruppo deve inoltre monitorare l'avanzamento mediante uso di \textbf{milestones}.

\paragraph{Revisione e Valutazione}
\mbox{} \\
Il gruppo deve eseguire verifica costante dei propri prodotti, secondo quanto riportato alla sezione LIIINK||||||

\paragraph{Rilascio e Completamento}
\mbox{} \\
Ogni rilascio sarà effettuato tramite \textit{release} nella \textit{repository} di lavoro.
Dovrà essere rilasciato sia codice sorgente che documentazione.
La struttura interna della cartella che sarà rilasciata dovrà includere le seguenti sezioni:
\begin{itemize}
    \item Documenti interni 
    \item Documenti esterni
\end{itemize}

\subsubsection{Metriche di qualità}
Il processo di Fornitura non fa uso di particolari metriche di qualità.

\subsubsection{Strumenti}
Gli strumenti da utilizzare in questo processo sono:

\begin{itemize}
    \item Google Sheets: per la creazione di tabelle e grafici \\
    https://www.google.com/sheets/about/
    \item Trello: per la gestione delle \textit{tasks} da svolgere nel progetto.\\
    https://trello.com/it
    \item Gantt Project: per la creazione dei grafici di Gantt relativi alla pianificazione di progetto.
    https://www.ganttproject.biz/
\end{itemize}

\subsection{Sviluppo}
Nella seguente sottosezione vengono normati i seguenti aspetti per il 
processo di \textit{Sviluppo}:
\begin{itemize}
    \item \textbf{Procedure}
    \item \textbf{Metriche di qualità}
    \item \textbf{Strumenti}
\end{itemize}

\subsubsection{Procedure}

Le attività individuate dallo standard \textit{ISO/IEC 12207:1995} in questo processo sono:
\begin{itemize}
    \item Analisi dei requisiti
    \item Progettazione
    \item Codifica
\end{itemize}

\paragraph{Analisi dei requisiti}
\mbox{} \\
Dopo aver individuato una prima pianificazione dei requisiti, il gruppo \textit{Club Swendwich}
deve procedere allo sviluppo del documento:
\begin{itemize}
    \item \textbf{Analisi dei requisiti}
\end{itemize}

\subparagraph{Casi d'uso}
\begin{itemize}
    \item   \textbf{Codice identificativo:}
            \par \centerline{\textbf{UC[Numero caso d'uso].[Eventuale caso d'uso figlio]-[Titolo caso d'uso]}}
\end{itemize}
Per la realizzazione degli Use Case diagram si utilizza \textbf{Draw.io} con le seguenti norme:
\begin{itemize}
\item Carattere del testo: Helvetica 12pt.
\item Titolo frame: UC[\#]
\item Titolo Use Case: UC[\#], a capo "Nome del caso d'uso".
\item Descrizione Extend with Condition: "Condition: \{descrizione\}" a capo "Extension Point: visualizzazione (descrizione di cosa viene visualizzato)".
\end{itemize}
Le componenti grafiche del diagramma seguono lo standard Use Case di UML analizzato nella lezione P4 del docente Cardin. \\

\noindent Per la descrizione degli Use Case si utilizza sempre lo standard visto nella lezione P4:
\begin{itemize}
	\item \textbf{Attore primario:} Chi è l'attore primario del caso d'uso.
	\item \textbf{Precondizioni:} Condizioni necessarie all'inizio del caso d'uso.
	\item \textbf{Postcondizioni:} Effetti/garanzia di ciò che succede in seguito.
	\item \textbf{Scenario principale:} Sequenza di passi che descrive le interazioni, si utilizza una lista \textit{itemize} in LaTeX per descriverne le fasi.
	\item \textbf{Estensioni:} Descrizione dei possibili ``extend``.
	\item \textbf{Generalizzazioni:} Descrizione di possibili aggiunte o modifiche delle caratteristiche base.
\end{itemize}


\subparagraph{Struttura dei requisiti}
\mbox{} \\
La struttura delle tabelle relative alla sezione \textit{Requisiti} nel documento \textit{Analisi dei requisiti} è strutturata nel modo seguente:
\begin{itemize}
    \item   \textbf{Codice identificativo:}
            \par \centerline{\textbf{R[Peso][Tipo][Codice]}}
            Dove \textbf{Peso} indica un numero da 1 a 3 che rappresenta:
            \begin{enumerate}
                \item Requisito obbligatorio;
                \item Requisito apprezzabile ma non essenziale;
                \item Requisito opzionale.
            \end{enumerate}
            Con \textbf{Tipo} viene indicata la natura del requisito, le possibilità sono le seguenti:
            \begin{itemize}
                \item \textbf{V}: Vincolo;
                \item \textbf{F}: Funzionale;
                \item \textbf{Q}: Qualitativo;
                \item \textbf{P}: Prestazionale.
            \end{itemize}
            Con \textbf{Codice} si intende un identificativo univoco del requisito.
    \item \textbf{Classe:} Serve a rendere più esplicita la tabella, in essa viene infatti riportato il peso, già espresso nel codice identificativo, ma questa volta scritto a parole.
    \item \textbf{Descrizione:} Una breve descrizione del requisito.
    \item \textbf{Riferimenti:} Indica da quale fonte deriva il requisito in analisi.   
\end{itemize}

\paragraph{Progettazione}
\mbox{} \\
Basandosi sul documento \textit{Analisi dei requisiti} il gruppo deve sviluppare
un applicativo di valore.

\subparagraph{Requirements and Technology Baseline}
\mbox{} \\
Per questa fase di avanzamento il progettista deve occuparsi di:
\begin{itemize}
    \item Scelta di tecnologie, framework, e librerie per la realizzazione del prodotto.
    \item Progettazione di un \textbf{Proof of Concept}
\end{itemize}

\paragraph{Codifica}
\mbox{} \\
INSERIRE PARTE DI SAMUELE

\subsubsection{Metriche di qualità}

\paragraph{Metriche per la Funzionalità}
\mbox{} \\

\paragraph{Metriche per l' Affidabilità}
\mbox{} \\

\paragraph{Metriche per l' Efficienza}
\mbox{} \\

\paragraph{Metriche per l' Usabilità}
\mbox{} \\

\paragraph{Metriche per la Manutenibilità}
\mbox{} \\

\paragraph{Metriche per la Portabilità}
\mbox{} \\

\subsubsection{Strumenti}
