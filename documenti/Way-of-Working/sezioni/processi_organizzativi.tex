\section{Processi organizzativi}

\subsection{Gestione organizzativa}
Nella seguente sottosezione vengono normati i seguenti aspetti per il 
processo di \textit{Gestione organizzativa}:
\begin{itemize}
    \item \textbf{Procedure}
    \item \textbf{Strumenti}
\end{itemize}

\subsubsection{Procedure}
Le attività individuate dallo standard \textit{ISO/IEC 12207:1995} in questo processo sono:
\begin{itemize}
    \item Inizializzazione e definizione dello scopo
    \item Pianificazione
    \item Esecuzione e Controllo
    \item Revisione e Valutazione
\end{itemize}

\paragraph{Inizializzazione e definizione dello scopo}
\mbox{}\\
I ruoli di progetto sono quelli indicati nelle dispense T06 del docente Vardanega.
Ogni membro del gruppo dovrà a rotazione svolgere ogni ruolo all'interno del progetto.
\begin{itemize}
    \item \textbf{Responsabile}
    \begin{itemize}
        \item Sigla: \textbf{RE}
        \item Comunicazione con proponente e committente
        \item Pianificazione del lavoro
        \item Gestione e controllo
    \end{itemize}
    \item \textbf{Amministratore}
    \begin{itemize}
        \item Sigla: \textbf{AM}
        \item Controllo dell'ambiente di lavoro
        \item Direzione delle infrastrutture di supporto
        \item Controllo di versioni e configurazioni
        \item Gestione della documentazione
        \item Risoluzione di problemi relativi alla gestione dei processi
    \end{itemize}
    \item \textbf{Analista}
    \begin{itemize}
        \item Sigla: \textbf{AN}
        \item Studio e analisi del dominio applicativo
        \item Definizione dei requisiti e delle complessità
        \item Particolare importanza nella redazione del documento \textit{Analisi dei requisiti}
    \end{itemize}
    \item \textbf{Progettista}
    \begin{itemize}
        \item Sigla: \textbf{PT}
        \item Progettazione dell'architettura di sistema in base alle tecnologie scelte
        \item Controllo dello sviluppo tecnico del prodotto
    \end{itemize}
    \item \textbf{Programmatore}
    \begin{itemize}
        \item Sigla: \textbf{PG}
        \item Codifica secondo le specifiche del progettista
        \item Gestione delle componenti di supporto per la verifica e validazione del codice
    \end{itemize}
    \item \textbf{Verificatore}
    \begin{itemize}
        \item Sigla: \textbf{VE}
        \item Verifica dei prodotti secondo le \textit{Norme di progetto}
        \item Segnalazione degli errori riscontrati all'autore del prodotto oggetto di verifica
    \end{itemize}
\end{itemize}

\paragraph{Pianificazione}
\mbox{}\\
Per la gestione delle attività da svolgere bisogna sempre guardare la pianificazione presente sulle bacheche di \textit{Trello}.
Sono presenti due bacheche:
\begin{itemize}
    \item \textbf{Documenti}, con le seguenti sezioni: 
    \begin{itemize}
        \item Questions for the next meeting
        \item Backlog
        \item Next Sprint
        \item Sprint Backlog
        \item In progress
        \item To be verified
        \item To be approved
        \item Done - Current Sprint
        \item Done - All time
    \end{itemize}
\end{itemize}

\begin{itemize}
    \item \textbf{Coding}
    \begin{itemdescript}
        \item [Backlog] Lista delle card relative a issues che sono in attesa di essere assegnate ad uno sprint.
        \item [Next Sprint] Lista delle card assegnate allo sprint successivo. 
        \item [Sprint Backlog] Lista delle card in attesa di essere prese in carico.
        \item [Code Review] Lista delle card relative a eventuali bug o verifiche non superate in attesa di essere prese in carico non appena possibile.
        \item [In Progress] Lista delle card attualmente nella fase di codifica. A ogni card corrisponde un branch di lavoro.
        \item [Testing] Lista delle card in attesa o attualmente nella fase di verifica.
        \item [Waiting for approval] Lista delle card in attesa di approvazione e merge nel branch padre.
        \item [Done - Current Sprint] Lista delle card approvate e delle quali è stato effettuato il merge durante lo sprint attuale.
        \item [Done - All Time] Lista onnicomprensiva delle card approvate durante tutti gli sprint.
    \end{itemdescript}
\end{itemize}
Tutto il codice prodotto deve essere mantenuto all'interno della repo git hostata su GitHub.
%La struttura della repository è la seguente:
%\begin{itemize}
%    \item \texttt{...}: ...
%\end{itemize}

\subparagraph{Gestione degli Sprint}
\mbox{}\\
Ogni 2 settimane inizia un nuovo sprint. Nello sprint precedente sarà già stato preparato, dal responsabile del gruppo, il backlog per lo sprint in partenza, e tali card saranno state posizionate nella lista \textit{Next Sprint}. \\
Con questa condizione soddisfatta, si comincia il nuovo sprint premendo il pulsante \textit{Next Sprint}. Questo causa lo spostamento delle card da \textit{Next Sprint} a \textit{Sprtint Backlog} e da \textit{Done - Current Sprint} a \textit{Done - All Time}.\\
Queste ultime schede vengono inoltre rinominate automaticamente nel seguente modo: \texttt{Sprint \{data inizio sprint\}\footnote{La data corrisponde all'inizio del primo sprint di appartenenza della card dato che potrebbe accadere, per varie motivazioni, che una card sia incompleta alla fine dello sprint e venga inclusa in quello successivo.} to \{data fine sprint\} - \{titolo della card\}}.\\
A questo punto si procede con l'assegnazione delle cards (e delle corrispondenti issues) ai vari componenti.

\subparagraph{Questions for the next meeting}
\mbox{}\\
La seguente scheda di \textit{Trello} deve essere aggiornata nei seguenti casi:
\begin{itemize}
    \item \textbf{Prima} di un \textbf{incontro interno} al gruppo, per segnalare agli altri membri del gruppo
    la presenza di questioni da discutere.
    \item \textbf{Prima} di un \textbf{incontro esterno} con il proponente/committente. In questo caso il responsabile
    di progetto, a seguito di una riunione interna con i membri del gruppo, si occuperà di segnare le
    domande che dovranno esssere poste al successivo colloquio esterno.
\end{itemize}

\paragraph{Esecuzione e Controllo}

\subparagraph{Comunicazioni}
\mbox{}\\

Le comunicazioni devono essere gestite nel seguente modo:
\begin{itemize}
    \item \textbf{Comunicazioni interne}
    \begin{itemize}
        \item Le comunicazioni di carattere generale avvengono sul canale \textit{Telegram}
        che include tutti i membri del gruppo.
        \item Le comunicazioni che invece includono temi come: 
        \begin{itemize}
            \item Pianificazione
            \item Decisioni
            \item Confronto
            \item Stesura di domande per il proponente/committente
        \end{itemize}
        devono avvenire sul canale \textit{Discord}.
    \end{itemize}
    \item \textbf{Comunicazioni esterne}
    \begin{itemize}
        \item Tutte le comunicazioni con il proponente e con il committente sono gestite dal Responsabile di progetto.
        \item Vengono svolte su \textit{Zoom} in caso di colloqui.
        \item Vengono svolte tramite posta elettronica del gruppo \href{mailto:clubswendwich@gmail.com}{clubswendwich@gmail.com} in tutti gli altri casi.
    \end{itemize}
\end{itemize}

Ogni incontro avvenuto su Discord/Zoom deve produrre un relativo verbale interno/esterno.

\subparagraph{Gestione dei Rischi}
\mbox{}\\
I rischi sono suddivisi per categoria a seconda della tipologia del problema che si può verificare e
sono indicati tramite codici univoci così generati:

\begin{center}
    \textbf{R[Categoria][Numero progressivo]}
\end{center}

I rischi rientrano nelle seguenti categorie:
\begin{itemize}
    \item \textbf{Tecnologie di lavoro e di produzione software:} T;
    \item \textbf{Rapporti interpersonali:} I;
    \item \textbf{Organizzazione del lavoro:} O;
    \item \textbf{Costi e tempi:} C;
\end{itemize}


\paragraph{Revisione e Valutazione}

\subparagraph{Richiedere la verifica di un documento}
\mbox{}\\
La richiesta di verifica dei documenti deve essere effettuata in uno dei seguenti modi, 
a seconda della dimensione del documento in questione:
\begin{itemize}
    \item Per richiedere la verifica di un \textbf{documento di piccole dimensioni} (es. verbali) è sufficiente:
    \begin{itemize}
        \item spostare la card di Trello nella lista "To be verified"; 
        \item togliere la partecipazione (azione eseguita automaticamente) in modo da rendere chiaro agli altri 
        componenti che l'issue ha bisogno di una verifica, e che tale verifica non è ancora stata presa in carico;
        \item il verificatore si assegna poi la card e procede con il lavoro.
    \end{itemize}
    \item Per richiedere la verifica di un \textbf{documento di grandi dimensioni} (es. documenti esterni nei quali è presente 
    una checklist con le sezioni da produrre/modificare) è necessario:
    \begin{itemize}
        \item creare una copia della card;
        \item spostare quest'ultima in "To be verified" (questo per permettere una verifica in parallelo alla 
        creazione/modifica dei punti della lista);
        \item man mano che i punti della checklist nella card "In progress" verranno spuntati, si procederà con 
        l'aggiunta di essi nella card "To be verified", in modo da iniziarne la verifica.
    \end{itemize}
\end{itemize}

\subparagraph{Richiedere l'approvazione di un documento}
\mbox{}\\
Per richiedere l'approvazione di un documento è sufficiente:
\begin{itemize}
    \item Spostare la card da "To be verified" a "To be approved";
    \item Togliere la partecipazione dei componenti.
\end{itemize}  
Automaticamente verrà assegnata l'etichetta \texttt{To be merged} alla card (tale etichetta serve a ricordare all'approvatore 
di eseguire il merge del branch dell'issue, una volta approvato il documento).\\
Quando questa azione viene eseguita basterà premere il pulsante "Merged" presente nella card: tale pulsante sostituirà la label 
\texttt{To be merged} con quella \texttt{Merged} e sposterà la card nella lista "Done - Current Sprint", togliendo contemporaneamente tutti i 
componenti assegnati alla card.

\subparagraph{Richiedere la verifica di una funzionalità}
\mbox{}\\
Per richiedere la verifica di una funzionalità bisogna:
\begin{itemize}
    \item Spostare la card di Trello nella lista \textit{Testing}.
\end{itemize} 
I partecipanti vengono tolti automaticamente, in modo da rendere chiaro agli altri componenti che l'issue ha bisogno di una verifica, e tale verifica non è ancora stata presa in carico.\\
Il verificatore si assegna poi la card e procede con il lavoro.

\subparagraph{Segnalare un problema nel codice}
\mbox{}\\
Se, durante la fase di verifica, si dovessero presentare dei problemi, bisognerà procedere nel seguente modo:
\begin{enumerate}
    \item Spostare la card nella lista \textit{Code Review}. Alla card verrà automaticamente assegnata la label \texttt{Bug}.
    \item Assegnare la card e l'issue corrisponde agli autori originali del codice. È possibile trovare tale informazione nei dettagli della card. In caso non siano visibili, premere il bottone \textit{Mostra Dettagli}.
    \item Commentare la card con una descrizione precisa dell'errore. Tale descrizione deve essere quanto più completa possibile, comprensiva di situazione in cui si verifica l'errore e metodo per ripetere l'errore. Se possibile allegare alla card anche un file di log dell'errore.
    \item È consigliato avvisare gli autori del problema rilevato tramite messaggio.
\end{enumerate}

\subparagraph{Richiedere l'approvazione di una funzionalità}
\mbox{}\\
Per richiedere l'approvazione di una funzionalità si effettuano i seguenti passi:
\begin{itemize} 
    \item Spostare la card da \textit{Testing} a \textit{Waiting for approval};
    \item I partecipanti vengono tolti automaticamente, in modo da rendere chiaro agli altri componenti che l'issue necessita dell'approvazione finale, e tale compito non è ancora stato preso in carico;
    \item L'approvatore si assegna poi la card e procede con il lavoro.
    \item Automaticamente viene assegnata l'etichetta \texttt{To be merged} alla card (tale etichetta serve a ricordare all'approvatore di eseguire il merge del branch dell'issue, una volta approvato il documento);
    \item Quando questa azione viene eseguita, si preme il pulsante "Merged" presente nella card: tale pulsante sostituisce la label \texttt{To be merged} con quella \texttt{Merged} e sposta la card nella lista \textit{Done - Current Sprint}, togliendo contemporaneamente tutti i componenti assegnati alla card.
\end{itemize}

\subsubsection{Strumenti}

\begin{itemize}
    \item Trello: strumento di gestione delle \textit{tasks} da svolgere nel progetto. \\
    \href{https://trello.com/it}{https://trello.com/it}
    \item GitHub: repository ufficiale del gruppo \textit{Club Swendwich}.\\
    \href{https://github.com/Club-Swendwich}{https://github.com/Club-Swendwich}
    \item Telegram: chat di gruppo per le comunicazioni ed eventuali votazioni. \\
    \href{https://web.telegram.org/}{https://web.telegram.org/}
    \item Discord: canale vocale per le riunioni interne.\\
    \href{https://discord.com/}{https://discord.com/}
    \item Zoom: software utilizzato per i meeting esterni con proponente e committente.\\
    \href{https://zoom.us/}{https://zoom.us/}
\end{itemize}