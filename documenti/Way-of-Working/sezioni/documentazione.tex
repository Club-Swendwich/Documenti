\section{Documentazione}

Lo strumento scelto dal gruppo per la stesura della documentazione è \LaTeX,
tutta la documentazione deve essere mantenuta all'interno della repo git
\href{https://github.com/Club-Swendwich/Documenti}{hostata su GitHub (LINK)}.
La struttura della repository è la seguente:
\begin{itemize}
    \item \texttt{assets}: Contiene tutti gli elementi grafici comuni ai documenti
        prodotti dal team.
    \item \texttt{verbali}: Contiene i resoconti di tutti gli incontri del team sia
        interni che con entità esterne al team.
    \item \texttt{presentazioni}: Contiene tutte le presentazioni basate su slide.
    \item \texttt{documenti}: Contiene tutti i documenti che non sono verbali o
        presentazioni.
    \item \texttt{template}: Contiene tutti i template da utilizzare quando si va a
        creare un nuovo documento.
\end{itemize}

\subsection{Creazione e Modifica di un documento}

\subsubsection{Creazione Issue GitHub e gestione del workflow tramite Card Trello}
Il primo passaggio da effettuare, quando si decide di creare o modificare un documento, è quello dell'apertura dell'issue relativa ad esso.
Tale issue dovrà essere aperta su GitHub nell'apposita sezione "Issues", della repository Documenti e dovrà essere compilata nel seguente modo:
\begin{itemize}
    \item Se il documento dovrà essere creato, il titolo dovrà essere
    \texttt{Nome Documento}
    \item Se, invece, si tratta di una modifica, il titolo dovrà essere
    \texttt{Nome Documento - Modifica da effettuare (in breve)}
    \item Nella descrizione è opportuno spiegare in breve in cosa consisterà tale documento o le modifiche che dovranno essere apportate
    \item È consigliato l'utilizzo della label \texttt{documento}
    \item Se necessario, assegnare l'issue a una milestone fissata
    \item Se la produzione da effettuare è di mole importante, è consigliato suddivedere la produzione in punti elenco, tramite checklist, in modo da distribuire e tracciare più facilmente il lavoro
\end{itemize}

Le issues create in questo modo saranno automaticamente trasformate in \textit{Trello cards} grazie all'integrazione con Zapier.
Quando le cards vengono create gli viene assegnata in automatico l'etichetta \texttt{To Be Branched}, per ricordare al collaboratore di eseguire il branch.

Prima di procedere con la produzione del documento è fondamentale assegnare sia issue che card ad uno o più componenti che lo hanno preso in carico, ed eseguire il branch per la risoluzione dell'issue.

Il branch deve seguire lo standard \texttt{NomeUtente/Issue\#}, dove "\#" sarà sostituito dal numero della issue.

Una volta eseguita tale operazione, è necessario cambiare l'etichetta della card in \texttt{Branched}, manualmente o tramite il bottone "Branched", che è possibile trovare all'interno della card.
È consigliato usare il bottone della card dato che, in caso sia stato dimenticato il passaggio fino a questo momento, la card verrà assegnata automaticamente al collaboratore che premerà il bottone e la card verrà spostata nella list "In progress".

A questo punto è possibile procedere con la creazione o modifica del documento.

Per migliorare il workflow è possibile scegliere già verificatori e approvatori che, una volta terminate le modifiche, procederanno con la verifica e l'approvazione del documento. Questi componenti potranno seguire la card (tramite il pulsante apposito), in modo da essere notificati su eventuali modifiche sulla stessa.

\subsubsection{Creazione del Documento}
Nel processo di produzione di un nuovo documento è fondamentale utilizzare i templates già
disponibili, essi sono il miglior modo per mantenere la documentazione uniforme e
risparmiare tempo.
La versione di partenza di un documento è la \texttt{0.1.0}.

\subsubsection{Scelta template}

Sono disponibili 3 tipi di template:
\begin{itemize}
    \item \texttt{documenti}: Tutti i documenti \LaTeX che non sono verbali, interni ed esterni.
    \item \texttt{presentazioni}: Tutte le presentazioni.
    \item \texttt{verbali}: Tutti i verbali.
\end{itemize}
Qualora non fosse presente il template desiderato è necessaria una discussione con il gruppo prima di procedere a crearne di nuovi.

\subsubsection{Redattori, Verificatori, Approvatori}
In ogni documento e verbale in prima pagina viene inserito un riassunto dei ruoli svolti da ogni componente. I ruoli vengono svolti in più fasi, e queste sono visibili nel dettaglio nella tabella ``Registro delle modifiche''. Può succedere che un componente sia inserito sia come redattore che come verificatore, in quanto può aver svolto il ruolo in fasi differenti del documento.


\subsubsection{Gestione nominativi}
Per dare maggiore coerenza al testo si è deciso di utilizzare le seguenti regole per i nominativi:
\begin{itemize}
	\item \texttt{Cognome Nome} : ordine da seguire nelle tabelle, nel registro delle modifiche, e nelle prime pagine dei documenti in corrispondenza dei ruoli attribuiti.
	\item \texttt{Assenti} : da indicare con "-" nel caso non fossero presenti assenti durante i meeting del gruppo o nelle call.
\end{itemize}

\subsection{Versionamento}
Ogni documento che verrà aggiornato nel tempo deve essere versionato anche
internamente (non sempre git basta), la versione è nel formato \texttt{X.Y.Z} dove
\texttt{X}, \texttt{Y}, \texttt{Z} sono numeri interi e positivi:
\begin{itemize}
    \item il numero al posto di \texttt{X} va incrementato solo se si tratta di una
    modifica/verifica pre-rilascio.
    \item il numero al posto di \texttt{Y} va incrementato solo se la modifica
    interessa la creazione, eliminazione o modifica totale di una sezione.
    \item il numero al posto di \texttt{Z} va incrementato solo se la modifica
    interessa la creazione, eliminazione o modifica totale di una sottosezione
    anche di secondo livello.
\end{itemize}
Ogni modifica che apporta un cambio di versione deve essere accompagnata da una
opportuna nuova riga nel registro delle modifiche.
\\

\noindent
\textbf{Lo scatto di versione avviene solo su modifiche andate a buon fine e verificate.}
\\

\noindent
\textbf{Se la modifica è solo una piccola correzione, specialmente se grammaticale o di
spelling, non va riportata nel registro modifiche.}\\


\noindent
La colonna di \textbf{approvazione} viene aggiornata solo nel momento in cui il documento viene approvato. Naturalmente nei documenti che avranno più revisioni di avanzamento, e che contestualmente richiederanno più versioni, l'approvazione verrà inserita per la fase di avanzamento da superare (che andrà segnalata nella descrizione), e sarà aggiunta una nuova riga di approvazione per il rilascio successivo.

\subsection{Richiedere la verifica di un documento}
La verifica dei documenti deve essere effettuata in uno dei seguenti modi, a seconda della dimensione del documento:
\begin{itemize}
    \item Per richiedere la verifica di un documento di piccole dimensioni (es. verbali) è sufficiente spostare la card di Trello nella lista "To be verified" e togliere la partecipazione (azione eseguita automaticamente), in modo da rendere chiaro agli altri componenti che l'issue ha bisogno di una verifica e tale verifica non è ancora stata presa in carico. Il verificatore, poi, si assegnerà la card e procederà con il lavoro.
    \item Per richiedere la verifica di un documento di grandi dimensioni (es. documenti esterni nei quali è presente una checklist con le sezioni da produrre/modificare) è necessario creare una copia della card e spostare quest'ultima in "To be verified". Questo per permettere una verifica in parallelo alla creazione/modifica dei punti della lista. Man mano che i punti della checklist nella card "In progress" verranno spuntati, si procederà con l'aggiunta di essi nella card "To be verified", in modo da iniziarne la verifica.
\end{itemize}

\subsection{Effettuare la verifica di un documento}
La fase di verifica non è meramente un controllo di spelling e grammatica, esso
deve verificare anche che il contenuto sia completo e coerente al documento.

\subsubsection{Richiedere l'approvazione di un documento}
Per richiedere l'approvazione di un documento è sufficiente spostare la card da "To be verified" a "To be approved" e togliere la partecipazione dei componenti. Automaticamente, verrà assegnata l'etichetta \texttt{To be merged} alla card. Tale etichetta serve a ricordare all'approvatore di eseguire il merge del branch dell'issue, una volta approvato il documento.
Quando questa azione viene eseguita, basterà premere il pulsante "Merged" presente nella card. Tale pulsante, sostituirà la label \texttt{To be merged} con quella \texttt{Merged} e sposterà la card nella lista "Done" togliendo, contemporaneamente, tutti i componenti assegnati alla card.


\subsubsection{Segnalare un problema}
Un errore trovato durante la fase di verifica va segnalato tramite le issue su Github
seguendo le seguenti regole:
\begin{itemize}
    \item Il titolo della issue deve essere
    \texttt{Nome Documento - Descrizione problema}
    \item Nella descrizione va spiegato più approfonditamente la natura del
    problema e eventuali puntatori alle righe interessate.
    \item  Come assignee invece va indicato l'ultimo redattore che ha toccato il file
    interessato.
    \item Fondamentale l'utilizzo delle label per indicare se si tratta
    di un bug o di un miglioramento, e se il documento interessato è un verbale.
\end{itemize}

\subsubsection{Richiedere l'approvazione di un documento}
Per richiedere l'approvazione di un documento è sufficiente spostare la card in "To be approved" e togliere la partecipazione. Automaticamente, verrà assegnata l'etichetta \texttt{To be merged} alla card. Tale etichetta serve a ricordare all'approvatore di eseguire il merge del branch dell'issue, una volta approvato il documento.
Quando questa azione viene eseguita, basterà premere il pulsante "Merged" presente nella card. Tale pulsante, sostituirà la label \texttt{To be merged} con quella \texttt{Merged} e sposterà la card nella lista "Done" togliendo, contemporaneamente, tutti i componenti assegnati alla card.

\subsection{Grafici Use Case}
Per la realizzazione degli Use Case diagram si utilizza \textbf{Draw.io} con le seguenti norme:
\begin{itemize}
\item Carattere del testo: Helvetica 12pt.
\item Nome attore: Utente.
\item Titolo frame: UC\#
\item Titolo Use Case: UC\#, a capo "Nome del caso d'uso".
\item Descrizione Extend with Condition: "Condition: \{descrizione\}" a capo "Extension Point: visualizzazione (descrizione di cosa viene visualizzato)".
\end{itemize}
Le componenti grafiche del diagramma seguono lo standard Use Case di UML analizzato nella lezione P4 del docente Cardin. \\

\noindent Per la descrizione degli Use Case si utilizza sempre lo standard visto nella lezione P4:
\begin{itemize}
	\item \textbf{Attore primario:} Chi è l'attore primario del caso d'uso.
	\item \textbf{Precondizioni:} Condizioni necessarie all'inizio del caso d'uso.
	\item \textbf{Postcondizioni:} Effetti/garanzia di ciò che succede in seguito.
	\item \textbf{Scenario principale:} Sequenza di passi che descrive le interazioni, si utilizza una lista \textit{itemize} in LaTeX per descriverne le fasi.
	\item \textbf{Estensioni:} Descrizione dei possibili ``extend``.
	\item \textbf{Generalizzazioni:} Descrizione di possibili aggiunte o modifiche delle caratteristiche base.
\end{itemize}


