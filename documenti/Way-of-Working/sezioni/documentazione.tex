\section{Documentazione}

Lo strumento scelto dal gruppo per la stesura della documentazione è \LaTeX,
tutta la documentazione deve essere mantenuta all'interno della repo git
\href{https://github.com/Club-Swendwich/Documenti}{hostata su GitHub (LINK)}.
La struttura della repository è la seguente:
\begin{itemize}
    \item \texttt{assets}: Contiene tutti gli elementi grafici comuni ai documenti
        prodotti dal team.
    \item \texttt{verbali}: Contiene i resoconti di tutti gli incontri del team sia
        interni che con entità esterne al team.
    \item \texttt{presentazioni}: Contiene tutte le presentazioni basate su slide.
    \item \texttt{documenti}: Contiene tutti i documenti che non sono verbali o
        presentazioni.
    \item \texttt{template}: Contiene tutti i template da utilizzare quando si va a
        creare un nuovo documento.
\end{itemize}

\subsection{Creazione e Modifica di un documento}

\subsubsection{Creazione}
Quando si decide di creare un nuovo documento è fondamentale utilizzare i template già
disponibili essi sono il miglior modo per mantenere la documentazione uniforme e
risparmiare tempo.
La versione di partenza di un documento è la \texttt{0.1.0}.

\subsubsection{Scelta di un template}

Sono disponibili 3 tipi di template:
\begin{itemize}
    \item \texttt{documenti}: Tutti i documenti \LaTeX che non sono verbali.
    \item \texttt{presentazioni}: Tutte le presentazioni sia esterne che interne.
    \item \texttt{verbali}: Tutti i verbali.
\end{itemize}
Se non è presente il template di cui necessiti prima di prendere una brutta decisione
e potenzialmente perdere tempo parlane con il resto del team.

\subsection{Versionamento}
Ogni documento che deve si sa verrà aggiornato nel tempo deve essere versionato anche
internamente (non sempre git basta), la versione è nel formato \texttt{X.Y.Z} dove
\texttt{X}, \texttt{Y}, \texttt{Z} sono numeri interi e positivi:
\begin{itemize}
    \item il numero al posto di \texttt{X} va incrementato solo se si tratta di una
    modifica/verifica pre rilascio.
    \item il numero al posto di \texttt{Y} va incrementato solo se la modifica
    interessa la creazione, eliminazione o modifica totale di una sezione.
    \item il numero al posto di \texttt{Z} va incrementato solo se la modifica
    interessa la creazione, eliminazione o modifica totale di una sottosezione
    anche di secondo livello.
\end{itemize}
Ogni modifica che apporta un cambio di versione deve essere accompagnata da una
opportuna nuova riga nel registro delle modifiche.

\textbf{Lo scatto di versione avviene solo su modifiche andate a buon fine e verificate}

\textbf{Se la modifica è solo una piccola correzione specialmente se grammaticale o di
spelling non va riportata nel registro modifiche a quello ci pensa già git.}

\subsection{Richiedere la verifica di un documento}
Per richiedere la verifica di un documento è necessario aprire una issue su github con
i seguenti parametri:
\begin{itemize}
    \item Il titolo della issue deve essere
    \texttt{Nome Documento - Richiesta verifica}
    \item Nella descrizione è opportuno spiegare in breve cosa è stato modificato/creato
    così da dare al verificatore contesto.
    \item Fondamentale l'utilizzo della label \texttt{richiesta verifica} ed indicare
    sempre tramite esse se si tratta di un verbale.
\end{itemize}

\subsection{Effettuare la verifica di un documento}
La fase di verifica non è meramente un controllo di spelling e grammatica esso
deve verificare anche che il contenuto sia completo e coerente al documento.

\subsubsection{Segnalare un problema}
Un errore trovato durante la fase di verifica va segnalato tramite le issue su github
seguendo le seguenti regole:
\begin{itemize}
    \item Il titolo della issue deve essere
    \texttt{Nome Documento - Descrizione problema}
    \item Nella descrizione va spiegato più approfonditamente la natura del
    problema e eventuali puntatori alle righe interessate.
    \item  Come assignee invece va indicato l'ultimo redattore che ha toccato il file
    interessato.
    \item Fondamentale l'utilizzo delle label soprattutto indicare se si tratta
    di un bug o di un miglioramento e se il documento interessato è un verbale.
\end{itemize}

