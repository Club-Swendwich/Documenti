\section{Processi di supporto}
\subsection{Gestione documentazione}
Nella seguente sottosezione vengono normati i seguenti aspetti per il 
processo di \textit{Gestione documentazione}:
\begin{itemize}
    \item \textbf{Procedure}
    \item \textbf{Metriche di qualità}
    \item \textbf{Strumenti}
\end{itemize}

\subsubsection{Procedure}
Le attività individuate dallo standard \textit{ISO/IEC 12207:1995} in questo processo sono:
\begin{itemize}
    \item Pianificazione
    \item Progettazione e sviluppo
    \item Manutenzione
\end{itemize}

\paragraph{Pianificazione}
\mbox{}\\
Lo strumento scelto dal gruppo per la stesura della documentazione è \LaTeX.\\
Per permettere un avanzamento \textit{agile} e tracciare lo stato delle attività è stato deciso di utilizzare una Scrum board su Trello.
La struttura della board è la seguente:
\begin{itemdescript}
    \item [To do] Lista delle card relative a issues che devono ancora essere prese in carico. È il backlog della board.
    \item [In progress] Lista delle produzioni in esecuzione al momento. Ogni card corrisponde a un branch.
    \item [To be verified] Lista delle modifiche in attesa di verifica da parte del verificatore.
    \item [To be approved] Lista delle verifiche in attesa di approvazione e merge da parte del verificatore.
    \item [Done] Lista degli issues approvati e risolti.
\end{itemdescript}
È inoltre presente un archivio dove vengono automaticamente spostate le card dopo 15 giorni di permanenza nella lista \texttt{Done}. Questo viene fatto per evitare il sovraffollamento della board.\\


\paragraph{Progettazione e sviluppo}
\subparagraph{Creazione del Documento}
\mbox{}\\
Nel processo di produzione di un nuovo documento è fondamentale utilizzare i templates già
disponibili, essi sono il miglior modo per mantenere la documentazione uniforme e
risparmiare tempo.
La versione di partenza di un documento è la \texttt{0.1.0}.

\subparagraph{Creazione Issue GitHub e gestione del workflow}
\mbox{}\\
Il primo passaggio da effettuare, quando si decide di creare o modificare un documento, è quello dell'apertura dell'issue relativa ad esso.
Tale issue dovrà essere aperta su GitHub nell'apposita sezione "Issues", della repository Documenti e dovrà essere compilata nel seguente modo:
\begin{itemize}
    \item Se il documento dovrà essere creato, il titolo dovrà essere
    \texttt{Nome Documento}
    \item Se, invece, si tratta di una modifica, il titolo dovrà essere
    \texttt{Nome Documento - Modifica da effettuare (in breve)}
    \item Nella descrizione è opportuno spiegare in breve in cosa consisterà tale documento o le modifiche che dovranno essere apportate
    \item È consigliato l'utilizzo della label \texttt{documento}
    \item Se necessario, assegnare l'issue a una milestone fissata
    \item Se la produzione da effettuare è di mole importante, è consigliato suddividere la produzione in punti elenco, tramite checklist, in modo da distribuire e tracciare più facilmente il lavoro
\end{itemize}

\noindent Le issues create in questo modo saranno automaticamente trasformate in \textit{Trello cards} grazie all'integrazione con Zapier. \\

\noindent \textbf{Gestione del Branch:}
\begin{itemize}
    \item Quando le cards vengono create gli viene assegnata in automatico 
    l'etichetta \texttt{To Be Branched}, per ricordare al collaboratore di 
    eseguire il branch.
    \item Prima di procedere con la produzione del documento è fondamentale 
    assegnare sia issue che card ad uno o più componenti che lo hanno preso in carico, 
    ed eseguire il branch per la risoluzione dell'issue.
    \item Il branch deve seguire lo standard \texttt{NomeUtente/Issue\#}, 
    dove "\#" sarà sostituito dal numero della issue.
    \item Una volta eseguita tale operazione, è necessario cambiare l'etichetta 
    della card in \texttt{Branched}, manualmente o tramite il 
    bottone "Branched", che è possibile trovare all'interno della card (è consigliato usare il bottone della card).
\end{itemize}
\noindent A questo punto è possibile procedere con la creazione o modifica del documento.
Per migliorare il workflow è possibile scegliere già verificatori e 
approvatori che, una volta terminate le modifiche, procederanno con la 
verifica e l'approvazione del documento.\\ 
Questi componenti potranno seguire la card (tramite il pulsante apposito), 
in modo da essere notificati su eventuali modifiche della stessa.

\subparagraph{Scelta template}
\mbox{}\\
Sono disponibili 3 tipi di template:
\begin{itemize}
    \item \texttt{documenti}: Tutti i documenti \LaTeX che non sono verbali, interni ed esterni.
    \item \texttt{presentazioni}: Tutte le presentazioni.
    \item \texttt{verbali}: Tutti i verbali.
\end{itemize}
Qualora non fosse presente il template desiderato è necessaria una discussione con il gruppo prima di procedere a crearne di nuovi.

\subparagraph{Redattori, Verificatori, Approvatori}
\mbox{}\\
In ogni documento e verbale in prima pagina viene inserito un riassunto dei ruoli svolti da ogni componente. 
I ruoli vengono svolti in più fasi, e queste sono visibili nel dettaglio nella tabella ``Registro delle modifiche''.\\ 
Può succedere che un componente sia inserito sia come redattore che come verificatore, in quanto può aver svolto il ruolo in fasi differenti del documento.

\subparagraph{Gestione nominativi}
\mbox{}\\
Per dare maggiore coerenza al testo si è deciso di utilizzare le seguenti regole per i nominativi:
\begin{itemize}
	\item \texttt{Cognome Nome} : ordine da seguire nelle tabelle, nel registro delle modifiche, e nelle prime pagine dei documenti in corrispondenza dei ruoli attribuiti.
	\item \texttt{Assenti} : da indicare con "-" nel caso non fossero presenti assenti durante i meeting del gruppo o nelle call.
\end{itemize}

\subparagraph{Glossario}
\mbox{}\\
Per evitare ambiguità con la terminologia utilizzata nei documenti è stato redatto un glossario.
\begin{itemize}
    \item Ogni parola del Glossario è contrassegnata da una "G" ad apice.
    \item Le parole contenute nei titoli non fanno parte del glossario.
\end{itemize}

\subparagraph{Immagini cruscotto PdP}
\mbox{}\\
Per rappresentare i dati nel \textit{preventivo di fase} e nel \textit{consuntivo di periodo}
è importante inserire un cruscotto informativo sotto forma di grafico. Questo va creato con 
\textit{Google Sheets} rispettando le seguenti regole:
\begin{itemize}
    \item \textbf{Grafico a colonne in pila}
    \begin{itemize}
        \item Legenda: posizionata a destra; colore del testo nero; dimensione 16 pt.
        \item Nome asse verticale: Ore.
        \item Colori ruoli: sono mantenuti i colori di default.
        \item Nominativi: disposizione a 30°; carattere 16pt. 
    \end{itemize}
    \item \textbf{Grafico ad anello}
    \begin{itemize}
        \item Legenda: posizionata a destra; colore del testo nero; dimensione 16 pt.
        \item Circonferenza anello: 50\%.
        \item Colori ruoli: sono mantenuti i colori di default.
        \item Inserire la percentuale dentro ogni "fetta"; dimensione carattere 14 pt.
    \end{itemize}
\end{itemize}

\paragraph{Manutenzione}
\mbox{}\\
Tutta la documentazione prodotta deve essere mantenuta all'interno della repo git hostata su GitHub \href{https://github.com/Club-Swendwich/Documenti}{(https://github.com/Club-Swendwich/Documenti)}.
La struttura della repository è la seguente:
\begin{itemize}
    \item \texttt{assets}: Contiene tutti gli elementi grafici comuni ai documenti
        prodotti dal team.
    \item \texttt{documenti}: Contiene tutti i documenti che non sono verbali o
        presentazioni.
    \item \texttt{presentazioni}: Contiene tutte le presentazioni basate su slide.
    \item \texttt{template}: Contiene tutti i template da utilizzare quando si va a
        creare un nuovo documento.
    \item \texttt{verbali}: Contiene i resoconti di tutti gli incontri del team sia
        interni che con entità esterne al team.
\end{itemize}

\subparagraph{Versionamento}
\mbox{}\\
Ogni documento che verrà aggiornato nel tempo deve essere versionato anche
internamente (non sempre git basta), la versione è nel formato \texttt{X.Y.Z} dove
\texttt{X}, \texttt{Y}, \texttt{Z} sono numeri interi e positivi:
\begin{itemize}
    \item il numero al posto di \texttt{X} va incrementato solo se si tratta di una
    modifica/verifica pre-rilascio.
    \item il numero al posto di \texttt{Y} va incrementato solo se la modifica
    interessa la creazione, eliminazione o modifica totale di una sezione.
    \item il numero al posto di \texttt{Z} va incrementato solo se la modifica
    interessa la creazione, eliminazione o modifica totale di una sottosezione
    anche di secondo livello.
\end{itemize}
Ogni modifica che apporta un cambio di versione deve essere accompagnata da una
opportuna nuova riga nel registro delle modifiche.
\\

\noindent
\textbf{Lo scatto di versione avviene solo su modifiche andate a buon fine e verificate.}
\\

\noindent
\textbf{Se la modifica è solo una piccola correzione, specialmente se grammaticale o di
spelling, non va riportata nel registro modifiche.}\\


\noindent
La colonna di \textbf{approvazione} viene aggiornata solo nel momento in cui il documento viene approvato. 
Naturalmente nei documenti che avranno più revisioni di avanzamento, e che contestualmente richiederanno più versioni, 
l'approvazione verrà inserita per la fase di avanzamento da superare (che andrà segnalata nella descrizione), e sarà 
aggiunta una nuova riga di approvazione per il rilascio successivo.

\subsubsection{Metriche di qualità}
\paragraph{Usabilità}
\mbox{}\\
\setlength\extrarowheight{5pt}

\begin{table}[htb]
    \centering
    \begin{longtable}{p{0.17\linewidth} c p{0.35\linewidth} c}
    \rowcolor[RGB]{33, 73, 50}
        \multicolumn{1}{>{\centering\arraybackslash}c}{\textcolor{white}{\textbf{Nome}}} 
        & \multicolumn{1}{>{\centering\arraybackslash}c}{\textcolor{white}{\textbf{Codice}}} 
        & \multicolumn{1}{>{\centering\arraybackslash}c}{\textcolor{white}{\textbf{Descrizione}}}
		& \multicolumn{1}{>{\centering\arraybackslash}c}{\textcolor{white}{\textbf{Formula}}}\\[4pt]
    \rowcolor[RGB]{216, 235, 171}
        &   & Indice di leggibilità di un documento &   \\
    \rowcolor[RGB]{216, 235, 171}
        \textbf{Indice di Gulpease} & MPD1 &  \textbf{$N_F$ numero frasi $N_L$ numero lettere} &   $ 89 + \frac{300 \cdot N_F - 10 \cdot N_L}{N_P} $ \\
    \rowcolor[RGB]{216, 235, 171}
        &   &  \textbf{$N_P$ numero parole} &   \\[4pt]
    \rowcolor[RGB]{216, 235, 171}
        &   &  &   \\
    \rowcolor[RGB]{216, 235, 171}
        \textbf{Ortogtrafia} & MPD2 & Correttezza ortografica di un documento  &   - \\
    \rowcolor[RGB]{216, 235, 171}
        &   &  &   \\[4pt]
    
    \end{longtable}
    \caption{Metriche di usabilità}
\end{table}
    
\setlength\extrarowheight{0pt}
\subsubsection{Strumenti}
Gli strumenti da utilizzare in questo processo sono:
\begin{itemize}
    \item LaTeX: linguaggio per la stesura dei documenti.\\
    https://www.latex-project.org/
    \item Visual Studio Code: editor di codice utilizzato dal gruppo, scelto per l'implementazione con Github(in particolare
    per la gestione dei branch).\\
    https://code.visualstudio.com/
    \item Draw.io: per la creazione di diagrammi UML.\\
    https://app.diagrams.net/
    \item Google Sheets: per la creazione di tabelle e grafici.\\
    https://www.google.com/sheets/about/
\end{itemize}
%-------------------------------------------------------------------------------------------------

\subsection{Assicurazione della qualità}
Nella seguente sottosezione vengono normati i seguenti aspetti per il 
processo di \textit{Assicurazione della qualità}:
\begin{itemize}
    \item \textbf{Procedure}
\end{itemize}

\subsubsection{Procedure}
Le attività individuate dallo standard \textit{ISO/IEC 12207:1995} in questo processo sono:
\begin{itemize}
    \item Accertamento di Prodotto
    \item Accertamento di Processo
\end{itemize}

\paragraph{Accertamento di Prodotto}
\begin{center}
    \textbf{M[PD][ID numerico]}
\end{center}
Dove \textbf{PD} è abbreviazione di \textbf{Prodotto}

\paragraph{Accertamento di Processo}
\begin{center}
    \textbf{M[PC][ID numerico]}
\end{center}
Dove \textbf{PD} è abbreviazione di \textbf{Processo}

%--------------------------------------------------------------------------------------

\subsection{Verifica}
\label{sec:Verifica}
Nella seguente sottosezione vengono normati i seguenti aspetti per il 
processo di \textit{Verifica}:
\begin{itemize}
    \item \textbf{Procedure}
    \item \textbf{Metriche di qualità}
    \item \textbf{Strumenti}
\end{itemize}

\subsubsection{Procedure}
Le attività individuate dallo standard \textit{ISO/IEC 12207:1995} in questo processo sono:
\begin{itemize}
    \item Analisi
    \item Revisione
    \item Test
\end{itemize}

\paragraph{Analisi}
\subparagraph{Verificare un documento}
\mbox{}\\
La fase di verifica non è meramente un controllo di errori grammaticali o di battitura. È necessario, infatti, verificare anche la completezza e la coerenza del contenuto all'interno del documento.
I compiti che dovrà eseguire il verificatore saranno, quindi:
\begin{enumdescript}
    \item [Controllo grammaticale] Controllo morfologico, ortografico, sintattico e semantico del testo prodotto.
    \item [Controllo della coerenza delle parti del documento] Controllo della coerenza dei contenuti delle sezioni di testo aggiunte. Esse non devono essere in conflitto tra loro o con il testo già esistente. È necessario, inoltre, controllare la coerenza e continuità della suddivisione delle sezioni del documento.
    \item [Controllo dell'aderenza alle Norme di Progetto] Controllo dell'ottemperanza alle buone pratiche descritte nelle Norme di Progetto durante la produzione del documento.
    \item [Controllo del registro delle modifiche e delle informazioni del documento] Controllo dello stato di aggiornamento e coerenza del registro delle modifiche e delle informazioni del documento presenti in copertina.
    \item [Controllo dell'impaginazione] Controllo della corretta impaginazione del documento. In particolare, è necessario controllare gli allineamenti dei paragrafi, la disposizione delle immagini rispetto al testo, la posizione delle didascalie e il funzionamento di eventuali link.
\end{enumdescript}
Una volta che il documento supererà con successo questi passaggi, il verificatore aggiornerà lo stato del documento (\textbackslash docstatus) e procederà con la richiesta di approvazione.

\setlength\extrarowheight{5pt}

    \begin{longtable}{cm{0.28\linewidth}m{0.13\linewidth}m{0.31\linewidth}}
    \rowcolor[RGB]{33, 73, 50}
    \multicolumn{1}{>{\centering\arraybackslash}c}{\textcolor{white}{\textbf{Metodo}}} 
        & \multicolumn{1}{>{\centering\arraybackslash}c}{\textcolor{white}{\textbf{Obiettivo}}} 
        & \multicolumn{1}{>{\centering\arraybackslash}c}{\textcolor{white}{\textbf{Attori}}}
		& \multicolumn{1}{>{\centering\arraybackslash}c}{\textcolor{white}{\textbf{Descrizione}}}\\[4pt]
		\rowcolor[RGB]{216, 235, 171}
	    	\textbf{Walkthrough} 
            & Rilevamento di errori mediante letture ad ampio spettro.
            & Redattori Verificatori
            & Un errore trovato durante la fase di verifica va segnalato 
            tramite le issue su Github, come indicato nella sezione \textit{Segnalare un problema}.\\[4pt]
        \rowcolor[RGB]{233, 245, 206}
            \textbf{Inspection} 
            & Rilevamento di errori mediante letture mirate. 
            & Redattori Verificatori
            & Il verificatore utilizza una \textit{lista di controllo}, per eseguire una verifica mirata.\\[4pt]
        \caption{Tipologie di verifica di un documento}
        \end{longtable}
    
\setlength\extrarowheight{0pt}
\subparagraph{Verificare il codice}
\mbox{}\\
\setlength\extrarowheight{5pt}

\begin{table}[htb]
    \centering
    \begin{tabular}{cm{0.30\linewidth}m{0.17\linewidth}m{0.30\linewidth}}
    \rowcolor[RGB]{33, 73, 50}
    \multicolumn{1}{>{\centering\arraybackslash}c|}{\textcolor{white}{\textbf{Analisi}}} 
        & \multicolumn{1}{>{\centering\arraybackslash}c}{\textcolor{white}{\textbf{Obiettivo}}} 
        & \multicolumn{1}{>{\centering\arraybackslash}c}{\textcolor{white}{\textbf{Attori}}}
		& \multicolumn{1}{>{\centering\arraybackslash}c}{\textcolor{white}{\textbf{Descrizione}}}\\[4pt]
		
		\rowcolor[RGB]{216, 235, 171}
	    	\textbf{Statica} 
            & Verificare il rispetto delle norme dello standard di codifica.
            & \centering Programmatori Verificatori
            & Analisi di flusso dei dati, verifica formale del codice.\\[4pt]
        \rowcolor[RGB]{233, 245, 206}
            \textbf{Dinamica} 
            & Eseguire il prodotto e trovare errori \textit{Run-time}. 
            & \centering Programmatori Verificatori
            & Test di unità, di integrazione, di sistema, di regressione, di accettazione.\\[4pt]
        \end{tabular}
        \caption{Tipologie di verifica del codice}
    \end{table}
    
\setlength\extrarowheight{0pt}

\paragraph{Revisione}
\subparagraph{Segnalare un problema}
\mbox{}\\
Un errore trovato durante la fase di verifica va segnalato tramite le issue su Github
seguendo le seguenti regole:
\begin{itemize}
    \item Il titolo della issue deve essere
    \texttt{Nome Documento - Descrizione problema}
    \item Nella descrizione va spiegato più approfonditamente la natura del
    problema e eventuali puntatori alle righe interessate.
    \item  Come assignee invece va indicato l'ultimo redattore che ha toccato il file
    interessato.
    \item Fondamentale l'utilizzo delle label per indicare se si tratta
    di un bug o di un miglioramento, e se il documento interessato è un verbale.
\end{itemize}

\paragraph{Test}
Tutti i test eseguiti al codice devono essere ripetibili ed automatici (per quanto possibile).
SCRIVERE I CODICI E LE PROCEDURE DEI TEST EFFETTUATI

\subsubsection{Metriche di qualità}

Parametri utilizzati nella tabella:
\begin{itemize}
    \item \textbf{Budget at Completion (BAC)}:Valore iniziale previsto per la realizzazione del progetto.
    \item \textbf{Planned Value (BCWS)}: Costo pianificato per realizzare le attività di progetto alla data corrente (giorni o euro).
    \item \textbf{Estimate to Complete (ETC)}: Valore stimato per la realizzazione delle rimanenti attività necessarie al completamento del progetto.
    \item \textbf{Number of Changed (NoC)}: numero di requisiti variati.
    \item \textbf{Number of Deleted (NoD)}: numero di requisiti eliminati.
    \item \textbf{Number of Added (NoA)}: numero di requisiti aggiunti.
    \item \textbf{Total Number of Initial Requirement (TNIR)}: numero totale dei requisiti iniziali.
    \item \textbf{Number of Satisfied (NoS)}: numero totale di requisiti soddisfatti.
    \item \textbf{Total number of Requirement (TnR)}: numero totale di requisiti.
    \item \textbf{Line of Code Executed (LCE)}: linee di codice eseguite dagli algoritmi di test.
    \item \textbf{Line of Code(LC)}: linee di codice totali.
    \item \textbf{Number of Quality Metrics Satisfied (NQMS)}: numero di metriche di qualità soddisfatte.
    \item \textbf{Total number of Quality Metrics (TQM)}: numero totale di metriche di qualità.
\end{itemize}

\setlength\extrarowheight{5pt}

\begin{longtable}{P{0.1\linewidth}P{0.25\linewidth}P{0.25\linewidth}P{0.30\linewidth}}
    \rowcolor[RGB]{33, 73, 50}
        \textcolor{white}{\textbf{Codice}}
        & \textcolor{white}{\textbf{Nome}}
        & \textcolor{white}{\textbf{Descrizione}}
		& \textcolor{white}{\textbf{Formula}}\\[4pt]
	\rowcolor[RGB]{216, 235, 171}
	    MPC1 
        & Schedule Variance (SV) 
        & Indica se il gruppo è in linea (ritardo o anticipo) rispetto alla schedulazione delle attività di progetto pianificate.
        & $SV = (\frac{BCWP}{BCWS} - 1) \cdot 100$ \\[4pt]
        \rowcolor[RGB]{233, 245, 206}
            MPC2 
            & Budget Variance (BV) 
            & Indica se, alla data corrente, il gruppo ha speso di più o di meno rispetto a quanto previsto dal budget.
            & $BV = (1 - \frac{ACWP}{BCWS}) \cdot 100$	\\[4pt]
        \rowcolor[RGB]{216, 235, 171}
	    	MPC3 
            & Estimated at Completion (EAC) 
            & Corrisponde al BAC rivisto allo stato corrente del progetto.
            & $EAC = ACWP + ETC$	\\[4pt]
        \rowcolor[RGB]{233, 245, 206}
            MPC4 
            & Actual Cost of Work Performed (ACWP) 
            & Costo effettivamente sostenuto dal gruppo alla data corrente (giorni o euro).
            & -		\\[4pt]
        \rowcolor[RGB]{216, 235, 171}
	    	MPC5 
            & Budgeted Cost of Work Performed (BCWP) 
            & Valore delle attività realizzate alla data corrente (giorni o euro). 
            & $\%completamento \cdot EAC$		\\[4pt]
        \rowcolor[RGB]{233, 245, 206}
            MPC6 
            & Requirements Stability Index (RSI) 
            & Metrica per tracciare le variazioni dei requisiti nel tempo originariamente specificati per il progetto.  
            & $(1-\frac{NoC + NoD + NoA}{TNIR}) \cdot 100$	\\[4pt]
        \rowcolor[RGB]{216, 235, 171}
	    	MPC7 
            & Satisfied Obligatory Requirements (SOR) 
            & Indica la percentuale dei requisiti obbligatori soddisfatti. 
            & $\frac{NoS}{TnR} \cdot 100$		\\[4pt]
        \rowcolor[RGB]{233, 245, 206}
            MPC8 
            & Non-calculated Risk (NCR) 
            & Indica il numero di rischi verificati non preventivati. 
            & Numero intero		\\[4pt]
        \rowcolor[RGB]{216, 235, 171}
	    	MPC9 & Code Coverage (CC) 
            & Indica il grado di copertura del codice dai test dinamici.
            & $\frac{LCE}{LC} \cdot 100$		\\[4pt]
        \rowcolor[RGB]{233, 245, 206}
            MPC10 
            & Passed Test Cases Percentage (PTCP) 
            & Corrisponde alla percentuale di test superati rispetto al totale dei test dinamici.
            & $\frac{\textit{Test Passati}}{\textit{Test Totali}} \cdot 100$	\\[4pt]
        \rowcolor[RGB]{216, 235, 171}
	    	MPC11 
            & Failed Test Cases Percentage (FTCP) 
            & Corrisponde alla percentuale di test falliti rispetto al totale dei test dinamici.
            & $\frac{\textit{Test Falliti}}{\textit{Test Totali}} \cdot 100$ \\[4pt]
        \rowcolor[RGB]{233, 245, 206}
            MPC12 
            & Quality Metrics Satisfied (QMS) 
            & Indica la percentuale di metriche di qualità soddisfatte
            & $\frac{NQMS}{TQM} \cdot 100$		\\[4pt]
    \caption{Metriche di processo}
\end{longtable}
\setlength\extrarowheight{0pt}

\subsubsection{Strumenti}
 INSERIRE STRUMENTI DEI TEST