\section{Analisi dei Rischi}
Questa attività richiede una attenta analisi di tutte le parti della creazione del progetto,
per poi classificare i possibili rischi in base alla loro entità e elencando possibili soluzioni o 
metodi per mitigarne gli effetti.
Segue una tabella realizata con in mente le fasi di:
\begin{itemize}
    \item \textbf{Identificazione}: Fase in cui si possono identificare possibile problematiche che possono comportare rischi immediati o futuri.
    \item \textbf{Analisi}: Fase di studio degli errori e valutazione della loro gravità e conseguente impatto sul progetto.
    \item \textbf{Controllo}: La metodologia con cui si evita la verificazione dei rischi precedentemente individuati
    \item \textbf{Monitoraggio}: Durante il progetto viene continuamente eseguita una: 
            \begin{itemize}
                \item Rilevazione di nuovi tipi di rischio
                \item Aggiornamento dei rischi già rilevati
            \end{itemize}
\end{itemize}
I rischi sono raggruppati in diverse categorie:
\begin{itemize}
    \item \textbf{Tecnologie scelte}: Rappresentati con la lettera \textbf{T}
    \item \textbf{Rapporti interpersonali}: Rappresentati con la lettera \textbf{I}
    \item \textbf{Tecnologie scelte}: Organizzazione del lavoro \textbf{O} 
\end{itemize}
I rischi vengono quindi classificati con la formula:
\begin{center}
    \textbf{R\{Iniziale categoria\}\{Numero progressivo\}}
\end{center} 
\newpage
\subsection{Rischi relativi alle tecnologie}
{\renewcommand{\arraystretch}{1.5} \small
\begin{tabular}{p{0.15\linewidth}p{0.30\linewidth}p{0.30\linewidth}p{0.15\linewidth}}
	\rowcolor[RGB]{33, 73, 50}
	\textcolor{white}{\textbf{Codice}} & \textcolor{white}{\textbf{Descrizione}} & \textcolor{white}
	{\textbf{Identificazione}} & \textcolor{white}{\textbf{Grado}}\\

	\rowcolor[RGB]{216, 235, 171}
	\textbf{RT1} \newline Competenze 
    & 
    I membri del gruppo hanno una scarsa esperienza del campo del software developement e non hanno l'esperienza
    necessaria per affrontarlo con confidenza 
    & 
    Ciascun membro deve comunicare le difficoltà che riscontrano, nel caso che un collega sia più competente nel campo
    & 
    Pericolosità: \newline
    \textbf{Elevata} \newline
    Occorrenza: \newline
    \textbf{Elevata} 
    \\
    \rowcolor[RGB]{233, 245, 206} %Qui sotto è un macello, qualcuno che sa quello che fa ripari la multicolumn con il width sbagliato
	\multicolumn{4}{p{0.98\linewidth}}{\textbf{Piano di contingenza}: I punti qualdove si abbia difficoltà vengono colmati da compagni che
    potrebbero averne una migliore comprensione.}	
    \\

    \rowcolor[RGB]{216, 235, 171}
	\textbf{RT2} \newline Utilizzo di tecnologie
    & 
    I membri del gruppo hanno una scarsa esperienza nell'utilizzo delle librerie fornite e delle metodologie imposte
    & 
    Ciascun membro deve comunicare le difficoltà che riscontrano, nel caso che un collega sia più competente nel campo
    & 
    Pericolosità: \newline
    \textbf{Elevata} \newline
    Occorrenza: \newline
    \textbf{Elevata} 
    \\
    \rowcolor[RGB]{233, 245, 206} %Qui sotto è un macello, qualcuno che sa quello che fa ripari la multicolumn con il width sbagliato
	\multicolumn{4}{p{0.98\linewidth}}{\textbf{Piano di contingenza}: I punti qualdove si abbia difficoltà vengono colmati da compagni che
    potrebbero averne una migliore comprensione.}	
    \\
\end{tabular}	
\newpage
\subsection{Rischi relativi ai rapporti interpersonali}