
\section{Analisi dei Rischi}
Di seguito sono riportate le tabelle relative ai potenziali rischi che potrebbero occorrere durante lo sviluppo
del progetto. Per il codice utilizzato si rimanda al documento \textit{Norme di progetto}.
\subsection{Rischi relativi alle tecnologie di lavoro e di produzione di software: "T"}

{\renewcommand{\arraystretch}{1.5} \small
\begin{tabular}{ >{\centering}p{0.20\linewidth} | >{\centering}p{0.28\linewidth} | >{\centering}p{0.28\linewidth} | >{\centering}p{0.13\linewidth} }
	\rowcolor[RGB]{33, 73, 50}
	\textcolor{white}{\textbf{Codice}} & \textcolor{white}
	{\textbf{Descrizione}} & \textcolor{white}{\textbf{Identificazione}} & \textcolor{white}{\textbf{Grado}}\tabularnewline
	\rowcolor[RGB]{216, 235, 171}
	RT1 \par Apprendimento di nuove tecnologie 
    & Le tecnologie utilizzate per lo sviluppo del progetto dispongono di una vasta documentazione.
    L'apprendimento delle nozioni interessate può rivelarsi complicato e causare ritardi nel processo di sviluppo. 
    & Ogni membro del gruppo deve evidenziare con trasparenza le possibili difficoltà incontrate. 
    & Pericolosità: \par \textbf{Elevata} \par Occorrenza: \par \textbf{Media}\tabularnewline
	\rowcolor[RGB]{233, 245, 206}
    \multicolumn{4}{p{0.9718\linewidth}}{\textbf{Piano di Contingenza:} In caso di assigment troppo complessi per il singolo verrà eseguita una redistribuzione del carico di lavoro. } \tabularnewline
	\rowcolor[RGB]{216, 235, 171}
    RT2 \par Scarsa esperienza 
	& Il gruppo non ha mai affrontato esperienze di simile complessità, questo può portare a rallentamenti e all'insorgere di problemi operativi.
    & Il Responsabile ha il compito di controllare che lo sviluppo avvenga senza rallentamenti, analizzando con i membri del gruppo interessati la sorgente del problema.
    & Pericolosità: \par \textbf{Elevata} \par Occorrenza: \par \textbf{Elevata}\tabularnewline
    \rowcolor[RGB]{233, 245, 206}
    \multicolumn{4}{p{0.9718\linewidth}}{\textbf{Piano di Contingenza:} Nei casi di difficoltà maggiore, il lavoro verrà attribuito a più componenti del gruppo per favorire la collaborazione e verranno fissati degli incontri al fine di chiarire le incertezze.  } \tabularnewline
    \rowcolor[RGB]{216, 235, 171}
    RT3 \par Strumenti Software 
	& Il malfunzionamento degli strumenti software adottati dal gruppo può causare rallentamenti o perdita di dati.
    & Chiunque noti qualunque tipo di anomalia ha il compito di avvisare il resto del team. 
    & Pericolosità: \par \textbf{Media} \par Occorrenza: \par \textbf{Bassa}\tabularnewline
    \rowcolor[RGB]{233, 245, 206}
    \multicolumn{4}{p{0.9718\linewidth}}{\textbf{Piano di Contingenza:} Utilizzare frequentemente strumenti di backup per salvaguardare i progressi durante lo sviluppo del progetto.  } \tabularnewline
    \rowcolor[RGB]{216, 235, 171}
    RT4 \par Guasto hardware
	& In caso di problemi legati ai dispositivi personali dei componenti del gruppo, si possono manifestare disagi e non operatività.
    & È compito di ogni elemento del team di avvisare il Responsabile riguardo i problemi riscontrati.
    & Pericolosità: \par \textbf{Bassa} \par Occorrenza: \par \textbf{Bassa}\tabularnewline
    \rowcolor[RGB]{233, 245, 206}
    \multicolumn{4}{p{0.9718\linewidth}}{\textbf{Piano di Contingenza:} Il team deve rispettare l'utilizzo degli strumenti prestabiliti per ridurre al minimo la possibilità di perdere dati.    } \tabularnewline
	
\end{tabular}	
}

\subsection{Rischi relativi ai rapporti interpersonali: "I"}

{\renewcommand{\arraystretch}{1.5}
\begin{longtable}{ >{\centering}p{0.20\linewidth} | >{\centering}p{0.28\linewidth} | >{\centering}p{0.28\linewidth} | >{\centering}p{0.13\linewidth} }
	\rowcolor[RGB]{33, 73, 50}
	\textcolor{white}{\textbf{Codice}} & \textcolor{white}
	{\textbf{Descrizione}} & \textcolor{white}{\textbf{Identificazione}} & \textcolor{white}{\textbf{Grado}}\tabularnewline
	\rowcolor[RGB]{216, 235, 171}
	RI1 \par Conflitti decisionali
	& I componenti del gruppo potrebbero non concordare in alcune situazioni causando discussioni che riducono l'operatività.
	& In caso di dispute è necessario confrontarsi insieme al Responsabile per trovare la soluzione più adeguata.
	& Pericolosità: \par \textbf{Bassa} \par Occorrenza: \par \textbf{Media}\tabularnewline
	\rowcolor[RGB]{233, 245, 206}
	\multicolumn{4}{p{0.9718\linewidth}}{\textbf{Piano di Contingenza:}È fondamentale discutere e valutare delle opzioni proposte basandosi esclusivamente su ciò che risulta più adeguato al fine del progetto.  } \tabularnewline
	\rowcolor[RGB]{216, 235, 171}
	RI2 \par Comunicazione on-line
	& Il gruppo si trova a lavorare per lo più da remoto, questo rallenta la comunicazione interna e con il proponente, creando ripercussioni sulla produttività. 
	& È responsabilità di ogni membro del team mostrarsi quanto più reperibile ed utilizzare i canali di comunicazione stabiliti.
	& Pericolosità: \par \textbf{Media} \par Occorrenza: \par \textbf{Media}\tabularnewline
	\rowcolor[RGB]{233, 245, 206}
	\multicolumn{4}{p{0.9718\linewidth}}{\textbf{Piano di Contingenza:}Il team ha disposto diversi canali per comunicare sia internamente che con il proponente. In paricolare risulta fondamentale l'utilizzo di una piattaforma di messaggistica istantanea per comunicare all'interno del gruppo.} \tabularnewline
	\rowcolor[RGB]{216, 235, 171}
	RI3 \par Reperibilità interna
	& A causa di motivazioni personali e/o accademiche alcuni componenti del gruppo potrebbero risulare non reperibili, causando ritardi nella comunicazione relativa allo sviluppo del progetto.
	& Ogni elemento del gruppo deve avvisare con anticipo eventuali irreperibilità attraverso i canali di comunicazioni stabiliti.
	& Pericolosità: \par \textbf{Media} \par Occorrenza: \par \textbf{Media}\tabularnewline
	\rowcolor[RGB]{233, 245, 206}
	\multicolumn{4}{p{0.9718\linewidth}}{\textbf{Piano di Contingenza:}Il gruppo ha deciso di fissare incontri regolari per organizzare ed analizzare direttamente l'avanzamento del progetto. In oltre ogni membro del gruppo ha il dovere di avvisare con ragionevole anticipo gli impegni che possono inttaccare il regolare svolgimento delle attività.} \tabularnewline
	\rowcolor[RGB]{216, 235, 171}
	RI4 \par Reperibilità esterna
	& Le comunicazioni con il proponente potrebbero rivelarsi poco frequenti in caso di poca reperibilità o strumenti inadeguati, rallentando la produttività.
	& Il gruppo e il proponente devono comunicare possibili indisponibilità e preferenze relative algi strumenti da adottare, per rendere il rapporto più funzionale e di qualità. 
	& Pericolosità: \par \textbf{Media} \par Occorrenza: \par \textbf{Bassa}\tabularnewline
	\rowcolor[RGB]{233, 245, 206}
	\multicolumn{4}{p{0.9718\linewidth}}{\textbf{Piano di Contingenza:} Concordare tra gruppo e proponente gli strumenti più opportuni da utilizzare, ed organizzare con cadenza regolare incontri per esporre l'avanzamento del progetto e riportare eventuali dubbi.} \tabularnewline
	
\end{longtable}	
}

\subsection{Rischi relativi all'organizzazione del lavoro: "O"}

{\renewcommand{\arraystretch}{1.5}
\begin{tabular}{  >{\centering}p{0.20\linewidth} | >{\centering}p{0.28\linewidth} | >{\centering}p{0.28\linewidth} | >{\centering}p{0.13\linewidth}  }
	\rowcolor[RGB]{33, 73, 50}
	\textcolor{white}{\textbf{Codice}} & \textcolor{white}
	{\textbf{Descrizione}} & \textcolor{white}{\textbf{Identificazione}} & \textcolor{white}{\textbf{Grado}}\tabularnewline
	\rowcolor[RGB]{216, 235, 171}
	RO1 \par Impegni/ \par Indisponibilità 
	& Alcuni membri del gruppo potrebbero doversi esentare dal lavoro per molteplici motivazioni, causando problemi organizzativi e di sviluppo.
	& È fondamentale che ogni membro del gruppo avvisi per tempo il possibile impegno in modo da favorire l'organizzazione delle attività.
	& Pericolosità: \par \textbf{Bassa} \par Occorrenza: \par \textbf{Medio-Bassa}\tabularnewline
	\rowcolor[RGB]{233, 245, 206}
	\multicolumn{4}{p{0.9718\linewidth}}{\textbf{Piano di Contingenza:} Avvisare con anticipo il team dell'assenza e riportarla su un calendario accessibile a tutto il gruppo. } \tabularnewline
	\rowcolor[RGB]{216, 235, 171}
	RO2 \par Alterazione dei requisiti
	& Nel caso in cui l'azienda decidesse di modificare i requisiti in corso d'opera, lo sviluppo del software subirebbe problemi organizzativi e temporali. 
	& Il rapporto tra proponente e team deve risultare attivo in modo da percepire le intenzioni riguardo il prodotto finale. 
	& Pericolosità: \par \textbf{Elevata} \par Occorrenza: \par \textbf{Bassa}\tabularnewline
	\rowcolor[RGB]{233, 245, 206}
	\multicolumn{4}{p{0.9718\linewidth}}{\textbf{Piano di Contingenza:} Si veda \textbf{RI4}} \tabularnewline

	
\end{tabular}	
}

\subsection{Rischi relativi ai costi e ai tempi: "C"}

{\renewcommand{\arraystretch}{1.5}
\begin{tabular}{  >{\centering}p{0.20\linewidth} | >{\centering}p{0.28\linewidth} | >{\centering}p{0.28\linewidth} | >{\centering}p{0.13\linewidth}  }
	\rowcolor[RGB]{33, 73, 50}
	\textcolor{white}{\textbf{Codice}} & \textcolor{white}
	{\textbf{Descrizione}} & \textcolor{white}{\textbf{Identificazione}} & \textcolor{white}{\textbf{Grado}}\tabularnewline
	\rowcolor[RGB]{216, 235, 171}
	RC1 \par Costi previsti
	& I costi calcolati preventivamente dal gruppo potrebbero non risultare idonei allo sviluppo del progetto a causa di una mancaza di esperienza.
	& Ogni elemento del gruppo deve monitorare il proprio tempo speso tra ore di studio personale e di lavoro.
	& Pericolosità: \par \textbf{Elevata} \par Occorrenza: \par \textbf{Media}\tabularnewline
	\rowcolor[RGB]{233, 245, 206}
	\multicolumn{4}{p{0.9718\linewidth}}{\textbf{Piano di Contingenza:} Aggiornare con costanza il calcolo della quantità di lavoro svolto, collezionando le informazioni riportate da ogni componente del team.} \tabularnewline
	\rowcolor[RGB]{216, 235, 171}
	RC2 \par Tempi previsti
	& A causa dell'inesperienza del gruppo, lo sviluppo del progetto potrebbe vedere non rispettate le scadenze prefissate. Il calcolo delle risorse da impiegare potrebbe essere modificato più volte.  
	& È compito di ciascun componente di avvisare il team in caso di ritardi e/o difficoltà riguardo l'assigment di cui incaricato.
	& Pericolosità: \par \textbf{Elevata} \par Occorrenza: \par \textbf{Media}\tabularnewline
	\rowcolor[RGB]{233, 245, 206}
	\multicolumn{4}{p{0.9718\linewidth}}{\textbf{Piano di Contingenza:} Il Responsabile ha il dovere di assegnare nuove risorse per concludere le attività, nel caso in cui si presentino problematiche.} \tabularnewline

\end{tabular}	
}

