
\section{Analisi dei Rischi}
Di seguito sono riportate le tabelle relative ai potenziali rischi che potrebbero occorrere durante lo sviluppo
del progetto. I rischi sono suddivisi per categoria a seconda della tipologia del problema che si può verificare e
sono indicati tramite codici univoci così generati:

\begin{center}
    \textbf{R[Categoria][Numero progressivo]}
\end{center}

I rischi rientrano nelle seguenti categorie:
\begin{itemize}
    \item \textbf{Tecnologie di lavoro e di produzione software:} T;
    \item \textbf{Rapporti interpersonali:} I;
    \item \textbf{Organizzazione del lavoro:} O;
    \item \textbf{Costi e tempi:} C;
\end{itemize}

\subsection{Rischi relativi alle tecnologie di lavoro e di produzione di software: "T"}

{\renewcommand{\arraystretch}{1.5} \small
\begin{tabular}{ >{\centering}p{0.20\linewidth} | >{\centering}p{0.28\linewidth} | >{\centering}p{0.28\linewidth} | >{\centering}p{0.13\linewidth} }
	\rowcolor[RGB]{33, 73, 50}
	\textcolor{white}{\textbf{Codice}} & \textcolor{white}
	{\textbf{Descrizione}} & \textcolor{white}{\textbf{Identificazione}} & \textcolor{white}{\textbf{Grado}}\tabularnewline
	\rowcolor[RGB]{216, 235, 171}
	RT1 \par Apprendimento di nuove tecnologie 
    & Le tecnologie utilizzate per lo sviluppo del progetto dispongono di una vasta documentazione.
    L'apprendimento delle nozioni interessate può rivelarsi complicato e causare ritardi nel processo di sviluppo. 
    & Ogni membro del gruppo deve evidenziare con trasparenza le possibili difficoltà incontrate. 
    & Pericolosità: \par \textbf{Elevata} \par Occorrenza: \par \textbf{Media}\tabularnewline
	\rowcolor[RGB]{233, 245, 206}
    \multicolumn{4}{p{0.9718\linewidth}}{\textbf{Piano di Contingenza:} In caso di assigment troppo complessi per il singolo, verrà eseguita una redistribuzione del carico di lavoro. } \tabularnewline
	\rowcolor[RGB]{216, 235, 171}
    RT2 \par Scarsa esperienza 
	& Il gruppo non ha mai affrontato esperienze di simile complessità, questo può portare a rallentamenti e all'insorgere di problemi operativi.
    & Il Responsabile ha il compito di controllare che lo sviluppo avvenga senza rallentamenti, analizzando con i membri del gruppo interessati la sorgente del problema.
    & Pericolosità: \par \textbf{Elevata} \par Occorrenza: \par \textbf{Elevata}\tabularnewline
    \rowcolor[RGB]{233, 245, 206}
    \multicolumn{4}{p{0.9718\linewidth}}{\textbf{Piano di Contingenza:} Nei casi di difficoltà maggiore, il lavoro verrà attribuito a più componenti del gruppo per favorire la collaborazione.  } \tabularnewline
    \rowcolor[RGB]{216, 235, 171}
    RT3 \par Strumenti Software 
	& Il malfunzionamento degli strumenti software adottati dal gruppo possono causare rallentamenti o perdita di dati.
    & Chiunque noti qualunque tipo di anomalia ha il compito di avvisare il resto del team. 
    & Pericolosità: \par \textbf{Media} \par Occorrenza: \par \textbf{Bassa}\tabularnewline
    \rowcolor[RGB]{233, 245, 206}
    \multicolumn{4}{p{0.9718\linewidth}}{\textbf{Piano di Contingenza:} Utilizzare frequentemente strumenti di backup per salvaguardare i progressi durante lo sviluppo del progetto.  } \tabularnewline
    \rowcolor[RGB]{216, 235, 171}
    RT4 \par Guasto hardware
	& In caso di problemi legati ai dispositivi personali dei componenti del gruppo, si possono manifestare disagi e non operatività.
    & È compito di ogni elemento del team di avvisare il Responsabile riguardo i problemi riscontrati.
    & Pericolosità: \par \textbf{Bassa} \par Occorrenza: \par \textbf{Bassa}\tabularnewline
    \rowcolor[RGB]{233, 245, 206}
    \multicolumn{4}{p{0.9718\linewidth}}{\textbf{Piano di Contingenza:} Il team deve rispettare l'utilizzo degli strumenti prestabiliti per ridurre al minimo la possibilità di perdere dati.    } \tabularnewline
	
\end{tabular}	
}

\subsection{Rischi relativi ai rapporti interpersonali: "I"}

{\renewcommand{\arraystretch}{1.5}
\begin{tabular}{ >{\centering}p{0.20\linewidth} | >{\centering}p{0.28\linewidth} | >{\centering}p{0.28\linewidth} | >{\centering}p{0.13\linewidth} }
	\rowcolor[RGB]{33, 73, 50}
	\textcolor{white}{\textbf{Codice}} & \textcolor{white}
	{\textbf{Descrizione}} & \textcolor{white}{\textbf{Identificazione}} & \textcolor{white}{\textbf{Grado}}\tabularnewline
	\rowcolor[RGB]{216, 235, 171}
	RX1 & Descrizione & Identificazione & Grado\tabularnewline
	\rowcolor[RGB]{233, 245, 206}
	RX2 & Descrizione & Identificazione & Grado\tabularnewline
	\rowcolor[RGB]{216, 235, 171}
	RX3 & Descrizione & Identificazione & Grado\tabularnewline
	
\end{tabular}	
}

\subsection{Rischi relativi all'organizzazione del lavoro: "O"}

{\renewcommand{\arraystretch}{1.5}
\begin{tabular}{  >{\centering}p{0.20\linewidth} | >{\centering}p{0.28\linewidth} | >{\centering}p{0.28\linewidth} | >{\centering}p{0.13\linewidth}  }
	\rowcolor[RGB]{33, 73, 50}
	\textcolor{white}{\textbf{Codice}} & \textcolor{white}
	{\textbf{Descrizione}} & \textcolor{white}{\textbf{Identificazione}} & \textcolor{white}{\textbf{Grado}}\tabularnewline
	\rowcolor[RGB]{216, 235, 171}
	RX1 & Descrizione & Identificazione & Grado\tabularnewline
	\rowcolor[RGB]{233, 245, 206}
	RX2 & Descrizione & Identificazione & Grado\tabularnewline
	\rowcolor[RGB]{216, 235, 171}
	RX3 & Descrizione & Identificazione & Grado\tabularnewline
	
\end{tabular}	
}

\subsection{Rischi relativi ai costi e ai tempi: "C"}

{\renewcommand{\arraystretch}{1.5}
\begin{tabular}{  >{\centering}p{0.20\linewidth} | >{\centering}p{0.28\linewidth} | >{\centering}p{0.28\linewidth} | >{\centering}p{0.13\linewidth}  }
	\rowcolor[RGB]{33, 73, 50}
	\textcolor{white}{\textbf{Codice}} & \textcolor{white}
	{\textbf{Descrizione}} & \textcolor{white}{\textbf{Identificazione}} & \textcolor{white}{\textbf{Grado}}\tabularnewline
	\rowcolor[RGB]{216, 235, 171}
	RX1 & Descrizione & Identificazione & Grado\tabularnewline
	\rowcolor[RGB]{233, 245, 206}
	RX2 & Descrizione & Identificazione & Grado\tabularnewline
	\rowcolor[RGB]{216, 235, 171}
	RX3 & Descrizione & Identificazione & Grado\tabularnewline
	
\end{tabular}	
}

