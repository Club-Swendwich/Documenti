\section{Modello di sviluppo}
Il gruppo ha deciso di utilizzare un modello di sviluppo Agile.

\subsection {Sviluppo di tipo agile}
Verrà utilizzato il framework \textit{Scrum}$^{G}$, che si basa su i seguenti principi:
\begin{itemize}
    \item \textbf{Trasparenza:} Gli elementi chiave devono essere comprensibili e facilmente consultabili, in maniera tale che una terza parte ne abbia una fruibile comprensione.
    
    \item \textbf{Ispezione:} \textit{Scrum}$^{G}$ fa del suo punto di forza l'utilizzo di una suddivisione in \textit{sprint}$^{G}$ dalla durata prestabilita, questo per cercare di avere sempre una chiara visione dell'obbiettivo.
    
    \item \textbf{Adattamento:} Quando ci si muove verso un obbiettivo, se vengono rilevati dei rallentamenti inaspettati, è necessario reagire di conseguenza.
\end{itemize}
I vantaggi della metodologia Agile permettono di offrire un prodotto funzionante in breve tempo e, strutturando il workflow in sezioni brevi dalla durata prestabilita, è più semplice non perdere il focus sull'obbiettivo. \\
Grazie a questa suddivisione si ottiene anche una maggiore flessibilità e si riducono i possibili rischi dovuti a eventuali ritardi.