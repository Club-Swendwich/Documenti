\section{Introduzione}
\subsection{Scopo del documento}
Il seguente documento ha lo scopo di descrivere pianificazione e modalità di sviluppo relativi al progetto "Login Warrior".
Al suo interno vengono riportati:
\begin{itemize}
    \item preventivo dei costi attesi;
    \item costi osservati;
    \item obiettivi raggiunti/da raggiungere;
    \item potenziali rischi e mitigazioni.
\end{itemize} 
%aggiungere link alle sezioni quando ci saranno tutte

\subsection{Scopo del capitolato}
Il capitolato C5 si pone l'obiettivo di creare un'applicazione per la visualizzazione di dati di login 
per facilitarne l'analisi e l'interpretazione.\\ 
Il prodotto fornirà all'utente la possibilità di scegliere tra diverse
modalità di visualizzazione personalizzabili e algoritmi per l'analisi dei dati. 
Attraverso questo processo esplorativo, EDA \textit{(Exploratory Data Analysis)},
l'utente finale potrà studiare i dati in modo più fruibile ed immediato: obbiettivo ultimo 
di questo capitolato è lo studio di dati di login per indivudare potenziali accessi non autorizzati.

\subsection{Glossario}
Al fine di chiarire possibili ambiguità relative alla terminologia utilizzata all'interno dei documenti è stato redatto il \textit{Glossario} contenente i termini di particolare rilievo,
questi sono contrassegnati con una '\textit{G}' ad apice. 

\subsection{Riferimenti}
\subsubsection{Riferimenti normativi}
\begin{itemize}
    \item Norme di Progetto v2.0.0
    \item \href{https://www.math.unipd.it/~tullio/IS-1/2021/Progetto/Capitolati.html}{Regolamento organigramma}
\end{itemize}
\subsubsection{Riferimenti informativi}
\begin{itemize}
    \item Analisi dei Requisiti v1.0.0
    \item \href{https://www.math.unipd.it/~tullio/IS-1/2021/Progetto/C5.pdf}{Capitolato d'appalto C5 - Login Warrior}
    \item \href{https://www.math.unipd.it/~tullio/IS-1/2021/Dispense/PD2.pdf}{Regolamento del Progetto Didattico}
    \item Slide T5 del corso Ingegneria del Software - \href{https://www.math.unipd.it/~tullio/IS-1/2021/Dispense/T05.pdf}{Il ciclo di vita del SW}
    \item Slide T6 del corso Ingegneria del Software - \href{https://www.math.unipd.it/~tullio/IS-1/2021/Dispense/T06.pdf}{Gestione di progetto}
\end{itemize}

\subsection{Scadenze}
\textit{Club Swendwich} si impegna a rispettare le seguenti scadenze per lo sviluppo del progetto:
\begin{itemize}
    \item Requirements and Technology Baseline: 07/02/21.
    \item Product Baseline: xx/xx/xx.
    \item Customer Acceptance: xx/xx/xx.
\end{itemize}