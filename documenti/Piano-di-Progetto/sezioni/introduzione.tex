\section{introduzione}
\subsection{Scopo del documento}
Il seguente documento ha lo scopo di descrivere pianificazione e modalità relativi allo sviluppo del progetto.
Al suo interno vengono riportati anche altri fattori come potenziali rischi, costi e tempi attesi.

\subsection{Scopo del capitolato}
Il capitolato C5 si pone l'obbiettivo di creare un applicativo in grado di visualizzare dati di diverse dimensioni 
per facilitarne l'analisi e l'interpretazione. Il prodotto fornirà all'utente la possibilità di scegliere tra diverse
modalità di visualizzazione personalizzabili e algoritmi per l'analisi dei dati. Attraverso questo processo esplorativo,
l'utente finale potrà studiare i dati interessati in modo più fruibile ed immediato.

\subsection{Glossario}
Al fine di chiarire possibili ambiguità relative ai termini utilizzati in questo documento, è stato redatto il \textit{Glossario} contenente le terminologie di particolare rilievo.
Questi ultimi sono segnati all'interno del documento con una '\textit{G}' ad apice. 

\subsection{Riferimenti}
\subsubsection{Riferimenti normativi}
\begin{itemize}
    \item Norme di Progetto vX.X.X
    \item Regolamento organigramma
\end{itemize}
\subsubsection{Riferimenti informativi}
\begin{itemize}
    \item Analisi dei Requisiti vX.X.X
    \item Capitolato d'appalto C5 - Login Warrior: Riconoscere e visualizzare i tentativi di accesso non autorizzati: \par \url{https://www.math.unipd.it/~tullio/IS-1/2021/Progetto/C5.pdf}
    \item Slide T5 del corso Ingegneria del Software - Il ciclo di vita del SW: \par \url{https://www.math.unipd.it/~tullio/IS-1/2021/Dispense/T05.pdf}
    \item Slide T6 del corso Ingegneria del Software - Gestione di progetto: \par \url{https://www.math.unipd.it/~tullio/IS-1/2021/Dispense/T06.pdf}
\end{itemize}
\subsection{Scadenze}
\textit{Clubswendwich} si impegna a rispettare le seguenti scadenze per lo sviluppo del progetto:
\begin{itemize}
    \item Revisione dei Requisiti: 
    \item Revisione di Progettazione:
    \item Revisione di Qualifica:
    \item Revisione di Accettazione:
\end{itemize}