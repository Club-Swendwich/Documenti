\section{Pianificazione}
La pianificazione di progetto viene suddivisa nelle seguenti fasi:
\begin{itemize}
    \item \textbf{Analisi};
    \item \textbf{Technology baseline e codifica del PoC$^{G}$};
    \item \textbf{Progettazione di dettaglio e codifica};
    \item \textbf{Verifica validazione e collaudo}
\end{itemize} 
Al momento viene presentata una pianificazione dettagliata delle prime 2 fasi. Ogni fase viene suddivisa in \textit{sprint}$^{G}$ secondo le regole
del framework \textit{Scrum}$^{G}$ dalla durata di circa 1-2 settimane. 

\subsection{Analisi}
Nella fase di analisi il team \textit{Club Swendwich} si occupa di produrre i seguenti documenti: \textit{Analisi dei requisiti}, \textit{Piano di progetto}, \textit{Piano di Qalifica} e \textit{Glossario}.

\subsubsection{Backlog: Analisi}
{\renewcommand{\arraystretch}{1.5}
\begin{longtable}{p{0.27\linewidth}p{0.49\linewidth}p{0.15\linewidth}}
	\rowcolor[RGB]{33, 73, 50}
	\textcolor{white}{\textbf{Titolo attività}} & \textcolor{white}{\textbf{Descrizione}} & \textcolor{white}{\textbf{Data inizio}}\\
    
    \rowcolor[RGB]{216, 235, 171}
    Incremento \par \textit{Norme di Progetto} & Il gruppo prosegue con la stesura delle Norme di Progetto, concentrandosi sulle sezioni "Documentazione" e "Codifica". & 18-11-2021\\

    \rowcolor[RGB]{233, 245, 206}
    Verifica \par \textit{Norme di Progetto} & Il gruppo prosegue con la verifica del documento \textit{Norme di Progetto} & 18-11-2021\\
    
    \rowcolor[RGB]{216, 235, 171}
    Casi d'uso & Il gruppo esegue uno studio preliminare per individuare i casi d'uso principali. & 25-11-2021\\

    \rowcolor[RGB]{233, 245, 206}
    Requisiti & Il gruppo esegue uno studio preliminare per individuare i requisiti (funzionali, vincolo e prestazionali) principali. & 25-11-2021\\

    \rowcolor[RGB]{216, 235, 171}
    Incontro & Il gruppo fissa un incontro con il proponente per discutere dei casi d'uso individuati. & 30-11-2021\\ 
    
    \rowcolor[RGB]{233, 245, 206}
    \textit{Analisi dei requisiti} & Il gruppo redige il documento \textit{Analisi dei requisiti}. & 03-12-2021\\

    \rowcolor[RGB]{216, 235, 171}
    Verifica \par \textit{Analisi dei requisiti} & Il gruppo inizia la verifica del documento \textit{Analisi dei requisiti}. & 03-12-2021\\

    \rowcolor[RGB]{233, 245, 206}
    Analisi dei rischi & Il gruppo individua i rischi principali riscontrabili nello sviluppo del progetto. & 13-12-2021\\

    \rowcolor[RGB]{216, 235, 171}
    Modello di sviluppo & Il gruppo identifica il modello di sviluppo al quale fare riferimento. & 14-12-2021\\

    \rowcolor[RGB]{233, 245, 206}
    Pianificazione & Il gruppo identifica un metodo di pianificazione e individua le fasi principali e i relativi sprint$^G$ . & 14-12-2021\\

    \rowcolor[RGB]{216, 235, 171}
    \textit{Piano di Progetto} & Il gruppo inizia la stesura della documento \textit{Piano di progetto}. & 21-12-2021\\

    \rowcolor[RGB]{233, 245, 206}
    Verifica \par \textit{Piano di Progetto} & Il gruppo inizia la verifa del documento \par \textit{Piano di Progetto}. & 21-12-2021\\

    \rowcolor[RGB]{216, 235, 171}
    Preventivo & Il gruppo elabora i preventivi relativi alle fasi individuate e descritte nella sezione \par "\textit{Pianificazione}" del documento \textit{Piano di Progetto}. & 27-12-2021\\

    \rowcolor[RGB]{233, 245, 206}
    Consuntivo di periodo & Il gruppo elabora un consuntivo di periodo relativo al documento \textit{Piano di progetto}. & 27-12-2021\\

    \rowcolor[RGB]{216, 235, 171}
    Attuazione dei rischi & Il gruppo individua i rischi che si sono verificati nel corso del progetto & 30-12-2021\\

    \rowcolor[RGB]{233, 245, 206}
    \textit{Piano di Progetto} & Il gruppo prosegue con la stesura del documento \textit{Piano di progetto}. & 02-12-2022\\
    
    \rowcolor[RGB]{216, 235, 171}
    \textit{Glossario} & Il gruppo inizia a redigere il documento \textit{Glossario}. & 03-01-2021\\

    \rowcolor[RGB]{233, 245, 206}
    \textit{Piano di Qualifica} & Il gruppo inizia la stesura del documento \textit{Piano di Qualifica}. & 07-01-2022\\

    \rowcolor[RGB]{216, 235, 171}
    Verifica \par \textit{Piano di Qualifica,} \par \textit{Glossario} & Il gruppo inizia la verifica dei documenti \textit{Piano di Qualifica} e \textit{Glossario}. & 07-01-2022\\

    \rowcolor[RGB]{233, 245, 206}
    Verifica finale & Il gruppo inizia la verifica finale di tutti i documenti redatti & 31-01-2022\\

    \caption{Tabella riassuntiva del backlog relativo alla fase di analisi}
\end{longtable}
}

\subsubsection{Suddivisione in sprint$^G$ : Analisi}
\begin{itemize}
    \item \textbf{\textit{Sprint$^G$  1}: (dal 18-11-2021 al 03-12-2021)}\\
    Il gruppo si concentra sulle attività di: incremento delle \textit{Norme di Progetto} e relativa verifica, individuazione dei casi d'uso e dei requisiti.\\
    Il gruppo fissa una \textbf{milestone}$^G$  per il \textbf{03-12-2021} alla quale si aspetta di aver completato la parte relativa alla sezione "\textit{Casi d'uso}" e "\textit{Struttura requisiti}" nel documento \textit{Norme di progetto}.\\
    Il gruppo inoltre fissa il seguente incontro e redige i relativi verbali:
    \begin{itemize}
        \item Incontro con Zucchetti: 03-12-2021;
    \end{itemize}
    Il gruppo esegue la verifica dei verbali interni redatti in questo sprint.

    \item \textbf{\textit{Sprint$^G$  2}: (dal 03-12-2021 al 20-12-2021)}\\
    Il gruppo si concentra sulle attività di: stesura dell'\textit{Analisi dei requisiti} e relativa verifica, Individuazione dei rischi, scelta del modello di sviluppo e pianificazione delle fasi.\\
    Il gruppo fissa una \textbf{milestone}$^G$  per il \textbf{20-12-2021} alla quale si aspetta di aver completato per la maggior parte la stesura del documento \textit{Analisi dei requisiti}, e di essersi concordato sul contenuto delle prime sezioni relative al documento \textit{Piano di progetto}.\\
    Il gruppo inoltre redige e verifica i verbali relativi alle rinunioni verificatesi in questo sprint.

    \item \textbf{\textit{Sprint$^G$  3}: (dal 21-12-2021 al 06-01-2022)}\\
    Il gruppo si concentra sulle attività di: stesura del \textit{Piano di progetto} e relativa verifica, elaborazione di un preventivo e di un consuntivo di periodo, individuazione dei rischi e stesura del \textit{Glossario}.\\
    Il gruppo fissa una \textbf{milestone}$^G$  per il \textbf{01-01-2022} alla quale ci si aspetta di essersi concordati sul contenuto delle sezioni: "\textit{Preventivo}", "\textit{Consuntivo di periodo"} e "\textit{Attuazione dei rischi}".\\
    Il gruppo inoltre redige e verifica i verbali relativi alle rinunioni verificatesi in questo sprint.

    \item \textbf{\textit{Sprint$^G$  4}: (dal 07-01-2022 al 31-01-2022)}\\
    Il gruppo si concentra sulle seguenti attività: stesura del \textit{Piano di qualifica} e relativa verifica, verifica ed incremento del \textit{Glossario} e correzione generale dei documenti redatti.\\
    Il gruppo fissa una \textbf{milestone}$^G$  per il \textbf{31-01-2022} alla quale si aspetta di aver ultimato tutti i documenti necessari per la RTB e di averli verificati.\\
    Il gruppo inoltre redige e verifica i verbali relativi alle rinunioni verificatesi in questo sprint.

    \item \textbf{\textit{Sprint$^G$  5}: (dal 31-01-2022 al 07-02-2022)}\\
    Il gruppo si concentra sull'attività di verifica finale e di approvazione dei documenti redatti  e segnala le ore personali svolte per l'aggiornamento del "\textit{Consuntivo di periodo"}. \\
    Il gruppo fissa una \textbf{milestone}$^G$  per \textbf{07-02-2022} alla quale ci si aspetta di aver ultimato la verifica finale dei documenti e di averli approvati.\\
    Il gruppo inoltre redige e verifica i verbali relativi alle rinunioni verificatesi in questo sprint.
\end{itemize}
\newpage
\subsubsection{Diagramma di Gantt: Analisi}
\begin{figure}[h!]
    \centering
    \includegraphics[scale=0.22]{../../assets/Diagrammi_Gantt/Analisi.png}
    \caption{Diagramma di Gantt - Analisi}
\end{figure}

\subsection{Technology baseline e codifica del PoC}
\subsubsection{Backlog: Analisi}
{\renewcommand{\arraystretch}{1.5}
\begin{longtable}{p{0.27\linewidth}p{0.49\linewidth}p{0.15\linewidth}}
	\rowcolor[RGB]{33, 73, 50}
	\textcolor{white}{\textbf{Titolo attività}} & \textcolor{white}{\textbf{Descrizione}} & \textcolor{white}{\textbf{Data inizio}}\\
    
    \rowcolor[RGB]{216, 235, 171}
    Individuazione \par delle tecnologie & Il gruppo si accorda sulle tecnologie migliori con le quali sviluppare il \textit{Proof of concept}. & 07-01-2022\\

    \rowcolor[RGB]{233, 245, 206}
    Funzionalità \par da sviluppare & Il gruppo concorda le funzionalità più importanti da sviluppare nel \textit{Proof of concept}. & 07-01-2022\\

    \rowcolor[RGB]{216, 235, 171}
    Incontro & Il gruppo fissa un incontro con il proponente per discutere delle scelte tecnologiche intraprese relativamente al \textit{Proof of concept}. & 10-01-2022\\

    \rowcolor[RGB]{233, 245, 206}
    Codifica \par \textit{caricamento dataset$^{G}$} & Il gruppo procede alla codifica della \par funzionalità di "\textit{caricamento del dataset$^{G}$}". & 15-01-2022\\

    \rowcolor[RGB]{216, 235, 171}
    Codifica \textit{Parsing}$^G$  & Il gruppo procede alla codifica della \par funzionalità di \textit{Parsing}$^G$ . & 17-01-2022\\

    \rowcolor[RGB]{233, 245, 206}
    Codifica \par \textit{Rendering$^G$ scatterplot$^{G}$} & Il gruppo procede alla codifica della \par funzionalità di \textit{Rendering$^G$  dello scatterplot$^G$ } & 20-01-2022\\

    \rowcolor[RGB]{216, 235, 171}
    Codifica \par \textit{Associazione assi} & Il gruppo procede alla codifica della \par funzionalità di \textit{associazione degli assi alle dimensioni$^{G}$}. & 22-12-2022\\

    \rowcolor[RGB]{233, 245, 206}
    Verifica e collaudo & Il gruppo procede alla verifica e al collaudo del \textit{Proof of concept}. & 31-01-2022\\

    \caption{Tabella riassuntiva del backlog relativo alla fase di Technology baseline}
\end{longtable}
}

\subsubsection{Suddivisione in sprint: Technology baseline e codifica del PoC}
\begin{itemize}
    \item \textbf{\textit{Sprint$^G$  1}: (dal 07-01-2022 al 15-01-2022)}\\
    Il gruppo si concentra sulle attività di: individuazione delle tecnologie da utilizzare e individuazione delle funzionalità da sviluppare.
    Il gruppo inoltre fissa i seguenti incontri:
    \begin{itemize}
        \item Incontro con Zucchetti: 10-01-2022;
    \end{itemize}
    Il gruppo in aggiunta redige e verifica i verbali relativi alle rinunioni verificatesi in questo sprint.

    \item \textbf{\textit{Sprint$^G$  2}: (dal 15-01-2022 al 31-01-2022)}\\
    Il gruppo si concentra sulle attività di codifica, in particolar modo vengono sviluppate le seguenti funzionalità: \textit{caricamento dataset$^{G}$}, \textit{Parsing}$^G$ , \textit{Rendering$^G$  Scatter Plot$^G$ } e \textit{Associazione degli assi alle dimensioni$^G$ }.\\
    Il gruppo fissa una \textbf{milestone}$^G$  per il \textbf{31-01-2022} alla quale si aspetta di aver codificato tutte le funzionalità necessarie al \textit{Proof of concept}.\\
    Il gruppo in aggiunta redige e verifica i verbali relativi alle riunioni verificatesi in questo sprint.

    \item \textbf{\textit{Sprint$^G$  3}: (dal 31-01-2022 al 07-02-2022)}\\
    Il gruppo si concentra sull'attività di verifica e collaudo del \textit{Proof of concept}.\\
    Il gruppo fissa una \textbf{milestone}$^G$  per il \textbf{07-02-2022} alla quale si aspetta di aver terminato tutte le attività di verifca e collaudo.\\
    Il gruppo in aggiunta redige e verifica i verbali relativi alle rinunioni verificatesi in questo sprint.
\end{itemize}

\subsubsection{Diagramma di Gantt: Technology baseline e codifica del PoC}
\begin{figure}[h!]
    \centering
    \includegraphics[scale=0.22]{../../assets/Diagrammi_Gantt/TB.png}
    \caption{Diagramma di Gantt - Technology baseline e codifica del PoC}
\end{figure}

\newpage
\subsection{Progettazione di dettaglio e codifica}
Nella fase di "Progettazione di dettaglio e codifica" il team \textit{Club Swendwich} si occupa della produzione del documento \textit{Specifica Architetturale}, dell'aggiornamento dei documenti \textit{Piano di Progetto}, \textit{Piano di Qualifica}, \textit{Norme di Progetto}, \textit{Glossario}, e della codifica dell'applicativo.

\subsubsection{Backlog: Progettazione di dettaglio e codifica}
{\renewcommand{\arraystretch}{1.5}
\begin{longtable}{p{0.27\linewidth}p{0.49\linewidth}p{0.15\linewidth}}
	\rowcolor[RGB]{33, 73, 50}
	\textcolor{white}{\textbf{Titolo attività}} & \textcolor{white}{\textbf{Descrizione}} & \textcolor{white}{\textbf{Data inizio}}\\

    \rowcolor[RGB]{216, 235, 171}
    Correzione errori & Il gruppo svolge la correzione degli errori segnalati nella revisione RTB. & 12-03-2022\\

    \rowcolor[RGB]{233, 245, 206}
    Incremento \par Pianificazione & Il gruppo identifica la fase principale di avanzamento e i relativi sprint. & 16-03-2022\\

    \rowcolor[RGB]{216, 235, 171} 
    Verifica \par Pianificazione & Il gruppo verifica l'incremento nel documento \textit{Piano di Progetto}. & 16-03-2022\\

    \rowcolor[RGB]{233, 245, 206}
    Individuazione \par architettura & Il gruppo studia e valuta le possibili architetture per la progettazione dell' applicativo. & 18-03-2022\\

    \rowcolor[RGB]{216, 235, 171}
    Diagrammi delle \par classi & Il gruppo inzia a produrre i diagrammi delle classi della Vista e del Modello. & 19-03-2022 \\

    \rowcolor[RGB]{233, 245, 206}
    Incremento \par Preventivo & Il gruppo elabora il preventivo relativo alla fase individuata e descritta nella sezione "\textit{Progettazione di dettaglio e codifica}". & 21-03-2022\\

    \rowcolor[RGB]{216, 235, 171}
    Verifica \par Preventivo & Il gruppo verifica l'incremento nel documento \textit{Piano di Progetto}. & 21-03-2022\\

    \rowcolor[RGB]{233, 245, 206}
    \textit{Specifica Architetturale} & Il gruppo inizia la stesura del documento. \textit{Specifica Architetturale} & 23-03-2022\\

    \rowcolor[RGB]{216, 235, 171}
    Verifica \par \textit{Specifica Architetturale} & Il gruppo inizia la verifica del documento \textit{Specifica Architetturale}. & 23-03-2022\\

    \rowcolor[RGB]{233, 245, 206} 
    Incremento \par \textit{Norme di Progetto} & Il gruppo prosegue la stesura delle \textit{Norme di Progetto}, aggiornando il way of working per la codifica. & 25-03-2022\\

    \rowcolor[RGB]{216, 235, 171} 
    Verifica \par \textit{Norme di progetto} & Il gruppo inizia la verifica degli incrementi nel documento \textit{Norme di progetto}. & 25-03-2022\\

    \rowcolor[RGB]{233, 245, 206}
    Codifica & Inizio dell'attività di codifica. &  28-03-2022\\
    
    \rowcolor[RGB]{216, 235, 171} 
    Verifica codice & Inizio dell' attività di verifica del codice & 28-03-2022\\

    \rowcolor[RGB]{233, 245, 206}
    Incremento \par \textit{Glossario} & Incremento del documento di \textit{Glossario}. & 28-03-2022\\

    \rowcolor[RGB]{216, 235, 171} 
    Test codice & Inizio dell'attività di test del codice & 7-04-2022\\

    \rowcolor[RGB]{233, 245, 206}
    Incremento \par \textit{Piano di Qualifica} & Il gruppo incrementa il contenuto del documento \textit{Piano di Qualifica} con i test effettuati. & 11-04-2022\\
    
    \rowcolor[RGB]{216, 235, 171} 
    Attualizzazione \par dei rischi & Il gruppo individua i rischi che si sono verificati nel corso del progetto & 11-04-2022\\

    \rowcolor[RGB]{233, 245, 206} 
    Consuntivo di periodo & Il gruppo elabora un consuntivo di periodo relativo al documento \textit{Piano di progetto} & 11-04-2022\\

    \rowcolor[RGB]{216, 235, 171} 
    Verifica finale & Il gruppo inizia la verifica finale di tutti i documenti redatti & 11-04-2022\\

    \rowcolor[RGB]{233, 245, 206}
    Revisione requisiti & Il gruppo incontra il proponente Gregorio Piccoli per ridiscutere i requisiti del progetto  & 04-05-2022\\
    
    \rowcolor[RGB]{216, 235, 171} 
    Redistribuzione \par carico di lavoro & Il gruppo, a seguito dell'incontro con il proponente, riassegna gli incarichi per adattarli ai nuovi requisiti & 05-05-2022\\

\end{longtable}
}

\subsubsection{Suddivisione in sprint: Progettazione di dettaglio e codifica}
\begin{itemize}
    \item \textbf{\textit{Sprint$^G$  1}: (dal 12-03-2022 al 16-03-2022)}\\
    Il gruppo si concentra sulle attività di correzione degli errori segnalati dai docenti dopo la revisione \textit{RTB}.
    
    Il gruppo in aggiunta redige e verifica i verbali relativi alle rinunioni verificatesi in questo sprint.

    \item \textbf{\textit{Sprint$^G$  2}: (dal 16-03-2022 al 25-03-2022)}\\
    Il gruppo si concentra sulle attività di aggiornamento del documento \textit{Piano di Progetto}, e la redazione del documento \textit{Specifica Architetturale}, che include i dettagli riguardanti l'architettura dell'applicativo.\\
    Il gruppo fissa una \textbf{milestone}$^G$  per il \textbf{23-03-2022} alla quale si aspetta di aver aggiornato e verificato Pianificazione e Preventivo di fase, e si aspetta di aver selezionato il modello architetturale che si vuole implementare.\\
    Il gruppo in aggiunta redige e verifica i verbali relativi alle riunioni verificatesi in questo sprint.

    \item \textbf{\textit{Sprint$^G$  3}: (dal 25-03-2022 al 11-04-2022)}\\
    Il gruppo si concentra sull'attività di aggiornamento del documento \textit{Norme di Progetto}, includendo delle nuove sezioni inerenti alla codifica. Esegue inoltre codifica impostando l'architettura, verifica del codice (statica e dinamica), incremento del \textit{Glossario} e test.

    Prosegue inoltre la redazione del documento \textit{Specifica Architetturale}. 
    Il gruppo in aggiunta redige e verifica i verbali relativi alle rinunioni verificatesi in questo sprint.

    \item \textbf{\textit{Sprint$^G$  4}: (dal 11-04-2022 al 16-04-2022)}\\
    Il gruppo si concentra sull'attività di codifica impostando la struttura dell'architettura. 

    Il gruppo in aggiunta redige e verifica i verbali relativi alle rinunioni verificatesi in questo sprint.

    \item \textbf{\textit{Sprint$^G$  5}: (dal 16-04-2022 al 23-04-2022)}\\
    Il gruppo prosegue l'attività di codifica dell'architettura. Inizia lo sviluppo in parallelo dei quattro grafici. 

    Il gruppo in aggiunta redige e verifica i verbali relativi alle rinunioni verificatesi in questo sprint.

    \item \textbf{\textit{Sprint$^G$  6}: (dal 23-04-2022 al 30-04-2022)}\\
    Il gruppo continua l'attività di codifica e di testing. Prosegue inoltre la stesura della \textit{Specifica Architetturale} e viene esteso il \textit{Glossario}.

    Il gruppo in aggiunta redige e verifica i verbali relativi alle rinunioni verificatesi in questo sprint.
\end{itemize}
\medskip
\textbf{N.B}: I seguenti sprint non sono stati pianificati come quelli elencati in precedenza.
\\ A causa di una serie di ritardi che si sono accumulati il gruppo 
ha, dal settimo sprint in avanti, utilizzato un approccio correttivo più che di pianificazione. \\ Viene comunque riportata la suddivisione in sprint che è stata 
decisa in corso d'opera.

\begin{itemize}
    \item \textbf{\textit{Sprint$^G$  7}: (dal 30-04-2022 al 7-05-2022)}\\
    Il gruppo incontra il proponente Gregorio Piccoli per rivedere i requisiti a seguito delle complicanze riscontrate durante lo sviluppo del progetto.
    Il gruppo redistribuisce il lavoro sui nuovi requisiti stabiliti.

    Il gruppo in aggiunta redige e verifica i verbali relativi alle rinunioni verificatesi in questo sprint.

    \item \textbf{\textit{Sprint$^G$  8}: (dal 7-05-2022 al 14-05-2022)}\\
    Il gruppo svolge in parallelo attività di codifica relative a \textit{Scatter Plot}$^G$ e \textit{Sankey Diagram}$^G$. 
    Prosegue inoltre la stesura della \textit{Specifica Architetturale}.

    Il gruppo in aggiunta redige e verifica i verbali relativi alle rinunioni verificatesi in questo sprint.

    \item \textbf{\textit{Sprint$^G$  9}: (dal 14-05-2022 al 21-05-2022)}\\
    Il gruppo inizia la composizione dell'architettura testando i relativi componenti. Prosegue la scrittura dei test e la stesura della \textit{Specifica Architetturale}.

    Il gruppo in aggiunta redige e verifica i verbali relativi alle rinunioni verificatesi in questo sprint.
    \item \textbf{\textit{Sprint$^G$  10}: (dal 21-05-2022 al 28-05-2022)}\\
    Il gruppo si focalizza sul completamento di \textit{Scatter Plot}$^G$, testando il caricamento dell'intero dataset. 

    Il gruppo in aggiunta redige e verifica i verbali relativi alle rinunioni verificatesi in questo sprint.

    \item \textbf{\textit{Sprint$^G$  11}: (dal 28-05-2022 al 05-06-2022)}\\
    Il gruppo prosegue con la codifica di \textit{Sankey Diagram}$^G$ e il testing di \textit{Scatter Plot}$^G$. 
    
    Il gruppo in aggiunta redige e verifica i verbali relativi alle rinunioni verificatesi in questo sprint.

    \item \textbf{\textit{Sprint$^G$  12}: (dal 05-06-2022 al 13-06-2022)}\\
    Il gruppo svolge un'attività di refactoring di alcune parti di codice ed implementa la selezione del tipo di grafico da renderizzare.
    Prosegue inoltre lo sviluppo di \textit{Sankey Diagram}$^G$ e la stesura della \textit{Specifica Architetturale}
    
    Il gruppo in aggiunta redige e verifica i verbali relativi alle rinunioni verificatesi in questo sprint.

\end{itemize}

\subsubsection{Diagramma di Gantt: Progettazione di dettaglio e codifica}
\begin{figure}[h!]
    \centering
    \includegraphics[scale=0.18]{../../assets/Diagrammi_Gantt/DettaglioECodifica.png}
    \caption{Diagramma di Gantt - Progettazione di dettaglio e codifica (fino allo sprint 6)}
\end{figure}