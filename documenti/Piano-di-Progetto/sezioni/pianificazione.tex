\section{Pianificazione}
La pianificazione di progetto viene suddivisa nelle seguenti fasi:
\begin{itemize}
    \item \textbf{Analisi}
    \item \textbf{Progettazione logica}
    \item \textbf{Progettazione di dettaglio}
    \item \textbf{Codifica}
    \item \textbf{Verifica validazione e collaudo}
\end{itemize} 
Ogni fase viene poi suddivisa in attività che dovranno essere svolte e completate nei tempi previsti relativi alla fase stessa.

\subsection{Analisi}
Questa fase inizia dal momento della presentazione dei capitolati d'appalto e termina con la data scelta per la prima revisione RTB (\textit{Requirements and Technology Baseline}), ovvero dal 22-10-2021 al \#\#-\#\#-\#\#\#\#.\\ % TO DO: CAMBIARE LA DATA
In questo periodo verrà eseguita un'approfondita analisi sul capitolato scelto dal team \textit{clubswendwich} e verranno, di conseguenza, redatti i documenti necessari.

\subsubsection{Attività}
\begin{itemize}
    \item \textbf{Scelta del capitolato:} viene fatto uno studio dei capitolati proposti analizzandone gli aspetti negativi e gli aspetti positivi al fine d'identificare il capitolato per cui concorrere;
    \item \textbf{Impegni:} viene stimato un preventivo dei costi e l'ammontare di ore per ciascun ruolo e per ciascuna persona. Viene inoltre fissata la data di ultima consegna del progetto;
    \item \textbf{Norme di Progetto:} vengono definite tutte le norme che il gruppo \textit{clubswendwich} dovrà seguire durante lo sviluppo dell'intero progetto;
    \item \textbf{Piano di Progetto:} documento nel quale le attività, i compiti e le risorse precedentemente analizzate vengono distribuite tra i membri del gruppo \textit{clubswendwich}. Contiene inoltre il calcolo del preventivo per la realizzazione del progetto;
    \item \textbf{Analisi dei Requisiti:} attività nella quale vengono studiati ed individuati i requisiti e i casi d'uso relativi al capitolato scelto dal gruppo \textit{clubswendwich};
    \item \textbf{Piano di Qualifica:} si individuano ed enunciano i metodi con i quali il gruppo ha scelto di garantire la qualità del prodotto;
    \item \textbf{Glossario:} documento nel quale viene data una definizione sintetica a tutti i termini tecnici o ambigui incontrati durante lo svolgimento del progetto.
    \item \textbf{Proof of Concept:} una versione molto ridotta del progetto, codificata allo scopo di dimostrare la fattibilità del progetto stesso.
\end{itemize}

\subsubsection{Periodi}
La pianificazione di questa fase è stata organizzata nei seguenti periodi:
\begin{itemize}
    \item \textbf{Primo periodo:} (dal 22-10-2021 al 18-11-2021) scelta del nome del gruppo e del logo, creazione della mail di riferimento e individuazione degli strumenti per la comunicazione e collaborazione interna. Discussione dei capitolati proposti e conseguente stesura dei documenti \textit{Scelta capitolato} e \textit{Impegni}. \\
    Inizio stesura delle \textit{Norme di Progetto}.\\
    Stesura dei verbali interni relativi alle riunioni avvenute in questa fase.
    
    \item \textbf{Secondo periodo:} (dal 18-11-2021 al 06-01-2022) stesura parallela delle \textit{Norme di Progetto}, del \textit{Piano di Progetto}, dell'\textit{Analisi dei Requisiti}. Iniziata anche la stesura del \textit{Glossario}.\\
    Il gruppo ha fissato una milestone per il 03-12-2021 alla quale ci si aspetta di aver terminato la parte relativa agli use cases e ai requisiti, nel documento \textit{Norme di progetto}.\\
    Il gruppo ha fissato una milestone per il 06-01-2022 alla quale ci si aspetta di aver terminato la parte relativa alla codifica, nel documento \textit{Norme di progetto}.\\
    Redatti i verbali relativi alle riunioni avvenute in questa fase.
    
    \item \textbf{Terzo periodo:} (dal 07-01-2022 al 31-01-2022) stesura del \textit{Piano di Qualifica} e parallelamente sviluppo del \textit{Proof of Concept}.\\
    Termine dei documenti redatti nel secondo periodo.\\
    Il gruppo ha fissato una milestone per il 31-01-2022 alla quale ci si aspetta di aver concluso tutti i documenti necessari alla candidatura per la RTB, e di aver completato la codifica del \textit{Proof of Concept}.\\
    Redatti i verbali relativi alle riunioni avvenute in questa fase.
    
    \item \textbf{Quarto periodo:} (dal 01-02-2022 al 07-02-2022) vine svolta l'attività di verifica finale su tutti i documenti ed eventualmente si completano i documenti in ritardo. Viene controllata la conformità dei documenti rispetto a quanto scritto nelle \textit{Norme di progetto}.
    Si procede successivamente con la verifica del codice relativo al \textit{Proof of Concept} e con i necessari collaudi.\\
    Redatti i verbali relativi alle riunioni avvenute in questa fase. 
\end{itemize}

\newpage
\subsection{Diagramma di Gantt: Analisi}
% TO DO: INSERIRE IL DIAGRAMMA DI GANTT AGGIORNATO E CORRETTO

\subsection{Progettazione Logica}
