\section{Attualizzazione dei rischi}
\label{sec:AttualizzazioneRischi}

{\renewcommand{\arraystretch}{1.5} \small
\begin{tabular}{ >{\centering}p{0.20\linewidth} | >{\centering}p{0.36\linewidth} | >{\centering}p{0.36\linewidth}}
	\rowcolor[RGB]{33, 73, 50}
	\textcolor{white}{\textbf{Codice}} & \textcolor{white}
	{\textbf{Descrizione}} & \textcolor{white}{\textbf{Accorginmenti migliorativi}}\tabularnewline
	\rowcolor[RGB]{216, 235, 171}
	RT2
    & Il gruppo ha riscontrato alcuni problemi relativi alla mancanza di esperienza in alcuni ambiti, il che ha costretto ad un rallentamento delle attività.  
    & In alcuni casi la sola ridistribuzione del carico di lavoro non è stata sufficiente a mitigare il problema. È stato quindi necessario a volte fissare un incontro interno per chiarire lo svolgimento delle attività che destavano incertezze.
    \tabularnewline
    \rowcolor[RGB]{233, 245, 206}
	RI2
    & Per rispettare le norme anti-covid e garantire maggiore sicurezza, il gruppo comunica quasi esclusivamente da remoto, il che comporta, in alcune situazioni, un ritardo nella comunicazione o una sua cattiva interpretazione.
    & È stato necessario l'utilizzo di canali di comunicazione diretta (Es. Telegram) per velocizzare lo scambio di informazioni. \tabularnewline
    \rowcolor[RGB]{216, 235, 171}
    RI3
	& Alcuni membri del gruppo, a causa della sessione invernale, risultano essere meno produttivi e la loro disponibilità è frammentata. 
    & Nonostante la ridistribuzione del carico di lavoro, lo svolgimento del progetto ha comunque subito dei rallentamenti. Per tanto è stato fondamentale avvisare con largo anticipo di eventuali impegni, per favorire un'organizzione qualitativamente superiore.
    \tabularnewline
	\rowcolor[RGB]{233, 245, 206}
    RO1
	& Un membro del gruppo ha contratto il Covid-19 dovendosi così assentare a causa dei sintomi manifestati.
    & Nonostante l'imprevedibilità di questo evento si è cercato di aumentare momentaneamente il carico di lavoro degli altri componenti, in modo da ridurne gli effetti negativi.
    \tabularnewline
\end{tabular}	
}
\\
\noindent
A seguito di questo primo periodo si è deciso inoltre di incrementare il grado di pericolosità del rischio \textbf{RI3} da \textit{Medio} a \textit{Alto}.\\
Si è inoltre osservata la necessità di aggiungere un nuovo rischio \textbf{RT5}, relativo al mancato svolgimento del lavoro nei tempi previsti a causa di impegni accademici, in modo da tale da attuare accorgimenti migliorativi più efficaci nei prossimi periodi.

\subsection{conclusioni}
Come ben evidenziato nel riepilogo del consuntivo, il gruppo \textit{Club Swendwich} ha riscontrato un notevole ritardo rispetto a quanto preventivato, per questo la data di ultima scadenza 
per la candidatura alla revisione RTB è stata posticipata al 16/02/2022.
