\section{Attualizzazione dei rischi}
\label{sec:AttualizzazioneRischi}

{\renewcommand{\arraystretch}{1.5} \small
\begin{tabular}{ >{\centering}p{0.20\linewidth} | >{\centering}p{0.36\linewidth} | >{\centering}p{0.36\linewidth}}
	\rowcolor[RGB]{33, 73, 50}
	\textcolor{white}{\textbf{Codice}} & \textcolor{white}
	{\textbf{Descrizione}} & \textcolor{white}{\textbf{Soluzione}}\tabularnewline
	\rowcolor[RGB]{216, 235, 171}
	RT2
    & Il gruppo ha riscontrato alcuni problemi relativi alla mancanza di esperienza in alcuni ambiti, il che ha costretto ad un rallentamento delle attività.  
    & Il Responsabile di comune accordo con il team ha redistribuito il carico di lavoro in modo da favorire la produzione delle parti più onerose. \tabularnewline
    \rowcolor[RGB]{233, 245, 206}
	RI2
    & Per rispettare le norme anti-covid e garantire maggiore sicurezza, il gruppo comunica quasi esclusivamente da remoto, il che comporta, in alcune situazioni, un ritardo nella comunicazione o una sua cattiva interpretazione.
    & Il gruppo ha utilizzato frequentemente canali di comunicazione diretta per permettere una comunicazione più rapida. \tabularnewline
    \rowcolor[RGB]{216, 235, 171}
    RI3
	& I membri del gruppo, a causa della sessione invernale, risultano essere meno produttivi e la loro disponibilità è frammentata. 
    & Il carico di lavoro è stato distribuito in modo da compensare gli impegni accademici dei singoli cercando di mantenere una buona produttività. \tabularnewline
	\rowcolor[RGB]{233, 245, 206}
    RO1
	& Un membro del gruppo ha contratto il Covid-19 dovendosi così assentare per qualche giorno a causa dei sintomi manifestati.
    & I restanti membri del gruppo hanno intensificato la produzione per compensare l'assenza del compagno.   \tabularnewline
\end{tabular}	
}