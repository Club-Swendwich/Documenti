\section{Tecnologie in uso}

{\renewcommand{\arraystretch}{1.5}
\footnotesize
\begin{longtable}{>{\raggedright\arraybackslash}m{0.40\linewidth}m{0.10\linewidth}m{0.40\linewidth}}
	\rowcolor[RGB]{33, 73, 50}
    
    \multicolumn{1}{>{\centering\arraybackslash}m{0.40\linewidth}}{\textcolor{white}{\textbf{Tecnologia}}} 
    & \multicolumn{1}{>{\centering\arraybackslash}m{0.10\linewidth}}{\textcolor{white}{\textbf{Versione}}} 
    & \multicolumn{1}{>{\centering\arraybackslash}m{0.40\linewidth}}{\textcolor{white}{\textbf{Uso}}}\\
    
    \rowcolor[RGB]{47, 106, 73}
    \multicolumn{3}{>{\centering\arraybackslash}m{0.953\linewidth}}{\textcolor{white}{\textbf{Linguaggi}}}\\

    \rowcolor[RGB]{216, 235, 171}
    Typescript & 4.7.2 & Estensione di JavaScript, utilizzato per la produzione di codice Javascript per il maggior supporto alla tipizzazione.\\

    \rowcolor[RGB]{233, 245, 206}
    Javascript & 1.8.5 & Linguaggio della libreria D3.js su cui si basa il capitolato C5.\\

    \rowcolor[RGB]{216, 235, 171}
    HTML & 5.2 & Utilizzato assieme a React per la creazione dell'interfaccia.\\

    \rowcolor[RGB]{233, 245, 206}
    CSS & 3 & Utilizzato per la definizione dei fogli di stile da implementare nell' applicativo.\\

    \rowcolor[RGB]{47, 106, 73}
    \multicolumn{3}{>{\centering\arraybackslash}m{0.953\linewidth}}{\textcolor{white}{\textbf{Strumenti}}}\\    

    \rowcolor[RGB]{216, 235, 171}
    Npm & 8.5 & Gestore dei pacchetti utilizzato per la creazione e le build del codice Javascript.\\

    \rowcolor[RGB]{233, 245, 206}
    Babel & 7.13.14 & \textit{Transcompiler} di Javascript per convertire il codice ECMAScript 2015+(ES6+) in codice retrocompatibile con le versioni meno aggiornate di alcuni browser.\\

    \rowcolor[RGB]{47, 106, 73}
    \multicolumn{3}{>{\centering\arraybackslash}m{0.953\linewidth}}{\textcolor{white}{\textbf{Librerie e Framework}}}\\ 

    \rowcolor[RGB]{216, 235, 171}
    D3.js & 7.0 & Libreria JavaScript utilizzata per la creazione dei grafici richiesti dal capitolato C5, \textit{Login Warrior}, partendo da un insieme di dati organizzati.\\

    \rowcolor[RGB]{233, 245, 206}
    React & 1.9.2 & Libreria JavaScript per la creazione di \textit{user interface}, utilizzata per facilitare lo sviluppo di un'applicazione dinamica.\\

    \rowcolor[RGB]{216, 235, 171}
    WebGL & 2.0 & \textit{API} grafica per facilitare la renderizzazione dei grafici richiesti, in particolare nei casi in cui sia presente una grossa mole di dati.\\

    \rowcolor[RGB]{233, 245, 206}
    MobX & 6.6.0 & Libreria utilizzata per l'implementazione del patter mvvm e
    rendere i componenti di react reattivi ai cambiamenti di essi.\\

    \caption{Tecnologie utilizzate per lo sviluppo dell'applicativo richiesto dal capitolato C5, \textit{Login Warrior}.}
\end{longtable}
}


