\section{Tecnologie in uso}

{\renewcommand{\arraystretch}{1.5}
\footnotesize
\begin{longtable}{>{\raggedright\arraybackslash}m{0.40\linewidth}m{0.10\linewidth}m{0.40\linewidth}}
	\rowcolor[RGB]{33, 73, 50}
    
    \multicolumn{1}{>{\centering\arraybackslash}m{0.40\linewidth}}{\textcolor{white}{\textbf{Tecnologia}}} 
    & \multicolumn{1}{>{\centering\arraybackslash}m{0.10\linewidth}}{\textcolor{white}{\textbf{Versione}}} 
    & \multicolumn{1}{>{\centering\arraybackslash}m{0.40\linewidth}}{\textcolor{white}{\textbf{Uso}}}\\
    
    \rowcolor[RGB]{47, 106, 73}
    \multicolumn{3}{>{\centering\arraybackslash}m{0.953\linewidth}}{\textcolor{white}{\textbf{Linguaggi}}}\\

    \rowcolor[RGB]{216, 235, 171}
    Typescript & bla & Estensione di JavaScript, utilizzato per la produzione di codice Javascript per il maggior supporto alla tipizzazione.\\

    \rowcolor[RGB]{233, 245, 206}
    Javascript & bla & Linguaggio della libreria D3.js su cui si basa il capitolato C5.\\

    \rowcolor[RGB]{216, 235, 171}
    HTML & bla & Utilizzato assieme a React per la creazione dell'interfaccia.\\

    \rowcolor[RGB]{233, 245, 206}
    CSS & bla & Utilizzato per la definizione dei fogli di stile da implementare all'applicativo.\\

    \rowcolor[RGB]{47, 106, 73}
    \multicolumn{3}{>{\centering\arraybackslash}m{0.953\linewidth}}{\textcolor{white}{\textbf{Strumenti}}}\\    

    \rowcolor[RGB]{216, 235, 171}
    Babel & bla & bla\\

    \rowcolor[RGB]{47, 106, 73}
    \multicolumn{3}{>{\centering\arraybackslash}m{0.953\linewidth}}{\textcolor{white}{\textbf{Librerie e Framework}}}\\ 

    \rowcolor[RGB]{216, 235, 171}
    D3.js & bla & bla\\

    \rowcolor[RGB]{233, 245, 206}
    React & bla & bla\\

    \rowcolor[RGB]{216, 235, 171}
    WebGL & bla & bla\\

    \caption{Valori di riferimento per le metriche dei "Processi primari"}
\end{longtable}
}


