\section{Introduzione}
\subsection{Scopo del documento}
Il seguente documento ha lo scopo di descrivere pianificazione e modalità di sviluppo relativi alle specifiche architetturali adottate per la realizzazione del capitolato C5, \textit{Login Warrior}.
Al suo interno vengono riportati:
\begin{itemize}
    \item tecnologie coinvolte per la realizzazione;
    \item configurazione per l'utilizzo dell'applicazione;
    \item diagrammi delle classi e di sequenza;
    \item contestualizzazione al prodotto dei design pattern adottati.
\end{itemize}
%aggiungere link alle sezioni quando ci saranno tutte

\subsection{Scopo del capitolato}
Il capitolato$^G$  C5 si pone l'obiettivo di creare un'applicazione per la visualizzazione di dati di login$^G$
per facilitarne l'analisi e l'interpretazione.\\
Il prodotto fornirà all'utente la possibilità di scegliere tra diverse
modalità di visualizzazione personalizzabili e algoritmi per l'analisi dei dati.
Attraverso questo processo esplorativo, EDA$^G$  \textit{(Exploratory Data Analysis)},
l'utente finale potrà studiare i dati in modo più fruibile ed immediato: obbiettivo ultimo
di questo capitolato$^G$  è lo studio di dati di login$^G$  per indivudare potenziali accessi non autorizzati.

\subsection{Glossario}
Al fine di chiarire possibili ambiguità relative alla terminologia utilizzata all'interno dei documenti è stato redatto il \textit{Glossario} contenente i termini di particolare rilievo,
questi sono contrassegnati con una '\textit{G}' ad apice.