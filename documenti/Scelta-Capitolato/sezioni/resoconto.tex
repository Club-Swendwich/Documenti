\section{Resoconto incontri}
Resoconto dei capitolati e dei punti principali (favorevoli/sfavorevoli alla scelta del capitolato) che sono stati evidenziati negli incontri di gruppo e nelle conferenze con i proponenti.

\subsection{Capitolato C1 - Bot4Me}
\begin{description}
	\item [Breve descrizione]: creazione di un chatbot per aiutare le persone a interagire con la realtà aziendale.
\end{description}
\begin{table}[h]
\begin{tabularx}{\linewidth}{>{\parskip1ex}X@{\kern4\tabcolsep}>{\parskip1ex}X}
\hfil\bfseries Pro
&
\hfil\bfseries Contro
\\\cmidrule(r{3\tabcolsep}){1-1}\cmidrule(l{-\tabcolsep}){2-2}
% PROS, seperated by empty line or \par
Minore quantità di nuove tecnologie \par
Facile individuare MVP
&
% CONS, seperated by empty line or \par
Progetto di minore interesse tra i membri del gruppo \par
Difficoltà nel gestire le opzioni di interazione client/bot \par
Difficoltà nell'utilizzo del bot per adempire a specifiche di progetto
\end{tabularx}
\end{table}
\begin{description}
	\item [Conclusione]: Anche se il progetto presentava varie idee promettenti (gestione di chat totalmente automatizzate e smart) i punti sfavorevoli e le preferenze del gruppo hanno portato a scegliere un altro capitolato.
\end{description}

\vspace{1cm}

\subsection{Capitolato C2 - Shop Chain}
\begin{description}
\item [Breve descrizione]: realizzazione su blockchain di una piattaforma che si incarichi di ricevere criptovalute e che le trattenga fino a quando l'acquisto viene recapitato all’acquirente.
\end{description}
\begin{table}[h]
\begin{tabularx}{\linewidth}{>{\parskip1ex}X@{\kern4\tabcolsep}>{\parskip1ex}X}
\hfil\bfseries Pro
&
\hfil\bfseries Contro
\\\cmidrule(r{3\tabcolsep}){1-1}\cmidrule(l{-\tabcolsep}){2-2}
% PROS, seperated by empty line or \par
Proponente molto disponibile \par
Interesse molto alto riguardo argomenti e tecnologie impiegate \par
Sviluppo del progetto considerabile ”standard” \par
Sviluppo di parti simili ad un database
&
% CONS, seperated by empty line or \par
Tecnologia Solidity da imparare \par 
Requisiti obbligatori importanti a livello quantitativo \par
Richiesto uno studio di nuovi linguaggi di programmazione \par
Tempi di consegna prevedibilmente lunghi \par
MVP più difficili da individuare. \\
\end{tabularx}
\end{table}
\begin{description}
	\item [Conclusione]: Il gruppo è rimasto molto colpito e interessato dalla prospettiva di lavorare nell'ambito della blockchain, inoltre l'opinione sul proponente è stata molto positiva. Tuttavia dopo un'accurata analisi, considerando in particolar modo i punti sfavorevoli individuati, si è deciso di indirizzarsi verso il capitolato C5 come scelta definitiva.
\end{description}

\vspace{1cm}

\subsection{Capitolato C3 - CC4D}
\begin{description}
\item [Breve descrizione]: creazione di una web application che permetta di censire le macchine produttive e le relative caratteristiche da raccogliere e visualizzare.
\end{description}
\begin{table}[h]
\begin{tabularx}{\linewidth}{>{\parskip1ex}X@{\kern4\tabcolsep}>{\parskip1ex}X}
\hfil\bfseries Pro
&
\hfil\bfseries Contro
\\\cmidrule(r{3\tabcolsep}){1-1}\cmidrule(l{-\tabcolsep}){2-2}
% PROS, seperated by empty line or \par
 
&
% CONS, seperated by empty line or \par
Scarso interesse generale da parte del gruppo \\
\end{tabularx}
\end{table}
\begin{description}
	\item [Conclusione]: Il gruppo non si è soffermato a valutare nel dettaglio il seguente capitolato in quanto non ha suscitato interesse nei membri.
\end{description}

\vspace{1cm}

\subsection{Capitolato C4 - Guida Michelin @ social}
\begin{description}
\item [Breve descrizione]: creazione di una guida Michelin social sulla base di storie e post su TikTok e Instagram
\end{description}
\begin{table}[h]
\begin{tabularx}{\linewidth}{>{\parskip1ex}X@{\kern4\tabcolsep}>{\parskip1ex}X}
\hfil\bfseries Pro
&
\hfil\bfseries Contro
\\\cmidrule(r{3\tabcolsep}){1-1}\cmidrule(l{-\tabcolsep}){2-2}
% PROS, seperated by empty line or \par
Disponibilità per ore di formazione tecnica \par
Integrazione con social network 
&
% CONS, seperated by empty line or \par
Scarso interesse generale da parte del gruppo \par
Complessità del prodotto finale	\par
Elaborazione del prodotto basato su tecnologie dispersive \\
\end{tabularx}
\end{table}
\begin{description}
	\item [Conclusione]: Nonostante il capitolato fornisse spunti di lavoro interessanti, come l'analisi di storie/post su social network, la gran parte del gruppo non ha dimostrato particolare interesse; inoltre il progetto risultava difficile da immaginare nelle sue fasi di codifica a causa della vastità delle tecnologie di sviluppo.
\end{description}


\newpage

\subsection{Capitolato C5 - Login Warrior}
\begin{description}
\item [Breve descrizione]: costruzione di un sistema di analisi esplorativa dei dati ottenuti dalle login per poter costantemente studiare i pattern d’uso regolare e i pattern di attacco
\end{description}


\begin{table}[h]
\begin{tabularx}{\linewidth}{>{\parskip1ex}X@{\kern4\tabcolsep}>{\parskip1ex}X}
\hfil\bfseries Pro
&
\hfil\bfseries Contro
\\\cmidrule(r{3\tabcolsep}){1-1}\cmidrule(l{-\tabcolsep}){2-2}
% PROS, seperated by empty line or \par
MVP ben definiti e fasi di sviluppo del progetto chiare \par
Proponente molto disponibile \par
Progetto sperimentale e innovativo \par
Requisiti obbligatori più contenuti \par 
Maggiore spazio per approfondire ML \par
Familiarità con la maggior parte dei linguaggi di programmazione richiesti
&
% CONS, seperated by empty line or \par
Grande quantità di dati su cui lavorare \par
Studio di nozioni di statistica ed analisi dei dati \par
Progetto con una struttura inusuale \par
Parte di Machine Learning potenzialmente molto vasta e complessa \\
\end{tabularx}
\end{table}
\begin{description}
	\item [Conclusione]: Il capitolato è stato scelto dal gruppo come prima scelta, le motivazioni nel dettaglio sono elencate nella sezione successiva.
\end{description}

\vspace{1cm}

\subsection{Capitolato C6 - Smart4Energy}
\begin{description}
\item [Breve descrizione]: sviluppo di un’applicazione mobile in grado di visualizzare lo stato di funzionamento dell’UPS in tempo reale e sviluppo di un sistema di supporto remoto che faccia da ponte con l'assistenza Socomec.
\end{description}
\begin{table}[h]
\begin{tabularx}{\linewidth}{>{\parskip1ex}X@{\kern4\tabcolsep}>{\parskip1ex}X}
\hfil\bfseries Pro
&
\hfil\bfseries Contro
\\\cmidrule(r{3\tabcolsep}){1-1}\cmidrule(l{-\tabcolsep}){2-2}
% PROS, seperated by empty line or \par
Capitolato molto chiaro \par
Parti grafiche dell'applicazione già fornite
&
% CONS, seperated by empty line or \par
L'azienda non è una software-house \par
Tecnologie sconosciute \par
Emulatore in ambiente MinGW - Windows \\
\end{tabularx}
\end{table}
\begin{description}
	\item [Conclusione]: Nonostante la prontezza da parte dei proponenti nel fornire materiale e chiarimenti inerenti al capitolato, il blocco dell'emulatore in ambito MinGW e il maggior interesse per il capitolato C5 hanno portato il gruppo a non scegliere la seguente proposta.
\end{description}